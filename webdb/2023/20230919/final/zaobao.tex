\entryitemWithDescription{24小时103架次大陆军机台海出没 专家:台军须落实防卫战略}{https://www.zaobao.com/news/china/story20230918-1434662}{在今年7月的长春航空展,中国大陆空军展示吸睛的``军机天团''表演,让运油20(左)同时在空中为歼20匿踪战斗机(中)和歼16型歼击机(右)加油。(互联网) 台湾国防部星期一(9月18日)侦察到24小时内有103架次中国大陆军机在台海出没,创历来新高。受访专家分析,大陆空军已具备突破第一岛链的远程打击续航战力,台军须落实防卫战略,避免擦枪走火并加强吓阻力……}

\entryitemWithDescription{电动车贸易摩擦升级 王毅:中欧应坚定摒弃保护主义}{https://www.zaobao.com/news/china/story20230918-1434638}{中国外长王毅(左)上星期六(9月16日)在瓦莱塔会见马耳他外长博奇。(新华社) 欧盟与中国在电动车产业的贸易摩擦近期持续升级,中国外交部长王毅上星期六(9月16日)在欧洲呼吁中欧摒弃保护主义,坚定支持自由贸易。受访学者分析,王毅的最新表态意在劝说欧盟不要在电动车领域与中国开打贸易战,同时也表达对欧洲加速对华去风险化的担忧……}

\entryitemWithDescription{调查:年龄歧视难租屋、行人地狱难通行 台湾长者的痛点}{https://www.zaobao.com/news/china/story20230918-1434630}{台湾老龄化人口将于2025年超过20\%迈向超高龄社会。台湾人寿与大数据公司``网路温度计''星期一(9月18日)公布``2023高龄友善大调查'',结果显示``年龄歧视难租屋''、``行人地狱难通行'',是台湾老年人长期以来的痛点……}

\entryitemWithDescription{苹果中国官网``辫子客服''引辱华争议}{https://www.zaobao.com/news/china/story20230918-1434625}{苹果中国网站的客户服务人员图片留有长辫子,被部分中国网民认为图片涉嫌辱华,引发争议。(互联网) (上海/北京/长沙综合讯)苹果中国网站一张客户服务人员图片留有长辫子,有中国网民认为图片涉嫌辱华,引发网络争议。中国媒体指出,该客服人员形象并非仅在中国官网存在,而且这名员工是印第安人。 据《新闻晨报》报道,这张陷入争议的照片,来自苹果网站手表专家一对一选购页面;照片中,客服人员留着一条黑色的长辫子……}

\entryitemWithDescription{拐卖11名儿童 贵州人贩子余华英一审判死刑}{https://www.zaobao.com/news/china/story20230918-1434579}{(贵阳综合讯)中国贵州省拐卖11名儿童案备受瞩目,贵阳市法院一审以拐卖儿童罪判处被告人余华英死刑。 贵阳市中级法院官方微信公众号发布消息称,该院星期一(9月18日)公开宣判被告人余华英犯拐卖儿童罪一案,一审判处余华英死刑,剥夺政治权利终身,并处没收个人全部财产。余华英当庭表示要上诉……}

\entryitemWithDescription{于泽远:中美竞相拉拢越南}{https://www.zaobao.com/news/china/story20230918-1434303}{美国总统拜登9月10日访问越南后,中国总理李强9月16日在广西南宁会见了越南总理范明政。 美越、中越高层在一周内相继会面,显示美国和中国都在积极发展对越关系,越南也乐于同美中两个大国密切互动,并从中受益。 拜登访越期间与越共总书记阮富仲宣布,将美越关系从``全面伙伴关系''提升至``全面战略伙伴关系''。此前,越南仅与中国、俄罗斯、印度和韩国建立了``全面战略伙伴关系''……}

\entryitemWithDescription{刘燕玲:期待亚细安和中国自贸区升级后 可实现更深入经济合作}{https://www.zaobao.com/news/china/story20230917-1434340}{新加坡贸工部兼文化、社区及青年部政务部长刘燕玲(左五)星期天(9月17日)在南宁出席中国---东盟博览会开幕式,并参观新加坡展馆。新加坡太平船务执行主席张松声(左六)在展馆向她介绍公司在中国的业务。(新加坡工商联合总会提供) 新加坡贸工部兼文化、社区及青年部政务部长刘燕玲说,亚细安和中国牢固的伙伴关系,源自于双方为各自人民与企业利益而携手合作的共同信念……}

\entryitemWithDescription{李强:对未来有些``隐忧和焦虑'' 中国亚细安需下更大功夫实践亲诚惠容}{https://www.zaobao.com/news/china/story20230917-1434337}{中国国务院总理李强星期天(9月17日)出席第20届中国---东盟博览会开幕式致词时说,对中国与亚细安的未来充满信心和憧憬,也有一些隐忧和焦虑。 (林煇智摄) 中国国务院总理李强指出,中国与亚细安的关系已成为亚太区域合作中最成功和最具活力的典范。但他也不讳言,对未来也有一些``隐忧和焦虑'',呼吁各国在践行``亲诚惠容''下更大功夫……}

\entryitemWithDescription{深圳每天拦数十名可疑偷渡人员 中年人居多}{https://www.zaobao.com/news/china/story20230917-1434336}{(深圳/香港综合讯)香港媒体报道,深圳边防检查部门目前每天要拦下数十名经香港转飞南美洲偷渡美国的中国大陆人士,其中以失业或生意失败的中年人居多。 香港《明报》星期六(9月16日)引述广东省一名接近边检知情人士报道,最近有许多大陆人经深圳前往香港,其中一个目的是经香港飞去第三地,辗转偷渡到墨西哥。 报道称,由于厄瓜多尔对中国公民实施免签政策,因而成为偷渡美国的跳板……}

\entryitemWithDescription{侯友宜提出要把台后备军队人数从34万增加到71万}{https://www.zaobao.com/news/china/story20230917-1434328}{(纽约/台北综合讯)台湾在野的国民党籍总统参选人侯友宜,在美东时间上星期六向美国智库学者表示,他要提升台湾全民的国防意识,后备军队人数从34万增加到71万,还反批民进党政府没教育台人忧患意识。 综合《中国时报》、中天新闻综合报道,侯友宜在美东时间上星期六(9月16日)在纽约出席美国智库外交政策全国委员会(NCAFP)、外交关系委员会(CFR)的对话会,并接受各种提问……}

\entryitemWithDescription{郭台铭申请成为总统被连署人 但不排除在野阵营整合可能性}{https://www.zaobao.com/news/china/story20230917-1434325}{台湾鸿海集团创办人郭台铭(左)和副手搭档赖佩霞,星期天(9月17日)前往中央选举委员会,完成2024年台湾总统副总统选举被连署人登记程序 。(庄慧良摄) 鸿海集团创办人郭台铭和副手搭档赖佩霞,星期天(9月17日)连袂前往中央选举委员会,完成申请成为2024年台湾总统副总统选举被连署人登记程序。 郭台铭竞选办公室强调,``我们是为整合而来'',但郭未排除与在野国民党、民众党总统参选人整合的可能性……}

\entryitemWithDescription{缅甸再向中国移交逾百涉诈嫌犯 共同打击电骗让中缅关系升温}{https://www.zaobao.com/news/china/story20230917-1434312}{据中国公安部介绍,以往潜藏于东南亚国家的电信网络诈骗犯罪团伙,近年来逐渐向缅甸北部地区转移,主要盘踞在与中国云南省西双版纳、普洱、临沧、德宏等地相邻的缅甸一侧。图为9月16日缅甸将109名缅北电信网络诈骗犯罪嫌疑人集中移交中国。(中新社) (北京/内比都综合讯)缅甸再将超过百名电信网络诈骗嫌犯移交中国,前后移交近1500人。有分析指出,共同打击电信诈骗案件令中缅关系升温……}

\entryitemWithDescription{美智库:印太经济框架成员国近10年对华依赖渐升}{https://www.zaobao.com/news/china/story20230917-1434304}{(华盛顿综合讯)美国智库报告发现,印太经济框架(IPEF)成员国中的大多数国家,近10多年对中国贸易依存度逐渐提升,美国实现贸易多边化的目标难度较大。 总部位于美国华盛顿的非盈利智库彼得森国际经济研究所(PIIE)在分析了印太经济框架成员国2010年至2021年的贸易数据后,于9月6日发布上述报告……}

\entryitemWithDescription{G77和中国通过《哈瓦那宣言》 学者料对高科技领域规则制定影响不大}{https://www.zaobao.com/news/china/story20230917-1434291}{77国集团(G77)和中国峰会星期六通过《哈瓦那宣言》,反对科技垄断和其他阻碍发展中国家技术发展的不公平做法,并强调不应限制发展中国家获取信息和通信技术材料、设备和技术。 受访学者分析,该宣言凸显出中国在发展中国家的影响力,但预料对国际高科技领域的规则制定影响不大。 77国集团和中国峰会15日至16日在古巴首都哈瓦那举行……}

\entryitemWithDescription{美国专家:中国超级计算机能力或超所有国家}{https://www.zaobao.com/news/china/story20230917-1434274}{(俄亥俄综合讯)美国田纳西大学教授、图灵奖得主唐加拉(Jack Dongarra)说,中国投入运行的超级计算机数量可能比任何其他国家都要多,但由于受到美国制裁,外界对此知之甚少……}

\entryitemWithDescription{中国特稿:中国突破7纳米芯片技术 中美科技战进入2.0阶段?}{https://www.zaobao.com/news/china/story20230917-1433739}{(早报制图) 中国通讯巨头华为8月底突然开售新款智能手机,搭载据称是中国国产的7纳米芯片,网速达到5G水平,在美国政界引发震荡。中国突破7纳米芯片技术将产生哪些深远影响?华盛顿将如何应对? 中国通讯巨头华为今年8月底突然开售新款Mate 60 Pro智能手机,时逢美国商务部长雷蒙多访华。一个多星期后,华为再预售多两款新手机,在中美科技战开打四年后,释放可能强势重返5G手机市场的信号……}

\entryitemWithDescription{中联办主任首次检阅港警结业会操 称有敌对势力试图扰港}{https://www.zaobao.com/news/china/story20230916-1434050}{作为检阅官,香港中央政府驻港联络办主任郑雁雄(穿黑色西装者)星期六(9月16日)在香港警队结业会操上,检阅警员学员队伍。(中新社) (香港综合讯)香港中央政府驻港联络办主任郑雁雄星期六(9月16日)检阅香港警队结业会操时说,目前仍有外国敌对势力企图破坏香港发展稳定。 综合《星岛日报》和``香港01''报道,香港警察学院星期六举行结业会操,郑雁雄检阅28名毕业见习督察和127名警员……}

\entryitemWithDescription{IMF将敦促中国转向消费驱动增长模式}{https://www.zaobao.com/news/china/story20230916-1434048}{IMF总裁格奥尔基耶娃说,将在最新磋商报告中敦促中国转变经济增长方式,提振国内消费,并解决房地产困境、地方债高企等拖累中国及全球经济增长的问题。(路透社档案图) (华盛顿路透电)国际货币基金组织(IMF)将敦促中国转变经济增长方式,提振国内消费,并解决房地产困境、地方债高企等拖累中国及全球经济增长的问题……}