\entryitemWithDescription{黄循财与上海市长龚正会面}{https://www.zaobao.com/news/china/story20230514-1394648}{到访中国的我国副总理兼财政部长黄循财(左),星期天傍晚与上海市长龚正会面。(通讯及新闻部提供) 正在中国访问的我国副总理兼财政部长黄循财,星期天(5月14日)在上海与当地官员、企业家和旅沪新加坡人会面交流,探讨新沪两地深化合作的机遇。 黄循财此次访华的首个公开活动,是星期天上午参访中国金融科技巨头蚂蚁集团的上海办公楼,并与蚂蚁董事长兼首席执行官井贤栋等集团高层会谈……}

\entryitemWithDescription{山西男子因情感纠纷杀人后开车撞人 致七死11伤}{https://www.zaobao.com/news/china/story20230514-1394645}{中国山西省一名男子因情感纠纷,先杀人、后开车在路上撞人,导致七死11伤。 综合每经网、潮新闻、新京报报道,山西吕梁兴县人民政府新闻办公室星期天(5月14日)通报,一名27岁郭姓男子因感情纠纷,星期六(13日)下午约2时在兴县奥家湾乡沟门前村将一名21岁郭姓女子致伤,并将她的婆婆、丈夫和儿子杀害。 随后,郭姓男子驾驶一辆轿车逃窜,过程中将一名出警人员和13名行人撞倒。该案件共致七人死亡,11人受伤……}

\entryitemWithDescription{中国4月贷款和社融增量环比断崖式下降}{https://www.zaobao.com/news/china/story20230514-1394643}{中国央行公布的金融统计数据显示,中国4月份贷款和社会融资规模增量环比出现断崖式下降,下降幅度远超预期。 据《华尔街日报》报道,中国央行星期四(5月11日)公布的数据显示,4月份人民币贷款增加7188亿元(下同,约1383亿新元),社会融资规模增量为1万2200亿元,月底社会融资规模存量同比增长10\%,同时广义货币供应量(M2)同比增长12.4\%……}

\entryitemWithDescription{金庸创作遭大陆作家侵权终审获胜 获赔188万元人民币}{https://www.zaobao.com/news/china/story20230514-1394642}{金庸2015年发现在中国大陆发行的小说《此间的少年》,所描写人物的名称均来自他的四部作品,且人物间的相互关系、性格特征及故事情节与其作品实质性相似,认为作者江南侵权,将他告上法庭。(互联网) 香港已故武侠小说作家金庸,生前起诉中国大陆作家江南抄袭其小说人物的``同人作品案''二审判决结果出炉。江南权侵和不正当竞争罪名成立,被要求赔偿188万元(人民币,下同,约36万新元)……}

\entryitemWithDescription{广州深圳2022年常住人口出现负增长}{https://www.zaobao.com/news/china/story20230514-1394636}{受冠病疫情、企业用工需求下降、外来人口回流等因素影响,广东省两大城市广州和深圳在2022年均出现人口负增长。图为4月7日广州一家咖啡厅。(法新社) 受冠病疫情、企业用工需求下降、外来人口回流等因素影响,广东省两大城市广州和深圳在2022年均出现人口负增长……}

\entryitemWithDescription{诚品疑似个资外泄 台民众购书后称接统战电话}{https://www.zaobao.com/news/china/story20230514-1394633}{台湾诚品书店疑似出现顾客个人资料外泄情况,有台湾民众称在购书后接到``统战''电话。(互联网) 台湾诚品书店疑似出现顾客个人资料外泄情况,有台湾民众称在购书后接到``统战''电话。 综合《经济日报》《自由时报》等台媒报道,``台湾伫遮计划''副执行秘书杨欣慈2月购入《阿共打来怎么办》一书后,星期六(5月13日)接获自称来自诚品书局的回访市调电话,关切为何要买内容``不恰当''的书……}

\entryitemWithDescription{香港图书馆疑似下架政治学者著作}{https://www.zaobao.com/news/china/story20230514-1394624}{香港媒体报道,香港公共图书馆网站已搜索不到多名香港政治学者的著作,包括香港中文大学政治与行政学系副教授马岳、国际关系学者沈旭晖的作品疑似被下架。 ``香港01''星期六(5月13日)报道,在香港公共图书馆网站搜寻马岳、沈旭晖、香港立法会法律界前议员吴霭仪、前岭南大学文化研究系客席副教授许宝强以及香港政治学者方志恒的名字,均显示``没有符合的检索结果'',即有关作者的书籍可能已被下架……}

\entryitemWithDescription{郭台铭宴请国民党中常委 寻求挺郭反对征召侯友宜}{https://www.zaobao.com/news/china/story20230514-1394618}{鸿海创办人郭台铭积极争取国民党提名参加2024年台湾总统大选。图为郭台铭星期五(5月12日)在新北市参加宗教活动。(路透社) 积极争取国民党提名参选台湾总统的鸿海创办人郭台铭,星期天(5月14日)宴请部分国民党中常委和立委。据台媒报道,郭阵营的目的是,一旦来临的中常会提出征召新北市长侯友宜参选,挺郭的中常委能反对这项决议。 国民党未来一周预计将宣布总统参选人,星期三(17日)的中常会可能会有结果……}

\entryitemWithDescription{台剧《人选之人——造浪者》刻画政治幕僚运作 政客争相引述金句}{https://www.zaobao.com/news/china/story20230514-1394597}{《人选之人》选角出色,主要演员包括饰演文宣部副主任的金马影后谢盈萱(中)、文宣部幕僚王净(左)和主任黄健玮。(取自《人选之人》脸书) 刻画政治幕僚的台剧《人选之人------造浪者》通过串流平台在全球各地上线,在台湾社会掀起热潮,朝野政治人物争相引用剧中金句拼人气。 学者指出,台剧通过出色创意和选角,突破制作经费短缺和监管法规限制,借力串流平台热播,有助宣传台湾选举文化和民主自由的软实力……}

\entryitemWithDescription{【图集】西方媒体中大熊猫形象的转变}{https://www.zaobao.com/news/china/story20230514-1394479}{2010年代开始,随着西方对中国的戒心骤增,中国大熊猫在美国、英国、加拿大等西方媒体漫画中的形象出现较大变化。原本圆润矮小的身形变得壮硕高大,表情也变得更加凶悍……}

\entryitemWithDescription{从招财萌猫变功夫熊猫 中国熊猫外交的变迁与困境}{https://www.zaobao.com/news/china/story20230514-1394411}{去年底开始引发中国舆情的旅美大熊猫丫丫事件,令中国实行近70的熊猫外交成为中国内外的舆论焦点。熊猫外交在中国不同阶段的发展中发挥了什么角色?中国强硬的战狼外交风格受到质疑之际,熊猫能否继续扮演中国``亲善大使''的角色?西方媒体视角中的熊猫,是如何从原本温驯可爱的形象,步步演变成强悍的猛兽……}

\entryitemWithDescription{中西视角下熊猫形象分歧日益扩大}{https://www.zaobao.com/news/china/story20230514-1394430}{过去十几年来,中西视角下的大熊猫形象分歧日益扩大。在中国,熊猫是圆润可爱、穿上坚硬外壳的冰墩墩;但在西方媒体中,熊猫的形象已渐渐变成是浑身肌肉、巨大霸道的专制主义象征。~ 近70年的中国熊猫外交,使熊猫成为高度政治化的符号。随着中西之间的经济和地缘政治竞争加剧,西方媒体中的熊猫形象也出现了直观的改变……}

\entryitemWithDescription{中国首个mRNA冠病疫苗在石家庄接种}{https://www.zaobao.com/news/china/story20230513-1394326}{河北石药集团开发的中国国产mRNA(信使核糖核酸)冠病疫苗,星期六(5月13日)在石家庄一个社区卫生服务中心接种了获批以来的全国首针。 《河北日报》星期六披露了上述消息。 石药集团mRNA疫苗是于3月22日获中国国家卫生健康委员会批准,被纳入紧急使用,可用于18岁以上人群加强免疫接种,并在4月10日被国家卫健委列为加强免疫优先推荐疫苗。它也是目前唯一获批在中国大陆使用的mRNA疫苗……}

\entryitemWithDescription{中国原民生银行副行长退休三月后被查}{https://www.zaobao.com/news/china/story20230513-1394323}{中国又有一名曾在金融监管部门任职多年的官员被查,显示中国金融反腐态势逐步扩大。 中央纪委国家监委在中国银保监会派驻的纪检监察组和黑龙江纪委监委星期六(5月13日)发出消息,称中国民生银行管理咨询委员会原副主席邢本秀涉嫌严重违纪违法,目前正接受调查……}

\entryitemWithDescription{副总理兼财政部长黄循财星期六起访华}{https://www.zaobao.com/news/china/story20230513-1394322}{我国副总理兼财政部长黄循财5月13日至17日正式访问中国,这是他首次以副总理身份访华。(取自黄循财脸书) 我国副总理兼财政部长黄循财于5月13日至17日对中国进行正式访问,他将与中国总理李强、副总理丁薛祥等高层官员会面。 根据总理公署发布的文告,黄循财是应丁薛祥之邀访华,他将到访上海和北京。这是黄循财首次以副总理身份访华……}

\entryitemWithDescription{台民调:侯友宜支持度领先郭台铭12.7个百分点}{https://www.zaobao.com/news/china/story20230513-1394318}{最新民调显示,台湾新北市长侯友宜的支持度继续领先鸿海集团创办人郭台铭。分析称,最后关头郭台铭需要``弯道超车''才可超越侯友谊。 台湾民意基金会星期六(5月13日)发布最新民调结果,显示侯友宜支持度42.1\%,郭台铭支持度29.4\%,侯友宜领先郭台铭12.7个百分点……}

\entryitemWithDescription{``五个女博士''品牌涉发布低俗广告 被立案调查}{https://www.zaobao.com/news/china/story20230513-1394314}{功能饮料品牌``五个女博士''近期一则广告被指侮辱女性。图为该视频广告的部分截图。(互联网) 中国一家名为``五个女博士''的胶原蛋白饮料品牌因涉嫌发布低俗广告,被市场监管部门立案调查。 在``五个女博士''引发争议的电梯广告中,几名女性表情夸张地展示在不同情况喝下该品牌的胶原蛋白饮料:``老公气我,喝''\,``熬夜追剧,喝''\,``又老一岁,喝''\,``喝五个女博士,都是你们逼的……}

\entryitemWithDescription{中国领导人G7峰会前密集访欧 秦刚吁中欧共同反对``新冷战''}{https://www.zaobao.com/news/china/story20230513-1394305}{中国国务委员兼外长秦刚(左)5月12日在挪威奥斯陆同挪威外长维特费尔特举行联合记者会时,向欧洲国家隔空喊话。(路透社) 七国集团峰会即将召开,中国国务委员兼外长秦刚星期五(5月12日)在挪威强调新冷战只会带来更大灾难,并呼吁中欧共同反对新冷战,带头推进大国协调和良性互动……}

\entryitemWithDescription{尊子漫画停刊下架 香港新闻自由再引关注}{https://www.zaobao.com/news/china/story20230513-1394298}{香港《明报》刊登政治漫画专栏``尊子漫画''已有40年。图为5月10日的《明报》封面版与刊登在内页的``尊子漫画''。(法新社) 在香港《明报》已有40年历史的政治漫画专栏``尊子漫画'',突然宣布从星期日(5月14日)起停刊。该专栏过去半年多以来屡次被香港官员指责``抹黑政府'',这次停刊再次引起了业界担忧香港的新闻自由状况……}

\entryitemWithDescription{早点:一封电邮搅混2024台湾总统战局}{https://www.zaobao.com/news/china/story20230513-1394091}{台湾鸿海集团创办人郭台铭近日重提2021年采购BNT疫苗(也称辉瑞疫苗)往事,星期二(5月9日)爆料指时任总统府秘书长李大维曾替总统蔡英文传话:``大小姐说,你还是不要买了'',瞬间在台湾政坛掀起波涛。 政治圈众所皆知、总统府御用的陶姓记者当晚深夜发了一则独家消息,强调``郭台铭早知BNT不卖他!仍甩锅蔡英文挡疫苗'',并同时公布一封BNT大股东2021年6月16日写给郭台铭的私人电邮……}

\entryitemWithDescription{新闻人间:``95后董明珠''孟羽童离职格力}{https://www.zaobao.com/news/china/story20230513-1394102}{孟羽童 年仅22岁就在万众瞩目下成为中国家电巨头格力电器董事长董明珠的秘书、曾被赋予``接班人''厚望的孟羽童,不到两年就从格力离职了。 中国网民星期二(5月9日)注意到,格力电器旗下的``明珠羽童精选''抖音账号更名为``格力明珠精选'',并删除了与孟羽童有关的内容。格力次日公开证实,孟羽童已经从公司离职,并称这是正常的人员流动。 孟羽童离职的话题,一度冲上了新浪微博热搜榜首……}

\entryitemWithDescription{中国特别代表下周出访乌克兰等国斡旋 分析:``试水温''之旅暂难取得实质成果}{https://www.zaobao.com/news/china/story20230512-1393996}{中国政府欧亚事务特别代表、前驻俄罗斯大使李辉,将从下周一(5月15日)起访问乌克兰、波兰、法国、德国和俄罗斯,就政治解决乌克兰危机同各方沟通。 受访学者分析,李辉此行是``试水温''之旅,虽不太可能在解决问题上取得具体和实质的成果,但却是外交调解俄乌战争的重要一步。 中国外交部发言人汪文斌星期五(5月12日)在例行记者会上宣布李辉出访消息……}