\entryitemWithDescription{杨丹旭:孙任泽的非正常死亡}{https://www.zaobao.com/news/china/story20240117-1462534}{刚看到同事传来的这篇报道,我以为我穿越了。这样的事情在十几二十年前时有耳闻,没想到还在上演。 中国媒体《财新》上周六发表一篇题为《嫌疑人孙任泽之死》的特稿。这名当时不满31岁的年轻人在2018年3月因为涉嫌寻衅滋事,被新疆伊犁的公安机关刑事拘留。同年9月,孙任泽在看守所接受审讯时昏迷,辗转在多家医院接受救治后,最终在当年11月不治身亡……}

\entryitemWithDescription{AIT主席罗森伯格对瑙鲁与台断交表示``遗憾''}{https://www.zaobao.com/news/china/story20240116-1462531}{美国在台协会(AIT)主席罗森伯格(Laura Rosenberger,右)星期二(1月16日)由发言人邓艾德(Ed Dunn)陪同,在台北开记者会回应瑙鲁与台湾断交议题。(路透社) 美国在台协会(AIT)主席罗森伯格(Laura Rosenberger)星期二(1月16日)对南太平洋岛国瑙鲁决定与台湾断交表示``遗憾''(unfortunate)……}

\entryitemWithDescription{国民党评估拿下立院龙头可能性高 赵少康建议立院副院长让给民众党}{https://www.zaobao.com/news/china/story20240116-1462517}{台湾立法院龙头改选在即,以关键少数自居的在野民众党立委开出改革立法院四条件,但执政的民进党立院党团以立法院正、副院长是否适任为重,最大在野国民党也以尊重立院党团意见为由,均给民众党软钉子碰。 依立法院长选举办法,第一轮选举若未过半,可以有第二轮选举,故除非民众党八席立委集体支持民进党,否则作为第一大党的国民党不分区首席当选立委韩国瑜,笃定当选立法院长……}

\entryitemWithDescription{小马可斯罕见祝贺赖清德当选 学者:应审慎看待不过度解读}{https://www.zaobao.com/news/china/story20240116-1462512}{菲律宾总统小马可斯罕见发文祝贺赖清德当选台湾总统后,菲律宾外交部和总统府隔天相继重申其``一个中国''政策。不过,北京仍对小马可斯言论表达强烈不满,并召见菲律宾驻华大使提出严正交涉。 受访台湾学者认为,应审慎看待小马可斯此次发文,不应过度解读,菲律宾对台议题的政策大方针未变,也未见台菲关系有重大突破迹象。 台湾民进党主席赖清德星期六(1月13日)在总统选举中胜出……}

\entryitemWithDescription{北京就美日欧等国拟派团赴台致贺严正交涉}{https://www.zaobao.com/news/china/story20240116-1462504}{针对美国、日本和欧洲几个国家据报表示拟于台湾选举后派团赴台致贺,中国大陆外交部回应称,中国大陆已对有关国家表示强烈不满和坚决反对。图为台湾民众1月14日在台北著名景点中正纪念堂附近散步。(法新社) (北京/新加坡综合讯)针对台湾媒体报道,美国、日本和欧洲几个国家表示拟于台湾选举后派团赴台致贺,中国大陆外交部发言人在例行记者会上回应称,中国大陆已对有关国家表示强烈不满和坚决反对,并已提出严正交涉……}

\entryitemWithDescription{预计达90亿人次 中国今年春运流动量将创新高}{https://www.zaobao.com/news/china/story20240116-1462494}{(北京综合讯)中国2024年春运将在下星期五(1月26日)启动,预计今年春运中国跨区域人员流动量将创历史新高,达到90亿人次。 据中国青年网报道,中国交通运输部副部长李扬星期二(1月16日)在发布会上说,今年中国春运发生了结构性变化,传统营业性运输包括铁路、公路、民航、水运客运出行人次预计达18亿人次,其余80\%都将自驾车出行,自驾车出行将创历史新高。 今年中国春运将持续至3月5日……}

\entryitemWithDescription{新疆阿勒泰多处雪崩 逾千游客滞留景区一周}{https://www.zaobao.com/news/china/story20240116-1462479}{经过连日清雪作业,新疆阿勒泰地区布尔津县的雪阻路段于星期二(1月16日)疏通,受困近一周的喀纳斯景区游客纷纷开车离开。(新华社) (北京/武汉综合讯)受持续暴雪影响,新疆阿勒泰地区通往喀纳斯景区部分路段发生雪崩,积雪深度达数米,逾千名游客近一周来被困景区,雪阻路段星期二(1月16日)终于疏通……}

\entryitemWithDescription{【东谈西论】赖清德当选会如何冲击两岸关系?}{https://www.zaobao.com/news/china/story20240116-1462449}{民进党赖清德以558万票(40.05\%)当选台湾总统,被外界认为是``少数总统''领导一个弱势政府。 (法新社) 2024年台湾总统选举已经在1月13日落幕。民进党籍的候选人赖清德以四成选票在三角战中胜出。 迎接赖清德的绝非坦途。民进党失去了立法院多数优势,他将是一个弱势总统,领导一个弱势政府,面对来自内外的压力。美国总统拜登在第一时间就直言,美国不支持台独……}

\entryitemWithDescription{戴庆成:香港告别``三权分立''?}{https://www.zaobao.com/news/china/story20240116-1462300}{香港律政司宪制及政策事务科上个月在《基本法简讯》一篇探讨香港政制的文章中,提及特区权力全属中央政府转授,使用``三权分立''描述政制并不妥当。(路透社档案图) 台湾民进党在刚刚过去的周六(1月13日)如预料赢得总统大选,不过在立委选举的表现却不理想,由国民党取而代之成为立法院最大党……}

\entryitemWithDescription{被誉为``台湾曼德拉''的民进党前主席施明德过世}{https://www.zaobao.com/news/china/story20240116-1462314}{1993年11月,施明德接任民进党主席,并于1994年5月蝉联党主席。1996年3月,因民进党在台湾选举中大败,施明德请辞党主席。(自由时报) 被誉为``台湾曼德拉''的民进党前主席施明德星期一(1月15日)与世长辞,享寿83岁。 这一天也是施明德的生日,施明德的女儿施笳和施蜜娜在脸书发文说,施明德没有忌日,只有生日,一生所捍卫的精神和价值会在世界诞生、长长久久地茁壮……}

\entryitemWithDescription{美代表团选后访台 大陆表示坚决反对干涉台事务}{https://www.zaobao.com/news/china/story20240116-1462313}{美国跨党派代表团星期一(1月15日)在台北的民进党中央党部与民进党籍的正副总统当选人赖清德(右二)和萧美琴(右一)会面。美国代表团成员包括前国安顾问哈德利(右三)。(法新社) 由美国前高官组成的跨党派代表团,在台湾大选隔天高调抵台,并在星期一(1月15日)一天内会见台湾总统蔡英文、民进党籍的正副总统当选人赖清德和萧美琴,以及落败的两名在野党候选人。对此,中国大陆表示坚决反对……}

\entryitemWithDescription{瑙鲁与台湾断交 学者:北京给赖清德下马威 释出坚决反台独信号}{https://www.zaobao.com/news/china/story20240115-1462310}{瑙鲁星期一(1月15日)毫无预警宣布与台湾断交后,位于台北天母的使馆特区外悬挂的瑙鲁国旗同一天被撤下。(路透社) 台湾民进党主席赖清德赢得总统选举后不到48小时,台湾友邦瑙鲁星期一(1月15日)毫无预警宣布与台湾断交,与中国大陆复交。 台海两岸之间的``邦交国战'',一向是北京压缩台湾国际空间的手段之一,以及北京释放警告的方法……}

\entryitemWithDescription{王毅首度就红海紧张局势公开发声 分析:不点名吁胡塞及美英保持克制}{https://www.zaobao.com/news/china/story20240115-1462301}{中国外长王毅星期天(1月14日)在埃及开罗首度就红海紧张局势公开发声,呼吁停止袭扰民船。(路透社) 也门胡塞武装持续袭击红海船只近两个月之际,正在非洲出访的中国外长王毅在埃及首度就红海紧张局势公开发声,呼吁停止袭扰民船,也表示安理会从未授权任何国家对也门使用武力,应避免给局势火上浇油,不点名呼吁胡塞武装及美国和英国在红海保持克制……}

\entryitemWithDescription{台湾选举落幕 台股小涨 新台币走弱}{https://www.zaobao.com/news/china/story20240115-1462283}{台湾股市星期一(1月15日)小幅上涨。图为台湾证券交易所屏幕上显示的股票数据。(彭博社) (台北综合讯)台湾选举结果尘埃落定,台湾股市星期一(1月15日)小涨,新台币兑美元则走弱。 综合彭博社和《经济日报》报道,台股开盘时上涨超过百点,一度重返1万7600点大关,但最终收盘时仅上涨33.99点。 至于新台币兑美元汇率,盘中一度重贬逾一角,最后收在31.215元、贬值8.6分,汇价创近一个月新低……}

\entryitemWithDescription{李强:中国开放大门只会越开越大}{https://www.zaobao.com/news/china/story20240115-1462267}{中国总理李强(右二)与瑞士联邦主席阿姆赫德(左二)1月14日共同乘专列自苏黎世前往瑞士首都伯尔尼,并在途中进行茶叙。(新华社) 中国总理李强星期天(1月14日)起在瑞士达沃斯出席世界经济论坛2024年年会并对瑞士、爱尔兰进行正式访问,他与瑞士联邦主席阿姆赫德会面时,强调中国开放的大门只会越开越大,欢迎更多瑞士企业赴华投资兴业……}

\entryitemWithDescription{​蓝绿争台立法院龙头 民众党称改革立场比人选重要}{https://www.zaobao.com/news/china/story20240115-1462248}{(台北综合讯)台湾立法院龙头之争继续发酵,扮演关键少数的民众党提出具体条件,要求民进党和国民党有意角逐立法院正副院长的人士先表达改革立场,并称``我们根本不在乎是谁''。 综合台媒报道,台湾2024年大选落幕,立法院席次朝野三党均不过半,拥有八个立委席次的民众党成关键少数。外界关注在2月1日开议后的立法院正副院长选举,是否出现国民党和民众党(蓝白)合作或民进党和民众党(绿白)合作……}

\entryitemWithDescription{外交部:将在我国的``一个中国''政策基础上继续发展对台友好关系}{https://www.zaobao.com/news/china/story20240115-1462058}{我国外交部祝贺赖清德和民进党胜选,并强调将在新加坡的``一个中国''政策基础上,继续发展对台友好关系。 根据我国外交部官网1月14日发布的文告,外交部发言人在回应媒体询问时说,我们乐见台湾选举圆满结束,祝贺赖清德和他的政党胜选。发言人说:``新加坡会在我们的`一个中国'政策基础上,继续发展与台湾和台湾人民密切和长久的友好关系。'' 外交部说,新加坡一贯支持两岸关系和平发展……}

\entryitemWithDescription{于泽远:赖清德上台将引发台海战争?}{https://www.zaobao.com/news/china/story20240115-1462077}{台湾大选结果出炉,民进党候选人赖清德、萧美琴当选台湾下届正副总统。中国大陆有不少人认为,素有``台独金孙''之称的赖清德上台,将使两岸关系继续恶化,甚至引发台海战争。 实际上,去年11月国民党与民众党``蓝白合''破局后,这次台湾大选就没有了多少悬念。虽然岛内不希望民进党继续执政的民众占据多数,但无论国民党还是民众党都很难集合不支持民进党的选票,民进党靠基本盘就能击败另外两党的对手……}

\entryitemWithDescription{民进党流失大量年轻选票受冲击最大 分析:年轻人不满没改善高房价切身问题}{https://www.zaobao.com/news/china/story20240114-1462119}{民进党支持者星期天(1月13日)在台北的竞选总部外,为赖清德当选总统而欢呼。不过,民进党流失大量年轻选票,赖清德接棒的是总统得票、立委席次``双不过半''的弱势政府。(彭博社) 台湾大选结果显示,蓝绿两大党都流失大量年轻选票。学者分析,绿营的民进党若不回应和改善年轻人在乎的切身课题,在两年后的地方选举将持续面对威胁;蓝营的国民党则须考虑修正路线、争取主流民意,才有望争取年轻选民并重返执政……}

\entryitemWithDescription{台大选隔天美国高级代表团抵台 王毅:任何台独行动都将受到严厉惩罚}{https://www.zaobao.com/news/china/story20240114-1462115}{美国前国家安全顾问海德利(右二)和美国在台协会主席罗森博格(左二)星期天(1月14日)晚抵达台北,台湾外交部北美司司长王良玉(左一)在机场迎接。美国在台协会台北办事处处长孙晓雅(Sandra Oudkirk)(右一)也前往迎接。(台湾外交部网站) (台北综合讯)台湾大选结束次日,由前高官组成的美国高级代表团旋即抵台;多个西方国家也就台湾选举表态……}

\entryitemWithDescription{胡锡进反驳大陆舆论对台政策质疑声 称大选谁赢绝非最重要}{https://www.zaobao.com/news/china/story20240114-1462114}{民进党赢得台湾2024总统大选后,中国大陆网络舆论出现质疑大陆对台政策的声音。图为民众1月9日从台北剑潭山眺望台北101大楼。(法新社) (北京综合讯)台湾民进党组合赢得总统大选后,中国大陆网络舆论出现质疑大陆对台政策的声音。大陆《环球时报》前总编辑胡锡批评这种观点是短视和幼稚,他认为选举结果对两岸整体格局影响有限,强调谁胜出绝非最重要的事情,台湾问题应交由专业团队来处理……}

\entryitemWithDescription{民众党关键地位左右立法院龙头之争}{https://www.zaobao.com/news/china/story20240114-1462112}{台湾2024年大选落幕,立法院席次出现朝野三党不过半,2月1日开议的立法院龙头之争备受瞩目。o国民党不分区立委第一名的韩国瑜能否成为立法院长仍有悬念。 在野国民党(蓝)于大选中跃升为第一大党,拥有52席立委,加上有案在身以无党籍参选的国民党籍立委员陈超明,以及亲蓝的无党籍原住民立委高金素梅,共有54席,距离过半的57席只差3席,但国民党不分区立委第一名的韩国瑜能否成为立法院长仍有悬念……}