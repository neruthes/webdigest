\entryitemWithDescription{沈泽玮:中美进入深水区近战肉搏}{https://www.zaobao.com/news/china/story20250609-6626426}{中美元首通电话打破僵局后,两国新一轮谈判将于6月9日在英国伦敦登场。 中国外交部称它为``中美经贸磋商机制首次会议''。换句话说,双方5月初在瑞士日内瓦谈判建立起的经贸磋商机制,将迎来第一次会议。 从这个角度看,日内瓦谈判只是热身,休战90天只为缓冲,真正的深水区近战肉搏战将自伦敦谈判开打。全世界都等着看中美怎么谈、谈出什么来……}

\entryitemWithDescription{中美伦敦会谈前夕 中国称已批准一定数量的稀土出口许可申请}{https://www.zaobao.com/news/china/story20250608-6625081}{(北京综合讯)中国官方称已批准一定数量的稀土出口许可申请,在中美贸易会谈前夕,此举或有助于缓解两国之间的紧张局势。 中国商务部新闻发言人星期六(6月7日)晚在官网以答记者问的形式说,稀土相关物项具有军民两用属性,对其实施出口管制符合国际通行做法。中国依法对稀土相关物项实施出口管制,目的是更好维护国家安全和利益,履行防扩散等国际义务,体现了坚持维护世界和平与地区稳定的一贯立场……}

\entryitemWithDescription{监督机制欠缺 中国互联网企业贪腐案数量增加}{https://www.zaobao.com/news/china/story20250608-6468952}{中国官方针对互联网行业的反腐行动这些年来持续推进,官方数据显示,相关贪腐案件近年增加,且呈现人员年轻化、``小官巨贪''等特点。 受访学者指出,中国对于企业贪腐,尚未建立完善的社会治理机制,加上``流量为王''的时代给予互联网平台巨大的``平台软权力'',形成腐败温床。 北京市海淀区法院5月15日发布白皮书,通报海淀区过去五年审理的互联网企业贪腐案件概况,并分析这类犯罪背后的成因……}

\entryitemWithDescription{港媒:章立凡今年3月病逝}{https://www.zaobao.com/news/china/story20250608-6621755}{(香港讯)中国历史学者章立凡因病去世,享年74岁。 香港《明报》星期天(6月8日)报道上述消息,没有具体说明病因。报道引述知情者说,章立凡曾中风,今年3月去世,他的家人受到非常大的压力,因此他的死讯及丧事都处理得非常保密。报道没有说明章立凡的家人为何受到非常大的压力。 知情者还说,章立凡的骨灰星期六(7日)下葬在北京怀柔的九公山陵园……}

\entryitemWithDescription{台湾举行海安演习 应对大陆灰色地带威胁}{https://www.zaobao.com/news/china/story20250608-6625660}{(高雄综合讯)台湾军方与海巡署举行两年一度的海域安全军事演习,演练联合作战能力,应对来自中国大陆日益加剧的灰色地带威胁。 综合路透社、自由时报、风传媒等报道,台湾海洋委员会海巡署与国防部于星期天(6月8日)海洋日在高雄港举行``海安12号演习'',实兵操演海上拦截围捕、空中垂降救援,以及舰艇海空分列式三大项目……}

\entryitemWithDescription{李家超:香港不会向美国实施报复性关税}{https://www.zaobao.com/news/china/story20250608-6620899}{(香港综合讯)香港特区行政长官李家超承诺,即使中美地缘政治紧张局势进一步升级,香港也不会向美国征收报复性关税,以免破坏自身``成功基因''并危及自由港地位。 香港《南华早报》星期天(6月8日)刊登对李家超的专访。这名上任即将三年的特首说,香港的成功因素在于开放性,而香港作为贸易枢纽的地位得益于零关税和自由港政策,``你不会摧毁自己的成功基因''……}

\entryitemWithDescription{中国官方智库:南中国海周边国家要自己掌握解决问题钥匙}{https://www.zaobao.com/news/china/story20250608-6620837}{(北京综合讯)中国官方智库发布报告称,地区国家要把解决南中国海问题的钥匙掌握在自己手中,反对域外势力插手干涉。 据新华社报道,在第17个世界海洋日到来之际,新华社国家高端智库星期天(6月8日)发布中英文报告《中国将南海打造成和平、友谊、合作之海的实践》。 报告说,南中国海稳,则地区国家受益;南中国海乱,则地区国家遭殃;南中国海地区和平稳定是包括中国在内地区国家的共同愿望,符合各国利益……}

\entryitemWithDescription{中国特稿:晶片围困华 稀土将军美}{https://www.zaobao.com/news/china/story20250608-6581514}{``中东有石油,中国有稀土。'' 据中国媒体报道,这是已故中国改革开放总设计师邓小平在1992年说过的话。 邓小平或许没料到,33年后预言成真,稀土成了中国与美国地缘政治博弈的王牌……}

\entryitemWithDescription{美中6月9日伦敦会谈料聚焦稀土科技关税 学者:有望取得实质进展}{https://www.zaobao.com/news/china/story20250607-6605709}{美中领导人通话不到两天,美国总统特朗普星期五(6月6日)率先宣布,美方经贸官员将于星期一(6月9日)在伦敦与中方代表举行会谈。他也称,北京已同意恢复向美国供应稀土。 中国外交部发言人星期六(6月7日)晚随后证实,称中国国务院副总理何立峰将于6月8日至13日访问英国,其间,将与美方举行中美经贸磋商机制首次会议。 受访学者预料,中美在伦敦会谈将聚焦稀土、科技与关税等议题……}

\entryitemWithDescription{领导人峰会在即 中欧就电动车稀土等贸易问题取得进展}{https://www.zaobao.com/news/china/story20250607-6610128}{中欧领导人峰会在即,持续困扰双边关系的诸多贸易问题正取得进展,双方为中国产电动车制定最低价格的谈判进入最后阶段,北京也表态愿放宽对欧稀土出口,并将于7月5日前对欧盟白兰地反倾销案作出最终裁定……}

\entryitemWithDescription{民进党:最多可罢免10席以上国民党立委}{https://www.zaobao.com/news/china/story20250607-6606110}{(台北综合讯)台湾将在7月至8月启动罢免投票,执政的民进党内部评估认为,虽有多名党籍立委被连署提案罢免,但成功罢免的可能性不高;相较之下,国民党则可能有超过10席立委面临被罢免成功的风险。 综合《联合报》《上报》等报道,民进党人士分析,国民党原本认为原住民族选区``蓝大于绿'',因此率先对该选区的民进党立委陈莹与伍丽华发起罢免,未料两案第二阶段连署相继失败,重挫士气……}

\entryitemWithDescription{吉利董事长:汽车工业存在``严重的产能过剩''}{https://www.zaobao.com/news/china/story20250607-6609477}{(北京路透电)吉利控股集团董事长李书福说,当今世界的汽车工业存在``严重的产能过剩'',该公司已决定不再建设新的汽车生产工厂或扩大现有工厂的产能。 李书福星期六(6月7日)以视频方式参加在重庆举行的``中国汽车重庆论坛'',并发表上述讲话。 他说,吉利不搞重复建设,而要充分利用全球过剩的产能,尽最大可能地展开务实合作、资源重组……}

\entryitemWithDescription{台籍教师持中国大陆定居证遭废台湾身份 批陆委会非法滥权}{https://www.zaobao.com/news/china/story20250607-6606366}{(台北综合讯)在福建任教的台籍教师张立齐因持有中国大陆定居证被废止台湾身份,他随后批评台湾政府的大陆委员会``非法滥权迫害'',陆委会则回应称``少数人不要心存侥幸,挑战政府执法决心''。 综合《自由时报》《中国时报》报道,在福建华侨大学任教的张立齐,去年响应中国大陆融合发展政策,成为福建首名领取``台湾居民定居证''的台湾人……}

\entryitemWithDescription{美指控中国研究人员走私危险真菌引质疑 中领馆批政治操弄}{https://www.zaobao.com/news/china/story20250607-6604686}{(芝加哥综合讯)两名中国研究人员在美国被控走私``潜在农业恐袭武器''真菌入境后,农业专家质疑称相关真菌在美已广泛存在且风险极小,中国官方也批评美方借此搞政治操弄。 美国司法部6月3日宣布起诉密歇根大学一名中国籍研究生和她在中国大学进行同类研究的男友,指控他们涉嫌将能引发农作物灾害的禾谷镰刀菌走私到美国……}

\entryitemWithDescription{时隔两个月 波音恢复对华交付商用飞机}{https://www.zaobao.com/news/china/story20250607-6605169}{(西雅图综合讯)美国航空巨头波音自4月以来首次恢复向中国交付商用飞机,显示中美双边贸易在关税僵局的背景下正逐步回暖。 综合路透社和彭博社报道,FlightRadar24的飞行数据显示,一架注册编号为N230BE、涂有厦门航空标志的波音737 MAX客机于星期五(6月6日)早上10时从西雅图起飞,开启飞往中国交付中心的首段航程……}

\entryitemWithDescription{新闻人间:国民党的新``战神''蒋万安}{https://www.zaobao.com/news/china/story20250607-6590758}{向来温文有礼的国民党籍台北市长蒋万安,因不满行政院克扣地方政府的一般性补助款,继上周率先提起诉愿后,星期四(6月5日)又带着九位同党籍的县市长和代表赴行政院请愿,强调``我们拒绝任中央宰割''。 行政院长卓荣泰反驳说,行政院已经没钱缴电费,``我们自己都无力自保,何来宰割……}

\entryitemWithDescription{于泽远:中国再秀``核肌肉''}{https://www.zaobao.com/news/china/story20250607-6584360}{中国官媒央视6月2日公开了东风-5洲际导弹的具体参数,这是去年9月解放军向南太平洋海域发射携载训练模拟弹头的东风-31AG洲际弹道导弹后,中国再次秀出``核肌肉''……}

\entryitemWithDescription{中国和南亚国家贸易额10年间翻番}{https://www.zaobao.com/news/china/story20250606-6588580}{(北京综合讯)中国和南亚国家贸易额10年间翻番,在2024年达近2000亿美元(2570亿新元)。 综合新华社与央视新闻报道,中国商务部副部长鄢东星期五(6月6日)在发布会上介绍,近10年来,中国和南亚国家贸易额年均增长率约6.3\%,中国连续多年成为巴基斯坦、孟加拉国等国最大贸易伙伴。 他说,尼泊尔的羊绒制品、阿富汗的青金石、印度的珠宝、斯里兰卡的茶叶香料等南亚商品,广受中国消费者好评……}

\entryitemWithDescription{中美僵持 美国驻华大使安抚美企}{https://www.zaobao.com/news/china/story20250606-6585694}{(北京综合讯)中美经贸关系僵持不下,给商业界带来巨大的不确定性。美国驻中国大使庞德伟于大使馆接待在华美国商业代表,向美企给予支持和安抚。 庞德伟星期五(6月6日)在X平台发文,表示与美国业界协会美中贸易全国委员会、中国美国商会,及上海美国商会进行了``良好会面'',让他更了解在华美企面临的挑战……}