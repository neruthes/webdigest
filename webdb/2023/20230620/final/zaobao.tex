\entryitemWithDescription{浙江省公布首批60个共同富裕实践观察点}{https://www.zaobao.com/news/china/story20230619-1405981}{中国首个共同富裕示范区,浙江省公布首批60个共同富裕实践观察点,包括村(社区)、乡镇、街道(平台)、企业、社会组织、公共服务机构五大类。 据《浙江日报》报道,浙江首批观察点中,村(社区)占比最大,共24个;社会组织中包括共富工坊、养老服务中心、公益基金会、演员公会等;公共服务机构中有学校、市场、图书馆、医院、慈善综合体等。 报道称,首批观察点的主要特点是颗粒度小,能从条线上实时了解动态变化……}

\entryitemWithDescription{就职中国总理后首次出访 李强抵步德国开启``深化合作之旅''}{https://www.zaobao.com/news/china/story20230619-1405978}{中国总理李强乘包机于当地时间18日抵达柏林勃兰登堡机场。礼兵沿红地毯两侧列队致敬,德国政府代表等到机场迎接。(新华社) 中国总理李强当地时间星期天(6月18日)晚上飞抵德国首都柏林,展开他出掌国务院后的首次正式出访,并将与德国总理朔尔茨共同主持第七轮中德政府磋商。 受访学者评估,面对德国政府内部对华政策信号持续混乱,李强此行明显是寻求进一步稳定双边关系,同时释放积极争取欧洲的外交信号……}

\entryitemWithDescription{香港首办小学普通话水平考级}{https://www.zaobao.com/news/china/story20230619-1405971}{(香港综合讯)香港上星期六(6月17日)首次举办小学普通话水平等级测试活动,全港30所小学的114名小一至小六考生参加。 《人民日报》报道,首场考试在香港考试及评核局进行,由取得资格认证的中国国家级测试员面对面以口试形式进行考核。通过朗读音节、朗读字词、朗读句子或短文、聆听理解或复述、命题说话等项目,评定考生普通话的水平……}

\entryitemWithDescription{中驻英使馆:英台部长会面违反一中原则}{https://www.zaobao.com/news/china/story20230619-1405941}{中国大陆驻英国大使馆指,英国安全部长与台湾数位发展部长会面,``严重违反一个中国原则'',并警告英国,任何损害中国利益的行径,必将受到``有力反击''。 路透社上星期五(6月16日)引述消息人士报道,英国内政部安全事务国务部长图根哈特(Tom Tugendhat)星期三(14日)打破惯例,与到访英国出席伦敦科技周高峰会的台湾数位发展部长唐凤会面,讨论``共同的安全利益''……}

\entryitemWithDescription{国泰往返大陆航班 增设讲普通话空服}{https://www.zaobao.com/news/china/story20230619-1405937}{今年5月被卷入歧视非讲英语乘客风波的香港国泰航空宣布,将优先在往返中国大陆的航班安排会普通话的空服人员,并于7月启动在大陆进行招聘。 据香港01、界面新闻报道,国泰航空行政总裁林绍波星期一(6月19日)发布一封《致员工信》,阐明公司对歧视风波的检讨工作成果。他在信中指出,工作小组正积极查找问题症结、分析根本成因及探讨改善方案。 国泰航空空服员在今年5月下旬被指歧视非讲英语乘客……}

\entryitemWithDescription{于泽远:布林肯不会白跑一趟}{https://www.zaobao.com/news/china/story20230619-1405660}{经过多番波折,美国国务卿布林肯的专机终于6月18日早晨抵达北京。中方在机场没有用鲜花和红地毯迎接布林肯,或许让布林肯在北京炎热的天气中感到一丝冷意。 实际上,包括中美在内的各方舆论都不看好布林肯此次中国行,有台湾媒体甚至称布林肯是到北京``送死''。 的确,2018年以来一路下跌的中美关系很难让人对布林肯来华报以乐观……}

\entryitemWithDescription{【视频】长谈五个半小时 中美外长表示愿避免竞争滑向冲突}{https://www.zaobao.com/news/china/story20230619-1405687}{(视频来源:路透社) 中美外长星期天(6月18日)在北京举行长达五个半小时会谈,双方表达稳定双边关系、避免竞争滑向冲突的意愿。中国国务委员兼外长秦刚强调,北京致力于构建稳定、可预期、建设性的中美关系,美国国务卿布林肯则表明美国不想与中国脱钩。 美国国务卿布林肯星期天抵达北京开启备受瞩目的两天访华行。这是美国国务卿2018年以来首次访华,也是2021年初拜登入主白宫以来访华的最高阶美国官员……}

\entryitemWithDescription{最新民调:柯文哲首次超越蓝绿排第一 侯友宜垫底朱立伦不``换侯''}{https://www.zaobao.com/news/china/story20230618-1405680}{国民党籍总统参选人侯友宜在偏蓝媒体TVBS的最新民调中垫底,落后民众党的柯文哲10个百分点。国民党主席朱立伦称,``换侯''是``自毁长城'',绝对不可能。图为今年5月17日,朱立伦(左)正式宣布征召侯友宜(右)参选台湾总统大选时,两人相互拥抱。(路透社) (台北综合讯)距离台湾总统大选约七个月,一项最新民调显示,民众党参选人柯文哲的支持度首次超越蓝绿参选人,国民党侯友宜落居第三……}

\entryitemWithDescription{美官员就续签《中美科技合作协定》进行辩论}{https://www.zaobao.com/news/china/story20230618-1405669}{中美科技战持续升温,美国官员在辩论是否续签即将到期的《中美科技合作协定》,还是要与中方谈判,加入防止工业间谍活动等条款。 路透社报道,中美在1979年建交时签订了一份《中美科技合作协定》,所涵盖的合作领域包括,环境和农业科学,以及物理和化学的基础研究,为两国在过去40年的学术和商业交流奠定基础。这份每五年续签的协定即将在8月27日到期……}

\entryitemWithDescription{西宁官方两天内更新倡议书 删除干部邀游客留宿内容}{https://www.zaobao.com/news/china/story20230618-1405645}{中国青海西宁官方发布旅游旺季服务保障倡议书,提倡干部主动邀请游客到自己家中用餐、留宿,在网上引发争议,官方不到两天内删除相关内容。 西宁市文明办上星期四(6月15日)发布倡议书,提倡党员干部在资源紧张的情况下,主动让出自己的车位,主动贡献自己的车辆,主动邀请游客到自己家中用餐留宿,尽可能地为困难游客提供便利和帮助,把西宁人民的热心、亲切和友善传递给广大游客……}

\entryitemWithDescription{台外长吴钊燮访欧:想要台积电去投资 应加强与台关系}{https://www.zaobao.com/news/china/story20230618-1405621}{台湾外交部长吴钊燮上周起到欧洲进行访问, 并在6月14日出席由捷克智库``欧洲价值安全政策中心'' 在捷克首都布拉格举办的一场安全会议。(法新社) 台湾外交部长吴钊燮近日出访欧洲时表示,如果欧洲国家希望台湾的半导体产业去欧洲投资,就应该加强与台湾的关系。 吴钊燮从6月12日起启程访问欧洲……}

\entryitemWithDescription{中国年轻人掀``淘金潮'' 学者:反映消费观更谨慎}{https://www.zaobao.com/news/china/story20230618-1405616}{在中国最大珠宝批发市场的深圳水贝,各大商城的黄金店铺门庭若市,顾客群有近一半是年轻人。(林煇智摄) 黄金价格近年来一直处于高位,但中国消费者的``淘金''热潮不减,且趋向年轻化。保值性较高的黄金饰品,已成时下年轻人的理财新宠。受访学者认为,这折射出年轻人感到经济前景的不确定性在增加,消费观变得更实在和谨慎。 黄金作为硬通货,一直是最稳定的避险资产之一……}

\entryitemWithDescription{马云现身杭州 围观数学竞赛}{https://www.zaobao.com/news/china/story20230618-1405613}{马云星期六(6月17日)到``2023阿里巴巴全球数学竞赛''杭州赛事现场观赛。(达摩院公众号) 在日本东京教课的阿里巴巴创始人马云星期六(6月17日)现身杭州,到``2023阿里巴巴全球数学竞赛''决赛现场观赛。 据``达摩院''微信公众号星期六消息,马云在开赛前到杭州的赛事现场了解情况,并与参赛选手、命题老师在线交流,畅谈数学。 马云说,大赛会不断创新,给热爱数学的人们带来新的乐趣……}

\entryitemWithDescription{韩媒:中超韩籍外援孙准浩被批捕在意料中}{https://www.zaobao.com/news/china/story20230618-1405611}{中国山东泰山足球俱乐部的韩国外援孙准浩星期六(6月17日)刑拘期满,已被正式批准逮捕。(法新社) 韩媒引述消息人士称,中国山东泰山足球俱乐部的韩国外援孙准浩星期六(6月17日)刑拘期满,已被正式批准逮捕。 据韩联社报道,消息人士星期天(18日)透露,中国检察机关已批准逮捕孙准浩,警方通常会进行两个月左右的补充调查,若案情重大,可能要数月之后才会起诉……}

\entryitemWithDescription{中国特稿:碰上最难就业季 大学生求职路茫茫}{https://www.zaobao.com/news/china/story20230618-1405118}{刚走出三年冠病疫情的中国大学生,又迎来竞争激烈的最难就业季。图为5月19日湖南长沙为应届毕业生举办的专场招聘会。(新华社) 学者认为,中国当前就业市场的最大问题是经济增长乏力,应届大学毕业生面对严峻的就业形势。在6月12日东方航空公司四川招聘会上,大批毕业生排队进入初试现场。(中新社) 中国大学毕业季来临,创纪录的1158万应届毕业生正进入就业市场……}

\entryitemWithDescription{``鼠鸭之争''尘埃落定 调查组判定江西工职院食堂异物是鼠头}{https://www.zaobao.com/news/china/story20230617-1405415}{江西工业职业技术学院一名学生称6月1日在食堂吃到老鼠头,学校却辨称那是鸭脖,引发一场``鼠鸭之争''。右图为学生拍摄的食堂餐食异物,左图是网民为讥讽学校制作的``鸭身鼠头''图。(互联网) (南昌综合讯)闹得沸沸扬扬的``鼠头鸭脖''事件官方调查结果出炉,判定江西工业职业技术学院食堂餐食中的异物为老鼠类啮齿动物的头部……}