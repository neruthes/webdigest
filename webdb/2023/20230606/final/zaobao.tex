\entryitemWithDescription{民进党性骚扰案滚成风暴 蓝白卷入政治攻防泥巴战}{https://www.zaobao.com/news/china/story20230605-1401547}{国民党星期一(6月5日)召开记者会,表明坚决维护性别平等和女性权益。考纪会主委黄怡腾说,若调查确认党员犯下性骚扰行为,惩处方式包括开除党籍、停止党权和警告。(国民党提供) 民进党掩盖性骚扰案滚成台湾政治风暴,案件激增到两位数,政治攻防泥巴战也同时开打,在野的蓝白阵营都被拖下水。 这轮风波始于5月31日,一名22岁民进党前女党工在脸书具名控诉遭该党重用的导演摸胸非礼,求助后被冷处理,前后掩盖九个月……}

\entryitemWithDescription{美助理国务卿康达访华 将与中国副外长马朝旭会晤}{https://www.zaobao.com/news/china/story20230605-1401533}{美国寻求与中国加强沟通之际,中国外交部副部长马朝旭星期一(6月5日)在北京会见美国助理国务卿康达。 据中国外交部网站消息,中国外交部发言人汪文斌星期一在例行记者会上说,马朝旭当天下午将会见到访的美国国务院亚太事务助理国务卿康达,以及白宫国安会中国事务高级主任贝兰。 汪文斌还说,两人同日上午也与中国外交部北美大洋洲司司长杨涛举行会谈……}

\entryitemWithDescription{美国公布美中军舰险撞视频 称两军在台海``不安全互动''}{https://www.zaobao.com/news/china/story20230605-1401530}{美国海军发布视频,显示中国苏州号驱逐舰越过美国钟云号驱逐舰的航道。(法新社) 中美军舰上周末在台湾海峡险些相撞。美国海军星期天(6月4日)公布视频,显示中国一艘军舰越过美国军舰前方,称这是两军在台海的``不安全互动''。中国外交部星期一(5日)坚称,中国军队采取的行动是应对有关国家挑衅的必要举措,完全安全专业……}

\entryitemWithDescription{香港警方六四逮捕23人 联合国震惊并吁立即释放}{https://www.zaobao.com/news/china/story20230605-1401494}{``六四''天安门事件34周年,香港警察在维多利亚公园和铜锣湾一带,带走23名手持悼念标语或蜡烛等物品的民众,联合国人权专员对此表达震惊。中国外交部发言人则表示,坚决支持香港特区政府依法履行职权。 综合路透社、香港电台报道,在以往举行大规模六四纪念集会的维园附近,数百名警察严阵以待,实施拦截和搜查行动,并部署了装甲车和警车。维园的悼念活动今年虽被嘉年华会取代,仍有人在维园以各种方式悼念……}

\entryitemWithDescription{四川乐山市金口河区发生山体垮塌 造成19死5伤}{https://www.zaobao.com/news/china/story20230605-1401490}{中国四川省乐山市金口河区星期天(6月4日)发生山体垮塌,造成19人死亡,五人受伤。 综合新华网、中新社、中新网等报道,灾难发生在星期天上午6时许,山体垮塌发生在金口河区的永胜乡鹿儿坪国有林场。 山体垮塌体下方半山腰是金口河区金开源矿业公司施工驻地,垮塌掩埋了矿井平台部分生产生活设施、车辆等,并导致简易工棚被埋和受损……}

\entryitemWithDescription{美在台协会主席二访台湾 称华府要与北京保持沟通不让竞争演变为冲突}{https://www.zaobao.com/news/china/story20230605-1401475}{美国在台协会(AIT)主席罗森伯格(Laura Rosenberger)说,美国要管理好美中竞争关系,避免冲突升级,美国也不会在台湾大选中选边站。 星期一(6月5日)抵达台湾进行六天访问的罗森伯格,在今年3月接任AIT主席后,短短两个月内再度访问台湾。根据台湾外交部网站,罗森伯格将会见台湾政府高层及相关部会,就台美关系各项议题交换意见……}

\entryitemWithDescription{六四34周年 超过10人被香港警方带走}{https://www.zaobao.com/news/china/story20230605-1401244}{香港警方在六四34周年纪念日 逮捕逾十人(路透社) 在``六四''天安门事件34周年当天和前夕,香港警察以涉嫌``在公众地方扰乱秩序''或``作出具煽动意图的作为''的罪名,拘留超过十人,包括泛民主派、前记者、行为艺术家以及社会运动人士……}

\entryitemWithDescription{卢沙野再谈失言风波 称``有些人小题大做''}{https://www.zaobao.com/news/china/story20230604-1401236}{中国驻法国大使卢沙野(右)5月31日接受法国独立媒体``法律视角''采访。(视频截图) 中国驻法国大使卢沙野在失言风波一个半月后重提此事,认为在电视公开辩论中的言论自由应得到保证,并强调他的言论与中国官方外交政策并不抵触,``有些人小题大做了''。 中国驻法大使馆网站星期六(6月3日)晚间公布卢沙野5月31日接受法国独立媒体``法律视角''(Vu du droit)的采访报道……}

\entryitemWithDescription{【视频】李尚福警告若中美激烈冲突 将是``世界不可承受之痛''}{https://www.zaobao.com/news/china/story20230604-1401235}{中国国防部长李尚福说,中美制度不同,但不妨碍双方求同存异发展关系,并称中国一直寻求与美国建立新型大国关系。他也警告,如果中美发生激烈冲突,``那是世界不可承受之痛''。 李尚福星期天(4日)在香格里拉对话上发表题为``中国的新安全倡议''的演讲,就台湾、南中国海问题和中美关系等议题阐述中国立场。 65岁的李尚福今年3月接替魏凤和出任中国国务委员兼国防部长,这是他首次以防长身份在香会演讲……}

\entryitemWithDescription{汤家骅指若不改反共仇共心态 香港民主难有发展}{https://www.zaobao.com/news/china/story20230604-1401234}{香港行政会议成员汤家骅认为,香港过去十年与民主普选擦身而过,民主发展现在要``由零开始'',希望香港人改变反共、恐共及仇共心态,否则看不到香港民主制度会有发展。 综合香港电台和网媒``香港01''报道,汤家骅星期天(4日)在商台节目上说,他所创办的香港智库``民主思路''5月访问北京时,获得国务院港澳事务办公室主任夏宝龙接见……}

\entryitemWithDescription{神舟十五号返回舱内蒙古安全着陆 三名航天员健康状态良好}{https://www.zaobao.com/news/china/story20230604-1401228}{中国``神舟十五号''载人飞船返回舱星期天(6月4日)清晨在内蒙古成功着陆,三名航天员费俊龙、邓清明、张陆全部安全顺利出舱,健康状态良好。 综合中国载人航天工程网发出的新闻稿以及央视报道,神舟十五号载人飞船返回舱于星期天清晨6时33分在东风着陆场着陆,现场医监人员确认航天员费俊龙、邓清明、张陆身体状态良好。指令长费俊龙说:``我们三个人感觉都很好,状态很好,身体很好……}

\entryitemWithDescription{台学者:北京高度担忧台湾问题日趋国际化}{https://www.zaobao.com/news/china/story20230604-1401219}{中国国防部长李尚福在香会演讲中划下台湾问题红线,重申台湾是``中国核心利益中的核心'',不管付出多大代价,解放军坚决维护国家主权和领土完整。 他也引述中国歌曲《我的祖国》的歌词说,朋友来了有好酒,豺狼来了有猎枪,以此表明中国走和平发展道路,但决不牺牲核心利益。 李尚福也重申两岸同属一个中国,指责台湾民进党不承认九二共识、挟洋谋独……}

\entryitemWithDescription{台政坛性骚扰风波延烧 女媒体人指控国民党立委}{https://www.zaobao.com/news/china/story20230604-1401214}{时任花莲县长、现任国民党立委傅崐萁。(傅崐萁脸书) 台湾政坛性骚扰风暴烧至国民党,一名女媒体人指国民党立委傅崐萁性曾性骚扰她。对此,国民党表示即刻启动调查,决不宽贷。 镜文学总经理董成瑜上星期六(6月3日)在脸书发文,描述了自己遭一名``想要居间统整国民党选举的县长''性骚扰的经过。据描述,2014年她和这名县长一起餐叙的途中,不只被他``抓手'',对方还突然站起来``抱着我的头亲我''……}

\entryitemWithDescription{港特首政策组专家认为 当前急须讨论人口政策}{https://www.zaobao.com/news/china/story20230604-1401190}{香港特区政府上月成立特首政策组专家组;多名成员认为,人口政策是专家组当务之急最须讨论的课题。 据网媒``香港01''报道,由56名来自商业和金融界、专业人士、智库及学术界等不同背景的人士所组成的专家组,成员均表示仍未收到首次会议的通知,但各自准备就不同范畴事宜给予意见。 研究策略专家组成员、选委界立法会议员周文港认为,人口政策直接影响经济及发展动力,以及教育的长远政策……}

\entryitemWithDescription{美加军舰驶过台湾海峡 引发中美隔空口水战}{https://www.zaobao.com/news/china/story20230604-1401183}{美国和加拿大军舰上星期六(6月3日)穿越台湾海峡,引发中美隔空口水战。中国军方批评美加蓄意挑起风险,美国则反指中国军舰在美军舰附近进行危险航行。 中国人民解放军东部战区新闻发言人施毅星期六在``东部战区''微信公号称,美国``钟云号''驱逐舰和加拿大``蒙特利尔号''护卫舰当天穿越台湾海峡,解放军东部战区组织海空兵力全程跟监警戒……}

\entryitemWithDescription{乌克兰防长称若俄不撤军 无须调解人}{https://www.zaobao.com/news/china/story20230604-1401022}{乌克兰国防部长雷兹尼科夫在香格里拉对话特别讨论会上说,只要俄罗斯军队不撤离乌克兰,乌克兰就不需要任何调解人。(叶振忠摄) 乌克兰国防部长雷兹尼科夫说,只要俄罗斯军队不撤离乌克兰,乌克兰就不需要任何调解人;只有俄军撤出后,才需要调解人,而新加坡会是很好的调解人……}

\entryitemWithDescription{中国特稿:中国安宁疗护 — 让告别有尊严}{https://www.zaobao.com/news/china/story20230604-1400657}{安宁疗护在中国近年渐入公众视野,越来越多地区开始试点工作。图为走进病房的广州安宁疗护志愿者。(广州红房子社工服务中心提供) 安宁疗护在中国近年渐入公众视野,越来越多地区开始试点工作。随着人口老龄化程度的加深,中国临终关怀服务的需求也在不断增加,不过,中国的安宁疗护仍面对公众认知度低、服务体系不健全、人才匮乏等现实问题……}

\entryitemWithDescription{【香会】奥斯汀批中国拒对话 称台海和平稳定攸关全球}{https://www.zaobao.com/news/china/story20230603-1400979}{美国国防部长奥斯汀星期六(6月3日)在香格里拉对话首场全会上演讲时警告,台湾海峡一旦爆发冲突将是``毁灭性的''。但他强调,台海冲突非迫在眉睫,也不是无法避免。(唐家鸿摄) 美国国防部长奥斯汀警告,台湾海峡一旦爆发冲突将是``毁灭性的'',台海和平稳定攸关全球。他也批评中国拒绝两国防长对话,并喊话``现在就是对话的合适时机''……}

\entryitemWithDescription{民进党性骚扰风波延烧 目前爆出至少九起指控}{https://www.zaobao.com/news/china/story20230603-1400972}{台湾民进党性骚扰风波持续延烧,目前爆出至少九起指控。蔡英文、赖清德和陈建仁等党内高层连日来纷纷出面致歉,仍难阻挡舆情发酵。 台湾总统蔡英文星期五(6月2日)晚间在脸书发文称,这两天接连有受害者揭露曾在民进党遭受性骚扰,令她感到不舍和痛心,``身为前任党主席,我责无旁贷。我要向受害的朋友和社会大众,表达最深的歉意''。 蔡英文也呼吁外界不要做过多政治操作,对受害者造成二度伤害……}