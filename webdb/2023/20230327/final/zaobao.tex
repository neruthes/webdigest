\entryitemWithDescription{蔡英文上台七年第九个断交国 洪都拉斯宣布弃台湾与大陆建交}{https://www.zaobao.com/news/china/story20230327-1376646}{台湾外交部长吴钊燮说,台湾秉持最大诚意提出协助洪都拉斯方案,但洪方仍``恣意作态需索'',罔顾台湾多年情谊和大陆建交,令台湾痛心与遗憾。(路透社) 中国大陆外交部长秦刚(右)与洪都拉斯外长雷纳星期天在北京钓鱼台国宾馆,完成建交仪式后彼此祝贺……}

\entryitemWithDescription{最新民调: 台支持统独民意降低 要维持现状者增加}{https://www.zaobao.com/news/china/story20230327-1376647}{台湾最新民调显示,相较去年10月,最终支持统一、独立的民意都有所降低,但以支持独立者较明显;支持永远维持现状及以后再决定的,则增加了6.1个百分点。学者解读,在两岸关系不稳定时,倾向独立的民意会转化到维持现状,这是基于安全需求的权变。 台湾政府的大陆委员会长期委托政治大学选举研究中心,就民众当下对统独立场及两岸关系看法进行民意调查……}

\entryitemWithDescription{香港解封后首场获批准游行 反对将军澳填海}{https://www.zaobao.com/news/china/story20230327-1376648}{约80名示威者星期天(3月26日)参加``反对将军澳填海''和兴建垃圾处理场的游行,他们的颈部均挂上印有独立编号的卡牌,没有佩戴口罩。香港警方派出大批警员在场监视。(法新社) 香港约有80名示威者星期天参加``反对将军澳填海''和兴建垃圾处理场游行。这是香港在2020年实行《香港国安法》以来,首个获警方批准的抗议游行……}

\entryitemWithDescription{日男疑涉间谍活动北京被捕}{https://www.zaobao.com/news/china/story20230327-1376650}{据日本媒体报道,一名50多岁日本男子疑似因从事间谍活动,本月在中国首都北京被拘捕。 据日本放送协会(NHK)星期六(3月25日)报道,这名男子是一家日本公司的高级主管。日本政府消息人士称,他本月较早前因涉嫌违反中国法律,而被中国安全机关拘留。据报,该名男子可能因涉嫌从事间谍活动而被捕。 日本驻中国大使馆已提出与男子谈话的请求,并正在收集关于男子被捕原因的信息……}

\entryitemWithDescription{早 说}{https://www.zaobao.com/news/china/story20230327-1376651}{可以预见的是,美国一定会使用各种手段为台湾巩固``邦交''。但美国自己就与中国有外交关系,却阻挠其他国家与中国建交,令人匪夷所思。 ------上海大学特聘教授、拉美研究中心主任江时学3月26日接受香港中通社采访时表示,洪都拉斯决定和中国大陆寻求建交后,美国政府和台湾政府都对该国施加了不小压力……}

\entryitemWithDescription{中国国务院工作规则变化}{https://www.zaobao.com/news/china/story20230327-1376652}{中国政府网微信公众号近日发布了新一届国务院关于印发《国务院工作规则》的通知。与五年前上届国务院发布的《国务院工作规则》相比,新修订的规则最大变化是突出国务院的政治定位,强调国务院要``把党的领导贯彻落实到政府工作全过程各领域''……}

\entryitemWithDescription{中央财办副主任韩文秀: 中国须化解外压通过技术创新突破}{https://www.zaobao.com/news/china/story20230327-1376653}{韩文秀在北京举行的中国发展高层论坛上说,当前世界经济存在滞胀风险,全球产业链供应链面临重构,中国经济恢复的基础还不够稳固。他指出,中国要努力克服新的外部压力,以及人口负增长、老龄化对中国经济中长期发展的影响,其中要有效应对外部的遏制打压……}

\entryitemWithDescription{``鼎泰丰''创办人杨秉彝逝世 享年96岁}{https://www.zaobao.com/news/china/story20230327-1376655}{``鼎泰丰''创办人杨秉彝,日前安详辞世,享年96岁。(互联网) 台湾具代表性的连锁餐饮品牌``鼎泰丰''的创办人杨秉彝,日前安详辞世,享年96岁。 综合中时新闻网、《联合报》报道,鼎泰丰星期六(3月25日)对外发布消息,创办人杨秉彝日前安详辞世,家属盼低调办理治丧事宜,感谢各界关心。 根据鼎泰丰网站公布的信息,杨秉彝1927年出生于中国大陆山西省,21岁赴台湾闯荡……}

\entryitemWithDescription{中国国际时装秀 ``妇好鸮尊''绕``战马''}{https://www.zaobao.com/news/china/story20230327-1376656}{北京冬奥会服装设计师丁洁3月26日携``视界.边界''系列时装,亮相中国国际时装周(2023秋冬系列)。丁洁以``妇好鸮尊''为设计灵感,模特佩戴3D打印发饰,绕行秀场特别设置的``战马'',上演一场历史与未来的``时空对话''。``妇好鸮尊''是一件商代后期的青铜器,1976年出土于河南省安阳市殷墟妇好墓,共两件,分别藏于中国国家博物馆和河南博物院。妇好为商王武丁的配偶,曾多次出征……}

\entryitemWithDescription{劣质大陆旅团被指扰民 港府:解决措施研究中}{https://www.zaobao.com/news/china/story20230327-1376657}{有低素质的中国大陆旅游团被指在香港扰民,出现强迫购物等乱象。香港特区政府表示,已开始研究解决措施。 据《星岛日报》报道,有议员反映,素质较低的大陆旅行团近日重现在香港旧区街头。香港工联会的立法会九龙东议员邓家彪说,土瓜湾等旧区道路较狭窄,随着未来旅客增多,旅游巴士路边停泊可能造成堵塞,影响居民生活。 邓家彪说,劣质旅游团往往出现``强迫购物''的情况,损害香港旅游业的形象……}

\entryitemWithDescription{美中贸易全国委员会会长艾伦:中国数据监管与产业政策得更透明 以改善营商环境}{https://www.zaobao.com/news/china/story20230326-1376334}{外资企业在中国面对更高的政治、监管和商业风险,中国的科技自主战略也增加外企面对的不确定性,导致外企越来越难说服董事会加大在华投资;美中贸易全国委员会会长艾伦呼吁中国更透明地实施数据监管相关政策以及产业政策,以改善整体营商环境……}

\entryitemWithDescription{四年来首次访华 库克:享受与中国共生关系}{https://www.zaobao.com/news/china/story20230326-1376335}{苹果公司首席执行官库克2019年以来首次到访中国。他3月24日到北京三里屯参观了苹果在中国大陆开设的第一家零售店,并与在场顾客合照。图为库克与中国歌手黄龄在店内合影。(互联网) 在中美关系跌入历史低谷、许多美企高管避免出席北京举办的中国发展高层论坛之际,苹果公司首席执行官库克却四年以来首次访华,并强调苹果``享受与中国之间的共生关系''……}

\entryitemWithDescription{出访中美洲前视察军队 蔡英文:守护台湾捍卫民主是台军使命}{https://www.zaobao.com/news/china/story20230326-1376336}{台湾总统蔡英文(第二排左五)3月25日由国安会秘书长顾立雄(第二排左四)、国防部长邱国正(第二排右四)、陆军司令徐衍璞(第二排右三)的陪同前往嘉义视察陆军52工兵群,并与官兵合影留念。(路透社) 蔡英文3月29日出访中美洲友邦将过境美国,马英九则先一步于3月27日前往中国大陆祭祖,两人出访时间高度重叠。台湾国际战略学会星期六举行``当前台海安全''论坛,演讲者对此进行热烈交锋……}

\entryitemWithDescription{美驻华大使探视 在中国被囚三美公民}{https://www.zaobao.com/news/china/story20230326-1376337}{美国驻华大使伯恩斯上周罕见地亲赴监狱探视在中国被囚禁的三名美国公民,包括在2018年因间谍罪被判刑10年的美籍华商李凯。 据路透社报道,李凯的儿子哈里森·李(音译,Harrison Li)星期五(3月24日)说,伯恩斯3月16日到上海监狱探视父亲,两人透过玻璃隔板进行了一小时的会面。伯恩斯曾想跟李凯握手,但不获中方准许。 李凯在1989年后赴美,曾经营飞机零件出口公司和加油站……}

\entryitemWithDescription{两岸``小三通''开放客运中转 首日人流不如预期}{https://www.zaobao.com/news/china/story20230326-1376338}{往返中国福建省和台湾的``小三通''客船服务星期六(3月25日)起开放客运中转,但人流较预期低。 综合《联合报》和中时新闻报道,``小三通''恢复客运中转后的首班航线在星期六早上起航,从厦门与金门同时对开,入境台湾133人、出境114人。 负责往来金马和厦门航线的一名主管说,船班满载是280人,目前不算热烈,一旦开放陆客搭乘该船并恢复自由行后,有望恢复热烈景况……}

\entryitemWithDescription{中国16省去年人口萎缩}{https://www.zaobao.com/news/china/story20230326-1376339}{中国16个省份在2022年陷入人口负增长。 据第一财经星期五(3月24日)报道,中国至今有23个省区市公布了2022年人口出生率、死亡率和自然增长率。其中,有五个省份人口自然增长率由正转负,仅九省人口出生率超过6.77‰的全国平均值。这九省中,七个是中西部城镇化率较低的省份,只有海南和福建来自东部……}

\entryitemWithDescription{新西兰外长访华与秦刚举行会谈 关切新疆人权香港自由}{https://www.zaobao.com/news/china/story20230326-1376340}{中国国务委员兼外长秦刚(右)3月24日在北京与新西兰外长马胡塔会面。(新华社) 新西兰外长马胡塔在访问北京期间与中国国务委员兼外长秦刚会谈时,提到了台海局势、俄乌战争等多项议题。 综合法新社、彭博社和路透社报道,马胡塔星期六(3月25日)发表声明说,新西兰对新疆人权情况以及香港的自由和权利受到侵蚀深表关注……}

\entryitemWithDescription{举报中宣部文艺局长滥用职权 影视监制李学政微博账号遭 ``封杀''}{https://www.zaobao.com/news/china/story20230326-1376341}{中国知名影视监制李学政以敢言著称,多次就影视圈和公共议题高调发声。(互联网) 中国知名影视监制李学政实名举报中宣部文艺局局长刘汉俊,指他利用职权在网络大量卖书等。事发后,李学政和相关公司的微博账号及网页已无法搜到或登入……}

\entryitemWithDescription{台洪关系或生变 美在台协会:大陆经常不兑现承诺}{https://www.zaobao.com/news/china/story20230326-1376342}{在洪都拉斯转向与中国大陆建交之际,美国在台协会星期六(3月25日)向其喊话说,北京常以不会兑现的承诺换取外交承认。 据路透社报道,美国在台协会发言人说,虽然洪都拉斯可能与台北断绝关系并支持北京之举是一项主权决定,但北京并不总是履行其承诺,``重要的是要注意,中国(大陆)经常做出许多承诺换取外交承认,最终却未兑现''……}

\entryitemWithDescription{香港特稿:陆港全面通关民间矛盾升温}{https://www.zaobao.com/news/china/story20230326-1376343}{香港全面复常后,铜锣湾人潮如鲫。(戴庆成摄) 叶德平:陆港两地民众之间的碰撞,是融和的第一个阶段。(受访者提供) 香港与大陆因为冠病疫情封关三年后,终于在今年2月6日全面通关,访港的大陆旅客随即出现爆发式增长。然而,正当香港各界摩拳擦掌、希望生意实现``报复性反弹''之际,连月来不时有大陆旅客投诉在香港被歧视,港人也对越来越多陆客来港而衍生出各种问题出现不满的情绪……}

\entryitemWithDescription{早说}{https://www.zaobao.com/news/china/story20230325-1376088}{现在的大学存在许多问题。比如说,大学生不就业的问题\ldots\ldots 我们真正的大学,没有教他动手能力,那么他毕业以后就去考公务员,公务员不要技术的,就磨嘴皮就行了。 ------有``中国玻璃大王''之称的中国福耀集团董事长曹德旺近日接受凤凰网访问,谈到中国大学生热衷考公务员的现象时如是回应。曹德旺的观点在微博登上热搜,引起网民热议……}