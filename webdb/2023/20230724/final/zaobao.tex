\entryitemWithDescription{国民党全代会 韩国瑜力挺侯友宜选总统}{https://www.zaobao.com/news/china/story20230723-1416818}{国民党7月23日全代会再现最强三战将``秃子、汉子、燕子''(即前排左起的韩国瑜、侯友宜和台中市长卢秀燕)同台互挺画面,相信有助凝聚蓝营军心。(国民党提供照片) 国民党全台党代表大会(全代会)星期天正式提名新北市长侯友宜参选台湾总统,高人气的前高雄市长韩国瑜热情相挺。但未获征召的鸿海集团创办人郭台铭,会前在脸书贴文称``民意永远大于党意'',蓝军选情与在野整合依然充满挑战……}

\entryitemWithDescription{李家超展开三天访新之行 时隔六年再有港特首到访}{https://www.zaobao.com/news/china/story20230723-1416796}{我国国家发展部兼外交部高级政务部长沈颖(左)星期天(7月23日)在机场迎接到访的香港特首李家超。(外交部提供) 香港特首李家超星期天(7月23日)抵达新加坡展开三天访问。这是李家超出任香港特首后,首次访问新加坡,期间将与李显龙总理会面。 据新加坡外交部文告,李家超是应李总理的邀请,在7月23日至25日访问我国。 李家超将出席李总理所设的午宴,副总理兼财政部长黄循财也将与李家超共进早餐……}

\entryitemWithDescription{中国驻日使馆批北约介入亚太贩卖安全焦虑}{https://www.zaobao.com/news/china/story20230723-1416790}{针对北大西洋公约组织指中国挑战北约的安全与价值观,中国驻日本大使馆回应批评北约贩卖安全焦虑,唯恐天下不乱。 据中新社报道,中国驻日本大使馆发言人星期六(7月22日)就北约东进亚太发表谈话,指近期以来北约持续介入亚太事务,挑动阵营对抗,引起地区国家高度警惕……}

\entryitemWithDescription{浙江杭州富阳暴雨引发山洪 已致五死三失联}{https://www.zaobao.com/news/china/story20230723-1416782}{浙江省杭州市富阳区7月22日傍晚至23日凌晨,短时强降雨引发山洪致局部山体塌方和泥石流,已致五人死亡,三人失联。(中新社) 中国浙江省杭州市富阳区部分区域突降暴雨引发山洪,截至星期天(7月23日)下午2时,已致五人死亡、三人失联。 据``富阳发布''公众号公告,富阳区22日傍晚至23日凌晨短时强降雨引发山洪,致局部山体塌方和出现泥石流……}

\entryitemWithDescription{法国外交顾问称中国正向俄罗斯输送军事装备}{https://www.zaobao.com/news/china/story20230723-1416780}{法国外交顾问称,中国正在向俄罗斯提供可以用作军事装备的物品,这些物品也可用在俄乌战争。 综合路透社和法新社报道,法国总统马克龙首席外交政策顾问博纳(Emmanuel Bonne)星期四(7月20日)在巴黎举行的阿斯彭安全论坛上被问及西方是否看到任何证据,显示中国以任何形式为俄罗斯提供武装时,作出上述回应。 他说:``有迹象表明,他们正在做一些我们不希望他们做的事……}

\entryitemWithDescription{西安回流生事件警方再抓六人}{https://www.zaobao.com/news/china/story20230723-1416778}{中国西安回流生事件热度持续。继上周控制涉案20余人后,西安警方上星期六(7月22日)再抓获犯罪嫌疑人六名,刑事拘留五人。 陕西省西安市公安局官方微博星期天(23日)通报,西安市一所补习学校通过伪造印章,为外省回流生在西安参加中考提供虚假资格审查材料,涉嫌犯罪。 该事件引发社会关注,是因为网络传言称西安2023年10万中考生中,有多达4万名回流生……}

\entryitemWithDescription{排解焦虑 中国年轻人刮起算命风潮}{https://www.zaobao.com/news/china/story20230723-1416746}{中国刮起算命风潮,年轻人聚集的市集里,时常出现塔罗牌占卜摊位。图为摄于北京三里屯的一个临时市集。(黄小芳摄) 25岁的郭欣(化名)向来坚决不信命理,但她在报考研究生和求职受挫后,首次主动花钱算了八字------对她而言,算命不是迷信,而是排解内心焦虑的一种心理咨询。 在过去三年的疫情中,中国的命理产业经历一波爆炸式发展和重塑,算命不再是年长者的迷信活动,也成为备受年轻人追捧的潮流文化……}

\entryitemWithDescription{香港特稿:港人大湾区养老须跨越重重难关}{https://www.zaobao.com/news/china/story20230723-1416003}{在东莞黄江康湖护老院,医护人员陪同年长者进行护理。(林煇智摄) 粤港两地融合发展不断深化,香港政府这几年加大政策力度推进跨境养老,以破解香港人口老龄化问题;但粤港两地在医疗、福利保障制度上的差异,长期是香港年长者北上养老最大的顾虑。这样的情况是否已经改善?跨境养老还面临哪些挑战? 95岁的陈素贞五年前在步入鲐背之年时做了一个决定,从香港到广东东莞养老……}

\entryitemWithDescription{反间谍法引发外企忐忑 中国商务部出面安抚}{https://www.zaobao.com/news/china/story20230723-1416549}{本月生效的新版《反间谍法》一度引发在华外企不安。(路透社) 北京的反间谍行动数月来在驻华外企中引发寒蝉效应,中国商务部罕见地召集在华外国商会和企业代表,解释本月生效的新版《反间谍法》,以稳定在华外资信心。 受访学者指出,中国与美国等西方国家博弈加剧下,外企担心执法的模糊性,会让他们在华的经营活动成为地缘政治斗争的牺牲品,而单靠经济部门安抚,难以完全打消疑虑……}

\entryitemWithDescription{足坛反腐:中国足协又两位高层被查}{https://www.zaobao.com/news/china/story20230722-1416547}{中国足协技术部部长谭海(左)和战略规划部部长戚军涉嫌严重违纪违法被查。(互联网) 继11名中国足坛重磅人物相继落马后,中国足球协会两位部长谭海和戚军涉嫌严重违纪违法被调查。 据湖北省纪委监委网站星期六(7月22日)通报,中国足球协会技术部部长谭海和战略规划部部长戚军涉嫌严重违纪违法,目前正接受中央纪委国家监委驻国家体育总局纪检监察组和湖北省监委审查调查……}

\entryitemWithDescription{借鉴俄乌战况 台湾研发无人机备战}{https://www.zaobao.com/news/china/story20230722-1416545}{台湾雷虎科技公司的一名员工3月30日在嘉义的亚洲无人机人工智能创新应用研发中心搬运一台无人机进行演示。(路透社) (台北综合电)台湾借鉴俄乌战争中乌克兰用无人机获得空中优势的经验,正加快组建一支无人机国家队,缩小与中国大陆的无人机差距。 据路透社星期五(7月21日)报道,去年夏天在俄乌战争爆发数月后,台湾总统蔡英文曾召集民进党高官,探讨乌克兰是如何成功应对像俄罗斯这样的一个强敌……}

\entryitemWithDescription{希音在美国投入60万美元进行政治游说}{https://www.zaobao.com/news/china/story20230722-1416509}{中国线上时装巨头希音(Shein)今年第二季度在美投入了60万美元进行政治游说。(路透社) 因持续面临有关强迫劳动、环境问题和不透明供应链的质疑,中国线上时装巨头希音(Shein)今年第二季度在美投入了60万美元(79万8000新元)进行政治游说。 据彭博社报星期六(7月22日)报道,美国两党议员一直以新疆维吾尔族强迫劳动产品的问题,质询在美国运营的希音……}

\entryitemWithDescription{郭台铭柯文哲双双计划访美}{https://www.zaobao.com/news/china/story20230722-1416507}{传将独立参选总统的鸿海集团创办人郭台铭(左)以及民众党总统参选人柯文哲(右)均计划赴美访问。(自由时报) 继台湾民进党和国民党的总统参选人都据报会访美后,传将独立参选总统的鸿海集团创办人郭台铭,以及民众党总统参选人柯文哲也双双计划赴美访问……}

\entryitemWithDescription{香港加强检测日本进口海产}{https://www.zaobao.com/news/china/story20230722-1416487}{日本媒体报道,香港从上月中旬开始,对从日本进口的海产品实施了更严格的辐射检测。 日本共同社星期六(7月22日)引述消息人士称,香港从6月中旬起加强了对日本进口海产的放射性物质检查,清关时间较正常情况增加了约3小时。报道也说,香港是日本海产的第二大出口目的地,如果清关时间延长的情况持续,会对日本渔业工作者造成打击……}

\entryitemWithDescription{英国外长推迟访华 或与秦刚``消失''有关}{https://www.zaobao.com/news/china/story20230722-1416479}{英国外交部长克莱弗利推迟原定本月底访华的行程。(路透社) 在中国国务委员兼外交部长秦刚从大众视野消失近一个月后,英国外交部长克莱弗利推迟了他原定本月底访华的行程。 据彭博社星期五(7月21日)报道,两名知情人称,秦刚未公开露面将近一个月,是克莱弗利延后访华的主要原因。中英双方正在检视另一个替代的日期。 英国外交部没有回应置评的请求……}

\entryitemWithDescription{新闻人间:消失的``他''}{https://www.zaobao.com/news/china/story20230722-1416242}{由朱一龙、倪妮等人主演的悬疑电影《消失的她》6月下旬上映后,热度一直不减,并打破了中国国产悬疑片的票房纪录。 现实中也有一个``消失的他'',社会关注度已远超上述热映影片,但截至7月21日,``他''的下落仍是一个谜。 ``他''就是中国国务委员兼外交部长秦刚……}

\entryitemWithDescription{庄慧良:中国大陆尖子生游宝岛}{https://www.zaobao.com/news/china/story20230722-1416284}{北京、清华、复旦、武汉和湖南大学五所中国大陆高校37名师生,应马英九基金会之邀来台展开九天八夜之旅。这些名校顶尖学生的自信风采、落落大方的谈吐,立即攫取台湾民众目光。宝岛民众对他们的友善与热情,也深深烙印在这些学生心中。 此团先前被台湾官方冠上``统战''之名,差点无法成行。马英九基金会想方设法减人数、调行程,终于在出发前四天获得许可……}

\entryitemWithDescription{侯友宜规划9月上旬访美 或与佩洛西会面}{https://www.zaobao.com/news/china/story20230721-1416271}{国民党总统参选人侯友宜规划9月上旬访问美国。(法新社) 国民党总统参选人侯友宜规划9月上旬访问美国,计划安排至少一场公开演讲,同时可能与美国众议院前议长佩洛西会面。 据台湾《中国时报》星期四(7月20日)报道,国民党立委江启臣已提前赴美为侯友宜踩点,届时也将担任侯友宜访美团团长。据悉,侯友宜访美时间主要配合参众议院开议时间,计划发表至少一场公开演讲,并可能与佩洛西会面……}

\entryitemWithDescription{中国第二口万米深井在四川盆地开钻}{https://www.zaobao.com/news/china/story20230721-1416266}{继中国新疆塔里木盆地的``深地塔科1井''后,四川省广元市剑阁县星期四(7月20日)首次开钻``深地川科1井'',是中国开钻的第二口万米深井。 综合新华社、《人民日报》报道,``深地川科1井''位于四川盆地西北部剑阁潜伏构造,地面海拔717米,设计井深10520米。这片区域超深层叠置多套优质储层,成藏条件优越,一旦成功将有望发现新的超深层天然气增储目标区……}

\entryitemWithDescription{台湾调查CPTPP机密文件泄露事件}{https://www.zaobao.com/news/china/story20230721-1416240}{两名知情官员透露,台湾政府正调查一起官方机密文件泄露事件,泄露的文件包括有关台湾申请加入《跨太平洋伙伴全面进展协定》(CPTPP)的外交电报和机密报告……}