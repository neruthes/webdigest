\entryitemWithDescription{台湾特稿:全台最高生育率地区 科技带来幸``孕'' 竹科人生生不息}{https://www.zaobao.com/news/china/story20240331-3226059}{新竹古迹城楼迎曦门和市中心全景。``台湾硅谷''新竹科学园区就位于该市。(Smchen43摄) 全球面对少子化挑战,带动人工智能(AI)和半导体晶片需求飙升,处于全球先进制造研发最核心的``台湾硅谷''新竹科学园区一带,却出现反潮流的多子化、人口增长现象。《联合早报》走访竹科,近距离感受``添丁发财''的荣景,了解其中的原因和挑战……}

\entryitemWithDescription{中国年轻人开始流行``上班恶心穿搭''}{https://www.zaobao.com/news/china/story20240330-3236579}{中国年轻人近期掀起``上班恶心穿搭''风潮,不少网民将自己身穿绒毛睡衣、运动裤上班的照片上传至社交平台,以较量谁的上班穿搭``更恶心''。(早报制图) (北京综合讯)中国年轻人近期掀起``上班恶心穿搭''风潮,不少网民将自己身穿绒毛睡衣、运动裤上班的照片上传至社交平台,以较量谁的上班穿搭``更恶心''……}

\entryitemWithDescription{美国修订对中国人工智能芯片和设备的出口限制}{https://www.zaobao.com/news/china/story20240330-3236627}{美国星期五(3月29日)修订实施了五个月的对华人工智能(AI)芯片和设备的出口限制,使中国更难以获得美国制造的AI芯片。(路透社档案照) (华盛顿综合讯)美国星期五(3月29日)修订实施了五个月的对华人工智能(AI)芯片和设备的出口限制,使中国更难以获得美国制造的AI芯片……}

\entryitemWithDescription{华为突破美国制裁 去年净利润翻倍}{https://www.zaobao.com/news/china/story20240330-3236345}{华为在深圳的一家零售店外张贴着Mate60 Pro的宣传海报。(林煇智摄) 专家:得益于生产与供应链精准规划 中国科技巨头华为去年强势回归5G智能手机市场后,营收创下2019年以来最大增幅,净利润更是同比翻倍,一度崩溃的手机等终端业务对收入涨幅贡献最大,意味着长期受美国施压的华为很大程度上已突破了制裁限制……}

\entryitemWithDescription{著有《巨流河》台湾文学家齐邦媛辞世 享年101岁}{https://www.zaobao.com/news/china/story20240330-3236404}{台湾作家、学者、教育者齐邦媛生前曾表示自己一旦离世,希望不要特别为她举办追思会。(互联网) (台湾综合讯)著有《巨流河》的知名台湾作家、学者、教育者齐邦媛星期四(3月28日)辞世,享年101岁。 综合台湾《联合报》《中国时报》等报道,前台笔会会长高天恩说,齐邦媛近期因高龄身体况状不佳反复进出医院,时常陷入昏迷。长期照顾她的看护证实,齐邦媛已于星期四凌晨辞世……}

\entryitemWithDescription{吉利汽车董事长提六四 直播短暂被禁}{https://www.zaobao.com/news/china/story20240330-3236128}{中国吉利汽车董事长李书福(左)3月28日邀请新东方创办人俞敏洪(右)一起做直播为吉利新车宣传。(直播截图) (北京综合讯)中国吉利汽车董事长李书福为吉利新车直播宣传期间,讲述自己的生意在1989年出现变动时,三次提到``六四''一词,直播随后被短暂封禁。 为推广新车,李书福星期四(3月28日)邀请新东方教育科技集团创办人俞敏洪到吉利在浙江台州的卫星超级工厂做直播……}

\entryitemWithDescription{美国将对多名香港官员实施新签证限制}{https://www.zaobao.com/news/china/story20240330-3236056}{(华盛顿/香港综合讯)美国将对多名``负责打压香港权利和自由''的香港官员实施新的签证限制,香港特区政府对此表示``嗤之以鼻,无惧任何威吓''。 据路透社报道,美国国务院星期五(3月29日)向国会递交一年一度的《香港政策法报告》。国务卿布林肯在随附的声明中称,过去一年,中国继续针对香港的高度自治、民主制度以及权利和自由采取打压行动,包括近期生效的、针对香港《基本法》第23条的立法……}

\entryitemWithDescription{新闻人间:``赖清德爱将''性招待风暴过关等升官?}{https://www.zaobao.com/news/china/story20240330-3233467}{上任18天就因性招待丑闻下台的台湾行政院前发言人陈宗彦,最近在监察院的弹劾案获压倒性票数否决,沉寂一年后看来或有可能重回政坛。 57岁的陈宗彦是民进党主席、候任总统赖清德一手提拔的爱将。赖清德2010年当台南市长第一天,陈宗彦就出任该市民政局长,过后再当新闻及国际关系处长,即与市长关系密切的发言人。2017年赖清德升任行政院长,一年后陈宗彦也火箭冲天高升为内政部次长……}

\entryitemWithDescription{台湾防长因儿子涉嫌召妓请辞获慰留}{https://www.zaobao.com/news/china/story20240329-3234144}{台湾国防部长邱国正的儿子邱晃年传出涉嫌召妓丑闻。 总统府证实,邱国正因家人所涉事宜为团队带来困扰等,星期四(3月28日)晚上向总统蔡英文表达歉意,并口头请辞,已获慰留。 总统府称,考量当前区域安全形势复杂,又值政府团队交接期间,蔡英文认为邱国正应坐镇岗位,确保各项重要国防事务稳健周全;同时也肯定邱国正上任以来综理国防事务、戮力从公 。 至于邱的家人所涉事宜,由所属机关查察,一切依法办理……}

\entryitemWithDescription{恐袭后中国暂停在巴基斯坦两水坝工程}{https://www.zaobao.com/news/china/story20240329-3233997}{(白沙瓦综合讯)据巴基斯坦官员透露,在五名中国工程师日前遭遇恐怖袭击身亡后,中国承包商已暂停在巴基斯坦两个大型水坝项目的施工。 法新社星期五(3月29日)报道,星期三以来,承建达苏水电站的中国葛洲坝集团公司,以及承建迪迈尔-巴沙大坝的中国电力公司已经停止施工,并要求巴基斯坦当局在复工前制定新的安保计划。 中国封面新闻报道,日前恐袭当天下午,达苏水电站工地便已全部停工……}

\entryitemWithDescription{台北素食餐厅中毒案 重症与死亡病例全部验出米酵菌酸}{https://www.zaobao.com/news/china/story20240329-3233890}{(台北综合讯)台北市素食餐厅食品中毒案目前累计通报21例,两名死者与六名住院患者体内全部验出米酵菌酸。 这是台湾首次检出这一毒素。据《联合报》报道,台湾卫福部次长王必胜星期五(3月29日)通报,目前宝林茶室案的死亡或重症患者体内都检出米酵菌酸阳性,可以确定此次食品中毒事件是米酵菌酸引起……}

\entryitemWithDescription{台湾首艘自制潜艇 有望2025年11月前交船}{https://www.zaobao.com/news/china/story20240329-3233769}{(台北讯)台湾首艘自制潜艇``海鲲号''目前正进行泊港测试,只要能够完成所有测试,有望在2025年11月前交付台湾海军。 据《自由时报》星期五(3月29日)报道,台湾国际造船公司董事长郑文隆在接受该报专访时透露,海鲲号的泊港测试目前大致按照规划正常进行,但海上测试还没有一个明确的时间表。 郑文隆说,海鲲号在泊港测试期间验证倾斜测试时,左右倾斜角度差不到一度,代表潜艇的设计概念与工艺实践完全符合……}

\entryitemWithDescription{关注赖清德就职演说 美国在台协会主席再访台}{https://www.zaobao.com/news/china/story20240329-3233643}{(台北讯)据知情人士报道,美国在台协会(AIT)主席罗森伯格将在下周访问台湾。这是她时隔两个月后再度访台,应当是华盛顿关注台湾总统当选人赖清德的就职演说,以把控台湾权力交替期间的两岸形势。 台湾《联合报》星期五(3月29日)引述知情人士报道,罗森伯格将于下周抵台,与台湾总统蔡英文、正副总统当选人赖清德和萧美琴等会唔……}

\entryitemWithDescription{学者:中央加强地方集权对科技创新帮助不大}{https://www.zaobao.com/news/china/story20240329-3231798}{穆迪主权风险部副总裁佩奇(左一)与新加坡国立大学东亚研究所高级研究员郭良平(左二)3月19日在新加坡国际事务研究所举办的研讨会上讨论中国中央和地方政府间的政策协调。(卓祾祎摄) 学者认为,中国政府加强了对地方的控制,对解决地方债问题、防止系统性风险有帮助,但对经济发展,尤其是中美较量过程中的科技创新而言,帮助不大……}

\entryitemWithDescription{民企过冬?专家:中国经济结构调整 民企必须接受}{https://www.zaobao.com/news/china/story20240329-3230795}{凯德投资(中国)首席执行官潘子翔(左起)、南洋理工大学经济学荣誉教授陈光炎博士、华侨银行亚洲地区研究与策略主管谢栋铭,在《联合早报》与通商中国举办的``解读两会''论坛上参加对谈,主持讨论的是《联合早报》副总编辑韩咏红(右一)……}

\entryitemWithDescription{陈光炎:未来一两年国际贸易关系将更趋紧张}{https://www.zaobao.com/news/china/story20240329-3230094}{南洋理工大学经济学荣誉教授陈光炎研判,国际贸易关系未来一两年会更趋紧张。(邝启聪 摄) 中国``新质生产力''产能或向亚非拉转移 中国发展``新质生产力''引发美国对中国输出过剩产能扰乱国际市场的疑虑,长期关注国际政经局势的本地学者陈光炎预判,国际贸易关系未来一两年必然更趋紧张,美国和欧盟将采取更多贸易保护主义措施……}

\entryitemWithDescription{傅海燕:RCEP处于早期发展阶段 仍有巨大潜力}{https://www.zaobao.com/news/china/story20240328-3230754}{永续发展与环境部部长傅海燕,3月28日在博鳌亚洲论坛2024年年会``打造亚洲增长中心''分论坛上发言。(中新社) 主管贸易关系的永续发展与环境部长傅海燕指出,区域全面经济伙伴关系协定(RCEP)还处于早期发展阶段,仍有巨大发展潜力,新加坡对RCEP扩大阵容保持开放态度……}

\entryitemWithDescription{美国众议院军事委员会跨党派访问团}{https://www.zaobao.com/news/china/story20240328-3230360}{台湾总统蔡英文和副总统赖清德,星期四(3月28日)分别接见美国众议院军事委员会``情报暨特种作战''小组主席伯格曼(Jack Bergman)所率领的跨党派访问团。也是候任总统的赖清德表示,期待台美持续合作,让印太区域更和平稳定……}

\entryitemWithDescription{药明康德据报未经同意与北京分享美国知识财产权}{https://www.zaobao.com/news/china/story20240328-3229792}{美国情报官员据报曾在2月底告诉参与制定生物安全法案的参议员,中国制药公司药明康德未经同意将美国知识财产权转让给中国。图为位于美国圣地亚哥的药明康德工厂。(路透社) (北京综合讯)美国情报官员据报曾在2月底告诉参与制定生物安全法案的参议员,中国制药公司药明康德在未经同意的情况下,将美国知识财产权转让给中国……}