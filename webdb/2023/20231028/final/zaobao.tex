\entryitemWithDescription{台湾半导体产值年减12\% 明后年有望靠AI带动成长}{https://www.zaobao.com/news/china/story20231028-1446287}{台湾半导体产业协会(TSIA)新任理事长、台积电欧亚业务及技术研究资深副总经理侯永清,被业界视为台积电热门接班人。(温伟中摄) 台湾半导体产业协会(TSIA)新任理事长侯永清说,由于经济趋缓和前景不确定,台湾半导体产值预估年减12\%,但明后年有望靠人工智能(AI)带动成长。他也评估,美国扩大AI晶片(也称芯片)销售中国大陆的禁令,对台短期影响较小且可控……}

\entryitemWithDescription{中国原总理李克强因突发心脏病逝世}{https://www.zaobao.com/news/china/story20231028-1446282}{李克强(中)今年3月11日在全国人民代表大会期间,与新当选的国务院总理的李强握手表示祝贺。(法新社) 中国原总理李克强因突发心脏病于星期五(10月27日)凌晨逝世,享年68岁,引起中国舆论震惊。中国官方预计将遵循追悼已故前总理李鹏的前例,为李克强治丧。 央视新闻星期五一早报道,李克强近日在上海休息,星期四(26日)因突发心脏病,经全力抢救无效,于星期五0时10分在上海逝世……}

\entryitemWithDescription{李成:中美在台海交战比美国内战概率更低}{https://www.zaobao.com/news/china/story20231027-1446272}{知名中国观察者李成星期五(10月27日)在慧眼中国论坛上说,中美之间在台海发生战争的概率,比美国发生内战的概率还低。(邝启聪摄) 知名中国政情观察学者李成认为,中美之间在台海发生战争的概率,比美国发生内战的概率还低。 在星期五(10月27日)进行的慧眼中国论坛``重塑中的中国经济与外交之路''主题讨论中,香港大学政治与公共行政学系教授、当代中国与世界研究中心创始主任李成发表了上述观点……}

\entryitemWithDescription{从盐碱荒滩到绿色之城 中新天津生态城获通商中国企业奖}{https://www.zaobao.com/news/china/story20231027-1446268}{通商中国青年奖得主张济徽(左起)、成就奖得主黄山忠,以及领取企业奖的中新天津生态城投资开发有限公司总裁张永昌,星期五(10月27日)晚间出席通商中国奖颁奖礼。(邝启聪摄) 15年的开发建设,让一片盐碱荒滩成为了一座拥有超过13万常住人口的绿色之城。作为新加坡与中国政府合作的重要成果之一,中新天津生态城获颁第11届通商中国企业奖……}

\entryitemWithDescription{温和改革派代表 经济思路被总结为``李克强经济学''}{https://www.zaobao.com/news/china/story20231027-1446259}{(北京综合讯)因突发心脏病去世的中国前国务院总理李克强,专业涵盖法律和经济,是中国第一位拥有经济学博士学位的总理,其经济发展思路曾被总结为``李克强经济学''(Likonomics)。 李克强在2013年3月被任命为国务院总理~,当时中国经济增长放缓,但新一届政府在近三个月的时间里,并未释放明确的经济政策调整信号。同年6月,时任英国巴克莱资本亚洲首席经济学家黄益平提出了``李克强经济学''一词……}

\entryitemWithDescription{李克强与新加坡的渊源}{https://www.zaobao.com/news/china/story20231027-1446260}{2018年11月13日,到访我国的中国总理李克强(前排左二)在时任财政部长王瑞杰(前排左一)的陪同下,参观专门研发癌症细胞疗法的本地生物科技公司Tessa Therapeutics。(~邝启聪摄) ``新-加-坡,代表着永远的创新!永远做加法!'' 已故中国前总理李克强在2018年访问新加坡,在一个晚宴上向本地工商界致辞时如是妙解新加坡……}

\entryitemWithDescription{李克强骤逝引民间悼念 学者:任内形象务实亲民}{https://www.zaobao.com/news/china/story20231027-1446258}{李克强任内每年都会在中国全国``两会''闭幕后举行总理记者会回答中外记者的提问。图为他2014年3月13日在北京人民大会堂出席总理记者会。(法新社档案照片) 中国前总理李克强星期五(10月27日)凌晨在上海骤逝的消息震惊舆论,他生前的视频与语录也被大量转发。受访学者分析,李克强在过去10年任内的形象务实亲民,但实质政治影响力有限,中国民间对他的悼念情绪带有深深的惋惜……}

\entryitemWithDescription{新加坡前驻华大使:美欧在俄乌战争现疲惫迹象}{https://www.zaobao.com/news/china/story20231027-1446232}{资深外交官、新加坡前驻中国大使陈燮荣,星期五(10月27日)在慧眼中国环球论坛的开幕主题论坛上说,美国对俄乌战争的目标``极具野心、前所未闻'',反映出美国的过度自信。(关俊威摄) 资深外交官、新加坡前驻中国大使陈燮荣分析,美国对俄乌战争的目标野心过大,西方开始对持续的战争感到疲惫,中国则间接参与,在中俄合作中获益……}

\entryitemWithDescription{中国民航第三季度复苏强劲 中国国际航空雨过天晴}{https://www.zaobao.com/news/china/story20231027-1446229}{(北京综合讯)中国民航客运市场需求旺盛,行业客运规模创季度历史新高。中国国际航空公司实现前三季度盈利,打响扭亏第一枪。 中国民航局星期五(10月27日)举行新闻发布会,民航局航空安全办公室副主任李勇介绍说,中国民航第三季度呈安全有序恢复态势。全行业共完成旅客运输量1.8亿人次,同比增长108.3\%,较2019年同期增长2.6\%,恢复水平较第二季度提高6.3个百分点……}

\entryitemWithDescription{韩咏红:李尚福落马凸显的反腐难题}{https://www.zaobao.com/news/china/story20231027-1446019}{在10月24日闭幕的全国人大常委会会议上,李尚福原本担任的国务委员、国防部长以及国家中央军委委员职务,都被免去。(法新社) 在消失了近两个月后,原中国国防部长李尚福被官宣落马了。 在本星期二(10月24日)闭幕的全国人大常委会会议上,李尚福原本担任的国务委员、国防部长以及国家中央军委委员职务,都被免去,概括起来是``三免''……}

\entryitemWithDescription{台湾蓝白合继续僵持 进入政党协商延长赛}{https://www.zaobao.com/news/china/story20231026-1446005}{侯友宜23日表态不介意当副手,并通过比民调和初选投票各半的方式,以侯正柯副和柯正侯副的搭配让选民抉择,盼柯文哲25日前回复是否愿意搭档。(法新社) 台湾在野``蓝白合''继续僵持,双方星期四(10月26日)宣告无法就比民调或初选投票方式选出总统参选人达成共识,进入政党协商的延长赛……}

\entryitemWithDescription{中国国安部护航《反间谍法》 指未强制收集外企数据}{https://www.zaobao.com/news/china/story20231026-1445988}{(北京综合讯)中国官方说,国安机构调取数据受到严格制约,并且不涉及外资企业和外籍人员的合法数据活动。 中国施行新修订的《反间谍法》以来,部分外企对正常商务行为可能被针对并进行惩罚感到担忧。 中国国家安全部星期四(10月26日)在微信公众号发文说,认为``《反间谍法》过度强调数据安全并使国家安全机关有权查阅数据,从而给企业和个人带来`数据安全风险'\,''是一种误解……}

\entryitemWithDescription{神舟十七号载人飞船升天 与空间站成功对接}{https://www.zaobao.com/news/china/story20231026-1445983}{中国神舟十七号载人飞船星期四(10月26日)顺利发射升空,完成与空间站对接。图为神舟十七号三名航天员汤洪波(前右)、江新林(后右一)和唐胜杰(后右二),与驻扎空间站的神舟十六号航天员景海鹏(前左)、桂海潮(后左一)和朱杨柱(后左二)在空间站天和核心舱合影。(中国载人航天工程办公室) (酒泉/北京综合讯)中国成功发射神舟十七号载人飞船,完成与中国空间站核心舱的对接,将三名航天员送入空间站……}

\entryitemWithDescription{中国国务委员谌贻琴当选全国妇联主席}{https://www.zaobao.com/news/china/story20231026-1445952}{64岁的谌贻琴是白族人,是首位担任全国妇联主席的少数民族女性。(互联网) (北京综合讯)中国国务委员谌贻琴当选第十三届全国妇联主席,是首位兼任全国妇联主席的国务委员,打破35年来的人事惯例。 据中华全国妇女联合会(简称全国妇联)官网新闻稿,全国妇联第十三届执行委员会第一次会议星期三(10月25日)召开,以无记名投票方式选出全国妇联主席、副主席和常务委员……}

\entryitemWithDescription{俄总理:俄中关系达到前所未有水平并将继续加强}{https://www.zaobao.com/news/china/story20231026-1445912}{中国国务院总理李强当地时间星期三(10月25日)下午在比什凯克出席上海合作组织成员国政府首脑(总理)理事会第22次会议期间,与俄罗斯总理米舒斯京会面。(新华社) (比什凯克 /北京综合讯)中俄两国元首会面一周后,俄罗斯总理米舒斯京在与中国总理李强会面时说,俄中全面战略协作伙伴关系达到了前所未有的水平,并在继续加强……}

\entryitemWithDescription{王纬温:中美合作管控以哈冲突升级?}{https://www.zaobao.com/news/china/story20231026-1445754}{中美在管控以哈冲突上并没有根本分歧,近期还出现一些合作的迹象。图为以加边界上看到加沙地带的烟雾。(路透社) 中国外长王毅星期四(10月26日)将到美国访问三天,是今年2月气球事件引发中美紧张以来,访美的中国最高级别官员,明显是为中美元首可能11月在旧金山再度会晤铺路。在以哈冲突持续近三周之际,王毅还将与美国高层就国际地区问题深入交换意见,为两国可能加强合作管控中东乱局预留伏笔……}

\entryitemWithDescription{香港特首称将在2024年完成基本法第23条立法}{https://www.zaobao.com/news/china/story20231025-1445755}{香港特首李家超(前排)星期三在立法会发表《施政报告》,与他的领导团队一同穿戴由香港知专设计学院(HKDI)师生携手设计制作的绿色领带和领巾。不少议员、政治助理到立法会,也穿上绿色的衣服、配饰、领带等。(彭博社) 近年国际政治形势日益复杂,中国政府越来越重视国家安全,香港特区政府将于明年完成《基本法》第23条立法。有受访学者相信,这项立法不会影响香港的投资环境……}

\entryitemWithDescription{中美问题专家:美国不会在习拜会做出让步}{https://www.zaobao.com/news/china/story20231025-1445751}{美国的中美关系专家韩磊(Paul Haenle)认为,习拜会虽然不会解决中美关系中的所有问题,但提供了一个让中美关系变得更积极的机会。(互联网) 卡内基中国莫里斯·格林伯格荣誉主任韩磊(Paul Haenle)认为,随着中国外交部长王毅确定出访美国,下个月的习拜会相信能如期举行。他说,虽然美国不会在习拜会中做出让步,但两国领导人见面,能创造让中美互动变得更积极且具建设性的机会……}

\entryitemWithDescription{港府为提振楼市宣布``减辣'' 但投资者反应冷淡}{https://www.zaobao.com/news/china/story20231025-1445750}{香港特首李家超星期三宣布多项楼市``减辣''措施'',包括买家印花税减半等。(路透社) (香港综合讯)香港特首李家超星期三(10月25日)宣布多项楼市``减辣''措施'',包括买家印花税减半等。不过,投资者对此反应冷淡,香港主要地产开发商的股价星期三收跌……}

\entryitemWithDescription{中国``神舟十七号''将于10月26日发射}{https://www.zaobao.com/news/china/story20231025-1445736}{中国将于星期四上午发射``神舟十七''号载人飞船,执行神舟十七号载人飞行任务的航天员乘组汤洪波(中)、唐胜杰(右)、江新林(左)三名航天员星期三上午在酒泉卫星发射中心问天阁与中外媒体记者集体见面。 (中新社) (北京中新电)中国将于星期四(10月26日)上午发射``神舟十七''号载人飞船,并将首次在``天宫''空间站舱外进行试验性维修作业……}

\entryitemWithDescription{台湾朝野互批``亲中卖台'' 互相质疑对台湾的忠诚}{https://www.zaobao.com/news/china/story20231025-1445729}{国民党立委马文君(左一)和民进党立委赵天麟一同出席立法院外交及国防委员会。(自由时报) 台湾朝野互批``亲中卖台'',执政的民进党猛攻在野的国民党立委马文君泄露国防机密,要求她退选;国民党反质疑民进党立委赵天麟被中国大陆间谍色诱,呼吁暂停其立委职务展开调查。 台湾立法院外交及国防委员会星期三(10月25日)和星期四审查国防和机密预算两天,事关巨额预算和国安,也成了朝野选战攻防的战场……}

\entryitemWithDescription{分析:中国增发国债兼具稳增长、提信心和化风险作用}{https://www.zaobao.com/news/china/story20231025-1445726}{中国增发1万亿元人民币国债的利好消息带动陆港股市终结四连跌。(路透社) 中国增发1万亿元(人民币,下同,1894亿新币)国债的利好消息带动陆港股市终结四连跌。分析指出,在今年中国经济增速达标无忧的背景下,新一轮刺激措施主要为明年稳增长发力,同时起到提振信心和化解风险的作用。 沪深300指数星期三(10月25日)开盘后一度上涨近1.3\%,过后震荡回落,全天上涨0.5%……}