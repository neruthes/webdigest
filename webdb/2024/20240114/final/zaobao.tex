\entryitemWithDescription{接棒``双不过半''弱势政府 赖清德:台湾政治须走向协商合作}{https://www.zaobao.com/news/china/story20240114-1461980}{当选台湾总统的赖清德与搭档萧美琴胜选后出席在台北民进党总部外举行的记者会。台湾星期六(1月13日)举行的总统选举全球瞩目,吸引了逾400名国际记者赴台采访。(法新社) 当选台湾总统的赖清德说,民进党在立法院席次未过半,说明人民期待``有效率的制衡'',台湾政治须走向协商与合作。 赖清德星期六(1月13日)在胜选后的国际记者会上致辞说,民进党无法在立法院席次过半,代表努力不够,必须虚心检讨……}

\entryitemWithDescription{拼政党轮替失败 侯友宜眼眶泛红为败选负责 学者:国民党中央须改变官僚老旧习气}{https://www.zaobao.com/news/china/story20240114-1461975}{国民党正副总统候选人侯友宜(左二)、赵少康(右二),以及党主席朱立伦(左一)、侯友宜竞选办公室执行长金溥聪(右一)星期六(1月13日)晚在竞选总部,为没能实现政党轮替,向支持者鞠躬致歉。(法新社) 台湾最大在野国民党力拼政党轮替再次失败。国民党总统候选人侯友宜星期六(1月13日)败选后,三度向支持者鞠躬致歉,眼眶泛红表示,``我尊重台湾人民作出的最后选择''……}

\entryitemWithDescription{韩国瑜出任立法院长?民众党意愿成关键}{https://www.zaobao.com/news/china/story20240114-1461973}{主张废除死刑的社民党台北市议员苗博雅,挟带青年族群的高声量,在台北市大安区立委选举拿下逼近45\%的得票率,虽竞选失利,卻是历来泛绿阵营在该区最佳表现。(档案照片) 韩国瑜可能出任立法院长 台湾立委选举落幕,国会朝小野大局面底定。外界推测,未来立法院长可能由国民党不分区立委排名第一的前高雄市长韩国瑜出任,但仍要看民众党所扮演的关键少数,是否同意……}

\entryitemWithDescription{赖清德当选总统 分析:两岸关系将受冲击}{https://www.zaobao.com/news/china/story20240114-1461971}{台湾民进党主席赖清德(左)在星期六举行的大选中,以逾558万得票数胜出。图为赖清德和副总统当选人萧美琴(右)在胜选后召开国际记者会。(法新社) 自称务实台独工作者的台湾民进党主席赖清德在星期六(1月13日)举行的大选中,以逾558万得票数胜出,在三角战中以40.05%得票率当选总统。民进党由此首次得以连续执政超过八年,两岸关系和台海稳定前景则料受冲击……}

\entryitemWithDescription{民众党选后抓``战犯'' 柯文哲:我承担最大责任}{https://www.zaobao.com/news/china/story20240114-1461970}{台湾民众党总统候选人柯文哲、副总统候选人吴欣盈1月13日晚间出席记者会。(路透社) 台湾立法委员选举结果三党不过半,民众党拿下八席立委席次,被视为未来国会关键少数。不过,民众党内部在选后就开始找败选``战犯'',党主席柯文哲星期六(1月13日)晚间坦言,他要承担最大责任,但为避免党内分裂,暂时不会请辞党主席……}

\entryitemWithDescription{分析:新总统520就职前北京将加大对台军事与经济施压}{https://www.zaobao.com/news/china/story20240114-1461969}{台湾民进党候选人连续三届当选总统。受访学者分析认为,5月20日总统就职演说前对台的文攻武吓大概不可避免。 (路透社) 台湾民进党候选人连续三届当选总统,受访学者分析,台湾总统选举结果反映对两岸关系持有对抗情绪的台湾选民仍占大多数,赖清德的胜选感言预计也让北京``非常不满意'',接下来几个月北京对台文攻无吓估计不可免。 台湾自2000年以来,没有政党能连续执政超过八年……}

\entryitemWithDescription{投票日当天 微博屏蔽台湾选举有关话题}{https://www.zaobao.com/news/china/story20240113-1461930}{台湾总统和立法委员选举当天(1月13日)早上,``台湾选举''词条登上中国大陆最大社媒平台之一微博的热搜榜,但这一热门词条随后被屏蔽。(法新社) (北京/台北综合讯)台湾总统和立法委员选举星期六(1月13日)登场,中国大陆网民对大选结果高度关注。当天上午8时投票开始后,有关词条``台湾选举''便登上中国大陆最大社媒平台之一微博的热搜榜,但这一热门词条随后在微博被屏蔽……}

\entryitemWithDescription{台湾选举日再侦获中国大陆空飘气球}{https://www.zaobao.com/news/china/story20240113-1461929}{台湾在总统和立委选举日再侦获中国大陆空飘气球。(法新社) (台北综合讯)台湾在总统和立委选举日再侦获中国大陆空飘气球。 台湾国防部星期六(1月13日)即台湾总统和立委选举投票日在网站公布最新解放军台海周边海、空域动态,显示星期五(1月12日)早6时至星期六早6时,共侦获大陆军机八架次、军舰六艘次,其中一架运8反潜机进入西南空域……}

\entryitemWithDescription{【热点评论】台湾总统选后 台海风云叵测}{https://www.zaobao.com/news/china/story20240113-1461872}{注:若无法通过此页面观看直播,请点击这里。 堪称历来最诡谲多变、纠缠不清的台湾总统选举三角战形势,星期六(1月13日)终有定案。民进党候选人赖清德萧美琴组合自下午四时计票开始之后持续领先,执政党似乎抵住了面对高达六成民意要求``换政府''的压力成功保住政权,这是否预示绿营在台湾长期执政时代到来……}

\entryitemWithDescription{新闻人间:以历史视角看《繁花》}{https://www.zaobao.com/news/china/story20240113-1461456}{新闻人间------繁花 改编自同名小说的电视剧《繁花》本周收官,这部爆红剧集引发多层次讨论,除拍摄手法、明星演技、沪语传播外,从历史视角也有可谈之处。 电视剧展现的上世纪90年代,是中国历史转折期,上海扮演了重要角色。1989年以后,中国改革陷入僵局,全国展开姓资姓社大论战。1991年,上海《解放日报》以笔名``皇甫平''发表文章,捍卫改革开放;1992年,邓小平南巡,重要一站正是上海……}

\entryitemWithDescription{温伟中:台湾大选的热与冷}{https://www.zaobao.com/news/china/story20240113-1461744}{到台湾南部的高雄看选举造势,蓝绿白三党会场氛围比我预期的更热。 选前超级星期天(1月7日),蓝营的国民党、绿营的民进党、白营的民众党都到绿营铁票仓高雄拼场。绿营要凝聚基本盘,蓝营想突破同温层,白营也不想被比下去。 我骑共享脚踏车Youbike跑三摊、避车龙。当天蓝绿都宣称来了12万人、白营也喊出8万人相挺。国民党在梦时代广场的场子人潮满溢,许多人手挥青天白日满地红旗……}

\entryitemWithDescription{台湾大选选前之夜 蓝绿白造势大拼场}{https://www.zaobao.com/news/china/story20240113-1461798}{成立四年半的台湾民众党声势惊人,支持者星期五(1月12日)挤满总统府前的凯达格兰大道,现场宣称来了35万人。(台湾民众党提供) 台湾星期六(1月13日)举行总统与立法委员选举,选前之夜蓝绿白造势活动的人潮都爆满,民进党能否延续政权、在野党能否变天、立法院是否三党不过半,将在星期六傍晚以后陆续揭晓。 这场选举牵动两岸和中美关系,也将影响台海局势与区域稳定,备受世界瞩目……}

\entryitemWithDescription{力抗蓝绿夹杀 柯文哲喊话年轻选民投票}{https://www.zaobao.com/news/china/story20240112-1461786}{台湾民众党总统候选人柯文哲星期五(1月12日)晚间在台北凯达格兰大道举办选前之夜造势活动。前排左起为台湾民众党副总统候选人吴欣盈、柯文哲、柯文哲夫人陈佩琪、民众党市议员陈世轩。(台湾民众党提供) 台湾总统大选星期六(1月13日)登场,此前一晚,面对国民党、民进党两大阵营夹杀的台湾民众党总统候选人柯文哲,向年轻支持者喊话,要他们一定回家投票……}

\entryitemWithDescription{中加外长通电 分析:北京望降低抗中阵营规模}{https://www.zaobao.com/news/china/story20240112-1461784}{中国和加拿大外长星期四(1月11日)通电话,这是两国外交高层相隔10个月后再度交流对话。中国外长王毅说,当前中加关系困局并非中国所望,北京对同渥太华接触对话持开放态度。加拿大外长乔利则称加国将寻求务实外交,继续捍卫基于规则的国际秩序。 受访学者分析,中美高层今年加快恢复对话沟通,有利于中国与美国其他盟友恢复正常外交对话交流……}

\entryitemWithDescription{香港未明确提交23条立法草案具体时间}{https://www.zaobao.com/news/china/story20240112-1461770}{香港政府早前表明将于今年内完成《基本法》第23条立法,但当局星期五(1月12日)向立法会提交的本年度立法议程中却没有明确标明提交草案的时间。有分析指出,港府是要因应国际形势变化而适时立法,以免遭到外部因素阻挠。 港府提交的2024年度立法议程,共有29项议程,包括最受关注的``维护国家安全条例草案'',内容为落实《基本法》第23条完善相关法律以维护国家安全,并就相关事宜订定条文……}

\entryitemWithDescription{上海地铁明确非日常服饰进站将被核查身份}{https://www.zaobao.com/news/china/story20240112-1461754}{(上海综合讯)一名身着动漫服饰的女子近日在上海地铁进站时被安检人员拦下,引发热议。上海地铁回应说,安检人员会拦截穿着非日常服饰进站的乘客,并通知民警进行身份核查。 综合东方网、《大河报》报道,网传视频显示,这名女子当时的装扮配有帽子和紫色假发,头发上贴有字符装饰。她解释自己是名角色扮演(cosplay)爱好者,身上是她搭配的汉服。因头饰和帽子不易重新装配,她不愿摘下,担心耽误行程……}

\entryitemWithDescription{解振华因身体原因卸任中国气候变化事务特使}{https://www.zaobao.com/news/china/story20240112-1461751}{解振华卸下中国气候变化事务特使一职,他从事环保工作逾40年,长期参与全球气候谈判。图为他2023年12月出席阿联酋举行的联合国气候大会(COP28)。(路透社) (北京综合讯)中国官方批准,解振华由于身体原因卸任中国气候变化事务特使,由外交部原副部长刘振民接任。 中国生态环境部于星期五(1月12日)宣布上述消息……}

\entryitemWithDescription{【早现场】台湾选前之夜大造势}{https://www.zaobao.com/news/china/story20240112-1461741}{注:若无法通过此页面观看直播,请点击这里。 台湾星期六(1月13日)举行总统和立法委员选举。各组候选人在星期五(1月12日)晚的选前之夜举行最后一场大造势催选票。 蓝绿两大阵营都选择在新北市的板桥区造势,决战态势明显。 执政民进党的造势活动晚上7时在板桥第二运动场举行。在野国民党的造势晚会则在相隔不到10分钟车程的板桥第一运动场举行,时间晚半小时开始……}

\entryitemWithDescription{郭台铭动向不明 周孟蓉转述:郭不会支持柯文哲}{https://www.zaobao.com/news/china/story20240112-1461729}{挺郭大将、屏东县议长周典论之女周孟蓉称,鸿海集团创办人郭台铭不会支持民众党总统候选人柯文哲。图为郭台铭(右二)、柯文哲(右一)和国民党总统候选人侯友宜(左一)去年11月2日出席在台北葫芦寺举行的祭祀活动。(法新社) 台湾在野的国民党在鸿海集团创办人郭台铭退选后一再呼唤他回归,但直至星期五(1月12日)选前之夜,郭台铭依然未现身……}

\entryitemWithDescription{中国科学院院士:科兴疫苗停产是疫苗迭代必然结果}{https://www.zaobao.com/news/china/story20240112-1461727}{(北京综合讯)针对中国国产的科兴冠病疫苗停产,中国科学院院士认为,科兴疫苗退市是冠病疫苗迭代的必然结果。 中国社交媒体本周爆出科兴生物已停产冠病疫苗的消息,后来得到科兴证实,``科兴新冠疫苗已停产''话题冲上微博热搜。 中国科学院院士、免疫学家魏于全接受央广网专访时说,科兴冠病疫苗是针对原始、非变异的冠病病毒而研发的灭活疫苗,在疫情期间发挥重要作用,停产是疫苗迭代升级的结果……}

\entryitemWithDescription{韩咏红:台湾大选歹戏拖棚?}{https://www.zaobao.com/news/china/story20240112-1461572}{2024年台湾选举进入倒数两天,最后关头原本应是高潮迭起,观众心跳加速的时刻,但现实中台湾大选却彷如篮球赛进入垃圾时间,两队比分迟迟不见变化,只剩下歹戏拖棚……}

\entryitemWithDescription{对政治冷感 台青年只盼生活有保障}{https://www.zaobao.com/news/china/story20240112-1461567}{``你相信我家隔壁的7-Eleven(职员),起薪都比记者还要高吗?'' ``不是说不珍惜手上这张民主的票,是因为我觉得眼前所看到的根本都改变不了。'' 两名台湾青年------一人为就业发愁,一人对政治失望选择不投票。 台湾年轻和中间选民有约600万,他们在本届总统选举被视为最终的造王者,蓝绿白三大阵营都努力争取年轻选票……}