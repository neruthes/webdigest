\entryitemWithDescription{``蓝白合''政党协商登场 没谈总统选举合不合}{https://www.zaobao.com/news/china/story20231030-1446817}{民众党主席柯文哲(左)星期一(10月30日)与国民党主席朱立伦在台北市长官邸进行``蓝白合''政党协商。(国民党提供) 台湾在野``蓝白合''政党协商星期一(10月30日)登场,两党主席会谈后发表共同声明,但没谈到最受关注的总统选举合不合、怎么合。 台湾将在两个多月后、2024年1月13日举行总统与立委选举……}

\entryitemWithDescription{港媒:大陆出入境健康申报最快11月1日取消}{https://www.zaobao.com/news/china/story20231030-1446816}{港媒报道,北京计划取消已实施近四年、俗称``黑码''的出入境健康申报手续。图为香港福田口岸。(中通社) (香港综合讯)港媒报道,北京计划取消已实施近四年、俗称``黑码''的出入境健康申报手续。 综合《明报》、香港01等消息,中国全国人大常委李慧琼引述大陆海关总署表示,取消黑码已进入审批最后阶段,有信心星期三(11月1日)起实施……}

\entryitemWithDescription{俄防长:西方执意升级俄乌战争可能导致核大国军事冲突}{https://www.zaobao.com/news/china/story20231030-1446815}{观察家认为,美国和大多数西方盟国不派出最高层级国防官员出席香山论坛,以免为北京试图掌握国际话语权捧场。图为中国国防部发言人吴谦大校(左)星期一在香山论坛,与以专家身份与会的美国国防部前助理防长施灿德(Chad Sbragia)交谈。(路透社) 俄罗斯防长绍伊古警告,西方执意升级与俄罗斯的冲突,可能导致核大国直接发生军事冲突。他也表明,俄罗斯准备好就乌克兰危机在冲突之后的解决,进行政治讨论……}

\entryitemWithDescription{台湾首度曝光仿美陆军作战指挥系统}{https://www.zaobao.com/news/china/story20231030-1446805}{台军``长青17号''操演10月27日在嘉义县举行,仿美陆军指挥系统也首度曝光。(自由时报) (台北讯)台湾军方首度曝光仿美陆军指挥系统,战时可通过美军频道联络其地面部队。 《联合报》报道,台湾``长青17号''操演于10月27日结束,总统府披露总统蔡英文在操演中听取参演``锐指专案''简报,仿美陆军作战指挥系统架构也因此首度曝光。 台湾陆军内部的``锐指专案''装备也被称为``讯联专案''……}

\entryitemWithDescription{美参议员携枪抵港获撤控 准签保守行两年}{https://www.zaobao.com/news/china/story20231030-1446803}{(香港综合讯)美国华盛顿州参议员威尔逊(Stephen Jeff Wilson)因携带枪械入境香港被控,经法官审查认为他无意犯案,遂撤销控罪。 综合香港01、彭博社等报道,威尔逊于10月21日在香港旅行期间,海关人员在其行李中发现一支左轮手枪。他遂自行向海关申报,后被控无牌管有枪械罪。 威尔逊声称自己并不知情,且已在美国俄勒冈州机场顺利通过安检……}

\entryitemWithDescription{韩正会见美国飞虎队代表团}{https://www.zaobao.com/news/china/story20231030-1446791}{中国国家副主席韩正(右二)星期一(10月30日)在北京会见美国飞虎队代表团。(新华社) (北京综合电)中国国家副主席韩正星期一会见美国飞虎队代表团时表示,中国人民始终铭记飞虎队的英勇事迹,不会忘记老朋友。 据新华社报道,韩正星期一(10月30日)在北京会见美国飞虎队代表团表示,``80多年前中美为抗击法西斯并肩作战,飞虎队的佳话承载着中美两国人民用生命和鲜血铸就的深厚友谊……}

\entryitemWithDescription{美智库研究报告:2050年中国制造业就业岗位将占全球43\%}{https://www.zaobao.com/news/china/story20231030-1446784}{美国智库预计,到了2050年,中国在全球制造业就业岗位中所占比例将上升到43\%。图为9月14日江苏南通一家纺织厂工人操作工厂机器。(法新社) (华盛顿法新电)总部位于美国华盛顿的全球发展中心星期一(10月30日)发布的一项研究显示,尽管美国和欧盟希望减少对中国产品的依赖,但到了2050年,中国在全球制造业就业岗位中所占比例预计将上升到43\%……}

\entryitemWithDescription{北京大学发文缅怀校友李克强:师生沉浸在深深悲痛中}{https://www.zaobao.com/news/china/story20231030-1446777}{中国国务院前总理李克强上星期五突发心脏病逝世,他的母校北京大学校报星期天刊文缅怀。~文章也附上过往校刊关于李克强的剪报,``以寄托对李克强校友的哀思''。(北京大学校报微信公号) (北京/柏林综合讯)中国国务院前总理李克强上星期五突发心脏病逝世,他的母校北京大学校报星期天刊文,深切缅怀这位杰出校友……}

\entryitemWithDescription{恒大清盘呈请聆讯再押后 法官指是最后通牒}{https://www.zaobao.com/news/china/story20231030-1446734}{中国恒大集团清盘呈请聆听推迟至12月4日,再获35天延期重组巨额债务,不过香港高等法院明确表示,这是恒大最后的机会。(法新社档案照) (香港综合讯)中国恒大集团再获35天延期重组巨额债务,不过香港高等法院明确表示,这是恒大``最后的机会''。 中国恒大星期一(10月30日)在香港交易所公告,香港高等法院于2023年10月30日将呈请聆讯进一步延至2023年12月4日……}

\entryitemWithDescription{庄慧良:``蓝白合''方程式有解或无解?}{https://www.zaobao.com/news/china/story20231030-1446590}{台湾总统选举将在2024年1月13日举行。(彭博社) 台湾在野国民党(蓝)和民众党(白)协商2024年正副总统组合,上星期宣告失败,双方龃龉不断,本周将展开第二阶段政党协商,依旧阴霾满天。 国民党和民众党皆知凭一己之力绝无可能达到政党轮替目标,``蓝白合''势在必行,却浪费半年时间,拖至11月20日正式登记正副总统前一个月才展开协商,已失先机……}

\entryitemWithDescription{中国纪检监察机关前三季立案47万件 省部级处分人数较去年减少}{https://www.zaobao.com/news/china/story20231029-1446581}{(北京综合讯)官方数据显示,中国纪检监察机关2023年前三季度累计立案47万件,对40万5000人予以处分,涉及34位省部级领导干部。 据中央纪委国家监委星期日(10月29日)通报,今年1至9月全国纪检监察机关共接收信访举报261万7000件次,其中检举控告类信访举报81万9000件次。处置问题线索128万3000件,其中谈话函询26万6000件……}

\entryitemWithDescription{中国发现10万吨级铀矿床 为核能发展提供保障}{https://www.zaobao.com/news/china/story20231029-1446575}{(北京综合讯)中国核工业地质局称,发现了一批万吨至10万吨级铀矿床,初步确立了天然铀的供应保障。 据央视新闻报道,中国核工业地质局局长陈军利上星期六(10月28日)在首届国际天然铀产业发展论坛上表示,中国已建立``天空地深''一体化铀矿勘查技术,累计探明多个大型、特大型铀矿床。十年来,新增铀矿资源储量占累计查明总量的三分之一,中国完全能够保障核能发展对天然铀的需求……}

\entryitemWithDescription{富士康被查后郭台铭首次现身 幕僚否认退选传闻}{https://www.zaobao.com/news/china/story20231029-1446572}{独立参选台湾总统的鸿海集团创始人郭台铭,星期天(10月29日)在新北板桥的``北台湾妈祖文化节''巡安会香祈福活动中致辞。(路透社) (台北/新北综合讯)富士康集团在中国大陆被查,独立参选台湾总统的鸿海集团创始人郭台铭神隐多日引发退选传闻后,星期六首次公开露面。郭办发言人否认退选传闻,郭台铭本人则在宗教活动中向妈祖祈求三件事……}

\entryitemWithDescription{恒大星期一面临清盘听证会 恐成港史上倒闭最大地产商}{https://www.zaobao.com/news/china/story20231029-1446569}{中国房地产巨头恒大集团位于深圳的总部大楼。(路透社) (深圳综合讯)中国房地产巨头恒大集团面临关键时刻。若法院颁令清盘,不仅集团主席许家印可能失去控制权,恒大也将成为香港法律史上遭遇清盘的最大型地产商。 彭博社报道,负债约3270亿美元(4476亿新元)的恒大,星期一(10月30日)将在香港高等法院面临清盘呈请聆讯,曾支持其重组的部分债权人现已摇摆,令其清盘风险大增……}

\entryitemWithDescription{中国前驻美大使崔天凯:中美``躺平''不能自动避免冲突}{https://www.zaobao.com/news/china/story20231029-1446561}{北京香山论坛星期一(10月30日)将举行开幕式,图为论坛会场。(路透社) 中国前驻美国大使崔天凯说,中美对抗冲突对世界肯定是灾难,但不代表``躺平''就能自动避免冲突,仍需双方共同努力。 中国官媒《环球时报》引述崔天凯星期日(10月29日)在第10届北京香山论坛高端对话的发言称,中美两个大国若发生冲突对抗,对两国及世界肯定是灾难性的,两国大多数人都理性地意识到这一点……}

\entryitemWithDescription{加籍华裔学者续签被拒 港中大终止委聘}{https://www.zaobao.com/news/china/story20231029-1446525}{香港中文大学官网星期日(10月29日)显示何晓清``现正休假''。 (香港综合讯)加拿大籍华裔历史学者何晓清的香港工作签证延期未获批准,随后收到香港中文大学的解雇通知。 综合香港01、《明报》等报道,何晓清表示去年7月申请续签,直到上星期二(10月24日)被香港入境处通知拒签。香港中文大学三天后因其签证问题,即时终止她的副教授委聘。 何晓清对此感到无奈和可惜,且不确定签证被拒的具体原因……}

\entryitemWithDescription{新加坡国大校长陈永财:中西方竞争波及大学跨国研究}{https://www.zaobao.com/news/china/story20231029-1446524}{国大校长陈永财10月12日在新国大重庆研究院接受《联合早报》专访。(王纬温摄) 中西方地缘政治和科技竞争不断升级,一定程度波及大学跨国学术研究。本月到访重庆的新加坡国立大学校长陈永财教授表示,国际政治对科研合作及人与人交流造成的影响是``不幸的'',国大在芯片等领域的跨国研究,目前可能面临一些局限……}