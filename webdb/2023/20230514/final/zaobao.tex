\entryitemWithDescription{中国首个mRNA冠病疫苗在石家庄接种}{https://www.zaobao.com/news/china/story20230513-1394326}{河北石药集团开发的中国国产mRNA(信使核糖核酸)冠病疫苗星期六(5月13日)在石家庄一个社区卫生服务中心接种了获批以来的全国首针。 《河北日报》星期六披露了上述消息。 石药集团mRNA疫苗是于3月22日获中国国家卫健委批准,被纳入紧急使用,可用于18岁以上人群加强免疫接种,并在4月10日被国家卫健委列为加强免疫优先推荐疫苗。它也是目前唯一获批在中国大陆使用的mRNA疫苗……}

\entryitemWithDescription{中国原民生银行副行长退休三月后被查}{https://www.zaobao.com/news/china/story20230513-1394323}{中国又有一名曾在金融监管部门任职多年的官员被查,显示中国金融反腐态势逐步扩大。 中央纪委国家监委在中国银保监会派驻的纪检监察组和黑龙江纪委监委星期六(5月13日)发出消息,称中国民生银行管理咨询委员会原副主席邢本秀涉嫌严重违纪违法,目前正接受调查……}

\entryitemWithDescription{副总理兼财政部长黄循财星期六起访华}{https://www.zaobao.com/news/china/story20230513-1394322}{我国副总理兼财政部长黄循财5月13日至17日正式访问中国,这是他首次以副总理身份访华。(取自黄循财脸书) 我国副总理兼财政部长黄循财于5月13日至17日对中国进行正式访问,他将与中国总理李强、副总理丁薛祥等高层官员会面。 根据总理公署发布的文告,黄循财是应丁薛祥之邀访华,他将到访上海和北京。这是黄循财首次以副总理身份访华……}

\entryitemWithDescription{台民调:侯友宜支持度领先郭台铭12.7个百分点}{https://www.zaobao.com/news/china/story20230513-1394318}{最新民调显示,台湾新北市长侯友宜的支持度继续领先鸿海集团创办人郭台铭。分析称,最后关头郭台铭需要``弯道超车''才可超越侯友谊。 台湾民意基金会星期六(5月13日)发布最新民调结果,显示侯友宜支持度42.1\%,郭台铭支持度29.4\%,侯友宜领先郭台铭12.7个百分点……}

\entryitemWithDescription{``五个女博士''品牌涉发布低俗广告 被立案调查}{https://www.zaobao.com/news/china/story20230513-1394314}{功能饮料品牌``五个女博士''近期一则广告被指侮辱女性。图为该视频广告的部分截图。(互联网) 中国一家名为``五个女博士''的胶原蛋白饮料品牌因涉嫌发布低俗广告,被市场监管部门立案调查。 在``五个女博士''引发争议的电梯广告中,几名女性表情夸张地展示在不同情况喝下该品牌的胶原蛋白饮料:``老公气我,喝''\,``熬夜追剧,喝''\,``又老一岁,喝''\,``喝五个女博士,都是你们逼的……}

\entryitemWithDescription{中国领导人G7峰会前密集访欧 秦刚吁中欧共同反对``新冷战''}{https://www.zaobao.com/news/china/story20230513-1394305}{中国国家副主席韩正5月11日在荷兰海牙与荷兰首相吕特举行会谈。(新华社) 七国集团峰会即将召开,中国国务委员兼外长秦刚星期五(5月12日)在挪威强调新冷战只会带来更大灾难,并呼吁中欧共同反对新冷战,带头推进大国协调和良性互动……}

\entryitemWithDescription{尊子漫画停刊下架 香港新闻自由再引关注}{https://www.zaobao.com/news/china/story20230513-1394298}{香港《明报》刊登政治漫画专栏``尊子漫画''已有40年。图为5月10日的《明报》封面版与刊登在内页的``尊子漫画''。(法新社) 在香港《明报》已有40年历史的政治漫画专栏``尊子漫画'',突然宣布从星期日(5月14日)起停刊。该专栏过去半年多以来屡次被香港官员指责``抹黑政府'',这次停刊再次引起了业界担忧香港的新闻自由状况……}

\entryitemWithDescription{早点:一封电邮搅混2024台湾总统战局}{https://www.zaobao.com/news/china/story20230513-1394091}{台湾鸿海集团创办人郭台铭近日重提2021年采购BNT疫苗(也称辉瑞疫苗)往事,星期二(5月9日)爆料指时任总统府秘书长李大维曾替总统蔡英文传话:``大小姐说,你还是不要买了'',瞬间在台湾政坛掀起波涛。 政治圈众所皆知、总统府御用的陶姓记者当晚深夜发了一则独家消息,强调``郭台铭早知BNT不卖他!仍甩锅蔡英文挡疫苗'',并同时公布一封BNT大股东2021年6月16日写给郭台铭的私人电邮……}

\entryitemWithDescription{新闻人间:``95后董明珠''孟羽童离职格力}{https://www.zaobao.com/news/china/story20230513-1394102}{孟羽童 年仅22岁就在万众瞩目下成为中国家电巨头格力电器董事长董明珠的秘书、曾被赋予``接班人''厚望的孟羽童,不到两年就从格力离职了。 中国网民星期二(5月9日)注意到,格力电器旗下的``明珠羽童精选''抖音账号更名为``格力明珠精选'',并删除了与孟羽童有关的内容。格力次日公开证实,孟羽童已经从公司离职,并称这是正常的人员流动。 孟羽童离职的话题,一度冲上了新浪微博热搜榜首……}

\entryitemWithDescription{中国特别代表下周出访乌克兰等国斡旋 分析:``试水温''之旅暂难取得实质成果}{https://www.zaobao.com/news/china/story20230512-1393996}{中国政府欧亚事务特别代表、前驻俄罗斯大使李辉,将从下周一(5月15日)起访问乌克兰、波兰、法国、德国和俄罗斯,就政治解决乌克兰危机同各方沟通。 受访学者分析,李辉此行是``试水温''之旅,虽不太可能在解决问题上取得具体和实质的成果,但却是外交调解俄乌战争的重要一步。 中国外交部发言人汪文斌星期五(5月12日)在例行记者会上宣布李辉出访消息……}

\entryitemWithDescription{侯友宜郭台铭相遇两手紧握互动热络 蓝营重燃两人整合成功乐观情绪}{https://www.zaobao.com/news/china/story20230512-1393986}{鸿海集团创办人郭台铭(右二)和新北市长侯友宜(右一),星期五(5月12日)在板桥慈惠宫庆祝妈祖诞辰的活动相遇,全程两手紧握、互动热络,还说了几句悄悄话。(郭台铭办公室提供) 国民党预料下个星期三(5月17日)征召总统候选人,鸿海集团创办人郭台铭与新北市长侯友宜都在做最后冲刺,不过两人星期五(5月12日)相遇时两手紧握、互动热络,令近期声势低迷的蓝营重燃两大重量级人物可能整合的乐观情绪……}

\entryitemWithDescription{中国手机巨企终止芯片设计业务}{https://www.zaobao.com/news/china/story20230512-1393985}{随着全球智能手机市场持续下滑,中国最大的智能手机制造商OPPO停止其芯片业务。 OPPO公司星期五(5月12日)宣布,面对全球经济、手机市场的不确定性,公司决定终止芯片子公司哲库(ZEKU)的业务。 哲库曾开发过图像处理芯片,并于2021年末发布,此后一直是OPPO旗舰手机的固定配置……}

\entryitemWithDescription{中国外长秦刚将出访澳洲 释放两国关系可维持``斗而不破``信号}{https://www.zaobao.com/news/china/story20230512-1393982}{澳大利亚贸易部长法瑞尔(左三)和中国商务部部长王文涛(右三)5月12日在北京的中国商务部会面。(路透社) 中国国务委员兼外长秦刚将访问澳洲,同时也邀请中国商务部部长王文涛访澳。这将是2017年以来首次有中国外长访澳;受访学者分析,若秦刚成行,将显示两国关系可维持在斗而不破的范畴内……}

\entryitemWithDescription{前NBA球星霍华德为``台独''言论道歉}{https://www.zaobao.com/news/china/story20230512-1393961}{台湾中华文化总会邀请霍华德(右)和赖清德(左)合作拍摄宣传片``来去总统府住一晚'',霍华德在视频中表述台湾为``国家'',引发中国大陆网民不满。图为两人在总统府内拍摄视频现场互动。(自由时报) (宜兰综合讯)NBA前球星霍华德在台湾一则宣传片中称台湾为``国家'',激起中国大陆网民愤怒。霍华德表示他``为伤害到任何人道歉'',强调不想涉及政治,但拥有``言论自由''……}

\entryitemWithDescription{主播线上销售162万颗泰国榴梿 被指致零售价格上涨}{https://www.zaobao.com/news/china/story20230512-1393944}{近日,中国主播在直播平台卖出162万颗泰国榴梿后,被质疑垄断市场,导致中国市场榴梿零售价格上涨。 综合封面新闻和《北京青年报》报道,在短视频平台上拥有一亿粉丝的中国网络主播辛有志(网名辛巴),5月7日在直播平台预售泰国榴梿,销售额达3亿元(人民币,5757万新元)。后有水果商和消费者在社交网络上称,榴梿短时间大量销售,致市场库存减少,价格上涨……}

\entryitemWithDescription{韩咏红:中国经济``长冠效应''显现}{https://www.zaobao.com/news/china/story20230512-1393607}{也许在2023年里,大家更需要比拼的不是有多高速的增长,而是谁更能保住元气……}

\entryitemWithDescription{继咨询尽职调查公司之后 中国证券公司被要求加强控制敏感信息传播}{https://www.zaobao.com/news/china/story20230511-1393605}{中国加强监管敏感信息,继咨询企业、尽职调查公司之后,中国国内证券公司被要求加强控制敏感信息的传播。 综合路透社、彭博社报道,中国监管机构要求国内证券公司不得泄露和传播可能对市场造成重大影响的不实信息。 中国监管机构是在对45家证券公司的300份研究报告进行现场检查后,向各证券公司发出通报……}

\entryitemWithDescription{学者:特拉斯访台显示英国如今更关注台海和大陆发展}{https://www.zaobao.com/news/china/story20230511-1393594}{英国前首相特拉斯下星期二(5月16日)将到访台湾五天,是继撒切尔夫人在1992年和1996年两度访台后,暌违27年后再有英国前首相访台。受访学者指出,种种迹象显示,英国如今更关注台海局势和中国大陆发展。 特拉斯(Liz Truss)是英国对华鹰派,曾是英国1000年来首任女性大法官。去年9月6日当首相后推出引爆争议的减税计划,上任45天后辞职,目前是保守党的国会议员……}

\entryitemWithDescription{香港政治漫画家尊子漫画 刊载40年后遭停刊}{https://www.zaobao.com/news/china/story20230511-1393590}{香港《明报》从这个星期日(14日)起,停止刊登著名政治漫画家黄纪钧的``尊子漫画''专栏。 综合《明报》、HK01、星岛网报道,香港记者协会认为漫画停刊反映香港容不下批评声音,将加剧媒体自我审查;明报工会也表示遗憾和无奈。 香港保安局长邓炳强则指,若有平台被利用煽动市民,编辑有责任停止……}

\entryitemWithDescription{中国CPI涨幅降至两年多来最低水平}{https://www.zaobao.com/news/china/story20230511-1393587}{中国4月份的消费者价格指数(CPI)涨幅降至两年多来最低水平,图为上海市民今年3月在一家杂货店购买食品。(互联网) 中国4月份的消费者价格指数(CPI)涨幅降至两年多来最低水平,生产价格指数(PPI)则连续第七个月下降,这两项最新经济指标,连同本周疲软的贸易数据以及最近制造业活动萎缩的数据,显示中国国内需求仍然疲弱,中国经济复苏的力度可能正在减弱……}

\entryitemWithDescription{维权人士郭飞雄煽动颠覆国家罪成被判八年监禁}{https://www.zaobao.com/news/china/story20230511-1393584}{中国维权人士郭飞雄5月11日因煽动颠覆国家政权罪,一审被判八年监禁。(推特) (广州综合讯)中国维权人士郭飞雄星期四(5月11日)因煽动颠覆国家政权罪,在广州法院被一审判处有期徒刑八年,剥夺政治权利三年。 ``中国人权律师团''在其推特账号转发郭飞雄(本名杨茂东)哥哥杨茂全的信息透露上述消息,并说郭飞雄表示不服判决,将提起上诉。 郭飞雄的律师向路透社证实上述消息属实……}

\entryitemWithDescription{秦刚参观波茨坦会议旧址:战后秩序必须维护 国家统一必须实现}{https://www.zaobao.com/news/china/story20230511-1393583}{中国国务委员兼外长秦刚星期三(5月10日)在德国参观波茨坦会议旧址,并在签名簿上留言。(中国外交部网站) 本周开启德法之行的中国国务委员兼外长秦刚星期三(5月10日)在德国参观波茨坦会议旧址时说,战后国际秩序必须得到维护,中国国家统一必须要实现……}

\entryitemWithDescription{澳大利亚贸易部长访华 将举行中澳最高级别贸易官员会谈}{https://www.zaobao.com/news/china/story20230511-1393575}{澳大利亚贸易部长法瑞尔(右)5月11日起对华展开三天访问,图为法瑞尔抵华后离开北京首都国际机场。(路透社) 澳大利亚贸易部长法瑞尔星期四起(5月11日)访华,并将展开2019年以来中澳最高级别贸易官员面对面会谈。法瑞尔启程前称,将大力倡导全面恢复澳洲对中国畅通无阻的出口;中国商务部则呼吁两国推动中澳经贸务实合作向前发展……}