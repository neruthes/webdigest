\entryitemWithDescription{于泽远:赖清德上台将引发台海战争?}{https://www.zaobao.com/news/china/story20240115-1462077}{台湾大选结果出炉,民进党候选人赖清德、萧美琴当选台湾下届正副总统。中国大陆有不少人认为,素有``台独金孙''之称的赖清德上台,将使两岸关系继续恶化,甚至引发台海战争。 实际上,去年11月国民党与民众党``蓝白合''破局后,这次台湾大选就没有了多少悬念。虽然岛内不希望民进党继续执政的民众占据多数,但无论国民党还是民众党都很难集合不支持民进党的选票,民进党靠基本盘就能击败另外两党的对手……}

\entryitemWithDescription{民进党流失大量年轻选票受冲击最大 分析:年轻人不满没改善高房价切身问题}{https://www.zaobao.com/news/china/story20240114-1462119}{民进党支持者星期天(1月13日)在台北的竞选总部外,为赖清德当选总统而欢呼。不过,民进党流失大量年轻选票,赖清德接棒的是总统得票、立委席次``双不过半''的弱势政府。(彭博社) 台湾大选结果显示,蓝绿两大党都流失大量年轻选票。学者分析,绿营的民进党若不回应和改善年轻人在乎的切身课题,在两年后的地方选举将持续面对威胁;蓝营的国民党则须考虑修正路线、争取主流民意,才有望争取年轻选民并重返执政……}

\entryitemWithDescription{台大选隔天美国高级代表团抵台 王毅:任何台独行动都将受到严厉惩罚}{https://www.zaobao.com/news/china/story20240114-1462115}{美国前国家安全顾问海德利(右二)和美国在台协会主席罗森博格(左二)星期天(1月14日)晚抵达台北,台湾外交部北美司司长王良玉(左一)在机场迎接。美国在台协会台北办事处处长孙晓雅(Sandra Oudkirk)(右一)也前往迎接。(台湾外交部网站) (台北综合讯)台湾大选结束次日,由前高官组成的美国高级代表团旋即抵台;多个西方国家也就台湾选举表态……}

\entryitemWithDescription{胡锡进反驳大陆舆论对台政策质疑声 称大选谁赢绝非最重要}{https://www.zaobao.com/news/china/story20240114-1462114}{民进党赢得台湾2024总统大选后,中国大陆网络舆论出现质疑大陆对台政策的声音。图为民众1月9日从台北剑潭山眺望台北101大楼。(法新社) (北京综合讯)台湾民进党组合赢得总统大选后,中国大陆网络舆论出现质疑大陆对台政策的声音。大陆《环球时报》前总编辑胡锡批评这种观点是短视和幼稚,他认为选举结果对两岸整体格局影响有限,强调谁胜出绝非最重要的事情,台湾问题应交由专业团队来处理……}

\entryitemWithDescription{民众党关键地位左右立法院龙头之争}{https://www.zaobao.com/news/china/story20240114-1462112}{台湾2024年大选落幕,立法院席次出现朝野三党不过半,2月1日开议的立法院龙头之争备受瞩目。o国民党不分区立委第一名的韩国瑜能否成为立法院长仍有悬念。 在野国民党(蓝)于大选中跃升为第一大党,拥有52席立委,加上有案在身以无党籍参选的国民党籍立委员陈超明,以及亲蓝的无党籍原住民立委高金素梅,共有54席,距离过半的57席只差3席,但国民党不分区立委第一名的韩国瑜能否成为立法院长仍有悬念……}

\entryitemWithDescription{中国多地探索教师退出机制 教师职业不再是``铁饭碗''}{https://www.zaobao.com/news/china/story20240114-1462089}{中国北京丰台、贵州贵阳、浙江宁波等地正探索建立教师退出机制。图为河北邢台市平乡县幼儿园的老师和小朋友1月12日一起剪窗花。(新华社) (北京综合讯)中国北京丰台、贵州贵阳、浙江宁波等地正探索建立教师退出机制,以防止个别教师躺平,让教师职业不再是``铁饭碗''……}

\entryitemWithDescription{分析:大陆或将拿台湾对陆出口前30大产品开刀}{https://www.zaobao.com/news/china/story20240114-1462084}{民进党在台湾总统选举中胜出后,一般预测中国大陆将加大对台湾的经贸施压。图为台湾新北市1月13日人来人往的市场街道。(路透社) (台北综合讯)台湾民进党在总统选举中胜出后,一般预测中国大陆将加大对台湾的经贸施压。分析人士预测,大陆除了将扩大取消海峡两岸经济合作架构协议(ECFA)的优惠关税项目之外,未来也可能把目标延伸到台湾对陆出口的前30大产品,加大打击面……}

\entryitemWithDescription{美学者提醒:台湾选后应保持克制同时加强备战}{https://www.zaobao.com/news/china/story20240114-1462080}{美国政治学者戴雅门(Larry Diamond)认为,台湾在选后应该保持克制、降温、不挑衅,同时加强备战,建设台湾安全与韧性。(缪宗翰摄) 对于台湾选后可能产生的地缘政治风险,美国著名政治学者戴雅门(Larry Diamond)星期天(1月14日)指出,北京必然推动统一进程,台湾应该保持克制、降温、不挑衅,同时加强备战,建设安全与韧性……}

\entryitemWithDescription{港人盼大陆恢复``一签多行'' 吸引深圳居民南下消费}{https://www.zaobao.com/news/china/story20240114-1462066}{香港经济在疫情防控措施解除后表现未如理想,越来越多港人希望恢复``一签多行''政策,吸引深圳居民南下消费,激活香港经济。图为食客2023年12月20日在香港庙街夜市摊位找美食。(法新社) 香港去年解除疫情防控措施后,访港旅客人数一直未能恢复至疫前水平,当地舆论提出恢复深圳居民``一签多行''措施,吸引深圳居民南下消费,激活香港经济……}

\entryitemWithDescription{中国特稿:欧美进军路跌宕 中国电动车加速开往东南亚}{https://www.zaobao.com/news/china/story20240114-1461707}{随着中国电动车国内销量增长放缓,越来越多中国车企把目光投向海外市场。图为一辆比亚迪电动车在北京街道上行驶。(路透社) 随着中国新能源汽车国内销量增长放缓,越来越多汽车企业加速海外市场布局作为新的业务增长点。但欧美针对汽车产业的贸易保护主义近年有抬头的迹象,官方纷纷采取一系列措施遏制中国电动车产业的发展。这对中国新能源汽车出海的步伐有什么影响?中国车企在东南亚的部署会不会加快……}

\entryitemWithDescription{接棒``双不过半''弱势政府 赖清德:台湾政治须走向协商合作}{https://www.zaobao.com/news/china/story20240114-1461980}{当选台湾总统的赖清德与搭档萧美琴胜选后出席在台北民进党总部外举行的记者会。台湾星期六(1月13日)举行的总统选举全球瞩目,吸引了逾400名国际记者赴台采访。(法新社) 当选台湾总统的赖清德说,民进党在立法院席次未过半,说明人民期待``有效率的制衡'',台湾政治须走向协商与合作。 赖清德星期六(1月13日)在胜选后的国际记者会上致辞说,民进党无法在立法院席次过半,代表努力不够,必须虚心检讨……}

\entryitemWithDescription{拼政党轮替失败 侯友宜眼眶泛红为败选负责 学者:国民党中央须改变官僚老旧习气}{https://www.zaobao.com/news/china/story20240114-1461975}{国民党正副总统候选人侯友宜(左二)、赵少康(右二),以及党主席朱立伦(左一)、侯友宜竞选办公室执行长金溥聪(右一)星期六(1月13日)晚在竞选总部,为没能实现政党轮替,向支持者鞠躬致歉。(法新社) 台湾最大在野国民党力拼政党轮替再次失败。国民党总统候选人侯友宜星期六(1月13日)败选后,三度向支持者鞠躬致歉,眼眶泛红表示,``我尊重台湾人民作出的最后选择''……}

\entryitemWithDescription{韩国瑜出任立法院长?民众党意愿成关键}{https://www.zaobao.com/news/china/story20240114-1461973}{主张废除死刑的社民党台北市议员苗博雅,挟带青年族群的高声量,在台北市大安区立委选举拿下逼近45\%的得票率,虽竞选失利,卻是历来泛绿阵营在该区最佳表现。(档案照片) 韩国瑜可能出任立法院长 台湾立委选举落幕,国会朝小野大局面底定。外界推测,未来立法院长可能由国民党不分区立委排名第一的前高雄市长韩国瑜出任,但仍要看民众党所扮演的关键少数,是否同意……}

\entryitemWithDescription{赖清德当选总统 分析:两岸关系将受冲击}{https://www.zaobao.com/news/china/story20240114-1461971}{台湾民进党主席赖清德(左)在星期六举行的大选中,以逾558万得票数胜出。图为赖清德和副总统当选人萧美琴(右)在胜选后召开国际记者会。(法新社) 自称务实台独工作者的台湾民进党主席赖清德在星期六(1月13日)举行的大选中,以逾558万得票数胜出,在三角战中以40.05%得票率当选总统。民进党由此首次得以连续执政超过八年,两岸关系和台海稳定前景则料受冲击……}

\entryitemWithDescription{民众党选后抓``战犯'' 柯文哲:我承担最大责任}{https://www.zaobao.com/news/china/story20240114-1461970}{台湾民众党总统候选人柯文哲、副总统候选人吴欣盈1月13日晚间出席记者会。(路透社) 台湾立法委员选举结果三党不过半,民众党拿下八席立委席次,被视为未来国会关键少数。不过,民众党内部在选后就开始找败选``战犯'',党主席柯文哲星期六(1月13日)晚间坦言,他要承担最大责任,但为避免党内分裂,暂时不会请辞党主席……}

\entryitemWithDescription{分析:新总统520就职前北京将加大对台军事与经济施压}{https://www.zaobao.com/news/china/story20240114-1461969}{台湾民进党候选人连续三届当选总统。受访学者分析认为,5月20日总统就职演说前对台的文攻武吓大概不可避免。 (路透社) 台湾民进党候选人连续三届当选总统,受访学者分析,台湾总统选举结果反映对两岸关系持有对抗情绪的台湾选民仍占大多数,赖清德的胜选感言预计也让北京``非常不满意'',接下来几个月北京对台文攻无吓估计不可免。 台湾自2000年以来,没有政党能连续执政超过八年……}

\entryitemWithDescription{投票日当天 微博屏蔽台湾选举有关话题}{https://www.zaobao.com/news/china/story20240113-1461930}{台湾总统和立法委员选举当天(1月13日)早上,``台湾选举''词条登上中国大陆最大社媒平台之一微博的热搜榜,但这一热门词条随后被屏蔽。(法新社) (北京/台北综合讯)台湾总统和立法委员选举星期六(1月13日)登场,中国大陆网民对大选结果高度关注。当天上午8时投票开始后,有关词条``台湾选举''便登上中国大陆最大社媒平台之一微博的热搜榜,但这一热门词条随后在微博被屏蔽……}

\entryitemWithDescription{台湾选举日再侦获中国大陆空飘气球}{https://www.zaobao.com/news/china/story20240113-1461929}{台湾在总统和立委选举日再侦获中国大陆空飘气球。(法新社) (台北综合讯)台湾在总统和立委选举日再侦获中国大陆空飘气球。 台湾国防部星期六(1月13日)即台湾总统和立委选举投票日在网站公布最新解放军台海周边海、空域动态,显示星期五(1月12日)早6时至星期六早6时,共侦获大陆军机八架次、军舰六艘次,其中一架运8反潜机进入西南空域……}

\entryitemWithDescription{【热点评论】台湾总统选后 台海风云叵测}{https://www.zaobao.com/news/china/story20240113-1461872}{注:若无法通过此页面观看直播,请点击这里。 堪称历来最诡谲多变、纠缠不清的台湾总统选举三角战形势,星期六(1月13日)终有定案。民进党候选人赖清德萧美琴组合自下午四时计票开始之后持续领先,执政党似乎抵住了面对高达六成民意要求``换政府''的压力成功保住政权,这是否预示绿营在台湾长期执政时代到来……}

\entryitemWithDescription{新闻人间:以历史视角看《繁花》}{https://www.zaobao.com/news/china/story20240113-1461456}{新闻人间------繁花 改编自同名小说的电视剧《繁花》本周收官,这部爆红剧集引发多层次讨论,除拍摄手法、明星演技、沪语传播外,从历史视角也有可谈之处。 电视剧展现的上世纪90年代,是中国历史转折期,上海扮演了重要角色。1989年以后,中国改革陷入僵局,全国展开姓资姓社大论战。1991年,上海《解放日报》以笔名``皇甫平''发表文章,捍卫改革开放;1992年,邓小平南巡,重要一站正是上海……}

\entryitemWithDescription{温伟中:台湾大选的热与冷}{https://www.zaobao.com/news/china/story20240113-1461744}{到台湾南部的高雄看选举造势,蓝绿白三党会场氛围比我预期的更热。 选前超级星期天(1月7日),蓝营的国民党、绿营的民进党、白营的民众党都到绿营铁票仓高雄拼场。绿营要凝聚基本盘,蓝营想突破同温层,白营也不想被比下去。 我骑共享脚踏车Youbike跑三摊、避车龙。当天蓝绿都宣称来了12万人、白营也喊出8万人相挺。国民党在梦时代广场的场子人潮满溢,许多人手挥青天白日满地红旗……}

\entryitemWithDescription{台湾大选选前之夜 蓝绿白造势大拼场}{https://www.zaobao.com/news/china/story20240113-1461798}{成立四年半的台湾民众党声势惊人,支持者星期五(1月12日)挤满总统府前的凯达格兰大道,现场宣称来了35万人。(台湾民众党提供) 台湾星期六(1月13日)举行总统与立法委员选举,选前之夜蓝绿白造势活动的人潮都爆满,民进党能否延续政权、在野党能否变天、立法院是否三党不过半,将在星期六傍晚以后陆续揭晓。 这场选举牵动两岸和中美关系,也将影响台海局势与区域稳定,备受世界瞩目……}