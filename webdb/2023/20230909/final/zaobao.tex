\entryitemWithDescription{新闻人间:华为新手机的启示}{https://www.zaobao.com/news/china/story20230909-1431671}{0909 新闻人间 台积电董事长刘德音星期三(9月6日)出席``SEMICON Taiwan 2023国际半导体展''大师论坛(CEO Summit)发表演说,会后被媒体追问在美国设厂困境等问题,不厌其烦详加说明。但一被询及中国大陆电信巨擘华为新手机用中芯七纳米晶片(也称芯片),显示在美国制裁下,大陆仍突围而出,``这晶片有看到台积电的影子吗?''刘德音立马斩钉截铁地说:``没有……}

\entryitemWithDescription{香港遭遇百年一遇大暴雨 港府被质疑反应过慢}{https://www.zaobao.com/news/china/story20230908-1431675}{持续暴雨导致香港多地严重积水,导致车辆熄火被困,抛锚在水中。(彭博社) 受台风``海葵''残留低压槽影响,香港遭遇百年一遇的大暴雨,多区出现严重淹水,逾130人受伤。港府被质疑反应过慢,没有及早预警和准备。 据彭博行业研究估计,这场罕见的暴雨可能给香港造成超过1亿美元(约1.36亿新元)的损失……}

\entryitemWithDescription{李强与佐科会谈 印尼邀请中国企业建设新首都}{https://www.zaobao.com/news/china/story20230908-1431660}{中国总理李强(左)星期五(9月8日)在雅加达与印度尼西亚总统佐科会谈。(新华社) (雅加达综合讯)中国总理李强在与印度尼西亚总统佐科会面时,承诺扩大印尼大宗商品和农渔产品进口,佐科则邀请中国企业参与新首都努山达拉建设。 正在印尼访问的李强星期五(9月8日)在雅加达与佐科会面。据新华社报道,李强在会面中呼吁两国把握自身根本和长远利益,在涉及彼此核心利益和重大关切问题上加大相互支持……}

\entryitemWithDescription{深港两地遭受有记录以来最大暴雨袭击}{https://www.zaobao.com/news/china/story20230908-1431656}{(彭博社) 受台风``海葵''外围残留云系与季风影响,深港两地遭受有记录以来最大暴雨袭击。 香港在星期四(9月7日)晚上录得每小时158.1 毫米的降雨量,是当地自1884年有记录以来的最高降雨量。雨势导致香港部分街道被淹,有茶餐厅店员竖起防洪挡水板,避免雨水倒灌进店内。 截至星期五下午6时15分,暴雨已造成超过130人受伤送医,四人伤势严重……}

\entryitemWithDescription{美国正式对华为手机芯片展开调查}{https://www.zaobao.com/news/china/story20230908-1431651}{受美国关注的华为Mate 60 Pro系列手机星期五在中国预售,引发抢购热潮。(路透社) 中美手机之争越演越烈,美国政府已正式对华为新型手机内的先进中国制芯片展开调查;中国外交部则批评美国``滥用国家力量,无理打压中国企业''。 受访学者分析,美国料难以进一步收紧对华限制措施,因此力图通过调查的方式,对有关华为芯片的舆论作出回应……}

\entryitemWithDescription{广东多地遭暴雨袭击 深圳降雨量破多项极值}{https://www.zaobao.com/news/china/story20230908-1431649}{深圳星期四(9月7日)晚上遭暴雨袭击,高降雨量导致多个地区出现内涝,马路和人行道完全被雨水淹没,有车辆浸泡在路中央。(林煇智摄) 受暴雨影响,深圳火车站的负一楼进出站口被水淹,乘客无法正常进出站,一度有约百人被滞留在站内。(中新社) 中国广东珠江三角洲遭大暴雨袭击,泛滥洪水导致深圳、广州、东莞等城市出现严重内涝,部分地区被迫停课停工……}

\entryitemWithDescription{数百港人寻求政治庇护 获英国批核率不足一成}{https://www.zaobao.com/news/china/story20230908-1431641}{(伦敦/香港综合讯)英国内政部统计,自2020年近300宗香港人赴英寻求庇护个案,当局仅批出28宗,批核率不足一成。 英国广播公司(BBC)报道,反对《逃犯条例》修订风波期间,不少年轻港人离港赴英,部分人因无法通过英国国民(海外)护照(BNO)签证计划申请居留,需向英国政府申请政治庇护。 报道引述英国内政部数据,自2020年赴英寻求庇护的港人个案共有299宗,当中未成年人达44人……}

\entryitemWithDescription{中缅合力打击 1207名缅北涉诈嫌疑人移交中国}{https://www.zaobao.com/news/china/story20230908-1431631}{(北京综合讯)继本月初抓获269名缅北涉诈犯罪嫌疑人后,中缅警方继续合力打击电信网络诈骗犯罪,缅甸星期三(9月6日)再将1207名涉案嫌疑人移交中国。 据新华社消息,中国公安部部署云南等地警方持续推进缅北涉华电信网络诈骗打击行动。云南普洱警方与缅甸相关地方执法部门开展边境警务合作,最近移交中国1207名缅北涉诈犯罪嫌疑人,其中网上在逃人员41名。官方报道中没有说明这些嫌疑人的国籍……}

\entryitemWithDescription{黄小芳:昙花一现的大爷跳水}{https://www.zaobao.com/news/china/story20230908-1431339}{两年多前到北京常驻后不久,我在中国的第一个旅游目的地就是毗邻的天津。除了被导游带到特产店买麻花和鲍鱼干,我对天津没有留下太深刻的印象。没想到这个看似平平无奇的城市,最近突然成为中国最火的旅游打卡地。 今年8月底,大批天津大爷在狮子林桥跳水的视频在网络蹿红,但这股风潮来得快去得也快。不到一个月的时间,跳水大爷在星期三(9月6日)以天津狮子林桥跳水队的名义宣布停止这项活动……}

\entryitemWithDescription{【早知】华为新手机面世是否意味中国突破了``卡脖子''?}{https://www.zaobao.com/news/china/story20230907-1431374}{一名技术人员9月3日在渥太华的实验室里,分解华为Mate 60 Pro智能手机。(彭博社) 华为上周低调发售最新旗舰手机Mate 60 Pro,随即在中国内外掀起围绕这款手机芯片设计、技术背景的探察旋风。华为新手机横空出世,是否意味着中国已在芯片领域突破``卡脖子''? 华为最新手机的科技成色如何……}

\entryitemWithDescription{腾讯推出``混元大模型'' 正式加入中国人工智能竞赛}{https://www.zaobao.com/news/china/story20230907-1431353}{腾讯副总裁蒋杰星期四(9月7日)在年度全球数字生态大会上说,混元大模型具备中文创作和复杂语境下的逻辑推理能力,目前已接入公司旗下的50多个产品和服务。(林煇智摄) 中国互联网巨头腾讯发布人工智能(AI)大模型``混元大模型'',并推出类似ChatGPT的聊天机器人,正式加入中国科技企业在AI领域的竞赛……}

\entryitemWithDescription{美国议员呼吁加码制裁中国科企 学者:即便不升级也会延续现有制裁}{https://www.zaobao.com/news/china/story20230907-1431344}{搭载中国国产芯片的华为新手机面世后,美国国内关于加码制裁中国科技企业的呼声水涨船高。学者分析,这类呼声更多是从政治角度而非技术层面提出,但在当前政治氛围下,华盛顿即便不升级,也会延续现有制裁。 美国众议院中国问题特别委员会主席加拉格尔(Mike Gallagher)星期三(9月6日)发表声明,呼吁美国商务部停止向华为及其芯片供应商中芯国际出口所有技术……}

\entryitemWithDescription{美国延长部分中国产品加征关税豁免至今年底}{https://www.zaobao.com/news/china/story20230907-1431333}{(华盛顿/北京综合讯)美国决定对部分中国产品的豁免加征关税期限,延长至今年年底。获得延长的是352个中国进口产品项目,以及77个与应对冠病有关的中国产品项目。 综合路透社、法新社报道,美国贸易代表办公室星期三(9月6日)宣布,将与上述项目有关的``301条款''关税排除期限,从原定的9月30日延长至12月31日。 美国通货膨胀当下持续走高,拜登政府正面临来自企业和国会议员要求减轻关税负担压力……}

\entryitemWithDescription{中国对榴梿的需求使全球销售增400%}{https://www.zaobao.com/news/china/story20230907-1431328}{中国消费者对榴梿的需求增大,图为广西南宁市民今年5月在摊贩选购榴莲。(中新社) (北京/香港综合讯)中国消费者对榴梿的需求增大,促使全球榴梿销量增长400\%。 《南华早报》报道,汇丰银行星期一(9月4日)发布的报告显示,中国对榴梿的需求占全球市场的91\%,中国过去两年共进口价值60亿美元(约81亿新元)的榴梿。在中国市场的带动下,今年第一季度的全球榴梿销量与去年同期相比,大增400\%……}

\entryitemWithDescription{中澳总理会晤促关系升温 阿尔巴尼斯答应年内访华}{https://www.zaobao.com/news/china/story20230907-1431306}{中国总理李强(右)星期四(9月7日)在印度尼西亚雅加达出席亚细安系列峰会期间,与澳大利亚总理阿尔巴尼斯(左)举行会晤。(新华社) 中国总理李强与澳大利亚总理阿尔巴尼斯星期四(9月7日)在印度尼西亚出席亚细安系列峰会期间举行会晤。李强呼吁双方抓紧重启和恢复各领域交流,推动关系进一步改善;阿尔巴尼斯则表示,将于今年稍晚时访问中国……}

\entryitemWithDescription{陈杰豪:重庆和中国西部仍有结构性增长动力 能与新加坡互补}{https://www.zaobao.com/news/china/story20230907-1431030}{国家发展部兼通讯及新闻部高级政务部长陈杰豪,星期一(9月4日)下午出席在重庆举行的中国国际智能产业博览会开幕式,会前接受《联合早报》采访。(王纬温摄) 我国国家发展部兼通讯及新闻部高级政务部长陈杰豪认为,重庆和中国西部地区仍有一些结构性的经济增长动力,包括有年轻及愿意提升技能的人口,以及比沿海城市更巨大的增长潜力等……}

\entryitemWithDescription{担心美国打压? 中国官媒低调报道华为新机}{https://www.zaobao.com/news/china/story20230906-1431011}{专业人员从华为Mate 60 Pro手机中取出的芯片,为中芯国际在中国制造的七纳米制程芯片。(彭博社) 中国通讯巨头华为的新手机Mate 60 Pro销售火爆之际,中国官媒转而低调承认新机的``中国芯''与最先进技术还有差距。分析指出,随着更多专业拆机报告发表,官媒将引导民间情绪逐渐回归理性,避免引发美国加码打压。 华为上周二(8月29日)在毫无宣传的情况下推出Mate 60 Pro……}

\entryitemWithDescription{台积电美国厂量产时间延后 刘德音坦承第一次海外盖厂必经学习过程}{https://www.zaobao.com/news/china/story20230906-1431009}{台积电董事长刘德音星期三在``SEMICON Taiwan 2023国际半导体展''的大师论坛演讲。(庄慧良摄) 台湾积体电路制造公司(台积电)三年前在美国亚利桑那州设厂,但量产时间从2024年初延后至2025年。台积电董事长刘德音星期三坦承,台积电第一次在海外盖大规模的工厂,开始会有一个学习的过程……}

\entryitemWithDescription{李强:中国今年有望实现5\%左右经济增长}{https://www.zaobao.com/news/china/story20230906-1430994}{(雅加达综合讯)外界质疑中国经济复苏乏力之际,中国国务院总理李强在国际多边会议上重申,中国今年有望实现5\%左右的经济增长目标。 综合新华社和路透社报道,李强星期三(9月6日)在印度尼西亚雅加达出席亚细安与中日韩(亚细安加三)领导人会议时,在开幕词中做出上述表述。 李强说,今年中国经济有望实现年初设定的5\%左右的增长目标,经济发展前景可期,将为地区和世界各国不断提供新的更大机遇……}

\entryitemWithDescription{中国拟修法禁止``伤害民族感情''言论和服饰}{https://www.zaobao.com/news/china/story20230906-1430982}{中国拟修订《治安管理处罚法》,把``有损中华民族精神、伤害中华民族感情''的言行列入处罚范围。(彭博社) (北京综合讯)中国拟修订《治安管理处罚法》,把``有损中华民族精神、伤害中华民族感情''的言行列入处罚范围,但因缺乏细化阐释,在互联网上引起争议。 中国人大网显示,《治安管理处罚法(修订草案)》(简称《草案》)经过全国人大常委会会议审议后,从9月1日起向公众征求意见,为期30天……}