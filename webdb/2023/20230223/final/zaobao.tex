\entryitemWithDescription{44年来台外长第一次进入首都圈 台美国安高层会谈首次在华盛顿举行}{https://www.zaobao.com/news/china/story20230223-1365919}{台湾国际战略学者分析,不对外公开的会谈内容,相信包括如何加强台湾防卫力量、商讨美国众议长访台和台湾总统访美,借此向北京传达别因俄乌战争错判台湾局势的信号。 在俄乌战争一周年和台海局势持续紧张之际,台美国安高层会谈首次在华盛顿城郊举行七小时闭门会议。这是1979年台美断交后,44年来台湾外交部长第一次进入美国首都圈……}

\entryitemWithDescription{台军强化与美军事交流}{https://www.zaobao.com/news/china/story20230223-1365920}{据台湾《自由时报》及中央社引述知情人士报道,台军近年持续强化与美国军事交流事项,从过去台军以排级、连级单位赴美受训,已改制为驻防南台湾的陆军333旅、驻守北台湾的542旅,预计今年下半年将派营级单位赴美交流。这两个联兵旅原本都是装甲步兵旅、机械化步兵旅单位,但已于2019年起重新编组为联合兵种作战旅,让每个所属营级单位皆有独立作战能力,也能透过联络官联系海、空、陆军航特部直升机部队支援作战……}

\entryitemWithDescription{澳洲智库: 中国贸易胁迫19国家和地区}{https://www.zaobao.com/news/china/story20230223-1365922}{澳大利亚一份智库报告说,中国2020年至2022年间对19个国家和地区进行了威胁和贸易限制,而澳洲是这种``胁迫外交''的首要目标。 据路透社报道,由澳洲战略政策研究所发表的报告指出,中国的有关手法旨在改变目标政府的政策,但结果是成败参半。对澳洲而言,政策没有见效,反而凸显国家韧性……}

\entryitemWithDescription{民进党中常会通过 3月初选登记4月公告总统候选人}{https://www.zaobao.com/news/china/story20230223-1365924}{台湾执政的民进党星期三在中常会通过2024年总统及立法委员提名时程,3月中旬进行总统初选登记,4月12日公告总统候选人名单。 总统初选办法未确定 国民党变数较多 在党内重要人士相继表态支持后,民进党总统候选人以党主席赖清德为主;在野国民党目前民调支持度较高的是新北市长侯友宜,但该党连总统初选办法都尚未确定,变数较多……}

\entryitemWithDescription{大陆台办称愿在既有政治基础上 同国民党加强高层往来}{https://www.zaobao.com/news/china/story20230223-1365925}{中国大陆国务院台湾事务办公室发言人朱凤莲说,大陆愿意在既有共同政治基础上,同台湾在野的国民党加强高层往来,并呼吁民进党政府尽快恢复海峡两岸空中客运直航正常化……}

\entryitemWithDescription{早 说}{https://www.zaobao.com/news/china/story20230223-1365926}{当你认为不可能,也许就有危险了。 ------台湾国防部长邱国正接受《天下杂志》采访谈到中国大陆可能攻台的时间时说,他不能明说哪一天会开战,但是无论是下个钟头、明天还是下个星期,台湾军方都得准备。现在部队主官逢年过节,都不能离营区太远。因为军中传统是,当你认为不可能的时候,也许就会发生危险了……}

\entryitemWithDescription{难改的医保}{https://www.zaobao.com/news/china/story20230223-1365927}{``什么时候轮到咱们啊?看个病也太折腾了\ldots\ldots'' 在熙熙攘攘的候诊大厅里,我身旁一位头发花白的老阿姨,操着浓重的东北口音,有气无力地问陪同她来看病的年轻人。 前两天在上海一家公立医院挂了专科门诊号,按照预约时间到达,墙上叫号屏幕却显示,还有20个人排在我前面……}

\entryitemWithDescription{新港府首份预算案推出多项惠民措施 学者:应考虑推出更多开源措施}{https://www.zaobao.com/news/china/story20230223-1365928}{香港财政司司长陈茂波预料香港经济今年会明显反弹,全年实质增长介于3.5\%至5.5\%。图为香港中环街景。(法新社) 港府财政司司长陈茂波宣读预算案时指出,估计本财政年度会录得约238.92亿新元赤字,高于原来的预算,不过,陈茂波强调目前港府的财政储备水平稳健,并在预算案中宣布推出多项减轻市民经济压力的措施……}

\entryitemWithDescription{广州文化新馆 水乡园林迎宾}{https://www.zaobao.com/news/china/story20230223-1365929}{广州文化馆新馆以``十里红云一湾水,八桥画舫十六亭''为设计主题,用传统建筑和园林空间的组合再现岭南水乡园林的佳境,包含公共文化中心、翰墨园、曲艺园、广府园、广绣园等多组主题园林建筑,以及由它们共同形成的大型园林景观。图为2月22日航拍的广州文化馆新馆……}

\entryitemWithDescription{中国促国企逐步停用国际四大会计师事务所}{https://www.zaobao.com/news/china/story20230223-1365930}{知情人士透露,中国政府敦促国企逐步停用国际四大会计师事务所做审计。 据彭博社报道,熟悉事件的人士说,中国财政部等部门上个月向一些国有企业提供窗口指导,敦促它们结束和四大审计公司的合同。离岸子公司仍可使用美国审计师,但他们的母公司在签订合同时,被要求聘用中国大陆或香港会计师。政府尚未设定变更审计公司的日期,更换工作可能会随着合同到期逐步进行……}

\entryitemWithDescription{港预留1亿港元争取举办大型盛事}{https://www.zaobao.com/news/china/story20230223-1365931}{香港特区政府宣布,将预留1亿港元(1700万新元)争取举行更多大型活动和盛事,提升香港国际形象。 综合香港电台网站、中通社报道,香港财政司司长陈茂波星期三(2月22日)发表2023年财政预算案时作出上述宣布。他说,大型盛事、国际会展等活动对吸引高增值旅客尤其重要,可进一步提升香港盛事之都的竞争力……}

\entryitemWithDescription{中国旅游业预计今年大幅回弹}{https://www.zaobao.com/news/china/story20230223-1365932}{中国旅游业预计今年将大幅反弹。 据中国经济网报道,中国旅游研究院公布的数据显示,预计2023年中国国内旅游人数约45.5亿人次,同比增长约80\%,约恢复至2019年疫情前的76\%,实现国内旅游收入约4万亿元(人民币,下同,约8000亿新元),同比增长约95\%,约恢复至2019年的71\%。重庆、北京、成都、上海、广州等城市为热门旅游目的地……}

\entryitemWithDescription{因``基因编辑婴儿''被判刑 曾坐牢大陆科学家申请港高才通获批惹议}{https://www.zaobao.com/news/china/story20230222-1365549}{贺建奎在北京接受香港多家媒体采访时,确认自己申请``高才通''已经获批。(法新社) 参与统筹``高才通''计划的港府劳工及福利局局长孙玉菡,周二上午回应这起事件时说,不评论个案,但承认目前``高才通''申请人无须填写刑事犯罪记录,当局会不时调整及检视计划的申请流程,包括申请者填写的个人资料……}

\entryitemWithDescription{台湾总统大选民调:侯友宜支持度32.4\%领先赖清德27.7%}{https://www.zaobao.com/news/china/story20230222-1365550}{台湾民意基金会星期二公布2024年总统大选民调,结果显示,在野国民党籍的新北市长侯友宜获得32.4\%的支持度,领先执政的民进党主席赖清德的27.7%,以及在野民众党主席柯文哲的19.5\%;若柯文哲不参选,侯友宜更以47.4\%胜过赖清德的32.7\%。 台湾民意基金会董事长游盈隆指出,目前国民党和民众党加起来的社会基础已超过民进党,赖清德并非稳赢,民进党2024年将陷入艰困的政权保卫战……}

\entryitemWithDescription{大熊猫``香香''从东京送返中国}{https://www.zaobao.com/news/china/story20230222-1365552}{图为``香香''在成都双流国际机场海关顺利通关后,准备搭乘货车前往中国大熊猫保护研究中心雅安碧峰峡基地。(新华社) 图为日本民众2月19日前往上野动物园欢送``香香''。(法新社) 2017年6月日本东京上野动物园自然繁殖成功的第一只雌性大熊猫``香香'',深受日本民众喜爱。由于父母是从中国出借的大熊猫``比力''与``仙女'',``香香''所有权属于中国……}

\entryitemWithDescription{港媒:美国问题专家袁鹏 改名袁亦鲲出任国安部副部长}{https://www.zaobao.com/news/china/story20230222-1365553}{中国现代国际关系研究院院长袁鹏已出任国家安全部副部长,并改名袁亦鲲。(互联网) 香港《明报》星期二(2月21日)一篇文章指出,中国的美国问题专家、中国现代国际关系研究院(CICIR)院长袁鹏已出任国家安全部副部长,并改名袁亦鲲……}

\entryitemWithDescription{英外长:与秦刚通电话时 提到新疆人权及台海和平问题}{https://www.zaobao.com/news/china/story20230222-1365554}{英国外交部长克莱弗利说,他星期一(2月20日)在与中国外交部长秦刚通电话时,提到了新疆地区的人权议题。 克莱弗利(James Cleverly)周二在推特发文说,他和秦刚谈话时提到新疆人权问题,以及维持台海和平的需要。中英外长还同意共同努力解决气候和贸易问题。 中国外交部的新闻稿则没有提及新疆以及台海问题……}

\entryitemWithDescription{早说}{https://www.zaobao.com/news/china/story20230222-1365555}{美方是乌克兰战场最大的武器提供者,昨天还宣布再向乌克兰提供5亿美元军事援助,美方现在却不断散布中方提供武器的虚假信息,居心何在? ------美国国务卿布林肯近日表示,中国``知道''如果向俄罗斯提供致命性支援将面临的后果。中国外交部发言人汪文斌星期二(2月21日)在例行记者会作出以上回应……}

\entryitemWithDescription{中国发布《全球安全倡议概念文件》 秦刚吁各国对俄乌战争 停止拱火浇油推责鼓噪}{https://www.zaobao.com/news/china/story20230222-1365557}{秦刚2月21日在蓝厅论坛开幕式发表主旨演讲后离开现场。(路透社) 中国外长秦刚指中国倡导大国带头讲平等、讲诚信、讲合作、讲法治,坚决维护``核战争打不赢也打不得''共识,通过对话协商政治解决乌克兰危机等热点问题。 中国外交部长秦刚星期二(2月21日)敦促有关国家在俄乌战争问题上立即停止拱火浇油,停止向中国甩锅推责,停止鼓噪``今日乌克兰,明日台湾''……}