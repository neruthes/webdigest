\entryitemWithDescription{中国出口逆袭仅昙花一现?}{https://www.zaobao.com/news/china/story20230430-1388959}{4月中旬开幕的广交会,展览面积和参展企业数量都创下新高。但参展外贸企业反映,订单下降、需求不足仍是出口面临的主要困难。(新华社) 中国出口在3月大涨14.8%,打破市场悲观预期,也拉动中国第一季经济超预期增长。但多组出口数据``打架'',也引发外界争议和质疑。中国出口的逆袭是触底反弹,还是昙花一现?外贸承压之际,外资能否对经济带来更大推动作用?外贸企业和外资企业,又将如何应对疫后新局面……}

\entryitemWithDescription{美媒:在华外企频传遭突击搜查 五天近32亿美元外资撤离}{https://www.zaobao.com/news/china/story20230429-1388863}{美国媒体报道,中国政府近期频传对在华外企发动突击调查行动,导致大量外资撤离。 据美国《华尔街日报》 星期五(4月28日)报道,在过去五个交易日内,有价值31.7亿美元(约42.34亿新元)的外资从沪深港通跨境交易机制撤离,创下2022年11月以来,持续时间最长的撤资潮。 《华尔街日报》是根据财经研究顾问公司Exante Data的分析作出上述报道……}

\entryitemWithDescription{中国五一长假开启 预计超2.4亿人出游}{https://www.zaobao.com/news/china/story20230429-1388856}{星期六(4月29日)是中国五天劳动节假期的第一天,北京的一条小巷里,游客摩肩接踵。(法新社) 中国``五一''劳动节长假开启,出行和旅游需求强劲,第一天的海陆空客运量就同比激增151.8\%,预计五天假期将有2.4亿人次出游。 据中国中央电视台报道,星期六(4月29日)作为假期的第一天,航空、公路、水路和铁路的出行量增加到5699万人次,比去年同期激增151.8\%……}

\entryitemWithDescription{美国前国安顾问博尔顿访台 谈双重承认台湾和大陆}{https://www.zaobao.com/news/china/story20230429-1388854}{美国前国安顾问博尔顿(中)星期六演讲时,公开呼吁美国恢复与台湾的全面外交关系并双重承认两岸,赢得观众喝彩。左为演讲后的座谈会主持人、台湾人公共事务会总会长简明子;右为民进党立委罗致政。(温伟中摄) 对华鹰派、考虑代表共和党竞逐总统的美国前国家安全顾问博尔顿访台,呼吁美国双重承认台湾和大陆,但在媒体追问若当总统是否实践时,改口``还要再想想''……}

\entryitemWithDescription{中国移民从抖音获取步行穿越美墨边境攻略}{https://www.zaobao.com/news/china/story20230429-1388851}{一名来自中国的移民男子4月4日在美国得克萨斯州弗朗顿,向边境巡逻人员表明自己的身份。这群人从墨西哥偷渡到美国得克萨斯州弗朗顿。 (路透社) 近几个月来通过陆路经墨西哥进入美国的中国非法移民人数剧增,一些人更是从社交媒体抖音上获取路线指引,制定徒步赴美的计划……}

\entryitemWithDescription{新闻人间:``家电枭雄''顾雏军国赔案尘埃落定}{https://www.zaobao.com/news/china/story20230429-1388661}{20多年前叱咤中国家电行业的资本大鳄、科龙电器原董事长顾雏军申请国家赔偿一案,星期三(4月26日)有了最新进展。中国最高人民法院宣布,维持向顾雏军支付国家赔偿金43万元(人民币,下同,约8万3000新元)的决定。 这个赔偿金额是去年1月由广东省高级人民法院裁定,包括人身自由赔偿金约28.7万余元,精神损害抚慰金14.3万元,以及返还罚金8万元和利息。 这一决定去年一公布,就遭遇舆论的冰火两重天……}

\entryitemWithDescription{庄慧良:``汪辜会谈''30年有感}{https://www.zaobao.com/news/china/story20230429-1388662}{了解两岸关系发展历史,便知两岸定位问题是绝对无法回避,相对模糊的``九二共识''是现有唯一解方……}

\entryitemWithDescription{台妇联会主委雷倩:旧美元秩序已逼近``砍掉重练''临界点}{https://www.zaobao.com/news/china/story20230429-1388522}{台湾妇女联合会主任委员雷倩星期五(4月28日)指出,旧美元秩序已逼近``砍掉重练''的临界点,美元霸权的弱化不是他国对抗所致,是随着美国在全球经贸角色不再绝对重要,同时新货币体系伴随实体经济发展复式升起的趋势,值得关注。 雷倩是应中美经济合作策进会邀请,发表``从银行危机看金融与经济的体系脱勾''演讲时,作上述表示……}

\entryitemWithDescription{中国大陆无人机长距离绕台 学者:演练侦察台东部战力保存区}{https://www.zaobao.com/news/china/story20230428-1388514}{台湾国防部星期五称,监测到中国大陆解放军一架TB-001无人机绕台飞行。图为日本航空自卫队2021年8月拍摄的解放军TB-001无人机。(日本航空自卫队) (台北/北京综合讯)中国大陆解放军罕见派出察打一体大型无人机环绕台湾长距离飞行,展示军方作战能力。台湾学者分析,解放军可能借这类绕台飞行,演练在台海爆发战争时对台湾东部战力保存区进行侦查……}

\entryitemWithDescription{中国海军苏丹撤侨军舰 抵达沙特阿拉伯吉达港}{https://www.zaobao.com/news/china/story20230428-1388511}{(新华社) 中国海军导弹驱逐舰南宁舰和综合补给舰微山湖舰,载着678名从苏丹撤离人员,穿越红海抵达沙特阿拉伯西部吉达港,其中668人为中国公民,10人为外国籍。图为4月27日从苏丹撤离人员乘坐中国海军军舰抵达沙特阿拉伯吉达港……}

\entryitemWithDescription{中国对德国据报或限制芯片化学制品出口表达关切}{https://www.zaobao.com/news/china/story20230428-1388510}{有消息称德国政府讨论限制向中国出口芯片化学制品,中国商务部长王文涛与德国副总理哈贝克举行会谈时,对德方有关出口限制措施表达关切。 彭博社星期四(4月27日)引述知情人士消息称,随着柏林加紧减少对中国的经济敞口,德国正就限制向中国出口用于生产半导体的化学品进行商讨……}

\entryitemWithDescription{中企在美太阳能市场仍占据主导地位}{https://www.zaobao.com/news/china/story20230428-1388505}{尽管美国政府采取了多项措施促进国内太阳能设备生产,限制进口中国公司的太阳能组件,但最新数据显示,中国公司在美国太阳能市场仍占据主导地位。 《华尔街日报》星期四(4月27日)引用光伏行业媒体PV Tech的数据报道,随着整体太阳能板销量激增,中资太阳能板制造商今年会在美国销售更多产品,市场份额将从去年的42\%增加到45\%,接近17吉瓦……}

\entryitemWithDescription{中国防长李尚福访印 两国边境问题分歧依旧}{https://www.zaobao.com/news/china/story20230428-1388502}{在新德里出席上合组织成员国国防部长会议的中国防长李尚福(右一)星期四(4月27日)同印度防长辛格(左二)举行会谈。(路透社) 中国国防部长李尚福星期四(4月27日)在印度会见印度国防部长辛格(Rajnath Singh),双方在边境问题上分歧依旧……}

\entryitemWithDescription{中国就美韩联合声明``涉华错误表达''约见韩国使节严肃交涉}{https://www.zaobao.com/news/china/story20230428-1388493}{中国外交部亚洲司司长刘劲松(右)星期四约见韩国驻华使馆公使姜相旭(左),就美韩联合声明中的``涉华错误表述''提出严肃交涉。(中国外交部网站) 受访学者分析, 尹锡悦访美释放出美日韩加强三方协作的明确信号,中国在台海将面对更大的战略压力,并且会有所反应,台海紧张局势预料将加剧……}

\entryitemWithDescription{郭台铭称鸿海不会成为他在两岸议题的软肋}{https://www.zaobao.com/news/china/story20230428-1388464}{鸿海集团创办人郭台铭星期四(4月27日)认为,如果当选台湾总统,不担心鸿海在中国大陆的布局会成为他在两岸议题上的``软肋''。(自由时报档案图) 争取国民党提名参选2024年台湾总统选举的鸿海集团创办人郭台铭星期四(4月27日)表示,自己已交出鸿海经营权四年了;如果当选台湾总统,不担心鸿海在中国大陆的布局会成为他在两岸议题上的``软肋''……}

\entryitemWithDescription{国民党料5月20日前后 征召侯友宜选台总统}{https://www.zaobao.com/news/china/story20230428-1388110}{侯友宜星期四在新北市议会接受国民党团的市政总质询,承诺正式编列预算,明年1月1日成立新北市体育局。国民党团此举被视为``做球''给侯友宜,真正的压力测试将是5月5日开始的民进党团总质询。(温伟中摄) 国民党可能在5月20日前后征召新北市长侯友宜参选总统,鸿海集团创办人郭台铭也正到全台争取支持,他表示和侯友宜是兄弟登山各自努力,谁胜出都会全力团结……}

\entryitemWithDescription{海外警察站争议发酵 中国公安部批美起诉44名中国官员}{https://www.zaobao.com/news/china/story20230427-1388108}{两名男子被指在纽约唐人街为中国经营非法的秘密警察站,图为美国联邦地区法院提供给媒体,疑似秘密警察站所在地的照片。(法新社) 中国``海外警察站''风波持续发酵,美国上周起诉44名中国官员后,中国公安部批美国恶意炮制``跨国镇压'',并警告若美国继续一意孤行,中国必将坚决反制到底。 有分析指出,美国的执法行动可能产生连环效应,推动其他国家加强对类似事件的调查与执法……}

\entryitemWithDescription{``俄国通''中国欧亚事务特别代表李辉 将赴乌克兰等国}{https://www.zaobao.com/news/china/story20230427-1388104}{中国政府欧亚事务特别代表李辉被视为``俄国通''。他在前苏联时期就曾派驻莫斯科,也曾担任中国驻俄罗斯大使长达10年。图摄于2018年,李辉当时任驻俄大使。 (互联网) (北京综合讯)中乌元首在俄乌战争爆发以来首次通话后,中国将派政府欧亚事务特别代表、前驻俄罗斯大使李辉访问乌克兰……}

\entryitemWithDescription{大熊猫``丫丫''落地上海 结束20年旅美生活}{https://www.zaobao.com/news/china/story20230427-1388096}{旅美大熊猫``丫丫''星期四(4月27日)下午抵达上海浦东国际机场,结束20年旅美生活。图为中方人员在机场携带竹子和医药箱接机。(新华社) 备受中国舆论关注的旅美大熊猫``丫丫''星期四(4月27日)下午乘坐专机抵达上海,结束20年旅美生活。 综合中国官媒新华社和央视新闻报道,``丫丫''当地时间星期三(4月26日)从美国孟菲斯启程,经历约16小时飞行后顺利抵达上海浦东国际机场……}

\entryitemWithDescription{早说}{https://www.zaobao.com/news/china/story20230427-1388095}{(互联网) 泛民上一次``软性杯葛''新选制下的立法会选举,对自身的生存造成了极大伤害和冲击。如果这次再不参选区议会,就等于是自绝于香港的选举,在政治上是``自寻死路'',无可救药。 ------中国全国侨联副主席卢文端星期四(4月27日)在香港《明报》撰文谈香港新一届区议会选举改革。他认为,泛民本身需要参加区议会选举,应该根据区议会选举改革内容部署参选……}