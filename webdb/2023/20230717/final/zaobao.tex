\entryitemWithDescription{台湾年轻人凯道集会挺柯文哲 高喊民进党政府下台}{https://www.zaobao.com/news/china/story20230716-1414571}{台湾时代力量党前立委黄国昌和网红馆长陈之汉发起的``公平正义救台湾''活动7月16日在台北凯达格兰大道举行,约有三万年轻人与会。(路透社) 台湾在野政党主要政治人物星期天(7月16日)出席诉求司法改革和居住正义的集会。在场约三万年轻人顶着炙热烈阳,要求八年前同样提出``公平正义''却违背诺言的民进党政府下台。民众党总统参选人柯文哲则获年轻人支持,集会俨然成柯文哲造势大会……}

\entryitemWithDescription{美国气候特使克里抵达北京 重启中美气候谈判}{https://www.zaobao.com/news/china/story20230716-1414564}{美国气候特使克里抵达北京,以重启中美气候谈判。克里访华将考验中美全球两大温室气体排放国在其他课题存在严重分歧的同时,如何加强合作对抗全球暖化。 据中国央视新闻报道,克里星期天(7月16日)下午抵达北京,中美将在星期一(17日)就合作应对气候变化深入交换意见。 克里是继美国国务卿布林肯和财长耶伦后,近期第三位访华的美国高级官员……}

\entryitemWithDescription{民进党全代会上 赖清德接棒当``队长''}{https://www.zaobao.com/news/china/story20230716-1414561}{民进党主席、总统参选人赖清德(右)7月16日在台北圆山大饭店举行的民进党全代会上,与身穿同款棒球外套的立委提名人击掌打气。(民进党提供照片) 民进党全台党员代表大会(全代会)展现大团结作战气势,台湾总统蔡英文授战旗力挺民进党主席、总统参选人赖清德接棒当``台湾队队长'',赖清德称他已做好热身准备上场……}

\entryitemWithDescription{中俄去年军演次数创20年新高}{https://www.zaobao.com/news/china/story20230716-1414541}{有数据显示,中国和俄罗斯军队去年一共举行六次联合军演,是20年来最多的一次。 彭博社报道,根据美国国防大学中国军事事务研究中心编制的数据,中俄的联合军演占中国去年与外国军队演习总数的三分之二。 数据显示,中俄去年举行的六次联合军演中,有五次是在俄罗斯入侵乌克兰后举行。其中四次属于双边演习,两次是与伊朗和叙利亚等美国的宿敌举行联合演习……}

\entryitemWithDescription{香港理工大学冲突已有141人被判暴动罪成}{https://www.zaobao.com/news/china/story20230716-1414524}{香港理工大学2019年爆发的大规模警民冲突近四年后,再有18人被以暴动罪判刑,目前累计共有141人罪成。 综合中新社和中通社报道,香港区域法院星期六(7月15日)就18人判处监禁36至57个月。另有一名案发时17岁女被告,则须提取教导所报告,押后至8月5日才判刑。 此案的19名被告均被控于2019年11月18日在港理大外参与暴动,案发时年龄介于17岁至31岁间……}

\entryitemWithDescription{天主教教宗批准沈斌出任上海教区主教}{https://www.zaobao.com/news/china/story20230716-1414519}{中国天主教主教团主席沈斌出任上海教区主教三个月后,天主教教宗方济各批准了这一任命。但梵蒂冈也指责北京没有就任命提前协商,违反了双边协议。 据路透社报道,梵蒂冈星期六(7月15日)发布声明说,方济各已批准对沈斌的任命……}

\entryitemWithDescription{全球两强价值战 民进党大老邱义仁:陆美仍有意控管冲突}{https://www.zaobao.com/news/china/story20230716-1414487}{民进党大老邱义仁(左)与台湾前任驻WTO代表朱敬一(右),7月14日晚间在台北的座谈会上,就北京与华盛顿的价值战争如何影响世界贸易规则,以及台湾在其中的角色进行对谈。 (缪宗翰摄) 曾任台湾驻世界贸易组织(WTO)代表的学者朱敬一星期五(7月14日)在一场座谈会上感慨,WTO规范已赶不上科技快速发展,也加剧全球两强价值之争……}

\entryitemWithDescription{大陆学生抵台交流 舒缓两岸紧张形势}{https://www.zaobao.com/news/china/story20230715-1414260}{北京大学党委书记郝平(前排右)星期六(7月15日)率领北大等五所中国大陆高校37名师生抵台参访。奥运金牌得主、刚从北大毕业的退役乒乓球员丁宁(左)也随团访台。(中通社) (台北综合讯)应马英九文教基金会的邀请,一个中国大陆学生交流团星期六(7月15日)中午抵台,开始交流参访行程。马英九基金会执行长萧旭岑接机时表示,此次交流对舒缓两岸紧张形势意义重大……}

\entryitemWithDescription{中国国务院常务会议强调加强电力保供}{https://www.zaobao.com/news/china/story20230715-1414254}{上海星期二(7月11日)发布高温橙色预警,街上行人纷纷用手或穿上长袖衣遮挡酷日。(路透社) 中国严阵以待准备应对夏天旱季缺电问题,中国国务院总理李强星期五(7月14日)主持召开国务院常务会议,强调要加强高峰时段重点地区电力保供,坚决防范遏制重特大事故发生。 受访学者分析, 夏季能源供应短缺问题不只关乎中国国内的经济发展,也可能影响中国对外经贸合作……}

\entryitemWithDescription{郭台铭连续两晚造势 参选总统箭在弦上}{https://www.zaobao.com/news/china/story20230715-1414253}{郭台铭(中)星期五(7月14日)参加``董事长开讲''高雄站粉丝见面会,与主持人、国民党台北市议员徐巧芯(左)和美丽岛电子报董事长吴子嘉(右)同台。现场``选总统''呼声不绝于耳。(取自郭台铭脸书) 鸿海集团创办人郭台铭连续两晚造势,参选台湾总统箭在弦上。他强调台湾应该避战,但如果两岸开战,他准备把私人专机当军用运输机,送物资到前线……}

\entryitemWithDescription{香港谴责美通过或关闭港经贸办的法案}{https://www.zaobao.com/news/china/story20230715-1414251}{香港特区政府强烈谴责美国参议院外交委员会通过一项鼓吹取消香港驻美经贸办特权豁免待遇甚至关闭办事处的法案。 综合彭博社、《星岛日报》报道,港府是在星期五(7月14日)发布声明作出上述谴责。 涉事的《香港经济贸易办事处认证法案》可能关闭香港三个在美经贸办,它们被美国共和党议员指控为中国共产党的喉舌……}

\entryitemWithDescription{黄小芳:急速升温的气候变化挑战}{https://www.zaobao.com/news/china/story20230715-1413981}{2021年刚到北京的第一个夏天,36、37摄氏度的高温令我感到惊讶。和朋友聊起北京夏天的天气时,他笑称``不用担心,北京的最高温度不会超过39摄氏度''------当时没有明白他的笑点,后来才发现根据官方规定,若气温达到40摄氏度以上,当地就必须停止室外露天作业。 不过,朋友的说法并不可靠……}

\entryitemWithDescription{重庆万州新一轮强降雨 致1.6万余人受灾}{https://www.zaobao.com/news/china/story20230714-1414005}{重庆市万州区星期五(7月14日)再度遭遇新一轮暴雨袭击,停靠在江南新区路边的部分车辆陷在洪水淤泥里。(新华社) 10天前因暴雨发生洪涝导致重大伤亡事故的重庆市万州区,再度遭遇新一轮暴雨袭击。当地官方星期五(7月14日)清晨紧急升级防汛应急响应,要求做好断道、停工、停学、停业、停运、停游、停航等措施……}

\entryitemWithDescription{学者:中美交流恐成``相互指责''互动}{https://www.zaobao.com/news/china/story20230714-1413997}{美国丹佛大学国际关系教授赵穗生星期五(7月14日)在新加坡国立大学东亚研究所举办的线上研讨会上,分享对中美关系前景的看法。(线上研讨会截屏) 中美高层近期频密互动释放两国关系回暖信号,但长期研究中国外交政策的学者认为,这些互动不足以稳定两国关系;在双方不做出重大妥协的情况下,这些访问与沟通已变成了一种相互指责的互动……}

\entryitemWithDescription{华春莹图文并茂质问:是谁颠覆了国际秩序?}{https://www.zaobao.com/news/china/story20230714-1413952}{中国外交部部长助理华春莹星期四(7月13日)深夜在推特发图文质问:``是谁颠覆了国际秩序?''(取自华春莹推特账号) (北京综合讯)针对北大西洋公约组织维尔纽斯峰会公报对中国提出的指控,中国外交部部长助理华春莹在个人推特账号上图文并茂回怼:``是谁颠覆了国际秩序……}

\entryitemWithDescription{陆委会民调显示逾八成台湾民众不接受``一国两制''}{https://www.zaobao.com/news/china/story20230714-1413949}{据台湾政府的大陆委员会星期四(7月13日)公布今年度第二次民意调查结果,台湾八成以上的民众不赞成中国大陆提出的``一国两制'',近九成不认同解放军的军机军舰持续在台湾周边活动。 陆委会星期四在官网公布的新闻稿中指出,针对大陆持续对台综合施压,台湾人民坚决反对……}

\entryitemWithDescription{于泽远:福建舰终于要出海了?}{https://www.zaobao.com/news/china/story20230714-1413630}{网络近日流传的图片显示,仍在码头舾装的中国第三艘航母福建舰已拆除了位于斜甲板的三号电磁弹射器的安装工棚。军事学者分析,由于福建舰三条弹射器工棚的安装时间接近,三号弹射器安装工棚被拆除后,一号、二号弹射器安装工棚也将在近期被拆除,这意味着已下水一年多的福建舰距离出海海试的日子不远了……}