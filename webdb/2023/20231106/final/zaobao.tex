\entryitemWithDescription{香港全年增长预测将低于预期}{https://www.zaobao.com/news/china/story20231105-1448116}{港府正积极提倡``夜缤纷''活动以刺激消费,星期日(11月5日)落幕的香港美食嘉年华配合``夜缤纷'',开放时间比以往推迟一小时至晚上9时。(香港中通社) (香港综合讯)香港财政司司长陈茂波说,本周将发表的全年经济增长预测范围,将低于年初时的估算。 陈茂波星期日(11月5日)发表网志说,香港第三季经济增长4.1\%,幅度较预期温和……}

\entryitemWithDescription{新加坡两大咖啡粉厂商参展进博会}{https://www.zaobao.com/news/china/story20231105-1448110}{金源源创办人孙利顺(右)和孙昱轩父子兵齐上阵,将近年来推出的全线袋泡咖啡产品带到进博会参展。(陈婧摄) 新加坡两大咖啡粉厂商新隆利和金源源今年双双参加上海进博会,希望把新加坡袋泡咖啡介绍给广大中国消费者。 长期为餐饮业者供应咖啡粉的新隆利,在进博会上首次推出面向消费者研发的袋泡南洋咖啡。公司执行董事黄瀚禾告诉《联合早报》,这款耗时两年研发的咖啡明年才在新加坡发售,进博会访客将抢先尝到它的味道……}

\entryitemWithDescription{赖清德:若当选不改``中华民国''国号}{https://www.zaobao.com/news/china/story20231105-1448108}{民进党总统参选人赖清德(中)星期六(11月4日)回到他当过市长的台南市出席竞选总部成立大会,受到据称两万名台南乡亲夹道欢迎。(民进党提供) 台湾总统大选倒数十周,民调领先的民进党总统参选人赖清德主动出击,称若当选总统不会改``中华民国''国号,为各方疑虑的台独立场打预防针。 台湾将在2024年1月13日举行总统与立委选举,即将执政满八年的民进党(绿)派出副总统兼党主席赖清德参选……}

\entryitemWithDescription{中国财长:新增国债发行提速 稳妥化解地方债}{https://www.zaobao.com/news/china/story20231105-1448097}{中国金融中心上海外滩的夜景。(海峡时报) (北京综合讯)中国财政部长蓝佛安说,中国将加快推进新增国债发行使用。 中国官方媒体新华社星期天(11月5日)刊登对蓝佛安的专访。 蓝佛安表示,财政部将继续贯彻实施好积极的财政政策,在支出上持续发力,接下来会加快推进新增国债的发行使用,用好新增地方政府专项债券资金。同时,抓好一揽子化债方案落实,强化各级政府责任,积极稳妥推动化解地方政府债务风险……}

\entryitemWithDescription{李强:中国将积极扩大进口和放宽市场准入}{https://www.zaobao.com/news/china/story20231105-1448088}{中国总理李强(前排右)11月5日在上海出席第六届进博会开幕式前,参观企业展区。(新华社) 中国总理李强承诺,中国将以实际行动促进更大范围、更高水平和更深层次合作开放,包括扩大进口、进一步放宽市场准入、落实全面取消制造业领域外资准入限制等……}

\entryitemWithDescription{英媒:全球民调巨头盖洛普将关闭在华业务}{https://www.zaobao.com/news/china/story20231105-1448085}{美国咨询公司盖洛普(盖洛普官网) (北京/旧金山综合讯)全球民调巨头美国盖洛普咨询公司,据报在中国监管审查和地缘政治紧张的双重压力下,已决定撤出中国市场。 《金融时报》星期六(11月4日)引述三名消息人士透露,盖洛普(Gallup)已于本周告知客户上述消息。公司建议客户考虑将部分项目迁至海外,而其他项目将被取消。《金融时报》也取得一份公司的内部通知,显示盖洛普对结束在华运营表示遗憾……}

\entryitemWithDescription{中美举行外交政策磋商}{https://www.zaobao.com/news/china/story20231105-1448075}{(北京综合讯)中美两国上周除了举行首轮海洋事务磋商,也举行了外交政策磋商。 中国外交部官网星期六(11月4日)公布,外交部政策规划司司长苗得雨上星期三(1日)在维也纳,同美国国务院政策规划司司长艾哈迈德举行中美外交政策磋商。双方围绕国际地区形势、各自内外政策及其他共同关心的问题交换了意见。 据中国外交部官网,中美两国曾在2010年、2015年和2016年举行外交政策磋商……}

\entryitemWithDescription{专家示警中国人口负增长恐将延续至下世纪}{https://www.zaobao.com/news/china/story20231105-1448073}{北京一名女子9月12日推着一辆婴儿车,走在人流不多的街道。(法新社) (上海讯)中国人口问题专家说,中国人口负增长趋势可能延续到下世纪,应加速对人力资本的投资和需求结构转型。 第一财经报道,上海社科院原常务副院长兼经济研究所所长、国家自科与社科基金重大课题首席专家左学金星期六(11月4日)出席研讨会时指出,中国人口变动的主要风险已转向持续负增长、极低生育率与快速老龄化……}

\entryitemWithDescription{港媒:中国科研人员正研制新型远程``潜艇杀手''}{https://www.zaobao.com/news/china/story20231105-1448068}{(香港综合讯)香港媒体报道,中国科研人员正在研究将无人机和人工智能技术结合,通过火箭发射智能鱼雷,让它可以攻击距离中国海岸线200公里的潜艇。 据香港《南华早报》星期六(11月4日)报道,在中国科研人员电脑模拟军演中,中国解放军将鱼雷发射至一艘在国际水域巡航的核潜艇附近,由于鱼雷的射程仅为40公里,而附近又没有可以作为发射平台的军舰或飞机,鱼雷的出现让这艘潜艇措手不及,于是成功将其摧毁……}

\entryitemWithDescription{加拿大指中国军机不安全飞行 北京反指加军机恶意挑衅}{https://www.zaobao.com/news/china/story20231105-1448064}{(北京/渥太华综合讯)针对加拿大军方两周内第二次指责中国空军在南中国海不安全飞行,中国国防部回应称,加军机采取超低空飞行等挑衅动作,``别有用心的恶意挑衅''。 中国国防部官网星期六(11月4日)发布发言人张晓刚回应称,加拿大``渥太华''号护卫舰先后两架次不明意图的舰载直升机逼近中国西沙领空,``违反中国国内法及有关国际法''……}

\entryitemWithDescription{富士康在大陆被查 鸿海10月营收同比下滑4.56\%}{https://www.zaobao.com/news/china/story20231105-1448051}{一名戴着口罩的女子2022年11月9日路过鸿海位于台北的办公室。(路透社) (台北综合讯)富士康集团在中国大陆被查,其母公司台湾鸿海集团10月营收同比下滑4.56\%。 综合彭博社和《自由时报》报道,鸿海集团星期天(11月5日)公布10月营收为新台币7412亿元(约312亿新元),环比增长12.20\%,同比减少4.56\%……}

\entryitemWithDescription{港保安局长:美国制裁提案是黑社会行为}{https://www.zaobao.com/news/china/story20231105-1448036}{一批香港市民星期天(11月5日)自发到美国驻港总领事馆前,强烈谴责美国国会跨党派议员提出的法案,严重干预香港事务,严重违反国际关系法则。(香港中通社) (香港综合讯)针对美国国会议员将多名香港官员纳入制裁名单的提案,香港保安局局长邓炳强批评,西方国家此举是``黑社会行为'',目的是保护他们在香港的``走狗''……}

\entryitemWithDescription{台湾特稿:台湾大缺工之问:要开放外劳还是善用中老龄员工?}{https://www.zaobao.com/news/china/story20231105-1447541}{台湾老爷酒店将工作流程拆解成多种任务,方便中高龄员工完成任务(老爷酒店提供) 疫情过后,台湾旅宿餐饮业迎来报复消费潮,但旅宿业劳工缺口约有8000人,官方考虑向3000名外劳(台湾称移工)开放门户解燃眉之急。劳工团体和学者却持反对意见,呼吁政府和资方应正视台湾少子化的现实,且台湾将在2025年迈入超高龄社会,资方更应提升劳工薪资,改善劳动条件,为中高龄人口及妇女营造友善职场,创造多元共容的社会……}

\entryitemWithDescription{阿尔巴尼斯开启访华行程 吁中澳在可能的领域中合作}{https://www.zaobao.com/news/china/story20231104-1447936}{澳大利亚总理阿尔巴尼斯星期六(11月4日)起展开访华行程,这是七年来首次有澳洲总理访华。图为档案照。(路透社) 澳大利亚总理阿尔巴尼斯星期六(11月4日)起展开访华行程,他启程前形容此次访问是``非常积极的一步'',并呼吁两国应在可能的领域中合作,但在必要时持不同观点。 这是七年来首次有澳洲总理访华,受访学者分析,中澳正寻求回调关系,中国有可能宣布解除对部分澳洲商品的进口限制和惩罚性关税……}

\entryitemWithDescription{欧盟商会称注重形式多过实质 中国进博会被指是``政治秀''引发争议}{https://www.zaobao.com/news/china/story20231104-1447921}{第六届中国国际进口博览会将于11月5日至10日在上海举办。 (新华社) 中国重开国门后首届中国国际进口博览会开幕前夕,相关舆论争议再起。中国欧盟商会批评进博会已变成``政治秀'',上海学者则指进博会作为综合性国际公共交流平台的属性,决定了它不可能完全去政治化。 第六届进博会11月5日至10日在上海召开。中国总理李强将出席星期天(11月5日)上午的开幕式并发表主旨演讲……}

\entryitemWithDescription{留日港生发表支持港独言论被判入狱两个月}{https://www.zaobao.com/news/china/story20231104-1447914}{(香港综合讯)一名留学日本的香港大学生在社交平台发表支持港独的言论,被香港法院判处入狱两个月。 综合《明报》和``香港01''报道,在东京一所大学就读法学院政治及经济科二年级的袁静婷(23岁),被指从2018年9月起,在脸书和Instagram账号发布13个煽动帖文,包括主张港独的内容,以及含有``光复香港时代革命''口号的海报照片。其中11个帖文在日本发布……}

\entryitemWithDescription{中国将推动稀土产业高端化发展}{https://www.zaobao.com/news/china/story20231104-1447913}{美国地质调查局数据显示,去年中国稀土矿产量占全球产量70\%,其次是美国和澳大利亚。图为美国加州一处稀土矿场。(路透社档案照) (北京综合讯)中国将加大高端稀土新材料攻关和产业化进程,推动稀土产业高端化、高质量发展。 据新华社报道,中国总理李强星期五(11月3日)主持召开国务院常务会议,提出上述目标……}

\entryitemWithDescription{中国工信部长:工业经济企稳 新能源车成亮点}{https://www.zaobao.com/news/china/story20231104-1447902}{中国工信部长金壮龙说,电动汽车等新能源产业表现亮眼。图为中国电动车品牌比亚迪在11月1日的东京车展上。(路透社) (北京综合讯)中国工业和信息化部部长金壮龙说,今年前三季度中国工业经济企稳回升,电动汽车等新能源产业表现亮眼。 金壮龙接受中国官媒央视访问时说,中国工业``1到9月经过了企稳回升,回升向好态势保持住了,成绩单来之不易……}

\entryitemWithDescription{中国小学生课间无休息 教育部要求确保学生课间正常活动}{https://www.zaobao.com/news/china/story20231104-1447901}{中国教育部要求各地政府及学校坚决纠正剥夺学生课间休息的不当做法。图为2021年9月1日上海一所学校正举行降旗仪式。(路透社) (北京综合讯)针对中国中小学生课间休息10分钟时间被剥夺,只允许喝水和上厕所的现象,中国教育部回应称,要求各地政府及学校坚决纠正剥夺学生课间休息的不当做法……}

\entryitemWithDescription{爱丁堡大学归还四具台湾原住民遗骨}{https://www.zaobao.com/news/china/story20231104-1447900}{英国爱丁堡大学将四具死于南台湾战场的台湾原住民遗骨归还台湾。图为星期五在爱丁堡大学举行遗骨返还仪式。(台湾原住民族委员会网站) (台北综合讯)英国爱丁堡大学正式将四具近150年前死于南台湾战场的台湾原住民遗骨归还台湾。 综合法新社、台湾中央社和《自由时报》报道,爱丁堡大学星期五(11月3日)举行了遗骨返还仪式。返还仪式开始前,还举行了一场与祖灵沟通的传统仪式,台、英双方仅部分人员参与其中……}

\entryitemWithDescription{黄小芳:中国手机巨头从``遥遥领先''到``越级对标''}{https://www.zaobao.com/news/china/story20231104-1447753}{在上海的苹果手机专卖店里,一名女子手中握着一台新款iPhone 15 Pro和华为的Mate 60 Pro。(路透社) 华为8月底低调发布新款手机Mate 60 Pro后,中国网络上出现不少以``华为吃饱、苹果跌倒''为题的文章,称华为凭一机之力,夺走了苹果在中国的市场份额。 在中国不少舆论看来,苹果在美国时间星期四(11月2日)发布的第四季度财报,似乎印证了这些文章的判断……}