\entryitemWithDescription{中美元首就建交45周年互致贺信 学者:美国对华政策基调不太可能改变}{https://www.zaobao.com/news/china/story20240101-1459440}{时任中国副总理邓小平(左)在中美建交后到访美国,在1979年1月31日于白宫与时任美国总统卡特签订中美科技合作协定和文化协定。 (互联网) 中美两国元首在2024年元旦互致贺信,祝贺两国建交45周年。分析指出,中美元首会晤和建交45周年,为增进双边沟通与合作提供契机,但两国关系同时面对美国总统大选和台湾问题等风险压力……}

\entryitemWithDescription{澳门赌收逆势上扬 12月同比大涨433\%}{https://www.zaobao.com/news/china/story20240101-1459426}{澳门12月的博彩收入飙升逾四倍。图为大批旅客星期天(12月31日)在澳门知名景点大三巴牌坊前游览。(新华社) (澳门综合讯)澳门博彩业在中国经济放缓下依旧稳健,12月博彩收入逆势飙升逾四倍,年度总收入达至疫情前的六成以上……}

\entryitemWithDescription{台湾副总统电视辩论会 两岸问题再成交锋重点}{https://www.zaobao.com/news/china/story20240101-1459424}{台湾副总统候选人电视辩论会1月1日登场,左起为民进党候选人萧美琴、民众党候选人吴欣盈,以及国民党候选人赵少康。(台北市摄影记者联谊会提供) 距离台湾大选剩不到两星期,星期一登场的副总统候选人辩论会上,国民党候选人赵少康与民进党候选人萧美琴皆批评对方的两岸立场,将为台湾带来风险;民众党候选人吴欣盈则说蓝绿一个靠中、一个反中,都失去台湾主体性……}

\entryitemWithDescription{中国据报要求延后中日韩领导人峰会}{https://www.zaobao.com/news/china/story20240101-1459416}{(首尔/北京/东京综合讯)韩国外交消息人士称,中国要求将中日韩领导人峰会推迟到中国两会之后举行。 据韩国《中央日报》上周五(12月29日)报道,熟悉中日韩三国领导人会议协商情况的韩国外交部门消息人士透露,中国大陆预计本月13日台湾总统选举结束后,下来一两个月要集中处理台湾相关情况,因此要求在今年3月两会(全国人大与政协年会)后举行中日韩三国领导人峰会……}

\entryitemWithDescription{蔡英文元旦记者会称 中华民国宪法不是风险 连结九二共识才是}{https://www.zaobao.com/news/china/story20240101-1459389}{台湾总统蔡英文星期一(1月1日)在总统府发表任内最后一次新年谈话后接受媒体联访。她说,《中华民国宪法》不是风险,跟``九二共识''连结才是。(自由时报) 针对民进党总统候选人赖清德有关``中华民国宪法带来灾难''的说法,台湾总统蔡英文在元旦记者会上进一步解释,《中华民国宪法》不是风险,跟``九二共识''连结才是……}

\entryitemWithDescription{香港警方为市民提供骗案预警}{https://www.zaobao.com/news/china/story20240101-1459387}{(香港综合讯)香港警方将推出新措施,以手机短讯方式为市民提供骗案预警。 根据香港政府星期一(1月1日)发布的新闻公报,警务处反诈骗协调中心下的``骗案预警''计划自星期二(1月2日)起,新增以短讯方式联络潜在受害人,适时发出警示和给予建议……}

\entryitemWithDescription{特稿:中国一线城市租房市场入冬}{https://www.zaobao.com/news/china/story20240101-1459386}{中国楼市冷风从买卖市场吹到租赁市场,北京、上海等一线城市的房租也持续走低。图为上海市中心楼房。(彭博社) ``大家都以为(冠病疫情管控)放开后市场就会回暖,没想到2023年比2022年更惨。'' 上海租房中介刘燕(化名)告诉《联合早报》,2023年她经手的租约比2022年减少三成,租金水平同比下降20\%至30%。入行八年来,她第一次感到``上海的房子也愁租''……}

\entryitemWithDescription{中国旅游业复苏强劲 免签政策提振入境游}{https://www.zaobao.com/news/china/story20240101-1459383}{在去年12月底的哈尔滨冰雪节上,许多游客都在与展示的冰雕拍照。(路透社) (北京综合讯)中国旅游市场在去年实现快速复苏,官方数据显示,实施六国免签政策首月有超过11万人次免签入境中国。预计今年中国国内出游人次将突破60亿,旅游收入也有望显著增长……}

\entryitemWithDescription{于泽远:中国军队反腐令人震撼}{https://www.zaobao.com/news/china/story20240101-1459267}{2023年12月29日,中国全国人大常委会一口气罢免了九名军方高级将领的全国人大代表资格,意味着这九名将领都``出事''了,否则不会被罢免。 这九人中,有据可查的至少有三名上将、四名中将……}

\entryitemWithDescription{赖清德``中华民国宪法灾难''说冲击选情 但对蓝绿哪方加分仍难料}{https://www.zaobao.com/news/china/story20231231-1459272}{台湾民进党总统候选人赖清德的``中华民国宪法带来灾难''说,引起中国大陆国务院台湾事务办公室、台湾在野党强力抨击、赖清德再三灭火澄清,对两岸关系和选情的冲击持续发酵。 受访学者评估,赖清德若上台,中国大陆将加大反独力道,赖清德则将紧握主流民意为后盾,但不至于走向制宪建国,也避免为美国制造麻烦。 台湾将在不到两周后的1月13日迎来总统与立委选举……}

\entryitemWithDescription{大陆军事杂志:台湾``雄升''导弹的威胁有限}{https://www.zaobao.com/news/china/story20231231-1459254}{(香港/台北综合讯)中国大陆一家军事杂志的文章认为,台湾``雄升''地对地巡航导弹对大陆的威胁有限。 《南华早报》星期天(12月31日)报道,雄升导弹的打击范围可达大陆东部、南部和中部。不过,《兵工科技》11月下旬发表的署名文章写道,由于雄升导弹的体积较大、速度亚音速且缺乏隐身技术,很容易``被现代、灵敏和精确的防空雷达系统探测、跟踪和监控''……}

\entryitemWithDescription{中国加强安全管控 多城宣布不办跨年夜活动}{https://www.zaobao.com/news/china/story20231231-1459252}{(北京综合讯)2024年元旦前夕,中国官方要求各地加强跨年夜安全防范,多个城市宣布不组织跨年夜庆祝活动,包括上海、广州和武汉等。 中国应急管理部去年12月28日发布2024年元旦假期安全提示,指元旦假期节庆、旅游、娱乐等活动增多,人流物流聚集,需加强跨年夜等重大集会活动安全管控,防范人员密集场所发生火灾和踩踏等事故……}

\entryitemWithDescription{传民进党前主席施明德病危急需输血}{https://www.zaobao.com/news/china/story20231231-1459243}{(台北综合讯)民进党前主席施明德日前传出肝癌复发,状况危急入住加护病房抢救中;上星期六更传出他正使用叶克膜与死神搏斗,且血小板过低。台北荣民总医院星期天表示,施明德的血小板状况已回复稳定,请外界尽量给予空间。 台湾民意基金会董事长游盈隆星期六(2023年12月30日)在脸书表示,施明德正使用叶克膜与死神搏斗中,且出现血小板过低状况……}

\entryitemWithDescription{告别疫情 北京方舱变民宅出租}{https://www.zaobao.com/news/china/story20231231-1459234}{金盏七彩家园目前入住率只有约十分之一,不少楼栋仍完全空置,空房间里摆着崭新的家具。(黄小芳摄) ``一开始有种奇妙的感觉,像又被隔离了一样。但习惯以后,就觉得没什么。'' 北京金盏七彩家园的住户马识路(20岁,学生)坐在狭小的合租房里接受《联合早报》采访,房间乍看没有什么特别之处;但望出窗外,整齐排列的方舱建筑映入眼帘,抬头一看,则是有别于一般房屋的条纹铁皮天花板……}

\entryitemWithDescription{中国特稿:中国式催婚失效 大龄青年不将就结婚意愿降}{https://www.zaobao.com/news/china/story20231231-1458782}{中国年轻人的婚育年龄普遍推迟,在2020年中国平均初婚年龄为28.67岁,比2010年增加3.78岁。(中新社) 随着社会变迁,中国现代婚姻观念出现明显改变,越来越多中国年轻人结婚意愿不足,推迟甚至放弃婚姻,婚姻不再是人生必备大事。自冠病疫情以来,个体对未来发展的不确定与失控感,更加剧了这一趋势。 每逢佳节被催婚。年末将至,在一些中国家庭,长辈对年轻一代的``中国式催婚''又要开始了……}

\entryitemWithDescription{台总统候选人辩论 蓝绿白对决两岸与房地产议题}{https://www.zaobao.com/news/china/story20231231-1459154}{民进党总统候选人赖清德(左起)、国民党候选人侯友宜和民众党候选人柯文哲星期六(12月30日)在电视辩论会开始前,先礼后兵握手问好。(法新社) 台湾总统候选人电视辩论会星期六(12月30日)登场,绿营民进党的赖清德、蓝营国民党的侯友宜,与白营民众党的柯文哲正面对决,针对两岸关系、疫苗采购合约和房地产争议激烈交锋……}

\entryitemWithDescription{赖清德辩称``中华民国给台湾带来灾难''是口误 受访学者:那是赖清德的真心话}{https://www.zaobao.com/news/china/story20231230-1459148}{台湾执政的民进党正副总统候选人赖清德(左三)和萧美琴(左四)共同出席电视辩论会后的记者会。 (庄慧良摄) 台湾执政的民进党总统候选人赖清德在电视辩论会上提及``中华民国给台湾带来灾难'',当场被民众党总统候选人柯文哲抓到把柄。赖清德会后澄清称是口误,漏讲了``宪法''两字。但国民党主席朱立伦直指赖清德是``务实的台独工作者'' ,只是说出自己的心声……}

\entryitemWithDescription{中使馆促英方将钟翰林缉拿归案并遣返香港}{https://www.zaobao.com/news/china/story20231230-1459133}{钟翰林星期五(12月29日)在其伦敦卧室内的港独旗帜前留影。(法新社) (北京综合讯)中国驻英国大使馆敦促英国政府尽快将香港独派学生组织``学生动源''前召集人钟翰林缉拿归案,并将他遣返香港。 根据中国驻英使馆网站星期五(12月29日)消息,使馆发言人就钟翰林抵达英国寻求政治庇护一事答记者问时,作出以上呼吁,并称钟翰林违反监管令,潜逃英国,香港惩教署已发出召回令,对其依法通缉……}

\entryitemWithDescription{华为5G手机据报产能不足 未能改变中国市场格局}{https://www.zaobao.com/news/china/story20231230-1459129}{华为今年8月29日无预警开卖Mate 60 Pro手机,测评达到5G速度,被认为成功突破美国科技战的卡脖子。图为今年9月华为在北京的旗舰店售出的Mate 60机型。(路透社档案图) (深圳综合讯)中国通讯巨头华为今年8月无预警地推出5G手机,却因产能不足,出货量不如苹果和其他中国品牌,被中国科技媒体指未能改变中国手机竞争格局……}

\entryitemWithDescription{四川凉山百万粉丝网红因带货造假获刑}{https://www.zaobao.com/news/china/story20231230-1459128}{网红主播``赵灵儿''(右)和``曲布''在抖音开直播与观众互动。(互联网) (成都综合讯)曾是中国深度贫困地区的四川大凉山,爆出网红以``助农''名义直播销售假冒特色农产品牟利的案件,九人分别被判八个月至三年两个月不等的有期徒刑……}

\entryitemWithDescription{京东告阿里巴巴胜诉获赔10亿元人民币}{https://www.zaobao.com/news/china/story20231230-1459114}{业内人士称,中国电商巨头企业京东(左)起诉同行阿里巴巴(右)``二选一''案的判决有着创纪录的赔偿金额,具有示范效应。(互联网) (北京综合讯)中国电子商务巨企京东起诉同行阿里巴巴的``二选一''案一审胜诉,获赔10亿元(人民币,下同,1.87亿新元)……}