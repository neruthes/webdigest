\entryitemWithDescription{杨丹旭:中国对美舆论要调整?}{https://www.zaobao.com/news/china/story20240417-3455076}{美国财政部长耶伦最近访华,有中国学者注意到,中国媒体对耶伦此行的报道,与近年来对美国的报道有些不同,``这次舆论的松弛感更强一些''。 回想起来,4月4日下午耶伦飞抵广州,斜跨一个帆布袋、手拎一个公文包走下飞机,一身朴素的装扮在中国社交媒体引发热议。一些网民感叹,她就像一位邻家老太太。 随后,中国媒体曝光了耶伦访华后的第一餐------在广州陶陶居用餐的菜单……}

\entryitemWithDescription{赖清德内政大权一把抓 外交国防延续蔡英文路线}{https://www.zaobao.com/news/china/story20240416-3452088}{台湾候任总统赖清德4月10日宣布,委任民进党前主席卓荣泰出任行政院长。(法新社) 台湾候任总统赖清德的新政府人事布局,被视为内政大权一把抓,外交国防则延续现任总统蔡英文的路线。 现任民进党主席的赖清德将在5月20日宣誓就任总统,新内阁团队也将于同日上任。他在4月10日宣布,委任民进党前主席卓荣泰出任行政院长……}

\entryitemWithDescription{中国3月新房售价降幅为近九年最大 分析:北京或决心摆脱地产拖累}{https://www.zaobao.com/news/china/story20240416-3454763}{路透社基于中国国家统计局的数据计算得出,中国3月新建商品住宅价格同比下降2.2\%,为2015年8月以来的最大降幅。图为民众3月15日在北京一售楼处了解楼盘信息。(中新社) 中国房地产市场持续低迷,3月新建商品住宅售价降幅为近九年最大。分析认为,中国房地产风险仍在出清中,但中国第一季经济增长优于预期,可能会让北京坚定通过转换增长模式,逐步摆脱房地产对经济的拖累……}

\entryitemWithDescription{五角大楼:美中防长举行视讯通话}{https://www.zaobao.com/news/china/story20240416-3455107}{(华盛顿综合电)美国五角大楼发布文告称,美国国防部长奥斯汀与中国国防部长董军星期二(4月16日)举行视讯通话,是美中防长近18个月以来首度实质对话。 综和法新社和路透社的报道,两人讨论美中国防关系以及区域和全球安全议题。奥斯汀强调尊重国际法所保障的公海航行自由重要性,特别是在南中国海。 美国国防部指出,两人也谈及俄乌战争、朝鲜问题……}

\entryitemWithDescription{中越防长会晤 签署设立海军热线谅解备忘录}{https://www.zaobao.com/news/china/story20240416-3454244}{(北京综合讯)中国国防部长董军上星期率团出席中越边境国防友好交流活动,并与越南国防部长潘文江举行会谈,并签署设立中越海军热线的谅解备忘录。 据中国国防部官网消息,中越第八次边境国防友好交流活动上星期四至五(4月11日至12日),先后在越南老街省老街市及中国云南省红河州河口县国际口岸举行,董军与潘文江分别率团出席并举行会谈。 这是董军去年12月出任中国国防部长后,首次踏出国境进行军事外交访问……}

\entryitemWithDescription{王毅:相信伊朗能把握好局势 赞赏强调不针对周边国家}{https://www.zaobao.com/news/china/story20240416-3453901}{(北京综合讯)中国外交部长王毅星期一称,中国注意到伊朗表示对以色列采取的行动是有限度的。他相信伊朗能够把握好局势,避免中东局势进一步动荡。 伊朗上星期六(4月13日)对以色列发动大规模导弹与无人机袭击,以报复4月1日在伊朗驻叙利亚首都大马士革的使馆馆舍遇袭事件。国际社会对此高度关注,呼吁伊以两国保持克制……}

\entryitemWithDescription{中国经济首季增长5.3\%超预期 分析:增长势头仍不稳固}{https://www.zaobao.com/news/china/story20240416-3454359}{中国第一季经济同比增长5.3\%,明显超出市场预期。受访分析师认为强劲的制造业和出口活动拉动了增长,但3月出现的数据波动凸显当前的增长势头并不稳固。 中国国家统计局星期二(4月16日)公布的数据显示,第一季中国国内生产总值(GDP)为29.6万亿元人民币(5.58万亿新元),同比增长5.3\%,高于此前高盛、摩根士丹利机构预测的5\%……}

\entryitemWithDescription{百度AI聊天机器人文心一言用户数突破2亿}{https://www.zaobao.com/news/china/story20240416-3454640}{百度创始人李彦宏星期二在百度AI开发者大会上说,``文心一言''用户数已突破2亿。(彭博社) (香港路透电)中国互联网巨头百度说,该公司开发的人工智能(AI)聊天机器人``文心一言''用户数已突破2亿。 百度创始人、董事长兼首席执行官李彦宏星期二(4月16日)在百度AI开发者大会上说,文心一言目前每天应用程序编程接口(API)的调用量也突破了2亿次,服务客户数达到8.5万……}

\entryitemWithDescription{欧盟拟对中国医疗器材采购启动调查 迈瑞:一向依法合规}{https://www.zaobao.com/news/china/story20240416-3453855}{(华盛顿/广州综合讯)据报欧盟将对中国医疗器材采购启动调查,以消除对中国政府偏袒国内供应商的担忧。消息传出后,中国A股医疗器械板块星期二(16日)集体走低。 彭博社星期一(4月15日)引述知情人士报道,欧盟可能最早在4月中旬宣布对中国医疗器材采购启动调查。这一调查将先从企业和成员国收集信息,然后开始与中国就市场公平和开放进行谈判……}

\entryitemWithDescription{北京半马疑似出现造假 中国选手被指获``保送''夺冠}{https://www.zaobao.com/news/china/story20240416-3453318}{北京星期天举行的半程马拉松比赛上,三名非洲跑手在冲刺阶段被指刻意让中国选手何杰(右二)夺冠。(路透社) (北京综合讯) 北京半程马拉松比赛疑似出现造假丑闻,三名非洲跑手冲刺时被指故意留力,刻意让中国选手何杰夺冠。 陷入争议的这场赛事,是星期天(4月14日)举行的北京国际长跑节---北京半程马拉松男子组比赛……}

\entryitemWithDescription{北京强化中华民族论述应对``去中国化''}{https://www.zaobao.com/news/china/story20240416-3451118}{分析认为,北京藉由习马二会强化中华民族论述,意在应对民进党长期执政造成的``去中国化''现象。 图为台独团体2月在台北街头放置的宣传旗帜。(法新社) 习马二会侧重中华民族论述,引发两岸舆论关注,甚至传出台湾候任总统赖清德若能把握``中华共识'',就能促使两岸关系趋缓。不过学者分析认为,北京此时强化中华民族论述,意在应对民进党长期执政造成的``去中国化''现象……}

\entryitemWithDescription{戴庆成:以玉女剑法破解吸星大法}{https://www.zaobao.com/news/china/story20240416-3438350}{香港武侠小说名家金庸一生创作了15部武侠小说,作品具有深刻的人文、社会及艺术价值,陪伴了一代又一代香港人成长,可以说是许多港人的集体回忆。 今年适逢金庸百年诞辰,港府和民间团体近来纷纷举办各种不同的纪念活动,包括展出金庸当年手稿、小说人物雕像、漫画家作品,以及举行相关讲座、征文比赛、音乐会等,在社会上掀起了一股热潮。就连一些官员平日谈话,也提及金庸的作品……}

\entryitemWithDescription{朔尔茨访上海 要求中国不要倾销和生产过剩}{https://www.zaobao.com/news/china/story20240415-3438777}{德国总理朔尔茨(右)星期一(4月15日)上午在上海参观科思创亚太创新中心。(新华社) (上海综合讯)德国总理朔尔茨星期一强调,他支持开放和公平的市场,要求中国不要倾销和生产过剩,同时呼吁欧盟不要出于保护主义的私利而采取行动。 朔尔茨星期日抵达中国,是今年首位访华的西方大国领导人。他星期一(4月15日)上午飞抵上海,继续访华行程,外界密切关注他如何就经贸问题表态……}

\entryitemWithDescription{广交会开幕 加强``新三样''相关展区}{https://www.zaobao.com/news/china/story20240415-3438344}{广交会星期一(4月15日)在广州开幕,共有2万9000家企业参展。图为新能源汽车展区吸引采购商参观。 (中新社) (广州/北京综合讯)中国规模最大的国际贸易展会广交会星期一(4月15日)在广州开幕,在美国与欧盟关注中国外贸``新三样''产能过剩之际,本届展会将加强培育新能源汽车及智慧出行等展区……}

\entryitemWithDescription{总统就职典礼外宾锐减 学者直言台湾外在环境不容乐观}{https://www.zaobao.com/news/china/story20240415-3434555}{台湾总统当选人赖清德4月10日在台北举行的记者会上宣布,民进党前主席卓荣泰将在5月20日出任新政府的行政院长。(法新社) 今年5月20日将有超过400名外宾出席台湾新总统就职典礼。对比现任总统蔡英文八年前就职时有超过700名外宾出席,出席赖清德就职典礼的外宾人数明显锐减。 受访外交学者直言,台湾外在环境不容乐观,许多国家因现实考量在台海两岸之间做出取舍,不排除近期再有邦交国断交……}

\entryitemWithDescription{中国最高法院要求离婚案提示关爱未成年子女}{https://www.zaobao.com/news/china/story20240415-3437988}{(北京综合讯)中国最高人民法院要求各地法院在涉及未成年子女的离婚案件中,提示关爱未成年人,消除引发未成年人违法犯罪的消极因素,预防未成年人犯罪。 据央视新闻客户端报道,中国最高人民法院星期一(4月15日)在一份文件中要求,各地法院要引导离婚案件当事人正确处理婚姻自由与维护家庭稳定的关系,关心、关爱未成年子女,关注未成年子女健康成长的精神和物质需求……}

\entryitemWithDescription{中国吁伊以各方冷静克制避免局势升级 学者:中国能发挥的作用有限}{https://www.zaobao.com/news/china/story20240415-3437492}{联合国安理会星期日(4月14日)就伊朗袭击以色列在纽约联合国总部召开紧急会议。图为中国常驻联合国代表团临时代办戴兵(前排右二)发言。(中新社) 中国在伊朗袭击以色列后,接连呼吁``有关方面保持冷静克制'',避免紧张局势进一步升级。中国常驻联合国代表团临时代办戴兵在安理会紧急会议上,称伊朗的行动是对以色列袭击伊朗驻叙利亚外交馆舍的回应,``有关问题至此可视为已解决''……}

\entryitemWithDescription{中国外交部官员批美网安机构对华进行栽赃陷害}{https://www.zaobao.com/news/china/story20240415-3437797}{(北京综合讯)中国外交部官员批评美国网络安全机构和企业勾连腐败,就网络安全问题对中国进行栽赃陷害,并以此获得部门和经济利益,也为美国对华关系增添非理性因素。 据中国央视新闻报道,中国国家计算机病毒处理中心与360公司星期一(4月15日)共同发布《``伏特台风''一一美国情报机构针对美国国会和纳税人的合谋欺诈行动》报告……}

\entryitemWithDescription{于泽远:中国能否拉住欧洲?}{https://www.zaobao.com/news/china/story20240415-3422254}{德国总理朔尔茨4月14日(星期天)开始对中国进行为期三天的访问,是他两年内第二次访华。中国料将利用朔尔茨访华的机会,尽量维护中德正常的经贸、政治关系,从而拉住欧洲,不让欧洲在中美博弈中完全站队美国。 4月以来,中美关系表面上有所缓和。先是4月2日中美元首通话,双方认为去年11月旧金山会晤后中美关系取得了进展;接着美国财政部长耶伦访华,强调美国不会同中国``脱钩''……}

\entryitemWithDescription{朔尔茨抵重庆访华三天 分析:率庞大政商代表团加强德中贸易}{https://www.zaobao.com/news/china/story20240414-3422752}{德国总理朔尔茨(右一)星期天飞抵重庆,与到场迎接的中国驻德国大使吴垦(左二)握手,重庆市副市长张国智(中)也在场迎接。(路透社) 德国总理朔尔茨星期天抵达重庆开启访问中国的行程。专家分析,朔尔茨此次率庞大政商代表团访华,释放希望加强对华贸易的信号,但预计将与北京讨价还价,寻求更大程度的贸易平衡……}