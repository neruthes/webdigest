\entryitemWithDescription{中国科企抢跑人形机器人赛道}{https://www.zaobao.com/news/china/story20230810-1422546}{以扫地机器人闻名的追觅科技,今年3月推出第二代仿生机器狗和通用人形机器人。(陈婧摄) 步入苏州追觅科技的展厅,四周陈列着吸尘器和扫地机等招牌产品,摆在正中的却是一个和普通男子体型相仿的机器人,以及一只摇头摆尾地和访客互动的机器狗。 这家以扫地机器人闻名的独角兽企业,今年3月推出第二代仿生机器狗和通用人形机器人……}

\entryitemWithDescription{港警国安处拘捕10名涉违国安法``612基金''人士}{https://www.zaobao.com/news/china/story20230810-1422534}{香港社运人士叶宝琳(右)8月10日被香港警方带到她工作的天主教书局塔冷通心灵书舍蒐证后,再被押走。(路透社) 香港警务处国家安全处星期四(8月10日)拘捕10人,指他们涉嫌捐助流亡海外的港人,违反《香港国安法》的``串谋勾结外国或者境外势力危害国家安全''罪及煽动暴动罪。 被捕的四男六女,年龄介于26至43岁……}

\entryitemWithDescription{乌克兰驻华大使称中俄关系是``权宜联姻''}{https://www.zaobao.com/news/china/story20230810-1422529}{在中国罕见公开批评俄罗斯、中国参与乌克兰和平峰会,中俄乌三国关系出现微妙变化之际,乌克兰驻华大使形容,中俄关系是一场为了资源的``权宜婚姻''。 据彭博社报道,乌克兰驻华大使利亚比肯(Pavlo Riabikin)星期二(8月8日)接受乌克兰新闻通讯社(RBC)采访时说,北京根据自身利益制定外交政策,对中国来说,俄罗斯既是政治伙伴,也是资源来源……}

\entryitemWithDescription{美国将限制对华敏感技术投资 中国称将保留采取措施权利}{https://www.zaobao.com/news/china/story20230810-1422527}{美国总统拜登8月9日签署行政令,限制美国公司和个人投资中国的敏感技术。图摄于8月9日。(路透社) 美国总统拜登星期三(8月9日)签署行政命令,限制美国公司和个人投资中国的敏感技术,包括半导体、量子运算与人工智能;中国外交部和商务部隔天提出严正交涉和强烈不满,誓言保留采取措施的权利。 受访学者分析,行政令将带来外溢效应,美国盟友有可能会跟进采取类似措施,其他国家的企业对华投资也将更谨慎……}

\entryitemWithDescription{中国警方侦破79起利用AI换脸欺诈案件}{https://www.zaobao.com/news/china/story20230810-1422520}{中国公安部通报,在针对``AI(人工智能)换脸''欺诈问题的专项行动中,全国共侦破相关案件79起,抓获犯罪嫌疑人515名。 据中新经纬报道,中国公安部网络安全保卫局副局长李彤星期四(8月10日)在发布会上说,犯罪分子用于实施``AI换脸''的物料主要为照片,特别是身份证照片,同时结合人员姓名、身份证号来突破人脸识别验证系统……}

\entryitemWithDescription{中国就驻伦敦使馆新址之争向英国抗议}{https://www.zaobao.com/news/china/story20230810-1422519}{中国驻伦敦大使馆新址计划去年底被当地议会否决后,至今仍没有进展。图为使馆新址伦敦塔附近的景观。(路透社) 中国驻伦敦大使馆新址计划去年底被当地议会否决后,至今没有进展。中国因此向英国政府提出抗议,指其未能履行外交义务。 中国在2018年斥资2.55亿英镑(约4.4亿新元)购置了伦敦塔附近的皇家铸币厂旧址,以便将现在位于波特兰大街的使馆迁至该地……}

\entryitemWithDescription{《愿荣光》禁令上诉案 香港律政司:法庭应遵行政机关判断}{https://www.zaobao.com/news/china/story20230810-1422491}{香港律政司不服高等法院拒绝就传播反修例歌曲《愿荣光归香港》批出禁制令而提出上诉,列出的上诉理由包括,法庭应``遵从于行政机关的判断''。 此前,香港律政司在6月5日入禀高院申请禁制令,禁止公众在网上或任何平台传播2019年反修例运动期间出现的歌曲《愿荣光归香港》,但香港高院上月拒批临时禁制令。 香港律政司星期一(8月7日)提出上诉,并在星期三公开草拟上诉文件……}

\entryitemWithDescription{中国恢复对日韩美澳等国出境团队游}{https://www.zaobao.com/news/china/story20230810-1422481}{中国宣布恢复对日本、韩国、澳洲等国的出境团队游业务。图为人们6月14日在日本涩谷人行道上行走。(法新社) 中国文化和旅游部宣布恢复对日本、韩国、澳洲、英国、德国和美国等多国的出境团队游业务。消息发布后,中国多家旅行社与在线旅行服务平台陆续上线相关产品,游客咨询量迅速攀升……}

\entryitemWithDescription{中国大陆批麻生涉台言论是想把台民众推向火坑}{https://www.zaobao.com/news/china/story20230810-1422470}{针对日本前首相麻生太郎访问台湾时提出,为避免台海发生战争,要具备吓阻实力、执行吓阻决心,中国大陆外交部批评称,日本政客访台必言战,摆出唯恐台海不乱的架势,是想把台湾民众推向火坑。 中国大陆外交部发言人星期三(8月9日)在官网就日本自民党副总裁麻生太郎访台时的言论,以回答记者问形式作出上述回应……}

\entryitemWithDescription{中国首条直通中越边境高速铁路开始铺轨}{https://www.zaobao.com/news/china/story20230809-1422213}{中国铁路南宁局集团称,从广西防城港至东兴的防东铁路星期二已开始铺轨。图为7月20日,防东铁路西湾跨海双线特大桥在进行桥梁架设。(中新社) 据中国铁路南宁局集团消息透露,从广西防城港至东兴的铁路星期二开始铺轨,意味着中国首条直通中越边境的高速铁路------防东铁路开始铺轨……}

\entryitemWithDescription{侯友宜主张将核电列入能源选项 民进党讥``侯友宜参选人打脸侯市长''}{https://www.zaobao.com/news/china/story20230809-1422209}{台湾在野国民党总统参选人、新北市长侯友宜星期三(8月9日)批判执政的民进党``2025非核家园''目标不可能达成,主张将核电正式列入能源选项。 侯友宜承诺当选后,第一任期内将完成核一(第一核能发电厂)、核二、核三检查检修工作,安全延役,并邀请顶尖核能安全学者专家成立核四总体安全审查委员会,在安全无虞下,推动核四安全重启……}

\entryitemWithDescription{中国大陆再派大批军机舰台海警巡}{https://www.zaobao.com/news/china/story20230809-1422207}{台湾国防部星期三(8月9日)通报,中国大陆派出10架次军机越过台海中线,并配合五艘大陆军舰执行联合战备警巡,是本周内第二次大规模军机舰在台海周边活动。 通报说,星期三自早上9时起即陆续侦获大陆各型军机共计25架次出海活动,其中10架次逾越台海中线及延伸线进入台湾西南空域。 台国防部强调,台军运用联合情监侦手段绵密掌握,并检派任务军机军舰及岸置导弹系统,严密监控应处……}

\entryitemWithDescription{游轮交通接驳服务严重不足 分析指香港旅游业竞争力下降}{https://www.zaobao.com/news/china/story20230809-1422205}{启德邮轮码头位于九龙九龙城启德承丰道33号,即前启德机场跑道的末端。特区政府期望发展启德邮轮码头,可以帮助香港把握亚太区游轮旅游业市场增长所带来的机遇,将香港发展成为亚洲的游轮中心。(互联网) 近年香港致力打造成为亚洲区内游轮旅游中心,但有游轮上周重临香港时,却出现交通接驳服务严重不足问题,引起旅客强烈不满。有受访学者认为,事件反映香港旅游业竞争力下降,当局有必要检讨和改进……}

\entryitemWithDescription{【视频】33人在北京洪灾中死亡 部分受灾户未完成清理}{https://www.zaobao.com/news/china/story20230809-1422186}{北京市政府星期三(8月9日)发布过去一个多星期的防汛救灾情况,至少33人在洪灾中遇难。 据《联合早报》记者现场观察,北京灾情最严重的门头沟区市区已大致恢复原貌,但部分村庄里较严重的受灾户仍未完成后续清理工作……}

\entryitemWithDescription{中国多省跟进医药反腐 已有上市药企高层被查}{https://www.zaobao.com/news/china/story20230809-1422180}{中国医药领域反腐行动不断扩大,截止星期二(8月8日),已有八个省份跟进集中整治,一些上市药企的高层人员也被报立案调查。 据澎湃新闻报道,江苏、上海、北京、海南四省的卫生监管部门已针对医药领域采取反腐行动,西藏自治区、陕西、山西、山东四省纪委部门也就整治医疗领域发声……}

\entryitemWithDescription{杨丹旭:不能说的中国经济}{https://www.zaobao.com/news/china/story20230809-1421909}{前阵子到浙江休假,朋友聚会逃不过一个话题:经济。 疫情虽然过去了,中国经济没有大幅反弹,下行的压力却只见加大,连经济发达的浙江都在发愁如何能让GDP(地区生产总值)数字更好看一点。 一名在二线城市开发区工作的朋友在饭桌上吐苦水,眼下最大的烦恼是招不到商,地方上有GDP压力,层层下放指标给各个部门,但招商部门却没有拿得出手的优惠政策……}

\entryitemWithDescription{麻生太郎在台湾演讲:要有战争觉悟 也要发挥吓阻反制能力避战}{https://www.zaobao.com/news/china/story20230808-1421901}{日本自民党副总裁、前首相麻生太郎8月8日在台湾外交部与远景基金会合办的``凯达格兰论坛-2023印太安全对话'' 演讲时说,台海和平稳定对日本和世界各国都很重要,为避免战争,必须具备吓阻实力、执行吓阻决心。(路透社) 日本自民党副总裁、前首相麻生太郎说,台海和平稳定对日本和世界各国都很重要,为避免战争,必须具备吓阻实力、执行吓阻决心,并让对手了解自身不惜动武维护台海稳定安全的意志……}

\entryitemWithDescription{美军太平洋前副司令:美中关系正危险地渐行渐远}{https://www.zaobao.com/news/china/story20230808-1421891}{美军太平洋司令部前副司令立夫(Daniel Leaf)星期二(8月8日)认为,美中关系目前正危险地渐行渐远,两国都不能承担双边关系走偏所带来的后果。 立夫星期二在重庆举办的美国史迪威将军诞辰140周年纪念活动上致辞时表示,无论喜欢与否,美中两国都离不开彼此,不论是两国政府还是人与人交流层面的关系,都将对21世纪有着深远影响,两国承受不了双边关系的失败……}

\entryitemWithDescription{蓝营县议长促郭柯合 不挺郭台铭独立参选}{https://www.zaobao.com/news/china/story20230808-1421890}{鸿海集团创办人郭台铭星期二(8月8日)在台北诚品信义店发布父亲节新书,次女郭晓如(左一)和长女郭晓玲(右一)到场献花和拥抱感谢,令他和妻子曾馨莹(左二)感动拭泪。(温伟中摄) 为挺鸿海集团创办人郭台铭而退党的彰化县议长谢典林星期一(8月8日)公开反对郭台铭独立参选,期待他与民众党参选人柯文哲合作以提高胜算。但民众党明确排除``郭柯配'',郭台铭则没正面回应,只强调团结主流民意,让执政党下台……}