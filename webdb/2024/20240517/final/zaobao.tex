\entryitemWithDescription{韩咏红:中俄复杂关系下普京访华找活路}{https://www.zaobao.com/news/china/story20240517-3676473}{宣誓就职不到10天,俄罗斯总统普京就踏上了访问中国的旅程。这是普京第19次以总统身份访华,也是他开启新一个六年总统任期的外交首访,目的是在中俄建交75周年之际,深化两国``新时代全面战略协作伙伴关系''。 除了外长拉夫罗夫,普京此行还带了共六位副总理,以及财长、央行行长、银行与石油公司CEO、铝业大亨等……}

\entryitemWithDescription{中国预计周五宣布重磅楼市政策 乐观预期带动地产股大涨}{https://www.zaobao.com/news/china/story20240516-3676193}{中国多地近期加码推出宽松楼市政策,寄望加速消化存量住房,提振持续低迷的房地产市场。图为广西南宁的住宅楼。(彭博社) 市场对中国政府将收购存量住房的预期持续升温,官方预计最早在星期五(5月17日)宣布相关政策。受乐观情绪提振,陆港房地产股星期四全面大涨……}

\entryitemWithDescription{蓝绿专家:赖政府若展现善意 或可打开两岸新局}{https://www.zaobao.com/news/china/story20240516-3675297}{陆委会前主委夏立言(左)与海基会前董事长洪奇昌,星期四(5月16日)在台北出席研讨会,点评新政府与两岸关系新局面。(温伟中摄) 台湾陆委会前主委夏立言、海基会前董事长洪奇昌,星期四在研讨会乐观指出,只要候任总统赖清德对中国大陆展现善意,新政府或可打开两岸关系新局面。 夏立言和洪奇昌星期四(5月16日)在参加余纪忠文教基金会的``新政府与两岸关系新局面?''跨党派研讨会时,提出以上看法……}

\entryitemWithDescription{是否特赦陈水扁?台湾总统府:无此决定}{https://www.zaobao.com/news/china/story20240516-3676224}{台湾前总统陈水扁因贪污洗钱等案,2008年11月被羁押,2015年获准保外就医。(互联网) (台北综合讯)台湾政坛近期传出总统蔡英文将在5月20日卸任前特赦前总统陈水扁,台湾总统府星期四表示,实际上相关程序都必须依法进行,也无此决定。 综和《联合报》和《自由时报》报道,台湾总统府对于特赦陈水扁的议题,近期不断重申现阶段的立场,仍是希望确保陈水扁获得妥善的健康照顾,并依照相关的法律规范办理……}

\entryitemWithDescription{湖北官员为政策达标 私自登记村民为个体工商户}{https://www.zaobao.com/news/china/story20240516-3676002}{(北京综合讯)中国湖北省大悟县刘院村多位村民发现,在不知情情况下被注册工商营业执照,成为个体工商户,导致无法申请社会救助。据中国媒体报道,当地官员为达成政绩要求,私自替村民登记营业执照。 《中国青年报》旗下《冰点周刊》星期三(5月15日)报道,2011至2023年,在``市场主体增量行动''中,陆续有134名刘院村村民的个人信息被用于违规办理工商营业执照……}

\entryitemWithDescription{旅日中国教授据报涉间谍罪 被判六年有期徒刑}{https://www.zaobao.com/news/china/story20240516-3675263}{(北京综合讯)据日本媒体报道,旅日中国籍教授袁克勤暂时回国为母奔丧期间,因涉嫌间谍罪被拘留,已被判处六年有期徒刑。 袁克勤此前供职于日本北海道教育大学,2019年5月暂返中国后在吉林省长春市失联。中国外交部之后称,袁克勤已承认从事间谍活动并被拘留。 据日本共同社星期三(5月15日)报道,消息人士透露,袁克勤被判罪的罪名是违反《反间谍法》。吉林省长春市的中级人民法院(地方法院)今年1月底作出判决……}

\entryitemWithDescription{美国前国务卿蓬佩奥将出席赖清德就职典礼}{https://www.zaobao.com/news/china/story20240516-3674823}{美国前国务卿蓬佩奥宣布,他将以个人身份参加下星期一(5月20日)举行的台湾总统就职典礼。(路透社档案照) (华盛顿/台北综合讯)继美国跨党派代表团后,美国前国务卿蓬佩奥也宣布,他将以个人身份参加下星期一(5月20日)举行的台湾总统就职典礼。 曾在特朗普政府时期担任国务卿的蓬佩奥,星期三(5月15日)接受美国之音采访时说,他会以个人身份前往台北,参加台湾候任总统赖清德的就职典礼……}

\entryitemWithDescription{中国各大社交平台发公告 强调从严打击炫富拜金内容}{https://www.zaobao.com/news/china/story20240516-3674138}{(北京综合讯)中国各大社交平台在同一天发公告,强调要从严打击``炫富拜金''等不良价值导向内容,这意味着未来那些晒豪宅豪车、炫耀富二代身份的内容都可能被清理,相关账号也会被封禁。 综合封面新闻和财经网报道,腾讯、抖音、快手、微博、B站、小红书等平台星期三(5月15日)发布不良价值导向内容专项治理公告,针对近期网络上出现的奢靡浪费、炫富拜金等问题,从严打击,倡导理性、文明的消费观和价值观……}

\entryitemWithDescription{陈婧:迟来的刺激政策}{https://www.zaobao.com/news/china/story20240516-3670597}{``刺激政策终于找对了方向,力度也上来了。'' 最近访问的几名学者和经济师,不约而同发出这样的感慨。过去两周,中国各领域经济刺激政策如雨后春笋般层出不穷,力度也超出预期。 先是最受关注的房地产,两个一线城市北京和深圳在``五一''长假前后相继放宽限购政策,热点二线城市杭州和西安更全面解除限购,带动新一波楼市松绑潮。本周又有消息称,中央政府将要求地方出资收购存量住宅,帮助房企消化库存……}

\entryitemWithDescription{国台办宣布惩戒台湾五名嘴 学者:感觉有如拿明朝的剑斩清朝的官}{https://www.zaobao.com/news/china/story20240515-3669221}{民进党政策会执行长王义川曾说``大陆高铁没靠背'',引起反弹。(取自脸书) 台湾政论节目《关键时刻》主持人刘宝杰星期三(5月15日)主持节目时回应:``评论经济等同台独?大陆怕什么?''(视频截图) 台湾财经专家黄世聪在脸书回应时表示,他论及大陆的内容都来自媒体,并非凭空捏造,在台湾都是言论自由的范围……}

\entryitemWithDescription{中国如何反制美国加征关税 学者:可能以关税对关税}{https://www.zaobao.com/news/china/story20240515-3670468}{美国白宫星期二宣布,将对价值180亿美元(近244亿新元)的中国进口货品加征关税。图为摄于2021年1月21日在北京一家美国公司大楼外,飘扬着中美两国国旗。(路透社档案照) 针对美国宣布对中国电动车、锂电池和半导体等产品加征高额关税一事,中国外长王毅星期三(5月15日)直指这是``当今世界上最典型的霸道霸凌''。对于美国此举是否意味着中美关系倒退,中国外交部发言人在当天例行记者会上拒绝正面回应……}

\entryitemWithDescription{赖清德就职典礼在即 中国大陆军事活动越来越接近台湾}{https://www.zaobao.com/news/china/story20240515-3669582}{台湾候任总统赖清德星期三(5月15日)在台北出席台湾资安大会。(法新社) (台北/北京综合讯)台湾政府报告称,随着候任总统赖清德5月20日就职典礼的临近,中国大陆军方最近几周在台海周边活动的航行和飞行距离,比以往任何时候都要靠近台湾,一些军机还对进入台海的外国船只进行了模拟攻击……}

\entryitemWithDescription{中美举行首次人工智能政府间对话}{https://www.zaobao.com/news/china/story20240515-3669443}{(日内瓦综合讯)中国在中美人工智能政府间对话首次会议上,就美国在人工智能领域对华限制打压表明了严正立场。 据中国外交部北美大洋洲司官方微信公号``宽广太平洋''消息,会议于当地时间星期二(5月14日)在瑞士日内瓦举行,由中国外交部北美大洋洲司司长杨涛和美国国务院关键和新兴技术代理特使森特(Seth Center)、白官国安会技术和国家安全高级主任查布拉(Tarun Chhabra)共同主持……}

\entryitemWithDescription{YouTube在香港屏蔽《愿荣光》等视频 学者料其他网络供应商会采取类似做法}{https://www.zaobao.com/news/china/story20240515-3669802}{全球最大的搜索引擎谷歌母公司Alphabet,旗下影片分享平台YouTube是全球最大的影音平台。(路透社档案照) 谷歌母公司Alphabet旗下影片分享平台YouTube,由周三(5月15日)起,限制在香港境内浏览32条与歌曲《愿荣光归香港》有关的被禁制片段连结,以回应香港高等法院日前对《愿荣光归香港》颁布临时禁制令……}

\entryitemWithDescription{中国驻英大使:英方不要在危险道路上越走越远}{https://www.zaobao.com/news/china/story20240515-3668833}{(伦敦综合讯)香港驻伦敦经贸办行政经理被控违反英国《国家安全法》后,伦敦和北京摩擦不断。中国驻英国大使郑泽光警告英国,不要在破坏中英关系的危险道路上越走越远。 路透社此前报道,英国警方星期一(5月13日)起诉三名男子,指他们涉嫌协助香港情报部门和参与外国干预,违反英国《国家安全法》。香港驻伦敦经贸办行政经理袁松彪是被告之一……}

\entryitemWithDescription{杨丹旭:中国迎来涨价周期?}{https://www.zaobao.com/news/china/story20240515-3664575}{第二季度以来,中国经济出现了一些回暖迹象,一个重要的衡量经济热度的指标------消费者价格指数(CPI)连续三个月上涨。中国国家统计局最新数据显示,4月份CPI同比上涨0.3%,涨幅较前值略有扩大,也高于预期。 去年初中国解除疫情防控,与很多国家在走出疫情后经济迅速反弹,通货膨胀困扰执政者,甚至变成政治压力不同,中国经济反而陷入低迷期。消费需求疲弱、CPI上涨乏力,倒成了一件让官方很头痛的事……}

\entryitemWithDescription{香港驻伦敦经贸办行政经理被控违反英国《国家安全法》 英中外交部掀骂战}{https://www.zaobao.com/news/china/story20240515-3664207}{香港驻伦敦经贸办行政经理袁松彪5月13日被控协助香港情报部门后,离开伦敦威斯敏斯特地方法院。(路透社) 三名被告之一的Chi Leung Wai 5月13日被控协助香港情报部门后,离开伦敦威斯敏斯特地方法院。(路透社) 伦敦和北京间的摩擦升级,继中国外交部表示``严重关切''香港驻伦敦经贸办行政经理等三人被英国起诉协助香港情报组织后,英国外交部召见中国驻英大使,称有关间谍活动``不可接受''……}

\entryitemWithDescription{路透社爆美台海军4月曾``巧遇''军演}{https://www.zaobao.com/news/china/story20240514-3664315}{美国与台湾海军据报4月在太平洋进行不对外公开、``在正式名义下不存在''的联合军演。图为台湾海军1月31日在高雄军事基地附近的水域进行演习。(路透社) (台北 / 北京综合讯)距离台湾总统就职典礼倒数五天,路透社星期二(5月14日)引述四名知情人士报道,美国与台湾海军4月在太平洋进行不对外公开、``在正式名义下不存在''的联合军演……}

\entryitemWithDescription{青海海西州中院疑``垂帘听审''被批}{https://www.zaobao.com/news/china/story20240514-3664139}{青海海西州中级法院上周六(5月11日)疑似通过微信群组实时向下级法院传达指令,遥控指挥庭审。(互联网) 中国青海一县级法院近日公开审理涉及多名被告的寻衅滋事案,辩方律师当庭拍到上级法院领导通过微信实时遥控指挥庭审。事件曝光后引发中国舆论高度关注,有网民批评上级法院是在``垂帘听审'',也有律师批评事件严重破坏中国刑事诉讼制度,质疑两审终审制可能形同虚设,被现场揭发则实属罕见……}

\entryitemWithDescription{陈水扁特赦争议 台法务部报告:未判案件不在特赦范围内}{https://www.zaobao.com/news/china/story20240514-3663977}{台湾舆论近来围绕涉贪的前总统陈水扁是否该获得特赦的争议不断。(互联网) (台北综合讯)台湾舆论近来围绕涉贪的前总统陈水扁是否该获得特赦的争议不断。台湾法务部报告说明,未判案件不在特赦范围内,也意味着若陈水扁获得特赦,他所涉尚未审结的四起案件均无法特赦,未没收款项也不在特赦范围内。 《镜周刊》上星期二(5月7日)报道,总统蔡英文决定于5月20日卸任前特赦陈水扁……}

\entryitemWithDescription{中菲合作遣返160余名在菲律宾从事离岸博彩的中国公民}{https://www.zaobao.com/news/china/story20240514-3662955}{(马尼拉综合讯)160余名在菲律宾从事离岸博彩的中国公民,星期二被中菲两国执法部门合作遣返。 根据中国驻菲律宾大使馆星期二(5月14日)通报,这是中菲两国执法部门再次合作。去年12月和今年2月,中菲执法部门已先后联手遣返180名和40余名在菲律宾从事离岸博彩的中国公民……}