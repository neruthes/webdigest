\entryitemWithDescription{杨丹旭:中国冲击论再现}{https://www.zaobao.com/news/china/story20240605-3800958}{两位美国经济学家最近先后在不同场合,谈到拜登政府对华关税政策,两人的意见相左。 美国耶鲁大学讲师、摩根士丹利亚洲区前主席罗奇上周在北京参加一个活动时直言,中国在生产非碳替代能源产品上有比较优势,而受气候变化影响的世界迫切需要这些产品,对中国采取保护主义立场,``可能是历史性错误''。他担心,``拜登正在进一步深陷一场对华贸易的新`永久战争'\,''……}

\entryitemWithDescription{汪文斌候任中国驻柬埔寨大使}{https://www.zaobao.com/news/china/story20240605-3800986}{53岁的中国外交部新闻司副司长、发言人汪文斌,已等候出任中国驻柬埔寨大使。(互联网) (北京综合讯)中国外交部新闻司副司长、发言人汪文斌,已等候出任中国驻柬埔寨大使。 据澎湃新闻星期二引述中国国家国际发展合作署微信公众号消息,国合署署长罗照辉星期二(6月4日)会见候任中国驻柬埔寨大使汪文斌。上述消息显示,汪文斌已候任中国驻柬埔寨大使……}

\entryitemWithDescription{科技巨头黄仁勋苏姿丰现身台北国际电脑展 分析:固桩供应链}{https://www.zaobao.com/news/china/story20240604-3799612}{英伟达执行长黄仁勋星期二(6月4日)下午在台北国际电脑展展场举行全球媒体记者会,进一步证实英伟达将在台湾设立第二个类似在高雄Taipei-1的AI超级电脑中心。(彭博社) 超微半导体(AMD)首席执行长苏姿丰(前右一)6月4日下午到台北国际电脑展参观,所到之处都被厂商与媒体团团包围,现场人潮水泄不通……}

\entryitemWithDescription{达赖喇嘛本月赴美接受膝盖治疗}{https://www.zaobao.com/news/china/story20240604-3799647}{西藏精神领袖达赖喇嘛将在本月内前往美国治疗膝盖。图为达赖喇嘛今年2月24日在达兰萨拉寺庙举行的公众集会上发表讲话。(法新社) (新德里综合讯)西藏精神领袖达赖喇嘛本月将前往美国治疗膝盖,并将从6月20日起暂停例行公开活动。 路透社星期一(6月3日)报道,达赖喇嘛办公室在一份声明中宣布,他将前往美国接受膝盖治疗的计划,并称他将在返回后恢复正常活动……}

\entryitemWithDescription{赖清德:纪念``六四''因全球有共同信念}{https://www.zaobao.com/news/china/story20240604-3800302}{(台北讯)台湾总统赖清德星期二在社交平台发文纪念``六四''事件,指出纪念``六四''因为全球有共同信念,即唯有民主自由,才能真正保护人民。 星期二(6月4日)是1989年天安门事件35周年,赖清德在脸书写道:``35年前的今天,民主化浪潮席卷全球,亚洲各地的青年学子站出来,追求民主自由、呼吁国家改变。'' 他指出,一个真正令人尊敬的国家,是可以让人民大声说话……}

\entryitemWithDescription{香港启动``好客之道'' 倡导礼貌接待旅客}{https://www.zaobao.com/news/china/story20240604-3800115}{香港文化体育及旅游局局长杨润雄星期一(6月3日)宣布开启``好客之道''运动。图为三名游客4月29日在香港尖沙咀的海滨大道合照。(路透社) (香港综合讯)香港特区政府星期一在全港启动``好客之道''运动,倡导社会各阶层礼貌对待访港旅客。 综合香港政府新闻公报和政府新闻网等消息,香港文化体育及旅游局局长杨润雄星期一(6月3日)宣布开启``好客之道''运动,覆盖全港18区的学校、大小商户和各行业等……}

\entryitemWithDescription{香港连续五年没举办``六四''纪念活动}{https://www.zaobao.com/news/china/story20240604-3800118}{香港警察星期二(6月4日)在港岛铜锣湾靠近维多利亚公园的大街上,带走一名高呼口号的女子。(法新社) 多年来一直举办全球最大规模``六四''纪念晚会的香港,星期二(6月4日)在天安门事件35周年纪念日这一天,连续第五年没有举办相关活动。 当天,港岛铜锣湾维多利亚公园附近的警力明显增加,许多制服及便衣警员分散驻守,也有多辆警车停泊,反恐特勤队高调巡逻……}

\entryitemWithDescription{嫦娥六号完成月球背面采样返回地球 创世界首例}{https://www.zaobao.com/news/china/story20240604-3799395}{中国国家航天局星期二(6月4日)发布嫦娥六号着陆器着陆月球背面拍摄的系列影像图。(新华社) (北京综合讯)中国飞行探测器``嫦娥六号''成功完成从月球背面采样,并起飞启程返回地球。此举为世界首创,也是中国太空计划的一项重大成就。 中国国家航天局在官网发布消息说,星期二(6月4日)上午7时38分,嫦娥六号上升器携带月球样品自月球背面起飞,成功将上升器送入预定环月轨道……}

\entryitemWithDescription{戴庆成:澳门博彩业危与机并存}{https://www.zaobao.com/news/china/story20240604-3781652}{香港与澳门只有一海之隔,两地关系密切。在以前,香港由于在多方面给予澳门大力支持,一直被称为兄长。不过近年澳门这个小弟弟凭着自强不息,在部分领域已开始超越香港,例如过去一年发展``演唱会经济''有声有色,就令香港相形见绌。 直属中国国家文旅部的中国旅游研究院上星期公布的调查也显示,今年首季澳门首次成为中国大陆旅客最满意的目的地,香港则跌至第七名。出现这个结果并不令人感到意外……}

\entryitemWithDescription{巴基斯坦总理访华 中巴将重振经济走廊项目}{https://www.zaobao.com/news/china/story20240604-3782018}{巴基斯坦总理夏巴兹星期二(6月4日)起访问中国。(路透社) 巴基斯坦总理夏巴兹星期二(6月4日)起访问中国,中巴预计将重振因疫情、巴基斯坦国内经济困难等原因而停滞不前的中巴经济走廊项目。 中国《经济日报》上星期五(5月31日)报道,中国和巴基斯坦近期在中巴经济走廊联合合作委员会会议上,确定了该项目第二阶段的五条新走廊,包括增长走廊、民生走廊、创新走廊、绿色走廊和开放走廊……}

\entryitemWithDescription{中华电信证实民进党立委和外部单位曾要求加强手机讯号}{https://www.zaobao.com/news/china/story20240603-3781754}{台湾执政的民进党政策会执行长王义川,一席以手机讯号定位分析集会群众的谈话,引发``国家机器''监控人民疑云。台湾政府相关单位连日来极力撇清绝无此事,并称三大电信业者并未提供任何资料给王义川。 民进党先前强调,5月下旬立法院会审查立法院改革法案时,院外的青鸟行动都是自发性的,并非民进党策动……}

\entryitemWithDescription{传俄中天然气管道谈判陷僵局 俄罗斯称谈判仍在持续}{https://www.zaobao.com/news/china/story20240603-3781496}{(莫斯科综合讯)有消息指中俄就``西伯利亚力量二号''天然气管道协议的谈判陷入僵局,俄罗斯澄清称谈判仍在持续,各方维护自己的利益完全正常。 路透社星期一(6月3日)报道,克里姆林宫发言人佩斯科夫星期一告诉记者,俄罗斯将与中国就西伯利亚力量二号天然气管道达成协议,有关谈判仍在持续,但商业讯息不会公开。 另据俄罗斯卫星通讯社报道,佩斯科夫称,各方维护各自的利益是完全正常的……}

\entryitemWithDescription{李稻葵:中国应发行更多国债}{https://www.zaobao.com/news/china/story20240603-3781367}{(北京彭博电)中国著名经济学家李稻葵说,中国应发行更多中央政府债券,因为地方政府资金短缺,无力投入资金和推动经济增长。 李稻葵星期一(6月3日)接受彭博电视采访时说,中央政府债务与中国国内生产总值(GDP)的比率目前约为20\%,这个比率应是现有的一倍以上。 中国从去年开始表明愿意加大中央政府发债的力度……}

\entryitemWithDescription{中国今年国内游支出近1万亿美元 将超疫情前水平}{https://www.zaobao.com/news/china/story20240603-3781232}{今年中国``五一''劳动节假期的第一天,众多游客爬上北京市郊的八达岭长城。 (法新社) (北京综合讯)中国旅游业强劲复苏,国内游对中国大陆经济的贡献预计达到9380亿美元(1.29万亿新元),将首次超过冠病疫情前的水平。 据彭博社星期一(6月3日)消息,世界旅游及旅行理事会(WTTC)和牛津经济研究院发布的报告显示,今年中国国内旅游支出预计将比2019年高出11\%……}

\entryitemWithDescription{中国外交部:南中国海问题升温责任全在菲律宾}{https://www.zaobao.com/news/china/story20240603-3777174}{中国外交部星期一(6月3日)强调近期南中国海问题升温,责任完全在菲律宾。图为5月16日``巴加凯''号上的菲律宾海岸警卫队人员在有争议的南中国海观察一艘中国海警船。(法新社) (北京综合讯)针对菲律宾总统小马可斯近日在香格里拉对话会上阐述对南中国海的主权声索,中国外交部星期一指菲方有关表态罔顾历史和事实,歪曲渲染海上事态,并强调近期南中国海问题升温,责任完全在菲律宾……}

\entryitemWithDescription{于泽远:中国为何不参加乌克兰和平会议?}{https://www.zaobao.com/news/china/story20240603-3766440}{中国明确表示不参加6月15日由瑞士主办的乌克兰和平会议,这一决定难免会让乌克兰和西方国家失望。 中国外交部发言人毛宁5月31日对中国立场进行了解释。她说,中方坚持国际和会应当具备俄乌双方认可、各方平等参与、对所有和平方案进行公平讨论这三个重要的要素,否则难以为恢复和平发挥实质作用……}

\entryitemWithDescription{俄罗斯使馆回应泽连斯基 指和平会议是个``笑话''}{https://www.zaobao.com/news/china/story20240602-3767519}{俄罗斯驻新加坡大使馆星期天下午发脸书贴文,指泽连斯基参加香格里拉对话会让俄罗斯感到愤慨,并称所谓``和平会议''是个笑话。俄罗斯指泽连斯基以参加香会作为最后手段,试图说服全球南方参加``滑稽的会议'',意图制造并向俄罗斯发出最后通牒。 俄使馆又指香会主办方国际战略研究所(IISS)正让一个著名的论坛成为反俄言论平台,以及迫使东南亚国家支持北约和乌克兰的工具……}

\entryitemWithDescription{批民进党政府和外部势力 中国防长董军:台湾正被引向险境}{https://www.zaobao.com/news/china/story20240602-3767195}{中国大陆防长董军(台上致辞者)星期天(6月2日)在香格里拉对话上发表演讲,抨击民进党政府大搞渐进式台独。(法新社) 中国大陆国防部长董军在香格里拉对话上大篇幅阐述北京在台湾问题上的立场,批评民进党政府大搞渐进式台独 、外部干涉势力以``切香肠''方式虚化掏空``一中原则'',把台湾引向险境,导致两岸和平统一的前景被破坏……}

\entryitemWithDescription{【早知】探索月球背面有什么意义?}{https://www.zaobao.com/news/china/story20240602-3767525}{(早报图表) 月球是距离地球最近的自然天体,但人们在地球上无法看到月球背面,如今迅速发展的航天技术正在为人类解开月球背面神秘的面纱。 ● 为什么人们看不到月球背面? 数十亿年来,地球引力持续对月球施加了潮汐力的作用,导致月球自转速度越来越慢,自转周期越来越长。最终,月球自转周期变得与月球绕地球的公转周期相同,均为27.32天……}

\entryitemWithDescription{董军批菲律宾在南中国海``碰瓷''}{https://www.zaobao.com/news/china/story20240602-3767242}{中国防长董军星期天(6月2日)在香格里拉对话上批评菲律宾在南中国海``碰瓷''。(海峡时报) 中国防长董军星期天(6月2日)在香格里拉对话上批评菲律宾在南中国海``碰瓷'',并警告称中国对侵权挑衅行径的克制有限……}

\entryitemWithDescription{中国官媒:菲律宾坐滩仁爱礁军舰人员持枪在甲板活动}{https://www.zaobao.com/news/china/story20240602-3766938}{(北京综合讯)中国官媒称,菲律宾坐滩仁爱礁军舰上的人员上个月被发现持枪在甲板上活动。 中国环球电视网(CGTN)旗下``CGTN记者团''微博账号星期天(6月2日)发布一段29秒的视频,称菲律宾5月19日对坐滩仁爱礁(菲律宾称阿云津礁)的57号军舰进行空投补给,中国海警应对时发现,舰上至少有两名人员持枪在甲板活动,并用手、用枪指着中国海警方向……}

\entryitemWithDescription{中国嫦娥六号将从月球背面采样 创人类纪录}{https://www.zaobao.com/news/china/story20240602-3767422}{``嫦娥六号''在月球背面着陆当天,北京航天飞行控制中心的工作人员监测这架探测器的着陆器和上升器组合体工作情况。(新华社) 中国飞行探测器``嫦娥六号''星期天(6月2日)成功登陆月球背面,将实施人类首次从月背采集样本并返回的任务。一旦成功,这次行动将扩大中国在国际太空竞赛中的优势,朝着载人登月目标更近一步。 中国国家航天局星期天发布嫦娥六号在月背着陆的影像……}