\entryitemWithDescription{王纬温:中澳关系已完全雨过天晴?}{https://www.zaobao.com/news/china/story20231109-1448814}{中国和澳大利亚从2018年至2022年交恶长达四年后,去年5月在澳大利亚工党重新掌权迎来转机。经两国高层18个月的修补,实现澳洲总理时隔七年再度访华。 中澳总理星期二(11月8日)在北京重启暂停了四年的年度会晤,全面恢复两国战略对话及其他政府间机制。中国总理李强还称澳洲总理阿尔巴尼斯是``老朋友''\,``帅哥'',调整对上届澳洲政府的战狼外交姿态,两国关系表面上呈现``雨过天晴''的局面……}

\entryitemWithDescription{李显龙总理会见韩正 强调希望中美保持沟通建立互信}{https://www.zaobao.com/news/china/story20231108-1448823}{尚达曼总统星期三(11月8日)在总统府会见到访我国的中国国家副主席韩正,李显龙总理同日设午宴款待韩正。李总理在与韩正会面时重申了牢固的新中双边关系,并强调新加坡希望中美能保持沟通、建立互信,给不确定的全球环境带来稳定。 韩正星期二起对我国进行两天正式访问。根据我国外交部文告,尚达曼与韩正在会面时表示,乐见新加坡与中国基于多方面合作的密切和长久关系……}

\entryitemWithDescription{平安集团否认官方要求其收购碧桂园}{https://www.zaobao.com/news/china/story20231108-1448795}{碧桂园上月错过了支付1500万美元(2035万新元)息票的最后期限,市场认为,其总计约110亿美元的离岸债券面临违约的风险。(路透社) (北京/广州综合讯)消息人士称,中国政府要求保险业巨头中国平安保险集团控股陷入债务困境的全国最大房企碧桂园。中国平安星期三(11月8日)回应称,相关报道完全与事实不符。 路透社星期三引述四名消息人士作出上述报道……}

\entryitemWithDescription{韩正:中美互动释放积极信号 中国愿同美国加强沟通}{https://www.zaobao.com/news/china/story20231108-1448791}{中国国家副主席韩正星期三(11月8日)在新加坡出席彭博创新经济论坛时说,中国愿与美国加强各层次沟通对话。(梁麒麟摄) 中国国家副主席韩正说,中美近期开展的重要高层次互动,释放了积极信号,也提升了国际社会对中美关系改善的正面预期。 他表明,中国愿同美国加强各层次沟通对话,推进互利合作,妥善管控分歧。 正在新加坡访问的韩正星期三(11月8日)出席第六届彭博创新经济论坛并致辞……}

\entryitemWithDescription{中国公布甲烷减排计划 并未列减少排放目标}{https://www.zaobao.com/news/china/story20231108-1448778}{中国官方星期二(11月7日)公布甲烷减排计划,却并未列出减少排放的具体目标。图为中国山西省晋城市煤层气生产基地。(路透社) (北京综合讯)在中美气候变化会谈落幕当天,中国官方公布甲烷减排计划,提出将提高监测监管、统计核算的能力,但未列出减少排放的具体目标……}

\entryitemWithDescription{台湾蓝白合陷僵局 民众党抛三民调显示柯文哲赢侯友宜}{https://www.zaobao.com/news/china/story20231108-1448755}{(台北讯)台湾在野阵营蓝白合陷入僵局,民众党星期三(11月8日)公布三家民调结果,均显示该党主席兼总统参选人柯文哲的支持度领先。国民党总统参选人侯友宜竞选办公室批评,民众党自费委托执行的民调``误导视听'', 显见对蓝白合作缺乏诚意与善意。 据《联合报》报道,民众党公布的三家民调结果中,柯文哲的支持度不仅超出侯友宜8个百分点,也胜过台湾副总统、民进党总统参选人赖清德……}

\entryitemWithDescription{黄之锋曾求美领馆庇护被拒 美记者新书披露细节}{https://www.zaobao.com/news/china/story20231108-1448750}{两名美国记者在新书中透露,前香港众志秘书长黄之锋在《香港国安法》通过前后曾请求美国驻港澳总领事馆的庇护,但未获批。(互联网) (香港讯)两名美国记者在新书中披露,前香港众志秘书长黄之锋在《香港国安法》通过前后,曾寻求进入美国驻港澳总领事馆或离港赴美,以寻求政治庇护,但未获允许……}

\entryitemWithDescription{美学者:中国提三大全球倡议 旨在取代西方体系}{https://www.zaobao.com/news/china/story20231108-1448744}{亚洲协会政策研究所中国分析中心执行主任季北慈(屏幕发言者)11月8日指出,中国自2021年起陆续提出全球发展、安全、文明三大倡议,旨在取代西方带领的价值体系,并争取全球南方国家认同。(缪宗翰摄) 专研中国问题的美国学者季北慈(Bates Gill)星期三(11月8日)在台北出席论坛时指出,中国自2021年起陆续提出三大全球倡议,旨在争取全球南方国家认同,取代西方带领的价值体系……}

\entryitemWithDescription{英特尔明年将推出二纳米制程晶片电脑处理器}{https://www.zaobao.com/news/china/story20231107-1448591}{美国半导体龙头企业英特尔(Intel)总裁基辛格(Pat Gelsinger)星期二(11月7日)在台北透露,英特尔将在明年上半年推出20埃米(即二纳米)制程晶片的电脑处理器,不过据评估仍无法撼动台积电的技术领先地位。 英特尔在台北举行创新科技论坛,基辛格发表主题演讲时指出,人工智能个人电脑(AI PC)将成电脑产业转捩点……}

\entryitemWithDescription{李强:愿同澳洲进一步加强对话沟通}{https://www.zaobao.com/news/china/story20231107-1448584}{中国国务院总理李强星期二(11月7日)在北京会见澳大利亚总理阿尔巴尼斯,标志着中澳重启已暂停四年的两国总理年度会晤。李强表示中国愿同澳洲进一步加强对话沟通,深化政治互信,妥善处理分歧;阿尔巴尼斯则强调,中澳必须通过对话和谅解谨慎管理地缘战略竞争,并呼吁北京完全撤销对澳洲输华商品的限制。 阿尔巴尼斯在正式访问中国的最后一天,于北京人民大会堂与李强举行会谈……}

\entryitemWithDescription{美报告指``一带一路''债权逾万亿美元 中国外交部驳斥}{https://www.zaobao.com/news/china/story20231107-1448578}{(北京综合讯)美报告称中国通过``一带一路''项目拥有超过一万亿美元的债权,其对外贷款重心正在转移,中国外交部予以驳斥。 综合路透社和法新社报道,美国研究机构``援助数据''(AidData)星期一(11月6日)发表报告称,中国金融机构自2000年到2021年向发展中国家提供了1.34万亿美元(约1.82万亿新元)贷款,成为``世界上最大的官方收债者'',其中约80\%贷款流向了陷入财困的国家……}

\entryitemWithDescription{中国将加强管理重点资源产品 稀土等进出口须报告}{https://www.zaobao.com/news/china/story20231107-1448559}{(北京综合讯)中国商务部宣布,将加强管理稀土、铁矿石、铜精矿、原油等重点资源的进出口。 中国商务部星期二(11月7日)在官网发布《大宗产品进出口报告统计调查制度》的通知,将原油、铁矿石、铜精矿、钾肥纳入《实行进口报告的能源资源产品目录》,将稀土纳入《实行出口报告的能源资源产品目录》。 根据通知,对外贸易经营者在进口、出口上述产品时,应履行有关进出口信息报告义务,制度自10月31日起执行,为期两年……}

\entryitemWithDescription{黑龙江体育馆雪夜坍塌 网民质疑事因或非一日之寒}{https://www.zaobao.com/news/china/story20231107-1448548}{(佳木斯综合讯)黑龙江省佳木斯市桦南县一体育馆发生坍塌事故,造成三死一伤,营运体育馆的健身俱乐部负责人已被警方控制。 综合新华社和央视新闻报道,该体育馆是在星期一(11月6日)晚上7时许坍塌,搜救于星期二(11月7日)凌晨结束。事发时有七名初中生在馆内打篮球,其中三人自行脱险,一人轻伤,三人遇难。佳木斯市、桦南县两级纪检监察机关已在第一时间介入,成立调查组开展调查……}

\entryitemWithDescription{居住难题成台湾选战热点 刘太格提议建示范社区赢选票}{https://www.zaobao.com/news/china/story20231107-1448543}{85岁的新加坡规划大师刘太格11月3日在台北出席大稻埕170周年老街文化节国际论坛,会前接受《联合早报》专访。(商协Socium提供) 买房太难、老屋太多成了台湾选战热点,新加坡城市规划大师刘太格在台北接受《联合早报》专访时建议明年上任的台湾新政府,不妨先打造一个示范社区,一旦达标将有望赢得选民支持,突破八年政党轮替魔咒……}

\entryitemWithDescription{【东谈西论】中国进博会是一场政治秀?}{https://www.zaobao.com/news/china/story20231107-1448466}{第六届中国国际进口博览会11月5日在上海开幕。本届进博会吸引了128个国家和地区的3400多家企业参展。(中新社) 中国国际进口博览会11月5日在上海开幕,这是中国第六次举办以进口为主题的国家级大型博览会。 不过,就在进博会开幕前,中国欧盟商会公开批评,进博会已变成一场``雾里看花的政治秀''。 进博会到底是一个``中国市场买全球''的平台,还是一个``政治作秀的舞台''……}

\entryitemWithDescription{被赖清德指是共产党员 徐春莺否认反呛要证据}{https://www.zaobao.com/news/china/story20231106-1448355}{徐春莺在记者会上出示早已取得的``中华民国''护照,强调自己是``中华民国''国民。(自由时报) 台湾在野民众党拟纳中国大陆籍配偶徐春莺为不分区立委,成为热议话题。执政的民进党总统参选人赖清德公开指控她是共产党员。徐春莺否认自己曾加入共产党,反呛``政治人物讲话要有诚信,请问赖副总统有证据吗?'' 国安局长蔡明彦星期一(11月6日)表示,台湾政府对陆配的原则和立场很清楚,只要合法活动都予以尊重……}

\entryitemWithDescription{升级版新中自由贸易协定预计今年底签署}{https://www.zaobao.com/news/china/story20231106-1448348}{贸工部兼文化、社区及青年部政务部长刘燕玲(左)在上海出席新中经贸与投资论坛时,与新加坡工商联合总会执行总裁郭柄汛对话。(陈婧摄) 新加坡贸工部兼文化、社区及青年部政务部长刘燕玲披露,升级版新中自由贸易协定预计在今年底签署。 刘燕玲星期一(11月6日)出席新加坡工商联合总会在上海举办的新中经贸与投资论坛……}

\entryitemWithDescription{丁薛祥吁反对知识封锁及扩大科技鸿沟 学者:不点名批评美国}{https://www.zaobao.com/news/china/story20231106-1448342}{中国国务院副总理丁薛祥星期一(11月6日)在重庆出席首届``一带一路''科技交流大会开幕式。(王纬温摄) 中国国务院副总理丁薛祥星期一(11月6日)在重庆呼吁``一带一路''国家,共同反对知识封锁和人为扩大科技鸿沟,共同完善全球科技治理。 受访学者分析,西方近期在芯片和人工智能等高科技领域加码施压中国,北京高层最新表态明显是在批评美国,并希望通过一带一路施展外交应对西方科技围堵……}

\entryitemWithDescription{美国运通传在港裁员数十人}{https://www.zaobao.com/news/china/story20231106-1448338}{(香港讯)信用卡公司美国运通据报近期拟裁减香港数十个职位。 香港《信报》星期一(11月6日)引述匿名消息人士报道,受香港市场前景不确定性影响,以高端客户定位的美国运通(American Express)近期启动结构调整,计划精简涉及运营、市场推广及信贷风险监控等关键部门……}

\entryitemWithDescription{台春节前公布组团赴陆细节 包机旅游航点可望变直航}{https://www.zaobao.com/news/china/story20231106-1448336}{一架台湾中华航空公司的货机11月5日飞过德国法兰克福的球场。(法新社) (台北综合讯)台湾交通部长王国材说,台湾将在明年春节之前,公布解除赴大陆旅行团禁令的细节,若两岸包机点运量稳定,将考虑变成直航点……}