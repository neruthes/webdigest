\entryitemWithDescription{中国遇今年最强台风 ``杜苏芮''登陆福建}{https://www.zaobao.com/news/china/story20230728-1418514}{(小)台风``杜苏芮''在福建晋江沿海登陆后,福建省厦门市出现特大暴雨。图为一名警察走在被大雨淹没的街道上,水位高至大腿处。(法新社) 今年第五号台风``杜苏芮''星期五(7月28日)登陆中国福建晋江沿海,是今年以来登陆中国的最强台风,也是登陆福建第二强的台风……}

\entryitemWithDescription{知情人士:中国望金砖扩大阵容 遭巴西印度反对}{https://www.zaobao.com/news/china/story20230728-1418489}{知情官员透露,中国希望金砖国家组织快速扩大,通过壮大自身政治影响力来对抗美国,但遭到成员国印度和巴西的反对。 据彭博社星期五(7月28日)报道,金砖五国------巴西、俄罗斯、印度、中国和南非下个月将在约翰内斯堡召开金砖国家组织领导人峰会,讨论扩大金砖、接纳印度尼西亚和沙特阿拉伯的议题。 据知情官员称,中国在领导人峰会预备会中一再游说金砖扩大阵容,但印度和巴西提出反对意见……}

\entryitemWithDescription{传美国拟禁港特首出席APEC峰会 北京强烈不满提出严正交涉}{https://www.zaobao.com/news/china/story20230728-1418478}{香港特首李家超:任何国家或经济体主办APEC时,都应按指引、习惯及规则,邀请成员经济体领导人参与。(法新社) 《华盛顿邮报》引述消息指白宫决定不让香港特首李家超出席11月在旧金山举行的亚太经济合作组织(APEC)领导人非正式会议。中国表示强烈不满和坚决反对,并称或将对美实施制裁。 受访学者说,将李家超列入APEC的黑名单,显示中美关系基本上并没有缓和,事件将加剧华盛顿和北京之间的紧张……}

\entryitemWithDescription{香港高院驳回港府申请 拒禁《愿荣光归香港》}{https://www.zaobao.com/news/china/story20230728-1418466}{香港高等法院驳回了港府针对反修例运动歌曲《愿荣光归香港》提出的禁制令申请,称这可能会损害言论自由并造成潜在的``寒蝉效应''。 综合路透社、香港01报道,在多项国际体育赛事播出香港得奖选手的国歌时,误播了《愿荣光归香港》而不是中国国歌《义勇军进行曲》后,港府向法院申请禁止以任何方式传播《愿荣光归香港》……}

\entryitemWithDescription{美国六众议员致函副总统 吁亲会赖清德以示支持台湾}{https://www.zaobao.com/news/china/story20230728-1418444}{美国六名众议员致函副总统哈里斯,建议她在台湾副总统赖清德8月出访巴拉圭过境美国期间,亲自会晤赖清德,以展现对台湾的支持。 综合美国之音和台湾《中国时报》报道,共和党籍众议员蒂凡尼(Tom Tiffany)当地时间7月26日(星期三)领衔写信给哈里斯,呼吁她在赖清德过境美国时安排会晤行程……}

\entryitemWithDescription{美众院通过《台湾国际团结法》 学者:意在助台开拓国际空间 北京料将施压}{https://www.zaobao.com/news/china/story20230727-1418154}{美国众议院星期二(7月25日)通过《台湾国际团结法》,主张联合国第2758号决议没涉及任何有关台湾主权的声明。受访学者分析,此法案意在帮助台湾开拓国际空间,但未必能巩固``台湾地位未定论'',预料也将面对北京反弹和施压……}

\entryitemWithDescription{``杜苏芮''来袭 福建广东多地停课停工}{https://www.zaobao.com/news/china/story20230727-1418145}{台风``杜苏芮''逼近福建,漳浦县园林工人星期四(7月27日)赶紧加固大型苗木,以防给台风吹倒。(新华社) 台风``杜苏芮''即将登陆,福建福州市长乐区沿海一带星期四(7月27日)风力逐渐加强,海上波浪滚滚。(新华社) (北京/台北综合讯)台风``杜苏芮''袭击台湾造成一人死亡后,继续向西北方向移动,预计星期五(7月28日)登陆中国大陆,并将影响十余个省份……}

\entryitemWithDescription{调查:各国对中国的负面观感持续居高不下}{https://www.zaobao.com/news/china/story20230727-1418140}{美国智库皮尤研究中心(Pew Research Center)的最新民调显示,接受调查的24个国家,普遍上对中国的外交政策持负面评价,也很少国家相信中国在国际事务上会做正确的事情。 多数接受调查的国家认为中国在国际舞台上,是干涉主义作风。57\%的人认为中国大量或相当程度地干涉其他国家的事务。美国、加拿大、澳大利亚、西班牙、日本和韩国的受访者中,将近七成认为中国干涉别国事务……}

\entryitemWithDescription{中国反对恶意炒作秦刚被免职一事}{https://www.zaobao.com/news/china/story20230727-1418131}{中国官方至今未说明前外长秦刚被免职的原因。图为秦刚今年5月23日在北京与荷兰外长胡克斯特拉会面后出席记者会。(路透社档案照) 中国外交部持续回避回应关于前外长秦刚神隐的提问,同时也表明反对恶意炒作秦刚被免除外长职务一事……}

\entryitemWithDescription{除了软对抗 政协委员:香港还有远对抗}{https://www.zaobao.com/news/china/story20230727-1418125}{香港警方国安处星期四(7月27日)拘捕了一对涉嫌资助流亡民主活动人士罗冠聪的男女。港区全国政协委员张志刚说,大批反对派分子移居外地后``远对抗'',香港必须认真面对,但不能依靠平息``黑暴''的手段来应对……}

\entryitemWithDescription{被控``台谍'' 台商李孟居结束附加刑离境大陆}{https://www.zaobao.com/news/china/story20230727-1418102}{大陆央视在2020年10月11日播出``台谍案''报道,曝光台湾民众李孟居的认罪影片。(互联网) 遭控为``台谍''的台商李孟居,结束一年10个月刑期和剥夺政治权利附加刑两年后,于星期一(7月24日)顺利从中国大陆离境前往日本,预计之后再返台。台湾政府的大陆委员会表示,陆方并未通报李孟居获释,但陆委会与当事人家属保持密切联系,返台时间相信李孟居自有安排……}

\entryitemWithDescription{西双版纳``猛男餐厅''被禁演并罚款近两万元}{https://www.zaobao.com/news/china/story20230727-1418098}{因涉不雅表演,云南省西双版纳一家``猛男餐厅''被责令停止演出,没收违法所得,并处罚款逾十万元(人民币,1万8525新元)。 云南省文旅厅官微星期三(7月26日)发布消息称,经查,云南景洪泰鼓泰鼓泰国菜餐厅内举办的营业性演出活动内容,存在危害社会公德和民族优秀文化传统的情形,已违反了相关规定……}

\entryitemWithDescription{王纬温:秦刚去职后中国外交再调整?}{https://www.zaobao.com/news/china/story20230727-1417824}{中国外交部去年底由驻美大使秦刚回国挂帅,新外交团队不仅调整过去几年的强硬对外风格,也更重视外交官整体素养专业,还对美国释放更多善意,并加大力度改善与澳大利亚的关系,令外界有耳目一新的感觉。 但秦刚掌舵中国外交部仅半年,就在6月25日公开露面后神隐整整一个月,引发外界议论纷纷。最终,这位最年轻的副国级高官星期二(7月25日)被正式免去外长职务,由级别更高的前任外长王毅回锅掌舵……}

\entryitemWithDescription{台湾总统选战最新民调 侯友宜拉近与柯文哲落差}{https://www.zaobao.com/news/china/story20230726-1417823}{台湾最新民调显示,国民党总统参选人侯友宜正拉近与民众党参选人柯文哲的落差,支持度落后幅度从双位数缩短至4.1个百分点。在野领军之争牵动选战形势,备受关注。 台湾将在五个半月后的2024年1月13日举行总统与立委选举,多份民调显示民心思变,约六成民众希望政党轮替。台湾从2000年以来都由蓝绿轮流执政八年,反映民众倾向支持监督制衡,避免一党长期执政垄断资源……}

\entryitemWithDescription{武汉地震监测中心遭网攻 官媒称初步证据显示来自美国}{https://www.zaobao.com/news/china/story20230726-1417815}{中美近期围绕网络攻击议题展开新一轮较量。湖北省武汉市应急管理局地震监测中心报称部分设备遭境外网络攻击,官媒更将矛头指向美国。中国外交部作出谴责之余,也称将采取必要措施维护中国的网络安全。 据武汉市公安局江汉分局官方微博星期三(7月26日)发布,武汉市应急管理局地震监测中心报警称,该中心发现部分地震速报数据前端台站采集点网络设备被植入后门程序,对国家安全构成威胁……}

\entryitemWithDescription{港议员容海恩否认若丈夫涉危害国安将离婚}{https://www.zaobao.com/news/china/story20230726-1417805}{今年46岁的容海恩是香港建制派政治人物,频频在中国大陆或亲大陆媒体中亮相。(容海恩脸书) 香港立法会议员容海恩否认在社媒上发声明,称若丈夫袁弥昌涉及危害国家安全行为将离婚。 综合香港《明报》、香港01、星岛网等报道,星期二(25日)网上流传图片,称容海恩发声明表示``不排除与袁弥昌结束婚姻关系,并会循法律途径争取两名女儿抚养权''……}

\entryitemWithDescription{模仿泰国``猛男''表演 西双版纳一餐厅被立案调查}{https://www.zaobao.com/news/china/story20230726-1417794}{据称让消费者不出国门就能体验``泰式风情舞''的``泰鼓泰鼓''餐厅,已被令暂停营业。(互联网) 云南西双版纳一家餐厅雇佣多名``猛男''在餐厅内演出,向消费者表演挑逗性动作,并在互动时为女顾客嘴对嘴喂食,现被当地官方立案查处……}

\entryitemWithDescription{中国外交部网站删除秦刚信息 分析:显示可能犯严重错误}{https://www.zaobao.com/news/china/story20230726-1417792}{中国外交部网站7月26日已清除有关秦刚的信息,主页的外交部长活动栏目也留白。(白艳琳摄) 中国官方免去秦刚的外长职务后,中国外交部网站全面清除有关秦刚的信息。中国官方继续对秦刚免职原因三缄其口,但强调``中国外交活动都在稳步向前推进''。 受访学者分析,外交部网站罕见地删除秦刚的信息,显示他可能犯下严重错误;但同时也有矛盾的信号,包括秦刚仍继续担任国务委员……}

\entryitemWithDescription{金正恩参谒中国军人烈士陵园 向毛岸英之墓献花}{https://www.zaobao.com/news/china/story20230726-1417779}{据朝中社7月26日发出的照片,显示朝鲜最高领导人金正恩到位于桧仓的中国人民志愿军烈士陵园,在烈士塔前敬献花圈。(路透社) 朝鲜最高领导人金正恩星期三参谒了中国人民志愿军烈士陵园,并向在朝鲜战争中阵亡的已故中国领导人毛泽东儿子毛岸英之墓献花。中国外交部称,这体现了中朝传统友好在新的历史时期的传承和发展……}