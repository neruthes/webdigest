\entryitemWithDescription{陈婧:透支结婚}{https://www.zaobao.com/news/china/story20240319-3176407}{看到``结婚人数近10年来首次回升''话题高挂微博热搜榜上,二十几岁的中国朋友不以为然:``因为今年不吉利,大家才赶着在去年结婚。等着瞧,今年还会跌回来。'' 根据中国民政部上周公布的最新统计数据,2023年中国共办理结婚登记768万对,比2022年增加12.4\%。这是中国结婚人数自2013年以来首次止跌回升,不仅比前一年多出84万5000对,也超过2021年的763万6000对……}

\entryitemWithDescription{台湾一岁童遭认证保姆虐死 卫福部长道歉}{https://www.zaobao.com/news/china/story20240318-3171617}{台北市一岁10个月大男童刘皓剀(昵称剀剀)遭官方认证的保姆虐待致死,台湾卫生福利部长薛瑞元表示政府人员处理态度有偏差,承认卫福部督导不周,为此向社会道歉。 虐童案引起社会与朝野政党高度关注,薛瑞元星期一(3月18日)到立法院进行专案报告时说,他愿意私下向受害男童家属道歉、安慰并提供后续必要协助……}

\entryitemWithDescription{太阳花学运十周年 学运领袖林飞帆致歉:让大家失望了}{https://www.zaobao.com/news/china/story20240318-3176126}{当年担任太阳花总指挥的民进党前副秘书长林飞帆,在太阳花学运十周年晚会上,以个人身份向学运参与者致歉。他强调,十年下来,在不同的位置上,都有让参与者失望的地方。(自由时报) 台湾在2014年发生太阳花学运,两岸自此中断服务贸易、货物贸易谈判,当年参与学运的民间团体星期一(3月18日)举办学运十周年晚会,重申诉求。当年的学生领袖林飞帆则在会中致歉,称十年下来,在不同的位置上,都有让大家失望的地方……}

\entryitemWithDescription{金门钓客出海失联 中国大陆海警救起}{https://www.zaobao.com/news/china/story20240318-3175687}{(金门综合讯)两名金门钓客出海钓鱼后失联,后来被中国大陆海警救起并转交金门。 台湾《联合报》报道,两名钓客星期天(3月17日)上午11时半左右乘波特船出海,后来海雾袭来。波特船(Porta-bote)是一种可折叠的小型载具。 钓客家属称两人在下午3时多失联,后于傍晚6时半通报金门海巡队。由于海雾笼罩,金门海巡队在晚间无法立刻出海寻找,只能在附近岸际持续用强光与雾笛警示……}

\entryitemWithDescription{河北邯郸13岁男孩被害 嫌疑人为三名同班同学}{https://www.zaobao.com/news/china/story20240318-3174586}{(邯郸综合讯)中国河北省邯郸市一名初中生杀害,嫌犯是三名不满14岁的同班同学,由于案件犯罪手法极其残忍,且涉及到未成年人如何承担刑事责任、留守儿童等话题,在中国社会引发广泛关注。 综合《新京报》和澎湃新闻的报道,河北省邯郸市肥乡区一名13岁王姓学生长期遭受三名同班同学霸凌。这三人涉嫌在3月10日杀害王某某并填埋,涉事者均为父母在外地工作的留守少年……}

\entryitemWithDescription{中国首两个月工业生产数据优于预期 但房地产开发投资继续下降}{https://www.zaobao.com/news/china/story20240318-3173051}{中国国家统计局星期一(3月18日)发布的数据显示,今年头两个月,中国房地产开发投资同比下降9\%,其中住宅投资下降9.7\%。图为置业顾问2月28日在太原市一家房地产售楼部,为市民推介商品住宅。(中新社) (北京综合讯)中国今年前两个月工业生产数据和社会零售额的增长超出市场预期,标志2024年开局良好。不过,房地产开发投资继续下降,仍是拖累经济和信心的主要因素之一……}

\entryitemWithDescription{王毅访新西兰 强调加强两国战略沟通}{https://www.zaobao.com/news/china/story20240318-3171824}{新西兰副总理兼外长彼得斯(右)于3月18日在惠灵顿议会举行的双边会议上,与中国外长王毅(左)握手拍照。(法新社) (惠灵顿综合讯)中国外长王毅在新西兰访问时强调,面对错综复杂的国际局势,需要与朋友合作。新西兰副总理兼外长彼得斯则对南中国海和台海日渐加剧的紧张局势表达关切。 法新社报道,王毅星期一(3月18日)与彼得斯举行闭门会晤,开启了对新西兰与澳大利亚的``外交闪电战''……}

\entryitemWithDescription{于泽远:解放军正在与美军赛跑}{https://www.zaobao.com/news/china/story20240318-3167933}{中国军事装备发展近日频频曝出大动作。先是3月初两会期间解放军海军、空军负责人分别透露了航母、歼-35隐身战斗机以及轰-20的新进展,随后香港《南华早报》又报道了中国新型高超声速无人机将可直达美国本土的消息。这些迹象显示,在中美博弈不断加深的背景下,解放军正在加速与美军赛跑。 解放军目前有三艘航母下水,其中战机滑跃起飞的辽宁舰、山东舰早已服役,战机电磁弹射起飞的福建舰有望在今年海试并正式服役……}

\entryitemWithDescription{时隔五年张志贤再度访华 赴四川浙江北京}{https://www.zaobao.com/news/china/story20240317-3168301}{国务资政兼国家安全统筹部长张志贤星期天(3月17日)起对中国进行为期六天的正式访问,这是他时隔五年再度访华。 据我国外交部星期天发布的消息,张志贤将在3月17日至22日访问四川、浙江和北京。 访华期间,张志贤将与四川和浙江的省级领导会面,并到访位于四川的新川创新科技园和位于浙江的新加坡杭州科技园。他也将在北京与中国高层领导会面……}

\entryitemWithDescription{中国四川云南连发山火 三名扑火人员身亡}{https://www.zaobao.com/news/china/story20240317-3168597}{中国四川省雅江县呷拉镇白孜村3月15日发生一起森林火情,后因火场风力突然增大,火势迅速扩大蔓延。(新华社) (雅江/临沧综合讯)中国四川、云南两地过去几天接连发生山火。云南明火扑灭后隔天又复燃,三名扑火人员身亡;四川火势蔓延迅速,已转移民众4903人……}

\entryitemWithDescription{甘肃麻辣烫爆火 官方开会吁全市上下把握机遇}{https://www.zaobao.com/news/china/story20240317-3168649}{甘肃天水麻辣烫爆红后,为期一周的天水麻辣烫``吃货节''从星期六(3月16日)开始举行。除评选麻辣烫最佳口味、最具创意、最受欢迎等奖项外,麻辣烫``吃货节''也将推出多条包含麻辣烫体验的旅游线路。(中新社) (天水综合讯)红油满满的甘肃天水麻辣烫,近期在中国互联网上爆红,相关词条多次登上社交平台热搜榜,是继淄博烧烤、哈尔滨冰雪游后``出圈''的另一个网红地方特色……}

\entryitemWithDescription{中国国安部指钓鱼邮件是境外间谍机关惯用攻击手法}{https://www.zaobao.com/news/china/story20240317-3168147}{(北京讯)继提醒公众网恋或高薪工作可能是窃取情报的途径后,中国国家安全部星期天(3月17日)再发文称,钓鱼邮件是境外间谍机关实施网络攻击的惯用手法。 中国国安部星期天(3月17日)在``国家安全部''微信公众号发文称,这些境外间谍将中国的党政机关、涉密单位计算机网络作为窃密主渠道,并通过钓鱼邮件``假扮官方实施欺诈''……}

\entryitemWithDescription{中国证监会发布两文件加强监管上市公司 分析:旨在进一步稳市场}{https://www.zaobao.com/news/china/story20240317-3168567}{中国证监会3月15日发布了两份上市公司监管文件。图为警员2月8日在中国证监会位于北京金融街的大楼外巡逻。(路透社) 中国证监会同时发布两份上市公司监管文件,寄望通过持续加强监管提高上市公司质量。分析指出,新规旨在进一步稳市场,但要执行到位并不容易,也对首次公开售股(IPO)带来更大压力,当局应考虑调动市场力量参与监管……}

\entryitemWithDescription{港府批英媒BBC报道抹黑《香港国安法》}{https://www.zaobao.com/news/china/story20240317-3168276}{(香港/伦敦综合讯)香港政府抨击英国广播公司(BBC)的报道以虚假指控抹黑港府维护国家安全法律。 港府星期六(3月16日)发布新闻公报称,对BBC一篇有关2019年反修例示威者被判囚的报道,以虚假指控抹黑港府维护国家安全法律,表示不满和谴责。 港府发言人说,《香港国安法》第一条已清楚订明制定该法的目的,是为坚定不移并全面准确贯彻``一国两制''、``港人治港''、高度自治的方针……}

\entryitemWithDescription{叫停一年后 中国证实与斐济恢复警务合作}{https://www.zaobao.com/news/china/story20240317-3168083}{斐济在叫停与中国的警务合作一年后,决定恢复两国警务合作。图为一艘中国渔船2022年7月16日在斐济首都苏瓦的码头卸下金枪鱼。(路透社档案照) (苏瓦/北京综合讯)太平洋岛国斐济在叫停与中国的警务合作一年后,决定恢复两国警务合作。 中国驻斐济大使馆证实上述消息。据中国中央广播电视总台报道,该台记者星期天(3月17日)从中国驻斐济大使馆了解到,中斐警务合作已重回正轨……}

\entryitemWithDescription{中国登记结婚人数近十年首次回升}{https://www.zaobao.com/news/china/story20240317-3168347}{官方数据显示,中国去年登记结婚人数近十年首次回升。图为北京一对新人3月12日准备前往北京市西城区民政局婚姻登记服务中心进行结婚登记。(新华社) (北京综合讯)中国去年登记结婚人数近十年首次回升,同比增加约12.4\%。 中国民政部网站星期五(3月15日)公布《2023年四季度民政统计数据》,显示2023年中国结婚登记数为768万对……}

\entryitemWithDescription{香港年长者自杀人数上升 疫情与移民潮加剧孤独感}{https://www.zaobao.com/news/china/story20240317-3167982}{香港一名白发苍苍的男子2月28日坐在街道上阅读报纸。(法新社) (香港综合讯)香港专家分析,香港近年出现移民潮,加上冠病防疫措施,加剧了当地年长者的焦虑和孤独感,并导致相关群体的自杀人数有所上升。 综合《明报》、网媒``香港01''、香港电台报道,香港大学社会工作及社会行政学系讲座教授叶兆辉星期天(3月17日)在香港商业电台节目上说,香港年长者的自杀人数有上升趋势,尤其是独居者……}

\entryitemWithDescription{台学者:台湾应在金门常态化部署大型海巡船}{https://www.zaobao.com/news/china/story20240317-3168056}{(台北综合讯)中国大陆快艇在金门海域翻覆导致台海两岸灰色地带冲突隐忧加剧,有台湾学者建议,台湾应在金门常态化部署大型海巡船。 《自由时报》报道,台湾的国家政策研究基金会副研究员揭仲说,金门接近中国大陆海域,陆方进入金门的``禁止、限制海域''比跨越台海中线容易,同时金门多配属百吨小船,在对抗数量多且大的中国大陆海警船上有劣势……}

\entryitemWithDescription{前美国驻华代办:美国不会利用台湾问题引战削弱中国}{https://www.zaobao.com/news/china/story20240317-3164950}{美国耶鲁大学杰克逊全球事务学院全球安全研究项目执行主任维滕斯坦(左)与前美国驻中国临时代办阮大为(右),3月带领耶鲁大学访团先后访问北京、台湾,并于3月16日在台北出席长风基金会举办的座谈会……}

\entryitemWithDescription{香港特稿:当普通话日渐普通化 港味流失重寻自身定位}{https://www.zaobao.com/news/china/story20240317-3155993}{有大陆网民调侃,提议将名字富有殖民地色彩的香港维多利亚海港,改名为延安港。图为维港海滨长廊挤满游客。(中通社) 香港成为英国殖民地后,在百年间慢慢形成中西合璧的香港文化。自香港回归这20多年以来,无论是在社会文化、生活习惯,还是政治体制等方面,都越来越大陆化。有学者警示本来是深度嵌入区域、嵌入国际的香港正``快速孤岛化'',但也有专家认为,最重要的是香港要找到适合自己的定位和机遇……}

\entryitemWithDescription{中国规定不得引用或转译损害国家主权外语地名}{https://www.zaobao.com/news/china/story20240316-3165687}{(北京综合讯)中国民政部规定,不得直接引用或擅自转译可能损害中国领土主张和主权权益的外语地名。 根据``中国民政''微信公众号星期五(3月15日)消息,中国民政部近日公布《地名管理条例实施办法》(以下简称《办法》),自今年5月1日起施行,其中第13条明确作出上述规定……}