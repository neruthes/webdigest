\entryitemWithDescription{民众党拟列大陆配偶为不分区立委 在台湾成为热议话题}{https://www.zaobao.com/news/china/story20231106-1448355}{徐春莺在记者会上出示早已取得的``中华民国''护照,强调自己是``中华民国''国民。(自由时报) 台湾在野民众党拟纳中国大陆籍配偶徐春莺为不分区立委,成为热议话题。执政的民进党总统参选人赖清德公开指控她是共产党员。徐春莺否认自己曾加入共产党,反呛``政治人物讲话要有诚信,请问赖副总统有证据吗?'' 国安局长蔡明彦星期一(11月6日)表示,台湾政府对陆配的原则和立场很清楚,只要合法活动都予以尊重……}

\entryitemWithDescription{升级版新中自由贸易协定预计今年底签署}{https://www.zaobao.com/news/china/story20231106-1448348}{贸工部兼文化、社区及青年部政务部长刘燕玲(左)在上海出席新中经贸与投资论坛时,与新加坡工商联合总会执行总裁郭柄汛对话。(陈婧摄) 新加坡贸工部兼文化、社区及青年部政务部长刘燕玲披露,升级版新中自由贸易协定预计在今年底签署。 刘燕玲星期一(11月6日)出席新加坡工商联合总会在上海举办的新中经贸与投资论坛……}

\entryitemWithDescription{丁薛祥吁反对知识封锁及扩大科技鸿沟}{https://www.zaobao.com/news/china/story20231106-1448342}{中国国务院副总理丁薛祥星期一(11月6日)在重庆出席首届``一带一路''科技交流大会开幕式。(王纬温摄) 中国国务院副总理丁薛祥星期一(11月6日)在重庆呼吁``一带一路''国家,共同反对知识封锁和人为扩大科技鸿沟,共同完善全球科技治理。 受访学者分析,西方近期在芯片和人工智能等高科技领域加码施压中国,北京高层最新表态明显是在批评美国,并希望通过一带一路施展外交应对西方科技围堵……}

\entryitemWithDescription{美国运通传在港裁员数十人}{https://www.zaobao.com/news/china/story20231106-1448338}{(香港讯)信用卡公司美国运通据报近期拟裁减香港数十个职位。 香港《信报》星期一(11月6日)引述匿名消息人士报道,受香港市场前景不确定性影响,以高端客户定位的美国运通(American Express)近期启动结构调整,计划精简涉及运营、市场推广及信贷风险监控等关键部门……}

\entryitemWithDescription{台春节前公布组团赴陆细节 包机旅游航点可望变直航}{https://www.zaobao.com/news/china/story20231106-1448336}{一架台湾中华航空公司的货机11月5日飞过德国法兰克福的球场。(法新社) (台北综合讯)台湾交通部长王国材说,台湾将在明年春节之前,公布解除赴大陆旅行团禁令的细节,若两岸包机点运量稳定,将考虑变成直航点……}

\entryitemWithDescription{中国东北今冬最强暴风雪来袭 多地停课停航}{https://www.zaobao.com/news/china/story20231106-1448312}{中国黑龙江省哈尔滨市星期一(11月6日)被皑皑白雪覆盖。(法新社) (北京综合讯)中国极端天气再现,东北降雪量或突破历史同期极值,多地停课停航。 中国北方过去一周天气剧变,经历了严重雾霾到异常高温的10月,继而又在周末突遇气温骤降。 中国天气网报道,受冷涡系统及地面气旋影响,中国东北大部地区将经历今冬最猛烈雨雪天气……}

\entryitemWithDescription{湖南长沙村民集资修建毛泽东铜像被拆除}{https://www.zaobao.com/news/china/story20231106-1448304}{(香港/深圳综合讯)湖南长沙一尊由村民集资建立的毛泽东铜像,据报上星期六(11月4日)晚被强行拆除。这起事件继8月山东郯城汉白玉毛泽东铜像落成前``被盗''后,再度引发争议。 据《星岛日报》报道,中国大陆左派网站``红歌会网''发文称,湖南长沙市望城区铜官镇花实村10月1日竖立的一尊毛泽东铜像,被要求拆除。上星期六夜晚,``一伙黑衣人带着大型吊车'',强行拆走了铜像……}

\entryitemWithDescription{美中军舰上周迫近太平岛 曾相互喊话}{https://www.zaobao.com/news/china/story20231106-1448301}{(台北综合讯)台媒报道称,太平岛周边数海里内,上星期五(11月3日)一度出现美国、中国大陆军舰先后迫近太平岛的情形,双方当时曾相互喊话,从岛上瞭望可目视船影集结。 据《联合报》报道,太平岛的雷达信息显示,3日上午,先有美国海军杜威号驱逐舰航经太平岛周边12海里处水域,大陆军方驱逐舰紧接着出现在太平岛岸上肉眼可视的数海里内位置……}

\entryitemWithDescription{大陆推出入境新政 方便台民众通关落户福建}{https://www.zaobao.com/news/china/story20231106-1448258}{(厦门/台北综合讯)中国大陆政府出台新政策,为台湾民众在福建通关出行、落户审批等,提供更多便利。 据中国国家移民管理局官网星期一(11月6日)消息,为推进福建建设两岸融合发展示范区,中华人民共和国出入境管理局根据建设规划,综合考虑台湾居民和企业的需求与期望,计划于2024年1月1日起实施10项出入境新措施……}

\entryitemWithDescription{香港全年增长预测将低于预期}{https://www.zaobao.com/news/china/story20231105-1448116}{港府正积极提倡``夜缤纷''活动以刺激消费,星期日(11月5日)落幕的香港美食嘉年华配合``夜缤纷'',开放时间比以往推迟一小时至晚上9时。(香港中通社) (香港综合讯)香港财政司司长陈茂波说,本周将发表的全年经济增长预测范围,将低于年初时的估算。 陈茂波星期日(11月5日)发表网志说,香港第三季经济增长4.1\%,幅度较预期温和……}

\entryitemWithDescription{新加坡两大咖啡粉厂商参展进博会}{https://www.zaobao.com/news/china/story20231105-1448110}{金源源创办人孙利顺(右)和孙昱轩父子兵齐上阵,将近年来推出的全线袋泡咖啡产品带到进博会参展。(陈婧摄) 新加坡两大咖啡粉厂商新隆利和金源源今年双双参加上海进博会,希望把新加坡袋泡咖啡介绍给广大中国消费者。 长期为餐饮业者供应咖啡粉的新隆利,在进博会上首次推出面向消费者研发的袋泡南洋咖啡。公司执行董事黄瀚禾告诉《联合早报》,这款耗时两年研发的咖啡明年才在新加坡发售,进博会访客将抢先尝到它的味道……}

\entryitemWithDescription{赖清德:若当选不改``中华民国''国号}{https://www.zaobao.com/news/china/story20231105-1448108}{民进党总统参选人赖清德(中)星期六(11月4日)回到他当过市长的台南市出席竞选总部成立大会,受到据称两万名台南乡亲夹道欢迎。(民进党提供) 台湾总统大选倒数十周,民调领先的民进党总统参选人赖清德主动出击,称若当选总统不会改``中华民国''国号,为各方疑虑的台独立场打预防针。 台湾将在2024年1月13日举行总统与立委选举,即将执政满八年的民进党(绿)派出副总统兼党主席赖清德参选……}

\entryitemWithDescription{中国财长:新增国债发行提速 稳妥化解地方债}{https://www.zaobao.com/news/china/story20231105-1448097}{中国金融中心上海外滩的夜景。(海峡时报) (北京综合讯)中国财政部长蓝佛安说,中国将加快推进新增国债发行使用。 中国官方媒体新华社星期天(11月5日)刊登对蓝佛安的专访。 蓝佛安表示,财政部将继续贯彻实施好积极的财政政策,在支出上持续发力,接下来会加快推进新增国债的发行使用,用好新增地方政府专项债券资金。同时,抓好一揽子化债方案落实,强化各级政府责任,积极稳妥推动化解地方政府债务风险……}

\entryitemWithDescription{李强:中国将积极扩大进口和放宽市场准入}{https://www.zaobao.com/news/china/story20231105-1448088}{中国总理李强(前排右)11月5日在上海出席第六届进博会开幕式前,参观企业展区。(新华社) 中国总理李强承诺,中国将以实际行动促进更大范围、更高水平和更深层次合作开放,包括扩大进口、进一步放宽市场准入、落实全面取消制造业领域外资准入限制等……}

\entryitemWithDescription{英媒:全球民调巨头盖洛普将关闭在华业务}{https://www.zaobao.com/news/china/story20231105-1448085}{美国咨询公司盖洛普(盖洛普官网) (北京/旧金山综合讯)全球民调巨头美国盖洛普咨询公司,据报在中国监管审查和地缘政治紧张的双重压力下,已决定撤出中国市场。 《金融时报》星期六(11月4日)引述三名消息人士透露,盖洛普(Gallup)已于本周告知客户上述消息。公司建议客户考虑将部分项目迁至海外,而其他项目将被取消。《金融时报》也取得一份公司的内部通知,显示盖洛普对结束在华运营表示遗憾……}

\entryitemWithDescription{中美举行外交政策磋商}{https://www.zaobao.com/news/china/story20231105-1448075}{(北京综合讯)中美两国上周除了举行首轮海洋事务磋商,也举行了外交政策磋商。 中国外交部官网星期六(11月4日)公布,外交部政策规划司司长苗得雨上星期三(1日)在维也纳,同美国国务院政策规划司司长艾哈迈德举行中美外交政策磋商。双方围绕国际地区形势、各自内外政策及其他共同关心的问题交换了意见。 据中国外交部官网,中美两国曾在2010年、2015年和2016年举行外交政策磋商……}

\entryitemWithDescription{专家示警中国人口负增长恐将延续至下世纪}{https://www.zaobao.com/news/china/story20231105-1448073}{北京一名女子9月12日推着一辆婴儿车,走在人流不多的街道。(法新社) (上海讯)中国人口问题专家说,中国人口负增长趋势可能延续到下世纪,应加速对人力资本的投资和需求结构转型。 第一财经报道,上海社科院原常务副院长兼经济研究所所长、国家自科与社科基金重大课题首席专家左学金星期六(11月4日)出席研讨会时指出,中国人口变动的主要风险已转向持续负增长、极低生育率与快速老龄化……}

\entryitemWithDescription{港媒:中国科研人员正研制新型远程``潜艇杀手''}{https://www.zaobao.com/news/china/story20231105-1448068}{(香港综合讯)香港媒体报道,中国科研人员正在研究将无人机和人工智能技术结合,通过火箭发射智能鱼雷,让它可以攻击距离中国海岸线200公里的潜艇。 据香港《南华早报》星期六(11月4日)报道,在中国科研人员电脑模拟军演中,中国解放军将鱼雷发射至一艘在国际水域巡航的核潜艇附近,由于鱼雷的射程仅为40公里,而附近又没有可以作为发射平台的军舰或飞机,鱼雷的出现让这艘潜艇措手不及,于是成功将其摧毁……}

\entryitemWithDescription{加拿大指中国军机不安全飞行 北京反指加军机恶意挑衅}{https://www.zaobao.com/news/china/story20231105-1448064}{(北京/渥太华综合讯)针对加拿大军方两周内第二次指责中国空军在南中国海不安全飞行,中国国防部回应称,加军机采取超低空飞行等挑衅动作,``别有用心的恶意挑衅''。 中国国防部官网星期六(11月4日)发布发言人张晓刚回应称,加拿大``渥太华''号护卫舰先后两架次不明意图的舰载直升机逼近中国西沙领空,``违反中国国内法及有关国际法''……}

\entryitemWithDescription{富士康在大陆被查 鸿海10月营收同比下滑4.56\%}{https://www.zaobao.com/news/china/story20231105-1448051}{一名戴着口罩的女子2022年11月9日路过鸿海位于台北的办公室。(路透社) (台北综合讯)富士康集团在中国大陆被查,其母公司台湾鸿海集团10月营收同比下滑4.56\%。 综合彭博社和《自由时报》报道,鸿海集团星期天(11月5日)公布10月营收为新台币7412亿元(约312亿新元),环比增长12.20\%,同比减少4.56\%……}