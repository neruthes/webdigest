\entryitemWithDescription{``中国象棋第一人''王天一据报被捕}{https://www.zaobao.com/news/china/story20240824-4534811}{35岁的王天一从2014年起,连续11年居中国象棋棋手等级分全国第一名。(微博账号``王天一Alien'') (北京/武汉综合讯)因涉嫌在比赛中``买棋''\,``卖棋'',有``中国象棋第一人''之称的象棋特级大师王天一据报已被捕,同时被调查的还有其他中国象棋特级大师……}

\entryitemWithDescription{全红婵买苹果手机被指``不爱国'' 中国媒体吁勿道德绑架}{https://www.zaobao.com/news/china/story20240824-4534627}{(北京综合讯)巴黎奥运会跳水冠军全红婵因购买苹果手机,遭到部分中国网民质疑``不爱国''。有中国媒体评论称,将爱国与消费选择挂钩,是对爱国主义的误解和滥用,呼吁``别道德绑架全红婵''。 据《新京报》星期五(8月23日)报道,广州一家苹果手机店的员工称,全红婵星期三(8月21日)到店里购买了一台iPhone 15 Pro Max手机。网络上也流传全红婵在店内结账时的视频,引发中国网民争论……}

\entryitemWithDescription{长荣声明认同九二共识反台独 台湾政府和民进党谴责中国大陆}{https://www.zaobao.com/news/china/story20240824-4534610}{(台北综合讯)巴黎长荣桂冠酒店未挂五星红旗事件延烧,母公司长荣集团发声明致歉并强调认同九二共识、反对台独后,台湾外交部、陆委会和执政的民进党同日发声谴责中国大陆。 台湾运输业巨头长荣集团星期五(8月23日)发表声明,针对旗下巴黎桂冠酒店在奥运期间相关布置衍生争议、进而影响两岸人民情感表示歉意,并承诺加强员工教育训练,避免再发生类似情况……}

\entryitemWithDescription{中国公安部回应``持证上网'':没有网号网证也可正常上网}{https://www.zaobao.com/news/china/story20240824-4534327}{(北京综合讯)针对官方提出为民众统一签发网号、网证引发了``持证上网''、言论管控担忧一事,中国公安部澄清称,用户不是``持证''才能``上网'',没有网号、网证也可正常上网。 引起争议的《国家网络身份认证公共服务管理办法(征求意见稿)》,由中国公安部、国家网信办等部门研究起草,将在星期天(8月25日)结束公众意见征求。中国官媒新华社星期五(8月23日)发布详细解读,试图缓解公众疑虑……}

\entryitemWithDescription{黑利:台湾不需为自身防卫付费 挺台是美国两党共识}{https://www.zaobao.com/news/china/story20240824-4534272}{前美国驻联合国大使黑利8月21日在台北举行的凯达格兰论坛上发表演讲。(路透社) (台北综合讯)前美国驻联合国大使黑利说,台湾不需要为自身的防卫付费,并称支持台湾是美国两党间少有的共识。她还说,孤立主义是不健康的,呼吁共和党与台湾等伙伴站在一起,共同面对中国大陆。 黑利曾是美国前总统特朗普在竞逐共和党总统候选人提名时的竞争对手。她在今年3月宣布退选,上月更呼吁支持者票投特朗普……}

\entryitemWithDescription{抵御消费降级寒风 中国二三线城市消费逆势回暖}{https://www.zaobao.com/news/china/story20240824-4516674}{中国二线和三线城市的消费已成为拉动中国消费马车的最新引擎。图为西安大唐不夜城步行街8月6日人流如梭。 (新华社) 在好些上海居民捂紧荷包过日子的当下,家住苏州的张鹏(41岁)却感觉,他个人和家庭的消费并没有明显下滑。 张鹏受访时说,在家庭日常消费之外,他还会跟朋友包场踢足球,孩子补习课的开销也跟以前一样,``卖方市场没降级,买方总归是要掏这个钱''……}

\entryitemWithDescription{贺一诚:澳门任期最短的特首}{https://www.zaobao.com/news/china/story20240824-4531165}{澳门第六届行政长官选举下周四(8月29日)开始接受各方有志人士报名,现任特首贺一诚本周三(8月21日)突然以身体理由宣布不参选,成为澳门首位仅做一任五年、不争取连任的特首。连日来,澳门网民除了纷纷猜测谁是下任特首,也热议贺一诚任内功过。 祖籍中国浙江省的贺一诚,出生于澳门,曾担任澳门行政会委员、经济发展委员会委员、科学技术暨革新委员会委员……}

\entryitemWithDescription{庄慧良:房价飙涨都是新青安惹的禍?}{https://www.zaobao.com/news/china/story20240824-4531980}{台湾在野民众党主席柯文哲深陷政治献金风暴,此时又传出他的夫人陈佩琪数月前去看上亿豪宅(新台币,下同,约432万新元)。陈佩琪在脸书表示,去年婆婆跌倒骨折,公公年迈无法爬上二楼卧室,两老只得把床移到一楼客厅。有感于老年无电梯的困扰,去年就和儿子商量把其旧公寓换成有电梯的房子,可以作为未来婚房,她也想给自己另觅安静住所。但房价高涨,换新屋除了有电梯外,其他屋况条件都变差……}

\entryitemWithDescription{中白联合公报批单边制裁 分析:北京向明斯克送暖但不抱团}{https://www.zaobao.com/news/china/story20240823-4531481}{中国总理李强(左)星期四(8月22日)在白罗斯首都明斯克与白罗斯总统卢卡申科会面、握手。(路透社) 中国总理李强星期四(8月22日)到访因在俄乌战争中支持莫斯科而受到西方制裁的白罗斯,两国过后发表联合公报,共同谴责使用非法单边强制措施,并表示将在防务、执法安全、贸易等领域加强合作。 受访学者分析,中国选择此时提升双边军事经贸关系,明显是向日益被西方孤立的白罗斯送暖……}

\entryitemWithDescription{中国住建部:老百姓交了钱,就该拿到房}{https://www.zaobao.com/news/china/story20240823-4530703}{今年2月29日,上海市一处碧桂园正在建设中的住宅开发项目。(路透社档案照片) (北京综合讯)中国今年需交付近400万套已售在建住房。中国住房和城乡建设部副部长董建国星期五表示,将推进按时保质交付,``老百姓交了钱,就应该拿到房''……}

\entryitemWithDescription{批雷军亏钱卖车是倾销 极越汽车公关负责人被处罚}{https://www.zaobao.com/news/china/story20240823-4530452}{(北京综合讯)中国新能源车品牌极越的公关负责人徐继业发文批评小米创办人雷军亏钱卖车,是``倾销''行为,引发争议。 小米集团在星期三(8月21日)发布财报,显示小米二季度交付智能电动汽车2万7307辆,收入62亿元(人民币,下同,11.36亿新元),净亏18亿元。媒体据此推算,小米SU7的单车亏损额超6万元。``小米卖一辆车亏6万多''的话题迅速登上微博热搜……}

\entryitemWithDescription{金门代表团争取恢复两岸观光 宋涛:不会让你们空手而归}{https://www.zaobao.com/news/china/story20240823-4530433}{(台北综合讯)金门民意代表团在北京拜会中国大陆国台办主任宋涛,尝试争取恢复陆客赴金门旅游,宋涛释出善意回应称``不会让你们空手而归''。 综合《旺报》与《联合报》报道,金门县议会议长洪允典、国民党立委陈玉珍率金门县多名议员以及旅游观光业者,星期三至星期六(8月21至24日)访问北京,并在星期四(8月22日)下午拜会了宋涛……}

\entryitemWithDescription{流亡美国民运人士唐元隽 被控充当``中国政府代理人''}{https://www.zaobao.com/news/china/story20240823-4530774}{(华盛顿综合讯)美国司法部星期三(8月21日)指控中国前异议人士唐元隽,以民运人士身份在美国活动,却充当中国政府代理人,监视中国民主活动人士和异议人士。 综合美国之音与哥伦比亚广播公司新闻网报道,美国联邦检察官对唐元隽提出三项刑事指控,包括秘密为中国国家安全部工作、对支持民主活动团体进行间谍活动、对调查人员作出不实陈述……}

\entryitemWithDescription{360儿童手表语音答问内容被指诋毁中国人 周鸿祎道歉}{https://www.zaobao.com/news/china/story20240823-4529545}{(成都/上海综合讯)中国科技巨头360旗下一款儿童智能手表语音回答的内容被指诋毁中国人,360集团董事长周鸿祎星期四(8月22日)发文道歉,并称已完成整改。 综合红星新闻与《新民晚报》报道,有中国网民星期三(8月21日)上传视频,指女儿使用360智能手表的语音问答功能,提问``中国人是世界上最聪明的人吗?''结果智能手表就此问题给出数百字的回答,内容引发争议……}

\entryitemWithDescription{韩咏红:与谣言共舞的时代}{https://www.zaobao.com/news/china/story20240823-4526569}{距离美国总统大选只剩两个多月,选举白热化之际,看前总统特朗普和现任副总统哈里斯这两位候选人彼此攻伐,我时而有观赏``故事大会''之感。政客在台上的发言精彩生动,在真实性方面却不是同等讲究,很容易就从夸大其词滑向明明白白的谎言。 特朗普绝对是善于现场煽动听众情绪的高手,当选情胶着时,他以误导性的假信息来攻击对手,更是毫无顾忌……}

\entryitemWithDescription{台湾明年国防预算再创新高 同比增7.7\%}{https://www.zaobao.com/news/china/story20240822-4525353}{台湾明年国防预算再创新高,同比增加7.7\%。图为台军星期二(8月20日)在屏东军事演习中发射爱国者导弹。(路透社) (台北综合讯)台海局势日趋紧张之际,台湾2025年度国防预算再创新高,同比增加7.7\%,连续八年增长。 综合《经济日报》与彭博社报道,台湾行政院星期四(8月22日)通过了2025年度总预算,这是赖清德政府的首个总预算案……}

\entryitemWithDescription{双城论坛或推迟到9月中下旬 分析:两岸都认同可慢不能断}{https://www.zaobao.com/news/china/story20240822-4525564}{台北市长蒋万安(左)去年赴上海出席双城论坛期间,主动拿出手机,跟上海市长龚正在会场自拍。(台北市政府提供) 上海-台北双城论坛原预估8月底举办,但大陆方面尚未提供出席名单,台北市长蒋万安又将在9月初访美,外界预估论坛可能推迟到9月中下旬。而台湾政府的大陆委员会星期四(8月22日)则证实,已完成论坛将签署的两份备忘录核定。受访学者分析,这显示两岸间至少都认同双城论坛``可以慢,但不能断''……}

\entryitemWithDescription{终审院长岑浩辉是澳门特首热门人选}{https://www.zaobao.com/news/china/story20240822-4524585}{澳门特首贺一诚星期三(8月21日)宣布不竞逐连任后,谁是下届澳门特首备受外界关注。综合香港媒体的分析,澳门终审法院院长岑浩辉目前是顶头大热。 据澳门电台报道,岑浩辉星期四(8月22日)出席活动被传媒追问,是否有意参与特首选举时回应说,特首是服务澳门特区、服务居民的崇高职位,一直以来,他都有为澳门服务的愿望,也有些朋友鼓励他为澳门继续做贡献,对于传媒提出的问题,他正在考虑中,有进一步消息会告诉大家……}

\entryitemWithDescription{中国检方以间谍罪起诉一名日籍药企员工}{https://www.zaobao.com/news/china/story20240822-4524850}{中国检方以间谍罪起诉日本制药公司安斯泰来的一名日籍男员工。图为摄于2009年7月的安斯泰来东京总部。(路透社档案照) (东京综合讯)日本制药公司安斯泰来(Astellas Pharma)一名日籍男性员工去年3月在北京被拘留,据日本媒体消息,这名男子已被中国检察机关以间谍罪起诉……}