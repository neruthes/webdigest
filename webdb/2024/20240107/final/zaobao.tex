\entryitemWithDescription{中国特稿:跌势难逆 中国股市何时繁花似锦?}{https://www.zaobao.com/news/china/story20240107-1460159}{新年开市第一周,陆港股指延续下跌势头,但投资者仍期待中国股市今年能``脱熊入牛''。图为香港交易广场的铜牛。(中新社) 中国股市的3000点保卫战,从2023年一路打到了2024年。开年第一周,A股迎来的不是开门红,而是``开门绿''。中国股市新年里能否逆转跌势?官方救市措施还有多大发力空间?哪些板块更有潜力? ``想从股票上赚钱,先要学会输……}

\entryitemWithDescription{高端疫苗争议在台湾选前连环爆}{https://www.zaobao.com/news/china/story20240106-1460491}{针对民进党总统候选人赖清德承诺公开高端合约,台湾福部星期五(1月5日)在脸书澄清,无法``立即''公开采购合约,理由是高端合约附带五年的保密条款期限。(自由时报资料照) 高端疫苗争议在台湾选前连环爆,继高端股东名册被指有上千民进党人之后,卫生福利部长薛瑞元等高官被发现密会绿营竞选办公室团队,遭在野的国民党指控违反行政中立……}

\entryitemWithDescription{大量学者论文被撤稿 中国教育部要求开展论文自查工作}{https://www.zaobao.com/news/china/story20240106-1460479}{(北京综合讯)由于国际期刊撤回大量中国学者发表的论文,中国教育部发布通知,要求各高校开展撤稿论文自查工作。 综合澎湃新闻和极目新闻报道,武汉大学、山东大学、河南财政金融学院等高校发布消息,称去年以来,Hindawi等国外出版机构撤回大量中国学者发表的论文,对中国学术声誉和学术环境造成不良影响。中国教育部科学技术与信息化司已决定并通知各高校开展撤稿论文自查工作……}

\entryitemWithDescription{侯友宜频提联合政府 柯文哲批国民党操作弃保}{https://www.zaobao.com/news/china/story20240106-1460476}{台湾在野的国民党总统候选人侯友宜(中)称喊话组建联合政府,不是操作弃保。图为侯友宜星期四(1月4日)在基隆竞选造势活动上,向支持者喊话。(法新社) (台北综合讯)台湾进入大选前冲刺时间,国民党正副总统候选人候友宜和赵少康多次抛出若当选将组建联合政府的话题,候友宜还称赞民众党总统候选人柯文哲和鸿海创办人郭台铭是好的人才。柯文哲批评国民党是在操作弃保……}

\entryitemWithDescription{中国国安部提醒航空爱好者莫变为``窃密志愿者''}{https://www.zaobao.com/news/china/story20240106-1460468}{(北京综合讯)中国国安部发文称,有境外机构以免费提供设备、共享航空信息为诱饵,面向中国境内招募``志愿者'',以非法采集中国飞行器飞行数据。 中国国安部星期六(1月6日)在其公众号发布文章,揭露境外机构在中国网络社交平台伺机寻找活跃的航空爱好者,并通过信息和邮件发广告,以免费提供设备、共享航空信息为诱饵,开展招募……}

\entryitemWithDescription{中缅官员密集对话 中国强调全力维护中缅边境地区安全稳定}{https://www.zaobao.com/news/china/story20240106-1460465}{中国国务委员、公安部部长王小洪星期五(1月5日)同缅甸内政部部长雅毕视频通话。(新华社) 缅甸地方武装组织星期五(1月5日)宣称控制果敢首府,中国国务委员、公安部部长王小洪同日与缅甸内政部部长雅毕视频通话,强调愿同缅甸全力维护中缅边境地区安全稳定。 中国外交部副部长孙卫东也在同一周访缅,重申中国将继续发挥建设性作用,为缅北和平进程提供支持。受访学者分析,中国料将在缅甸停火问题上加大斡旋力度……}

\entryitemWithDescription{台国防部指大陆空飘气球威胁航安 意在认知作战影响民心}{https://www.zaobao.com/news/china/story20240106-1460462}{台湾将于下星期六(1月13日)举行总统和立委选举,台军连续多天发现大陆空飘气球飞越台湾本岛。图为摄于去年12月27日的台北101大楼。(彭博社) (台北综合讯)中国大陆空飘气球近日频频逾越台海中线甚至飞越台湾本岛,台湾国防部指气球路径已严重威胁多条国际航道的安全,并说北京是企图运用认知作战影响台湾民心士气……}

\entryitemWithDescription{新闻人间:负债海归送外卖被捅死的时代悲剧}{https://www.zaobao.com/news/china/story20240106-1460305}{青岛一名年轻外卖员一个月前被小区保安捅伤致死,引起社会哗然。《三联生活周刊》星期二报道揭开外卖员的海外留学回国就业人员(海归)背景后,再次掀起舆情,也给外卖员一家的命运蒙上一层更加悲情的色彩……}

\entryitemWithDescription{庄慧良: 台湾大选柔性``弃保''的可能性}{https://www.zaobao.com/news/china/story20240106-1460332}{从目前情势来看,柯文哲要突围而出的可能性相对低,三党立委席次若都未过半,民众党立委未来可扮演关键角色。图为柯文哲1月2日在新北市一场竞选活动中向支持者挥手致意。(法新社) 在台湾2024年大选倒数时刻,执政的民进党(绿)1月2日推出《在路上》\#交棒篇的竞选宣传影片,引起热烈讨论……}

\entryitemWithDescription{中国大陆介选成台湾选战攻防焦点}{https://www.zaobao.com/news/china/story20240105-1460323}{新北市校园割颈案引起社会震惊、家长不安,星期三(1月3日)全台家长会长联盟在台湾教育部外聚集陈情。有家长忧心孩子在校环境而落泪,呼吁赋予老师充分管教权,以维护校园与学生的安全。(中新社) 台湾新北市15岁少年遭同学割颈身亡,引发民间反废除死刑(废死)的讨论。民进党总统候选人赖清德星期四(1月4日)说,把此案连结为政府推动废死的结果,是中国大陆介选(介入选举)的手段……}

\entryitemWithDescription{美媒:华为新款笔记本电脑芯片产自台积电 打消中国技术再突破传言}{https://www.zaobao.com/news/china/story20240105-1460307}{美国在2020年对华为祭出制裁措施,阻断了华为获取先进芯片和芯片制造设备的渠道。图为华为去年6月在上海的展览标志。(路透社档案照) (深圳彭博电)美媒拆解中国电信通讯设备巨企华为的最新笔记本电脑,发现其中使用的芯片实际由台积电制造,打消了关于中国技术再取得新突破的传言……}

\entryitemWithDescription{忧移远通信有华军方背景 美议员促政府考虑列入黑名单}{https://www.zaobao.com/news/china/story20240105-1460297}{(华盛顿综合讯)美国两名议员以威胁国家安全为由,要求美国政府通报是否将中国上海移远通信列入受限军工企业名单。 综合路透社、彭博社等报道,美国众议院中国问题特别委员会共和党籍主席加拉格尔,及该委员会民主党籍成员克里希纳穆尔,在一封星期三(1月3日)写给美国财政部和国防部的信函中说,有重要证据表明,移远通信的技术或对中国国防工业基础有所贡献……}

\entryitemWithDescription{侯友宜承诺就任一年内重启两岸两会对话}{https://www.zaobao.com/news/china/story20240105-1460296}{台湾总统选举投票日逼近,在野的国民党总统候选人候友宜星期五(1月5日)在新北市扫街拜票。(路透社) (台北综合讯)台湾在野的国民党总统候选人侯友宜承诺,若当选就任台湾总统,将在一年内重启两岸两会对话协商,``让友宜恢复两岸友谊''……}

\entryitemWithDescription{中国网信办:2024年严防网络意识形态风险}{https://www.zaobao.com/news/china/story20240105-1460278}{中国网络舆论监管部门本周召开会议,强调今年维护网络意识形态安全的重要性。分析认为,随着内外部环境不确定性的加剧,监管部门对全年网络信息安全的工作,沿续了近年加大舆论维稳力度的整体调性。 据``网信中国''微信公众号消息,中国互联网信息办公室星期三至星期四(1月3日至4日)在北京召开主任会议。会议强调,要不断巩固壮大网上主流思想舆论,坚决防范化解网络意识形态风险挑战,持续提高网络综合治理效能……}

\entryitemWithDescription{中国影子银行巨头中植企业集团申请破产清算}{https://www.zaobao.com/news/china/story20240105-1460274}{中植在中国房地产业鼎盛期管理的资产价值超过1万亿元人民币。图为中植集团位于北京的办公大楼。(路透社) (北京综合讯)中国影子银行业巨头中植企业集团已向法院申请破产清算,北京市第一中级人民法院星期五(1月5日)在其微信公众号上宣布已裁定受理此案。 综合彭博社、法新社报道,去年11月,中国当局宣布对中植企业集团的资金管理业务进行立案侦查……}

\entryitemWithDescription{缅甸去年向中国移交逾四万名电信网络诈骗嫌疑人}{https://www.zaobao.com/news/china/story20240105-1460270}{(北京综合讯)中国去年与缅甸各方合作,逮捕多名涉及电信网络诈骗的犯罪集团头目。据官方最新数据,2023年前11个月缅甸共将超过四万名电诈嫌疑人移交至中国。 中国公安部网站星期五(1月5日)通报,截至去年11月,中国共破获39万1000起电诈案件,逮捕7万9000名犯罪嫌疑人,包括263名诈骗集团``金主''、头目和骨干成员。中国公安部还与多国开展国际执法合作,抓获3000余名犯罪嫌疑人……}

\entryitemWithDescription{解放军在南中国海进行海空联合演训}{https://www.zaobao.com/news/china/story20240105-1460269}{(广州综合讯)南中国海局势紧张之际,中国人民解放军南部战区星期五(1月5日)在南中国海进行海空联合演训,这是解放军一周内再一次在该海域有所行动。 美菲两军星期三(1月3日)在南中国海展开为期两天的联合巡逻,是两军两个月内的第二次联合巡逻。菲律宾军方称,美菲巡逻期间,两艘中国军舰跟踪了美菲的舰艇……}

\entryitemWithDescription{新疆宗教事务条例加入``宗教中国化''等内容}{https://www.zaobao.com/news/china/story20240105-1460263}{(乌鲁木齐综合讯)新疆修订自治区宗教事务条例,增加了坚持宗教中国化方向、宗教场所须体现中国特色等内容。 据《新疆日报》报道,新疆维吾尔自治区人大常委会星期四(1月4日)公布新修订的《新疆维吾尔自治区宗教事务条例》,自今年2月1日起施行。 相比现行2014年所修订的版本,新条例增加了12条,并在原有条例上增加新表述……}

\entryitemWithDescription{中国2035年中小学教师过剩 ``师范热''面临当局降温}{https://www.zaobao.com/news/china/story20240105-1460227}{(北京综合讯)中国面临少子化冲击,预计到2035年有近200万中小学教师面临过剩。鉴于近年``师范热''现象,多个省市已要求对高等院校教育类专业新增布点施加控制。 中国人口总量2022年较2021年减少85万,近61年来首次出现负增长……}

\entryitemWithDescription{韩咏红:大陆对台介选 只有大陆吗?}{https://www.zaobao.com/news/china/story20240105-1460097}{作者说,在过去半年左右的竞选时期,``大陆介选''疑云在台湾始终挥之不去。图为台北市景。(彭博社) 台湾大选投票日一天天逼近,政坛上的攻伐、爆料、乃至腥膻色故事也如期而至。新年的第二天(1月2日),疑似民进党籍人物罗致政的不雅私密影片外流,在台湾手机Line群组广传。 罗致政当晚公开否认自己是影片男主角,指控影片是深伪造假,为境外网军介选操作……}

\entryitemWithDescription{吴钊燮呼吁国际严防中国大陆操纵民主选举}{https://www.zaobao.com/news/china/story20240104-1460094}{台湾外交部长吴钊燮称,中国大陆通过在台湾的代理人设立虚假组织和虚假新闻网站,进行虚假民调,并使用数千个虚假社交媒体账户来操纵公众舆论。(法新社) 台湾将在1月13日举行总统和立委选举,台湾外交部长吴钊燮撰文称,中国大陆意图侵蚀台湾民主体制,呼吁国际社会严防大陆操纵民主国家的选举结果,共同捍卫以规则为基础的国际秩序。 吴钊燮接受英国《经济学人》杂志邀稿,以``即将来临的台湾大选:风险何在……}

\entryitemWithDescription{紧张局势开年持续 中美军舰同时在南中国海巡航}{https://www.zaobao.com/news/china/story20240104-1460091}{美菲两军星期三在南中国海展开为期两天的联合巡逻,美菲各派出四艘舰艇。图为菲律宾武装部队星期四在美国核动力航母卡尔文森号上对菲律宾海军直升机做检查。(法新社) 南中国海局势在2024开年持续紧张,中美军舰分别同时在这海域巡航两天,针锋相对意味浓……}