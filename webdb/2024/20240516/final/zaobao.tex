\entryitemWithDescription{陈婧:迟来的刺激政策}{https://www.zaobao.com/news/china/story20240516-3670597}{``刺激政策终于找对了方向,力度也上来了。'' 最近访问的几名学者和经济师,不约而同发出这样的感慨。过去两周,中国各领域经济刺激政策如雨后春笋般层出不穷,力度也超出预期。 先是最受关注的房地产,两个一线城市北京和深圳在``五一''长假前后相继放宽限购政策,热点二线城市杭州和西安更全面解除限购,带动新一波楼市松绑潮。本周又有消息称,中央政府将要求地方出资收购存量住宅,帮助房企消化库存……}

\entryitemWithDescription{国台办宣布惩戒台湾五名嘴 学者:感觉有如拿明朝的剑斩清朝的官}{https://www.zaobao.com/news/china/story20240515-3669221}{民进党政策会执行长王义川曾说``大陆高铁没靠背'',引起反弹。(取自脸书) 台湾政论节目《关键时刻》主持人刘宝杰星期三(5月15日)主持节目时回应:``评论经济等同台独?大陆怕什么?''(视频截图) 台湾财经专家黄世聪在脸书回应时表示,他论及大陆的内容都来自媒体,并非凭空捏造,在台湾都是言论自由的范围……}

\entryitemWithDescription{中国如何反制美国加征关税 学者:可能以关税对关税}{https://www.zaobao.com/news/china/story20240515-3670468}{美国白宫星期二宣布,将对价值180亿美元(近244亿新元)的中国进口货品加征关税。图为摄于2021年1月21日在北京一家美国公司大楼外,飘扬着中美两国国旗。(路透社档案照) 针对美国宣布对中国电动车、锂电池和半导体等产品加征高额关税一事,中国外长王毅星期三(5月15日)直指这是``当今世界上最典型的霸道霸凌''。对于美国此举是否意味着中美关系倒退,中国外交部发言人在当天例行记者会上拒绝正面回应……}

\entryitemWithDescription{赖清德就职典礼在即 中国大陆军事活动越来越接近台湾}{https://www.zaobao.com/news/china/story20240515-3669582}{台湾候任总统赖清德星期三(5月15日)在台北出席台湾资安大会。(法新社) (台北/北京综合讯)台湾政府报告称,随着候任总统赖清德5月20日就职典礼的临近,中国大陆军方最近几周在台海周边活动的航行和飞行距离,比以往任何时候都要靠近台湾,一些军机还对进入台海的外国船只进行了模拟攻击……}

\entryitemWithDescription{中美举行首次人工智能政府间对话}{https://www.zaobao.com/news/china/story20240515-3669443}{(日内瓦综合讯)中国在中美人工智能政府间对话首次会议上,就美国在人工智能领域对华限制打压表明了严正立场。 据中国外交部北美大洋洲司官方微信公号``宽广太平洋''消息,会议于当地时间星期二(5月14日)在瑞士日内瓦举行,由中国外交部北美大洋洲司司长杨涛和美国国务院关键和新兴技术代理特使森特(Seth Center)、白官国安会技术和国家安全高级主任查布拉(Tarun Chhabra)共同主持……}

\entryitemWithDescription{YouTube在香港屏蔽《愿荣光》等视频 学者料其他网络供应商会采取类似做法}{https://www.zaobao.com/news/china/story20240515-3669802}{全球最大的搜索引擎谷歌母公司Alphabet,旗下影片分享平台YouTube是全球最大的影音平台。(路透社档案照) 谷歌母公司Alphabet旗下影片分享平台YouTube,由周三(5月15日)起,限制在香港境内浏览32条与歌曲《愿荣光归香港》有关的被禁制片段连结,以回应香港高等法院日前对《愿荣光归香港》颁布临时禁制令……}

\entryitemWithDescription{中国驻英大使:英方不要在危险道路上越走越远}{https://www.zaobao.com/news/china/story20240515-3668833}{(伦敦综合讯)香港驻伦敦经贸办行政经理被控违反英国《国家安全法》后,伦敦和北京摩擦不断。中国驻英国大使郑泽光警告英国,不要在破坏中英关系的危险道路上越走越远。 路透社此前报道,英国警方星期一(5月13日)起诉三名男子,指他们涉嫌协助香港情报部门和参与外国干预,违反英国《国家安全法》。香港驻伦敦经贸办行政经理袁松彪是被告之一……}

\entryitemWithDescription{杨丹旭:中国迎来涨价周期?}{https://www.zaobao.com/news/china/story20240515-3664575}{第二季度以来,中国经济出现了一些回暖迹象,一个重要的衡量经济热度的指标------消费者价格指数(CPI)连续三个月上涨。中国国家统计局最新数据显示,4月份CPI同比上涨0.3%,涨幅较前值略有扩大,也高于预期。 去年初中国解除疫情防控,与很多国家在走出疫情后经济迅速反弹,通货膨胀困扰执政者,甚至变成政治压力不同,中国经济反而陷入低迷期。消费需求疲弱、CPI上涨乏力,倒成了一件让官方很头痛的事……}

\entryitemWithDescription{香港驻伦敦经贸办行政经理被控违反英国《国家安全法》 英中外交部掀骂战}{https://www.zaobao.com/news/china/story20240515-3664207}{香港驻伦敦经贸办行政经理袁松彪5月13日被控协助香港情报部门后,离开伦敦威斯敏斯特地方法院。(路透社) 三名被告之一的Chi Leung Wai 5月13日被控协助香港情报部门后,离开伦敦威斯敏斯特地方法院。(路透社) 伦敦和北京间的摩擦升级,继中国外交部表示``严重关切''香港驻伦敦经贸办行政经理等三人被英国起诉协助香港情报组织后,英国外交部召见中国驻英大使,称有关间谍活动``不可接受''……}

\entryitemWithDescription{路透社爆美台海军4月曾``巧遇''军演}{https://www.zaobao.com/news/china/story20240514-3664315}{美国与台湾海军据报4月在太平洋进行不对外公开、``在正式名义下不存在''的联合军演。图为台湾海军1月31日在高雄军事基地附近的水域进行演习。(路透社) (台北 / 北京综合讯)距离台湾总统就职典礼倒数五天,路透社星期二(5月14日)引述四名知情人士报道,美国与台湾海军4月在太平洋进行不对外公开、``在正式名义下不存在''的联合军演……}

\entryitemWithDescription{青海海西州中院疑``垂帘听审''被批}{https://www.zaobao.com/news/china/story20240514-3664139}{青海海西州中级法院上周六(5月11日)疑似通过微信群组实时向下级法院传达指令,遥控指挥庭审。(互联网) 中国青海一县级法院近日公开审理涉及多名被告的寻衅滋事案,辩方律师当庭拍到上级法院领导通过微信实时遥控指挥庭审。事件曝光后引发中国舆论高度关注,有网民批评上级法院是在``垂帘听审'',也有律师批评事件严重破坏中国刑事诉讼制度,质疑两审终审制可能形同虚设,被现场揭发则实属罕见……}

\entryitemWithDescription{陈水扁特赦争议 台法务部报告:未判案件不在特赦范围内}{https://www.zaobao.com/news/china/story20240514-3663977}{台湾舆论近来围绕涉贪的前总统陈水扁是否该获得特赦的争议不断。(互联网) (台北综合讯)台湾舆论近来围绕涉贪的前总统陈水扁是否该获得特赦的争议不断。台湾法务部报告说明,未判案件不在特赦范围内,也意味着若陈水扁获得特赦,他所涉尚未审结的四起案件均无法特赦,未没收款项也不在特赦范围内。 《镜周刊》上星期二(5月7日)报道,总统蔡英文决定于5月20日卸任前特赦陈水扁……}

\entryitemWithDescription{中菲合作遣返160余名在菲律宾从事离岸博彩的中国公民}{https://www.zaobao.com/news/china/story20240514-3662955}{(马尼拉综合讯)160余名在菲律宾从事离岸博彩的中国公民,星期二被中菲两国执法部门合作遣返。 根据中国驻菲律宾大使馆星期二(5月14日)通报,这是中菲两国执法部门再次合作。去年12月和今年2月,中菲执法部门已先后联手遣返180名和40余名在菲律宾从事离岸博彩的中国公民……}

\entryitemWithDescription{中国商飞正研发C939新型宽体客机}{https://www.zaobao.com/news/china/story20240514-3663745}{中国商飞现有两种飞机机型投入商业运营,其中C919型窄体客机曾在今年2月举行的新加坡航空展上亮相。(路透社) (北京综合讯)中国商用飞机有限责任公司(简称中国商飞)正研发C939新型宽体客机,以便同波音和空中客车竞争。 香港《南华早报》星期一(5月13日)报道引述知情人士指,中国商飞已勾画C939初步设计方案,但要从初步概念到制造出可进行测试的原型机,还需要多年时间……}

\entryitemWithDescription{深圳拟引入港铁东铁线 罗湖口岸将``一地两检''}{https://www.zaobao.com/news/china/story20240514-3663340}{(深圳/香港综合讯)香港铁路公司(港铁)九条铁路线之一的东铁线,未来有可能直接开进中国大陆深圳市罗湖区。罗湖口岸即将规划重建,以实现``一地两检''及50分钟车程……}

\entryitemWithDescription{安徽动物园20只东北虎死亡 当地成立调查组}{https://www.zaobao.com/news/china/story20240514-3662874}{安徽阜阳一野生动物园中一只东北虎长期被关在一间``小黑屋''里。(中国慈善家杂志微信公众号) (北京/阜阳综合讯)安徽阜阳一野生动物园被曝过去五年内20只东北虎死亡,以及大量野生动物非正常死亡,阜阳市成立调查组全面调查事件……}

\entryitemWithDescription{需求低迷 澳洲昆达士停飞上海}{https://www.zaobao.com/news/china/story20240514-3662366}{澳大利亚昆达士航空公司表示将继续密切关注澳中市场,并在需求恢复后重返上海。(路透社) (悉尼路透电)由于需求低迷,澳大利亚昆达士(Qantas)航空公司宣布将从7月28日起停飞其往返悉尼与上海的航线。 据路透社星期二(5月14日)报道,昆达士在九个月前复飞上述航线。不过,昆达士总裁华勒思(Cam Wallace)说,澳大利亚和中国之间的旅行需求并未像预期那样强劲恢复……}

\entryitemWithDescription{民进党前副秘书长林飞帆 将任台国安会副秘书长}{https://www.zaobao.com/news/china/story20240514-3662048}{民进党前副秘书长林飞帆将在5月20日后出任台湾国家安全会议副秘书长。(互联网) (台北讯)去年因绿营性骚扰案风波而退出立委选举的民进党前副秘书长林飞帆,将在5月20日后出任台湾国家安全会议副秘书长。 据《自由时报》报道,距离5月20日台湾总统就职典礼不到一周,总统副总统交接小组星期二(5月14日)再公布新人事。 在总统府和国安会这一波新人事中,今年36岁的林飞帆成为外界关注焦点……}

\entryitemWithDescription{中美人工智能政府间对话首次会议 周二将于瑞士日内瓦举行}{https://www.zaobao.com/news/china/story20240513-3657247}{中美围绕人工智能(AI)课题举行政府间对话首次会议,将于星期二(5月14日)在瑞士日内瓦召开。(路透社) (北京/华盛顿综合讯)中美围绕人工智能(AI)课题举行政府间对话首次会议,星期二(5月14日)将在瑞士日内瓦召开。美国官员指出,会议旨在降低这项新兴科技带来的风险,但强调美国不会与中国政府就针对中资企业的技术保护政策展开谈判……}

\entryitemWithDescription{蓝营智库提出国安战略法草案 应对大陆``蟒蛇困台战法''}{https://www.zaobao.com/news/china/story20240513-3655837}{台湾退役海军上将、国民党立委陈永康星期一(5月13日)在发表会提出``国家安全战略法草案''。(温伟中摄) 台湾候任总统赖清德下周宣誓就职,国民党智库国政基金会星期一(5月13日)提出``国家安全战略法草案'',为资源分配和跨部会协调提供法源基础,以应对中国大陆四路并进的``蟒蛇困台战法''……}