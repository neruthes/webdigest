\entryitemWithDescription{戴庆成:香港书展的变与不变}{https://www.zaobao.com/news/china/story20230725-1417092}{一年一度的香港书展上周三(7月19日)在湾仔会展中心开幕后,每天都吸引了大批书迷入场买书。但对我这种文字工作者来说,为期七天的香港书展更像是密集的见面会,在场内总会不期然遇上一些许久未见的同行和朋友,大家趁机相聚一番。 记得2019年下半年反修例运动爆发的时候,示威者在7月份天天上街闹事,香港书展的气氛也紧张起来……}

\entryitemWithDescription{防御中国大陆袭击 台湾启动五天汉光演习}{https://www.zaobao.com/news/china/story20230724-1417122}{全台湾22县市分阶段在星期一下午实施半小时的``万安演习防空警报'',车辆必须减速靠停、民众须疏散到附近的避难所。图为星期一下午的台北某处街道,在防空警报下车辆停止运行,街上空无一人……}

\entryitemWithDescription{李显龙总理会见李家超 讨论深化新港合作}{https://www.zaobao.com/news/china/story20230724-1417119}{李显龙总理(右)星期一(7月24日)在总统府会见到访的香港特首李家超。(海峡时报) 我国总理李显龙与香港行政长官李家超讨论,进一步深化新港在贸易与投资、创新与研究等领域的双边合作。李总理也表示,有信心香港会在``一国两制''下持续发展与繁荣。 李总理星期一(24日)与到访我国的李家超会面,并设午宴款待李家超。 根据外交部文告,李总理和李家超重申新港密切和长久的关系……}

\entryitemWithDescription{齐齐哈尔一中学体育馆坍塌致11死4伤 初步调查:施工商违规置放建筑材料}{https://www.zaobao.com/news/china/story20230724-1417086}{7月24日无人机拍摄的齐齐哈尔市第34中学体育馆楼顶坍塌事故搜救现场。事故共造成11人死亡,相关调查工作正在全面推进中。(新华社) 中国黑龙江省齐齐哈尔市一所中学的体育馆,星期天(7月23日)发生屋顶坍塌事故,导致11人死亡四人受伤。现场搜救工作在星期一上午结束,相关责任人已被警方拘留。 坍塌的体育馆位于齐齐哈尔市第三十四中学内。现场画面显示,体育馆整个屋顶倒塌,仅剩墙体,馆内一片废墟……}

\entryitemWithDescription{中国大陆海警台海执法常态化 近两月二度跨越海峡中线}{https://www.zaobao.com/news/china/story20230724-1417084}{台湾海军前舰长吕礼诗指出,中国大陆海警船近两个月来二度跨越海峡中线,显示大陆海警正将台湾海峡逐渐纳为``管辖海域'',将海峡执法权``常态化''。 据台湾《自由时报》报道,吕礼诗星期一(7月24日)发文指出,大陆海警舷号``2501''、排水量5000吨的大型海警船,近两个月来二度跨越海峡中线,在澎湖群岛西南海域活动……}

\entryitemWithDescription{中俄结束日本海联合军演}{https://www.zaobao.com/news/china/story20230724-1417068}{中国和俄罗斯结束在日本海举行的``维护海上战略通道安全''的联合军演。在四天的演习中,中俄双方派出10多艘舰艇和30多架飞机。 根据中国国防部官网,在名为``北部·联合-2023''演习中,中俄双方海空兵力联合筹划、指挥、行动,先后进行了海空护航、威慑驱离、锚地防御等多个课目的实兵演练,检验了远海远域实战能力……}

\entryitemWithDescription{打造摇滚之城 石家庄能否脱颖而出?}{https://www.zaobao.com/news/china/story20230724-1417066}{石家庄世纪公园7月22日举行摇滚之城户外免费演唱会,至少上千人聚集在公园草坪,台上的儿童摇滚乐队点燃现场气氛。(黄小芳摄) ``你该不会是专程来找摇滚演唱会的吧?'' 听闻记者从北京到石家庄找免费摇滚演出,出租车司机闫大哥(化名,47岁)显得惊讶;当记者询问石家庄的摇滚氛围是否浓厚,他笑着说,``不搞点噱头怎么吸引你来,但有活动的时候确实挺嗨''……}

\entryitemWithDescription{日本欢迎中国军事院校学生赴日交流}{https://www.zaobao.com/news/china/story20230724-1417029}{美国已将来自中国军事相关院校的学生拒之门外,但其盟友日本却非常欢迎这些中国留学生。 据《华尔街日报》星期一(7月24日)报道,日本一些政府官员和学术界管理人士表示,很高兴有学生在美国收紧限制之后,把目光投向美国之外的选项,并指出知识交流和科学交流自由的重要性……}

\entryitemWithDescription{于泽远:``31条''能否提振民企信心?}{https://www.zaobao.com/news/china/story20230724-1416752}{中国高层7月19日发布《关于促进民营经济发展壮大的意见(以下简称``民营经济31条'')》后,网络名人司马南的压力陡增,因为他前段时间大力指控著名企业联想集团涉嫌造成巨额国有资产流失,被不少人视为打压民企的``黑手''。 司马南7月22日发文辩解说:``除了对联想09年改制涉嫌巨额国有资产流失问题的揭露,我打压过哪一家子民企呢……}

\entryitemWithDescription{国民党正式提名侯友宜参选 蓝军选情与在野整合仍充满挑战}{https://www.zaobao.com/news/china/story20230723-1416818}{国民党7月23日全代会再现最强三战将``秃子、汉子、燕子''(即前排左起的韩国瑜、侯友宜和台中市长卢秀燕)同台互挺画面,相信有助凝聚蓝营军心。(国民党提供照片) 国民党全台党代表大会(全代会)星期天正式提名新北市长侯友宜参选台湾总统,高人气的前高雄市长韩国瑜热情相挺。但未获征召的鸿海集团创办人郭台铭,会前在脸书贴文称``民意永远大于党意'',蓝军选情与在野整合依然充满挑战……}

\entryitemWithDescription{李家超展开三天访新之行 时隔六年再迎来港特首}{https://www.zaobao.com/news/china/story20230723-1416796}{我国国家发展部兼外交部高级政务部长沈颖(左)星期天(7月23日)在机场迎接到访的香港特首李家超。(外交部提供) 香港特首李家超星期天(7月23日)抵达新加坡展开三天访问。这是李家超出任香港特首后,首次访问新加坡,期间将与李显龙总理会面。 据新加坡外交部文告,李家超是应李总理的邀请,在7月23日至25日访问我国。 李家超将出席李总理所设的午宴,副总理兼财政部长黄循财也将与李家超共进早餐……}

\entryitemWithDescription{中国驻日使馆批北约介入亚太贩卖安全焦虑}{https://www.zaobao.com/news/china/story20230723-1416790}{针对北大西洋公约组织指中国挑战北约的安全与价值观,中国驻日本大使馆回应批评北约贩卖安全焦虑,唯恐天下不乱。 据中新社报道,中国驻日本大使馆发言人星期六(7月22日)就北约东进亚太发表谈话,指近期以来北约持续介入亚太事务,挑动阵营对抗,引起地区国家高度警惕……}

\entryitemWithDescription{浙江杭州富阳暴雨引发山洪 已致五死三失联}{https://www.zaobao.com/news/china/story20230723-1416782}{浙江省杭州市富阳区7月22日傍晚至23日凌晨,短时强降雨引发山洪致局部山体塌方和泥石流,已致五人死亡,三人失联。(中新社) 中国浙江省杭州市富阳区部分区域突降暴雨引发山洪,截至星期天(7月23日)下午2时,已致五人死亡、三人失联。 据``富阳发布''公众号公告,富阳区22日傍晚至23日凌晨短时强降雨引发山洪,致局部山体塌方和出现泥石流……}

\entryitemWithDescription{法国外交顾问称中国正向俄罗斯输送军事装备}{https://www.zaobao.com/news/china/story20230723-1416780}{法国外交顾问称,中国正在向俄罗斯提供可以用作军事装备的物品,这些物品也可用在俄乌战争。 综合路透社和法新社报道,法国总统马克龙首席外交政策顾问博纳(Emmanuel Bonne)星期四(7月20日)在巴黎举行的阿斯彭安全论坛上被问及西方是否看到任何证据,显示中国以任何形式为俄罗斯提供武装时,作出上述回应。 他说:``有迹象表明,他们正在做一些我们不希望他们做的事……}

\entryitemWithDescription{西安回流生事件警方再抓六人}{https://www.zaobao.com/news/china/story20230723-1416778}{中国西安回流生事件热度持续。继上周控制涉案20余人后,西安警方上星期六(7月22日)再抓获犯罪嫌疑人六名,刑事拘留五人。 陕西省西安市公安局官方微博星期天(23日)通报,西安市一所补习学校通过伪造印章,为外省回流生在西安参加中考提供虚假资格审查材料,涉嫌犯罪。 该事件引发社会关注,是因为网络传言称西安2023年10万中考生中,有多达4万名回流生……}

\entryitemWithDescription{排解焦虑 中国年轻人刮起算命风潮}{https://www.zaobao.com/news/china/story20230723-1416746}{中国刮起算命风潮,年轻人聚集的市集里,时常出现塔罗牌占卜摊位。图为摄于北京三里屯的一个临时市集。(黄小芳摄) 25岁的郭欣(化名)向来坚决不信命理,但她在报考研究生和求职受挫后,首次主动花钱算了八字------对她而言,算命不是迷信,而是排解内心焦虑的一种心理咨询。 在过去三年的疫情中,中国的命理产业经历一波爆炸式发展和重塑,算命不再是年长者的迷信活动,也成为备受年轻人追捧的潮流文化……}

\entryitemWithDescription{香港特稿:港人大湾区养老须跨越重重难关}{https://www.zaobao.com/news/china/story20230723-1416003}{在东莞黄江康湖护老院,医护人员陪同年长者进行护理。(林煇智摄) 粤港两地融合发展不断深化,香港政府这几年加大政策力度推进跨境养老,以破解香港人口老龄化问题;但粤港两地在医疗、福利保障制度上的差异,长期是香港年长者北上养老最大的顾虑。这样的情况是否已经改善?跨境养老还面临哪些挑战? 95岁的陈素贞五年前在步入鲐背之年时做了一个决定,从香港到广东东莞养老……}

\entryitemWithDescription{反间谍法引发外企忐忑 中国商务部出面安抚}{https://www.zaobao.com/news/china/story20230723-1416549}{本月生效的新版《反间谍法》一度引发在华外企不安。(路透社) 北京的反间谍行动数月来在驻华外企中引发寒蝉效应,中国商务部罕见地召集在华外国商会和企业代表,解释本月生效的新版《反间谍法》,以稳定在华外资信心。 受访学者指出,中国与美国等西方国家博弈加剧下,外企担心执法的模糊性,会让他们在华的经营活动成为地缘政治斗争的牺牲品,而单靠经济部门安抚,难以完全打消疑虑……}

\entryitemWithDescription{足坛反腐:中国足协又两位高层被查}{https://www.zaobao.com/news/china/story20230722-1416547}{中国足协技术部部长谭海(左)和战略规划部部长戚军涉嫌严重违纪违法被查。(互联网) 继11名中国足坛重磅人物相继落马后,中国足球协会两位部长谭海和戚军涉嫌严重违纪违法被调查。 据湖北省纪委监委网站星期六(7月22日)通报,中国足球协会技术部部长谭海和战略规划部部长戚军涉嫌严重违纪违法,目前正接受中央纪委国家监委驻国家体育总局纪检监察组和湖北省监委审查调查……}

\entryitemWithDescription{借鉴俄乌战况 台湾研发无人机备战}{https://www.zaobao.com/news/china/story20230722-1416545}{台湾雷虎科技公司的一名员工3月30日在嘉义的亚洲无人机人工智能创新应用研发中心搬运一台无人机进行演示。(路透社) (台北综合电)台湾借鉴俄乌战争中乌克兰用无人机获得空中优势的经验,正加快组建一支无人机国家队,缩小与中国大陆的无人机差距。 据路透社星期五(7月21日)报道,去年夏天在俄乌战争爆发数月后,台湾总统蔡英文曾召集民进党高官,探讨乌克兰是如何成功应对像俄罗斯这样的一个强敌……}