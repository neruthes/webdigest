\entryitemWithDescription{戴庆成:香港出现尹光热潮的背后}{https://www.zaobao.com/news/china/story20231122-1451493}{近年香港乐坛风光不再,已经有很长时间没有出现令人眼前一亮的新晋歌手。反而是74岁的老牌歌星尹光,在最近一段日子突然愈发抢手,不仅频频亮相电视台表演,并再次在香港歌坛最具标志意义的演出场地红馆开唱,更首次成为今年香港商业电台的''叱咤乐坛我最喜爱的男歌手''比赛的十强候选人之一,成为全城热话。 说起尹光,现在的许多年轻人未必熟悉,但他本身的经历就是一个传奇……}

\entryitemWithDescription{四川公司将火锅汤油转成潜在飞机燃料}{https://www.zaobao.com/news/china/story20231121-1451516}{成都市一家餐厅的员工正在清理一锅火锅汤,这些油汤将被回收再利用。此图摄于10月21日。(法新社) (成都法新电)中国四川一家公司将火锅剩油再循环,转化成潜在飞机燃料,并运销国际。 据法新社星期二(11月21日)报道,中国企业四川金尚环保找到一个利用废弃火锅油并出口转化为航空燃料的商机。公司总经理叶彬说,公司的口号就是``让地沟油飞起来''……}

\entryitemWithDescription{中国国足世预赛不敌韩国 在球迷预料之中}{https://www.zaobao.com/news/china/story20231121-1451514}{中国足球队星期二(11月21日)晚上在深圳世界大学生运动会中心体育场举行2026年世界杯亚洲区预选赛36强C组第二场比赛,迎战韩国队。图为前来打气的一名中国球迷在赛前展示``中国足球不恐韩''的条幅。(路透社) 中国男子足球队在世界杯第二场预选赛中,主场以0比3不敌韩国队。针对长期表现不佳的中国队落败,大多数中国球迷表示在意料之中,并认为中国男足要达到世界一流水平,还有很长的路要走……}

\entryitemWithDescription{赖萧配称不走``一中''老路 蓝白仍在``吵''}{https://www.zaobao.com/news/china/story20231121-1451510}{民进党总统参选人赖清德暨副总统参选人萧美琴,星期二(11月21日)前往中央选举委员会完成登记。(中新社) 台湾执政的民进党总统副总统参选人赖清德和萧美琴,星期二(11月21日)前往中央选举委员会完成登记。赖清德强调,两人对台湾的未来方向有共同想法,要让台湾走入国际,不走回一中老路……}

\entryitemWithDescription{王乙康吁新中青年把握使命 打造充满希望新亚洲}{https://www.zaobao.com/news/china/story20231121-1451485}{在全球化时代,抵御外来文化影响是不切实际的。卫生部长王乙康在新中论坛上致辞时说,海纳百川是今时今日非常重要的特性,更是一种文化自信的表现。 由《联合早报》举办的第五届新中论坛,星期二(11月21日)在北京举行,论坛主题为``新中青年成长的挑战与机遇''……}

\entryitemWithDescription{传解放军出现第五军种 组建高超音速近空指挥部队}{https://www.zaobao.com/news/china/story20231121-1451479}{(香港讯)据中国军方科研论文披露,解放军组建了世界上首个近空指挥部队,战时负责执行``精准无情''的攻击任务。 香港《南华早报》星期一(11月20日)引述一篇论文报道,解放军继陆、海、空和火箭军外,出现第五个新军种。中国国防科技大学研究人员透露,中国建立了一支配备专业高超音速武器、直接听命于解放军最高层级的近空作战指挥部队。 目前尚不清楚该支部队确切建立的时间,《南华早报》还无法独立核实以上消息……}

\entryitemWithDescription{学者:处理青年就业问题 得消除就业壁垒}{https://www.zaobao.com/news/china/story20231121-1451476}{中国大学生面对的就业环境比过去困难,最重要是使用政策资源,消除大学生面对的就业壁垒,并在大学生找工作期间给予最低的生活保障。 中国青少年研究中心研究员郗杰英星期二(11月21日)在《联合早报》举办的第五届新中论坛上,参与圆桌讨论环节,并做出上述分析。 圆桌讨论环节题为``新中青年成长的烦恼''。郗杰英指出,中国年轻人目前面对的四大烦恼,即学业重、就业难、成家不易、养育孩子很艰辛……}

\entryitemWithDescription{王乙康:中国目前情况和1980年代的日本截然不同}{https://www.zaobao.com/news/china/story20231121-1451475}{卫生部长王乙康(台上左二起)在新中论坛上,与VIPKID创始人兼首席执行官米雯娟和澎湃新闻总裁兼总编辑刘永钢对谈。对谈环节由新报业媒体华文媒体集团社长李慧玲(左一)主持。(联合早报) 卫生部长王乙康说,中国目前情况和上世纪80年代的日本截然不同,接下来若能减少对房地产业过度依赖、促进国内消费、投资和发展科技,并继续推进改革开放,没有理由对中国未来经济感到沮丧……}

\entryitemWithDescription{福建舰暂离码头 专家:可望今年执行海试}{https://www.zaobao.com/news/china/story20231121-1451471}{(香港/纽约综合讯)最新卫星图像显示,中国第三艘航母福建舰星期天(11月19日)短暂离开港池码头,军事专家称,该舰有望今年执行首次海试,不会拖到明年。 据《明报》报道,美国军事专家沙顿(H I Sutton)发布的卫星图像显示,福建舰当日短暂离开港池码头,可能是执行倾斜试验或清理之前船底的淤泥,而且外侧航道的上海吴淞口零号锚地船隻也被清空大半……}

\entryitemWithDescription{缅甸向中国移交超过三万名电信网络诈骗嫌犯}{https://www.zaobao.com/news/china/story20231121-1451461}{缅甸北部木姐地区执法部门上星期六(11月18日)将571名电信网络诈骗嫌犯,陆续移交给中国云南省德宏公安机关。(中新社) (北京综合讯)中国公安部星期二(11月21日)宣布,经过多轮打击行动,缅甸北部执法部门累计向中方移交3万1000名电信网络诈骗犯罪疑犯,``打击工作取得显著成果''……}

\entryitemWithDescription{庄慧良:``蓝白合''还有胜算吗?}{https://www.zaobao.com/news/china/story20231121-1451288}{台湾执政的民进党总统参选人赖清德和驻美代表萧美琴组成的``赖萧配'',星期一开心亮相。但在野国民党(蓝)总统参选人侯友宜和民众党(白)总统参选人柯文哲,为了民调误差范围认知差异几近分手,外界高度关注两人能否在星期五(11月24日)中央选举委员会登记截止前搭档参选。 国民党和民众党是两个完全不同的政党,套句柯文哲的话,两党的DNA完全不同,合作本非易事……}

\entryitemWithDescription{民进党赖萧配登场 在野蓝白副手未决}{https://www.zaobao.com/news/china/story20231120-1451283}{台湾现任副总统、民进党主席兼总统参选人赖清德(左),星期一(11月20日)下午在台北的竞选总部记者会,正式宣布副手为台湾驻美代表萧美琴。(法新社) 民进党总统参选人赖清德星期一(11月20日)下午正式宣布,他的副总统搭档为台湾驻美代表萧美琴。赖萧配星期二(21日)率先登记,在野蓝白两党的副手仍未敲定。若蓝白合本周内破局,赖萧配备受看好能延续民进党的执政权……}

\entryitemWithDescription{两岸关系低迷 香港赴台学生人数骤降}{https://www.zaobao.com/news/china/story20231120-1451241}{(香港 / 台北综合讯)据台媒报道,当前两岸关系低迷,香港特区政府配合北京,限制台湾招生资讯进入香港校园,赴台就读的香港新生人数近年急剧下降。 《旺报》星期一(11月20日)引用台湾教育部十年数据报道,在马英九执政的2015年,赴台的香港学士与硕博士新生有5038人。2016年蔡英文上任后,这一数字逐年下降。不过在2020年,赴台香港学生人数突然大幅增长,2021年更达到6037人,创历史新高……}

\entryitemWithDescription{美共和党议员促军方履行对台军援承诺}{https://www.zaobao.com/news/china/story20231120-1451239}{(华盛顿彭博电)一群美国共和党议员警告说,台湾向美国采购的F-16战斗机项目面临进一步延误交付的``高风险''。他们敦促美国国防部``集中精力'',履行对台湾的军援承诺。 据彭博社报道,逾20名共和党众议员上星期五(11月17日)致函美国空军部长弗兰克·肯德尔(Frank Kendall),在信中赞赏拜登政府设法加快交付台湾军机项目的速度,但担心过去存在的生产与转让困难可能会持续下去……}

\entryitemWithDescription{中国据报拟在阿曼建军事基地 分析:或增加美处理台海问题的复杂性}{https://www.zaobao.com/news/china/story20231120-1451228}{(华盛顿综合讯)中国据报有意在阿曼兴建军事设施,分析认为,这可能迫使美国不得不重新评估在中东地区乃至全球的军事战略布局与资源配置,或增加其处理台海问题的复杂性。 美国之音星期天(11月19日)的报道引述台湾国策研究院资深顾问陈文甲说,阿曼基地可被视为中国在非洲吉布提后,补充和扩展其在印太地区的军事存在,进一步增强其全球军事布局,可能会对美国在中东地区的军事优势构成挑战与威胁……}

\entryitemWithDescription{分析:中国将持续以非军事手段 维持南中国海主动权}{https://www.zaobao.com/news/china/story20231120-1451220}{中菲两国海警船与民用船只在10月22日发生碰撞事件以来,双方在南中国海的摩擦持续。图为一艘中国海警船11月10日在南中国海有争议的阿云津礁(中国称仁爱礁),靠近执行补给任务的菲律宾海警船……}

\entryitemWithDescription{中国初婚人数九年间大减 去年首次不足1100万}{https://www.zaobao.com/news/china/story20231120-1451196}{(北京综合讯)中国的结婚人数近年持续下降,结婚年龄也普遍推迟,初婚人数从2013年到2022年已大减逾半。 第一财经星期天(11月19日)引述《中国统计年鉴2023》报道,中国2022年全年结婚登记数降至683.5万对,较上年减10.6\%。初婚人数为1051.76万人,比2021年减少9.16\%至106.04万人,这也是多年来初婚人数首次低于1100万人……}

\entryitemWithDescription{中国首只全流程国产克隆猫诞生}{https://www.zaobao.com/news/china/story20231120-1451192}{中国首只全流程采用国产设备、试剂和耗材培育的体细胞克隆猫星期天(11月19日)在青岛农业大学哺乳动物体细胞克隆基地诞生。(中新社) (青岛综合讯)中国首只全流程采用国产设备、试剂和耗材培育的体细胞克隆猫诞生,标志着中国在动物克隆领域拥有完整的产业链。 综合《科技日报》和青岛新闻报道,青岛农业大学星期天(11月19日)发布消息说,克隆猫当天在该校的哺乳动物体细胞克隆基地诞生……}

\entryitemWithDescription{美研究:中国政策性银行正将绿色能源拒之门外}{https://www.zaobao.com/news/china/story20231120-1451044}{(波士顿讯)美国研究发现,尽管中国曾承诺向发展中国家提供更多绿色和低碳能源项目的资金支持,但其政策性银行尚未实现这一承诺。 《南华早报》星期天(11月19日)报道,波士顿大学全球发展政策中心研究发现,包括中国进出口银行和中国国家开发银行在内的中国政策性银行,在2021至2022年间并未向能源部门注入资金……}

\entryitemWithDescription{国民党推韩国瑜领军不分区立委 盼柯文哲回心转意任侯友宜副手}{https://www.zaobao.com/news/china/story20231119-1451043}{台湾在野国民党(蓝)星期天举行临时中常会,通过以高雄市前市长韩国瑜领军的不分区立委名单,希望民众党(白)总统参选人柯文哲回心转意,担任国民党总统参选人侯友宜的副手,也要抢回支持柯文哲和鸿海集团创办人郭台铭的韩粉票源。 民众党同日(11月19日)也举行誓师大会,柯文哲回应支持群众的热烈要求,强调会继续用台湾民众党总统候选人的身份拼战到底……}