\entryitemWithDescription{杨丹旭:中国防长去向的谜团}{https://www.zaobao.com/news/china/story20230920-1434993}{中国国务委员兼防长李尚福已连续三周没有公开露面,他的去向成为中国政坛最新的谜团。 路透社上周四(9月14日)报道,李尚福原定要在9月7日至8日参加一场在中越边境召开的防务合作年度会议,但越南在会前被告知,因为李尚福的``健康状况'',得推迟这次会议。 越南方面对这项活动颇为重视,当地媒体8月底曾报道,除了双方代表团会晤,原本安排的议程还包括在两国的口岸互为双方代表团举行欢迎仪式、种植纪念树等等……}

\entryitemWithDescription{学者:王毅在莫斯科的表态显示 中国不寻求与俄罗斯结盟来对抗美国}{https://www.zaobao.com/news/china/story20230920-1435007}{俄罗斯外长拉夫罗夫(前排左)星期一(9月18日)在莫斯科会见到访的中国外长王毅(前排右)。(中新社) 中俄元首今年10月将在北京会晤,两国外长星期一(9月18日)先在莫斯科会面。中国外长王毅呼吁中俄继续加强战略协作,但强调双方合作不针对第三方;俄外长拉夫罗夫则称,中俄对美国在国际场合中的反俄和反华行动``立场非常接近''……}

\entryitemWithDescription{台农业部长为进口蛋争议请辞获准 在野党指民进党政府想避``仲''救``清''}{https://www.zaobao.com/news/china/story20230919-1434997}{(台北综合讯)台湾进口蛋争议不断,农业部长陈吉仲星期二(9月19日)晚间第二次向行政院长陈建仁请辞获准,将从星期四(21日)起生效。 据台湾行政院新闻稿,行政院发言人林子伦透露,陈吉仲近日多次向陈建仁表达歉意,称今年初为因应禽流感导致台湾鸡蛋供给不足,而启动进口鸡蛋专案,达到稳定供给、稳定物价、稳定蛋农生计之政策效益……}

\entryitemWithDescription{【早知】围绕元首会晤中美双方最关切的问题}{https://www.zaobao.com/news/china/story20230919-1434973}{继去年11月的峇厘岛会晤后,中美元首今年内最可能会面的机会就是11月的亚太经济合作组织(APEC)峰会期间。目前距离APEC峰会还有不到两个月的时间,双方高层三天内接连进行了两轮``坦诚''\,``建设性''的讨论,中国还以``战略沟通''来形容王毅与沙利文上周末的会晤。 哪些是中美从现在到元首会晤时最需要``沟通''的突出问题……}

\entryitemWithDescription{中国要求各国驻港总领馆一个月内提供当地雇员资料}{https://www.zaobao.com/news/china/story20230919-1434971}{(香港综合讯)中国外交部驻港特派员公署星期一(9月18日)向所有驻港总领事馆发信,要求各领馆在一个月内,申报在香港聘用的本地雇员资料。 综合路透社、《明报》、英文网媒``香港自由新闻''报道,公署外交信函中称,须申报资料的领事馆香港本地雇员,是与驻港总领馆订立雇佣合约的人员,包括香港永久性居民,或持任何签证的香港非永久性居民……}

\entryitemWithDescription{布林肯与韩正会谈 中美未来几周将继续高层接触}{https://www.zaobao.com/news/china/story20230919-1434962}{美国国务卿布林肯(左)与中国国家副主席韩正(右)在美东时间9月18日下午于纽约举行会谈。(新华社) 美国国务卿布林肯与中国国家副主席韩正会面,美国称双方重申将致力保持开放的沟通渠道,包括在下来几周进行后续的高层接触。韩正则强调,中国的发展对美国是机遇而不是挑战,敦促美国推动中美关系重回健康稳定轨道……}

\entryitemWithDescription{马赛地首席执行官:与中国脱钩将带给全球汽车供应链风险}{https://www.zaobao.com/news/china/story20230919-1434958}{(纽约彭博电)欧盟正对中国电动车发起反补贴调查之际,德国马赛地---奔驰集团董事会主席康林松(Ola Kallenius)说,追求与中国脱钩将给良性竞争和全球汽车供应链带来风险。 康林松在纽约接受彭博电视采访时说,开放市场是驱动增长和创造财富的动力,``让我们保持市场开放,让市场参与者为之奋斗''……}

\entryitemWithDescription{港社协调查:李家超表现得分下滑}{https://www.zaobao.com/news/china/story20230919-1434946}{(香港综合讯)在香港特首李家超下月发表《施政报告》之前,香港社区组织协会一份调查发现,基层市民对李家超的评分较去年下滑。 综合``香港01''、《信报》报道,港社协星期一(9月18日)发布的调查显示,以10分为满分,李家超今年的平均评分为6.4分,相比去年他上任后的6.8分,低了0.4分。 上述调查于8月16日至本月11日在网上进行,共访问644人……}

\entryitemWithDescription{【东谈西论】中国防长去哪儿?}{https://www.zaobao.com/news/china/story20230919-1434898}{中国国防部长李尚福已有三个星期没有公开露面,这是继中国前外交部长秦刚之后,第二个在公众视线中突然消失的中国高层官员。(法新社) 中国国防部长李尚福已经有三个星期没有公开露面。据英国媒体报道,李尚福没有出席 9月7日至8日在中越边境主办的防务合作年度会议。美国政府也认为李尚福正在接受调查,而且已经被解除职权……}

\entryitemWithDescription{民众党暂停高虹安党权 直至她获判无罪}{https://www.zaobao.com/news/china/story20230919-1434909}{(新竹综合讯)台湾在野的民众党籍新竹市长高虹安,日前因被控担任立法委员期间诈领助理费,而被检察官依违反《贪污治罪条例》起诉。民众党星期二宣布,在判决她无罪之前,暂停她的党权。对此,高虹安回应表示,一切以市政优先……}

\entryitemWithDescription{24小时103架次大陆军机台海出没 专家:台军须落实防卫战略}{https://www.zaobao.com/news/china/story20230918-1434662}{在去年7月的长春航空展,中国大陆空军展示吸睛的``军机天团''表演,让运油20(左)同时在空中为歼20匿踪战斗机(中)和歼16型歼击机(右)加油。(互联网) 台湾国防部星期一(9月18日)侦察到24小时内有103架次中国大陆军机在台海出没,创历来新高。受访专家分析,大陆空军已具备突破第一岛链的远程打击续航战力,台军须落实防卫战略,避免擦枪走火并加强吓阻力……}

\entryitemWithDescription{电动车贸易摩擦升级 王毅:中欧应坚定摒弃保护主义}{https://www.zaobao.com/news/china/story20230918-1434638}{中国外长王毅(左)上星期六(9月16日)在瓦莱塔会见马耳他外长博奇。(新华社) 欧盟与中国在电动车产业的贸易摩擦近期持续升级,中国外交部长王毅上星期六(9月16日)在欧洲呼吁中欧摒弃保护主义,坚定支持自由贸易。受访学者分析,王毅的最新表态意在劝说欧盟不要在电动车领域与中国开打贸易战,同时也表达对欧洲加速对华去风险化的担忧……}

\entryitemWithDescription{调查:年龄歧视难租屋、行人地狱难通行 台湾长者的痛点}{https://www.zaobao.com/news/china/story20230918-1434630}{台湾老龄化人口将于2025年超过20\%迈向超高龄社会。台湾人寿与大数据公司``网路温度计''星期一(9月18日)公布``2023高龄友善大调查'',结果显示``年龄歧视难租屋''、``行人地狱难通行'',是台湾老年人长期以来的痛点……}

\entryitemWithDescription{苹果中国官网``辫子客服''引辱华争议}{https://www.zaobao.com/news/china/story20230918-1434625}{苹果中国网站的客户服务人员图片留有长辫子,被部分中国网民认为图片涉嫌辱华,引发争议。(互联网) (上海/北京/长沙综合讯)苹果中国网站一张客户服务人员图片留有长辫子,有中国网民认为图片涉嫌辱华,引发网络争议。中国媒体指出,该客服人员形象并非仅在中国官网存在,而且这名员工是印第安人。 据《新闻晨报》报道,这张陷入争议的照片,来自苹果网站手表专家一对一选购页面;照片中,客服人员留着一条黑色的长辫子……}

\entryitemWithDescription{拐卖11名儿童 贵州人贩子余华英一审判死刑}{https://www.zaobao.com/news/china/story20230918-1434579}{(贵阳综合讯)中国贵州省拐卖11名儿童案备受瞩目,贵阳市法院一审以拐卖儿童罪判处被告人余华英死刑。 贵阳市中级法院官方微信公众号发布消息称,该院星期一(9月18日)公开宣判被告人余华英犯拐卖儿童罪一案,一审判处余华英死刑,剥夺政治权利终身,并处没收个人全部财产。余华英当庭表示要上诉……}

\entryitemWithDescription{于泽远:中美竞相拉拢越南}{https://www.zaobao.com/news/china/story20230918-1434303}{美国总统拜登9月10日访问越南后,中国总理李强9月16日在广西南宁会见了越南总理范明政。 美越、中越高层在一周内相继会面,显示美国和中国都在积极发展对越关系,越南也乐于同美中两个大国密切互动,并从中受益。 拜登访越期间与越共总书记阮富仲宣布,将美越关系从``全面伙伴关系''提升至``全面战略伙伴关系''。此前,越南仅与中国、俄罗斯、印度和韩国建立了``全面战略伙伴关系''……}

\entryitemWithDescription{刘燕玲:期待亚细安和中国自贸区升级后 可实现更深入经济合作}{https://www.zaobao.com/news/china/story20230917-1434340}{新加坡贸工部兼文化、社区及青年部政务部长刘燕玲(左五)星期天(9月17日)在南宁出席中国---东盟博览会开幕式,并参观新加坡展馆。新加坡太平船务执行主席张松声(左六)在展馆向她介绍公司在中国的业务。(新加坡工商联合总会提供) 新加坡贸工部兼文化、社区及青年部政务部长刘燕玲说,亚细安和中国牢固的伙伴关系,源自于双方为各自人民与企业利益而携手合作的共同信念……}

\entryitemWithDescription{李强:对未来有些``隐忧和焦虑'' 中国亚细安需下更大功夫实践亲诚惠容}{https://www.zaobao.com/news/china/story20230917-1434337}{中国国务院总理李强星期天(9月17日)出席第20届中国---东盟博览会开幕式致词时说,对中国与亚细安的未来充满信心和憧憬,也有一些隐忧和焦虑。 (林煇智摄) 中国国务院总理李强指出,中国与亚细安的关系已成为亚太区域合作中最成功和最具活力的典范。但他也不讳言,对未来也有一些``隐忧和焦虑'',呼吁各国在践行``亲诚惠容''下更大功夫……}

\entryitemWithDescription{深圳每天拦数十名可疑偷渡人员 中年人居多}{https://www.zaobao.com/news/china/story20230917-1434336}{(深圳/香港综合讯)香港媒体报道,深圳边防检查部门目前每天要拦下数十名经香港转飞南美洲偷渡美国的中国大陆人士,其中以失业或生意失败的中年人居多。 香港《明报》星期六(9月16日)引述广东省一名接近边检知情人士报道,最近有许多大陆人经深圳前往香港,其中一个目的是经香港飞去第三地,辗转偷渡到墨西哥。 报道称,由于厄瓜多尔对中国公民实施免签政策,因而成为偷渡美国的跳板……}