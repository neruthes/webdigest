\entryitemWithDescription{AI热潮大赢家台积电业绩超预期 带动台湾科技业}{https://www.zaobao.com/news/china/story20240418-3465386}{全球最大晶片制造商台积电预估,第二季营收将比去年同期大增27.6\%。图为去年7月5日,公众在新竹科学园区的台积电创新馆观看晶圆介绍视频。(法新社) 因人工智能(AI)半导体需求激增,全球最大晶片制造商台湾积体电路制造公司(简称台积电)今年第一季净利同比增长8.9\%,超出市场预期。受访专家指出,AI浪潮下,台湾以台积电为中心的供应链上中下游厂商都将获益,前景备受看好……}

\entryitemWithDescription{民进党中生代``大阿哥''郑文灿 据报任海基会董事长}{https://www.zaobao.com/news/china/story20240418-3466657}{受台湾政府委托处理两岸事务的海峡交流基金会(简称海基会)董事长一职,据报将由台湾行政院副院长郑文灿出任。(互联网) (台北综合讯)受台湾政府委托处理两岸事务的海峡交流基金会(简称海基会)董事长一职,据报将由台湾行政院副院长郑文灿出任。 《镜周刊》星期三(4月17日)报道,台湾候任总统赖清德阵营的内部已锁定,海基会董事长将推派郑文灿出任……}

\entryitemWithDescription{中国整体就业形势稳定 25至29岁非在校生青年失业率上升}{https://www.zaobao.com/news/china/story20240418-3466669}{中国官方数据显示,3月份中国城镇不含在校生的16岁至24岁劳动力失业率为15.3\%,与前一个月持平,但25岁至29岁劳动力失业率上升,显示青年人就业压力仍然存在。图为人们2月19日在中国河南省郑州市参加一场招聘会……}

\entryitemWithDescription{区域竞争加剧 香港港口吞吐量首次跌出全球十大之列}{https://www.zaobao.com/news/china/story20240418-3466051}{曾连续11年蝉联第一的香港首次跌出全球港口吞吐量排行前十。图为堆放在香港葵青货柜码头的集装箱。(彭博社) 香港货柜码头曾连续11年蝉联全球港口货柜吞吐量排行第一,但近年不断萎缩,去年更首次跌至全球第11位。受访学者指出,香港航运成本高,除非港府降低土地税,否则有关问题短期内难以改善……}

\entryitemWithDescription{布林肯据报将在下周访华四天}{https://www.zaobao.com/news/china/story20240418-3466044}{美国国务卿布林肯据报将在下个星期访华四天。图为布林肯4月18日在意大利卡普里岛出席七国集团外长会议第二天的会议。(路透社) (华盛顿/北京综合讯)继美国财政部长耶伦等美国高级官员后,美国国务卿布林肯据报将在下个星期访华四天。这是他时隔不到一年再度访华……}

\entryitemWithDescription{赖清德入选《时代》周刊百大最具影响力人物}{https://www.zaobao.com/news/china/story20240418-3465067}{美国《时代》周刊公布2024年100位最具影响力人物名单,台湾候任总统赖清德入选。图为赖清德今年4月10日在台北举行的新闻发布会上讲话。(法新社) (台北/纽约综合讯)美国《时代》周刊公布2024年100位最具影响力人物名单,台湾候任总统赖清德入选……}

\entryitemWithDescription{陈婧:中国经济乍暖还寒}{https://www.zaobao.com/news/china/story20240418-3461490}{中国经济今年第一季迎来``开门红'',打破外界的悲观预期。但随之而来的疑问,是这样的强劲表现能持续多久? 中国国家统计局星期二(4月16日)发布的数据显示,第一季中国国内生产总值(GDP)同比增长5.3\%。这不仅高于去年第四季的5.2%,更远超高盛、摩根士丹利等投行,以及路透社和彭博社调查分析师的预测,为实现5%左右的全年增长目标开了个好头……}

\entryitemWithDescription{中美防长通话后数小时 美军机飞越台海}{https://www.zaobao.com/news/china/story20240418-3461732}{美国海军第七舰队发布的声明称,一架P-8A``海神''巡逻机(也用于执行反潜任务)当地时间星期三(4月17日)在国际空域飞越台湾海峡。(互联网) 一架美国海军巡逻机星期三飞越敏感的台湾海峡;此前数小时,中美防长刚刚举行一年多来的首次实质性接触。受访学者认为,这是当前中美关系既有分歧竞争,又需要对话合作的生动写照……}

\entryitemWithDescription{中美再次就产能过剩交锋 学者指美国为对华实施更多制裁铺路}{https://www.zaobao.com/news/china/story20240417-3461419}{美国财政部长耶伦(中)星期二(4月16日)在华盛顿与主持中美经济工作组会议的中方代表中国财政部副部长廖岷(右),以及中国人民银行副行长宣昌能(左)会面。(法新社) 中美经济工作组会议再次就中国产能过剩问题进行交锋,美国重申担忧中国工业产能过剩,中国则关切美国对华经贸限制措施……}

\entryitemWithDescription{香港国安法落地近四年291人被捕 学者:社会政治化氛围过去几年大减}{https://www.zaobao.com/news/china/story20240417-3461703}{香港保安局星期三(4月17日)披露,截至3月8日,共有291人涉嫌从事危害国家安全的行为和活动被捕。图为摄于1月18日香港维多利亚港旁的海滨长廊。(法新社) 《香港国安法》实施近四年,共有291人因涉嫌违反相关法律被拘捕。受访学者认为,香港社会的政治化氛围在过去几年确实大大减少,这有赖于国安法的实施,但当局未来仍有必要加强国安教育宣传……}

\entryitemWithDescription{台湾自造潜舰专案召集人黄曙光请辞 蔡英文尚未批准}{https://www.zaobao.com/news/china/story20240417-3460534}{台湾``总统府国造潜舰专案小组''召集人黄曙光上将,以身心俱疲为由请辞,声称与他人和政治因素都无关。(台湾国防部海军司令部提供) 台湾第一艘自造潜舰仍在测试中,``国造潜舰''专案召集人黄曙光却向台湾总统蔡英文请辞,声称身心俱疲,辞职无关政治。但他未来还得厘清针对他的贪污指控,与他发起的诽谤官司,台湾海军能否建构不对称战力也成关注焦点……}

\entryitemWithDescription{中国田协计划出台指导意见 规范路跑赛事商业竞争}{https://www.zaobao.com/news/china/story20240417-3460520}{中国马拉松选手何杰(右二)与三名非洲选手,在上星期天(4月14日)的北京半程马拉松比赛中,几乎是并排跑向终点。(路透社) (北京综合讯)北京半程马拉松比赛疑似造假风波持续发酵之际,中国田径协会星期二召开会议,提出将尽快研究出台规范路跑赛事商业竞争的指导性意见,加强市场规范化等方面的指导和培训力度。 中国田协在官网发布消息称,今年春季以来,中国各地路跑赛事集中举办,群众参赛热情高涨……}

\entryitemWithDescription{缅北冲突再起 解放军中缅边境空防实弹演习}{https://www.zaobao.com/news/china/story20240417-3460874}{(北京综合讯)中国军队从星期三起在中缅边境中方一侧,举行空防实兵实弹演习。随着缅甸北部局势近期再出现不确定性,北京借此次演习向各方发出休战的明确信号。 解放军南部战区星期三(4月17日)在微信公众号上宣布,根据年度训练计划,自当天起,南部战区组织陆军、空军部队,在中缅边境中国一侧举行空防实兵实弹演习。 南部战区称,此次演习旨在检验战区部队侦察预警、立体封控、警示驱离、防空打击能力……}

\entryitemWithDescription{中国国台办副主任罕见会见美国务院高官 就台湾问题阐明立场}{https://www.zaobao.com/news/china/story20240417-3460340}{国台办星期二(4月16日)在官网发布消息称,国台办副主任仇开明(图中)星期一(15日)与美国分管东亚与太平洋事务助理国务卿康达等人会面。(互联网) (北京/台北综合讯) 美国助理国务卿康达访华期间,罕有地与中国对台系统官员会面,双方就台湾问题进行讨论。这一不同寻常之举,显示台海议题未来一段时间对中美关系走势的重要性……}

\entryitemWithDescription{中国驻欧盟使团前团长傅聪 出任常驻联合国代表}{https://www.zaobao.com/news/china/story20240417-3460196}{中国新任常驻联合国代表、特命全权大使傅聪(右)星期二(4月16日)在纽约联合国总部,向联合国秘书长古特雷斯递交了全权证书。(中新社) (纽约综合讯)中国驻欧盟使团前团长傅聪,出任中国常驻联合国代表、特命全权大使。 根据中国常驻联合国代表团网站的消息,傅聪星期二(4月16日)在美国纽约的联合国总部,向联合国秘书长古特雷斯递交了全权证书。 傅聪表示,中国始终是联合国事业的坚定支持者和积极贡献者……}

\entryitemWithDescription{杨丹旭:中国对美舆论要调整?}{https://www.zaobao.com/news/china/story20240417-3455076}{作者说,中国媒体对耶伦热衷中华美食的报道,淡化了火药味,拉升了民间对她的好感。(法新社) 美国财政部长耶伦最近访华,有中国学者注意到,中国媒体对耶伦此行的报道,与近年来对美国的报道有些不同,``这次舆论的松弛感更强一些''。 回想起来,4月4日下午耶伦飞抵广州,斜跨一个帆布袋、手拎一个公文包走下飞机,一身朴素的装扮在中国社交媒体引发热议。一些网民感叹,她就像一位邻家老太太……}

\entryitemWithDescription{赖清德内政大权一把抓 外交国防延续蔡英文路线}{https://www.zaobao.com/news/china/story20240416-3452088}{台湾候任总统赖清德4月10日宣布,委任民进党前主席卓荣泰出任行政院长。(法新社) 台湾候任总统赖清德的新政府人事布局,被视为内政大权一把抓,外交国防则延续现任总统蔡英文的路线。 现任民进党主席的赖清德将在5月20日宣誓就任总统,新内阁团队也将于同日上任。他在4月10日宣布,委任民进党前主席卓荣泰出任行政院长……}

\entryitemWithDescription{中国3月新房售价降幅为近九年最大 分析:北京或决心摆脱地产拖累}{https://www.zaobao.com/news/china/story20240416-3454763}{路透社基于中国国家统计局的数据计算得出,中国3月新建商品住宅价格同比下降2.2\%,为2015年8月以来的最大降幅。图为民众3月15日在北京一售楼处了解楼盘信息。(中新社) 中国房地产市场持续低迷,3月新建商品住宅售价降幅为近九年最大。分析认为,中国房地产风险仍在出清中,但中国第一季经济增长优于预期,可能会让北京坚定通过转换增长模式,逐步摆脱房地产对经济的拖累……}

\entryitemWithDescription{五角大楼:美中防长举行视讯通话}{https://www.zaobao.com/news/china/story20240416-3455107}{(华盛顿综合电)美国五角大楼发布文告称,美国国防部长奥斯汀与中国国防部长董军星期二(4月16日)举行视讯通话,是美中防长近18个月以来首度实质对话。 综和法新社和路透社的报道,两人讨论美中国防关系以及区域和全球安全议题。奥斯汀强调尊重国际法所保障的公海航行自由重要性,特别是在南中国海。 美国国防部指出,两人也谈及俄乌战争、朝鲜问题……}

\entryitemWithDescription{中越防长会晤 签署设立海军热线谅解备忘录}{https://www.zaobao.com/news/china/story20240416-3454244}{(北京综合讯)中国国防部长董军上星期率团出席中越边境国防友好交流活动,并与越南国防部长潘文江举行会谈,并签署设立中越海军热线的谅解备忘录。 据中国国防部官网消息,中越第八次边境国防友好交流活动上星期四至五(4月11日至12日),先后在越南老街省老街市及中国云南省红河州河口县国际口岸举行,董军与潘文江分别率团出席并举行会谈。 这是董军去年12月出任中国国防部长后,首次踏出国境进行军事外交访问……}