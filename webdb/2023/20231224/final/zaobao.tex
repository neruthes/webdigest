\entryitemWithDescription{中国特稿:中国青年下乡寻找``市外''桃源}{https://www.zaobao.com/news/china/story20231224-1457662}{面对高压力的大城市生活,以及经济放缓下的就业压力,新一代中国青年掀起``下乡''风潮,重返农村追求慢节奏生活。但大部分农村地区仍然凋敝``空心'',缺乏就业机会,年轻人回农村是无奈的选择,还是更好的出路? 在城市打拼10年的刘斌去年从上海一家科技公司离职后,今年3月来到福建的农村,过上向往已久的乡村生活……}

\entryitemWithDescription{夏宝龙:把``爱国者治港''制度优势转化为治理效能}{https://www.zaobao.com/news/china/story20231223-1457931}{中国国务院港澳办主管的全国港澳研究会星期五(12月22日)在北京举行成立十周年庆祝大会,港澳办主任夏宝龙出席并致辞。(中新社) (香港/北京综合讯)中国国务院港澳办主任夏宝龙指出,如何把爱国者治港的制度优势转化为治理效能,是一国两制实践行稳致远必须要回答好的时代课题……}

\entryitemWithDescription{中国考研人数首次下滑 疫后考研潮结束}{https://www.zaobao.com/news/china/story20231223-1457922}{中国黑龙江省哈尔滨市考生在哈尔滨理工大学考点排队进入考场。(新华社) 中国硕士研究生招生考试初试星期六(12月23日)开考,今年考试报名人数为438万人,是在连续八年递增后首次下降。 2020至2022年三年疫情期间,中国高校生通过考研延迟就业的情况尤为突出。受访学者指出,这显示疫情后中国考研潮已结束,低迷的就业形势戳破了多年来形成的学历泡沫……}

\entryitemWithDescription{清华铊投毒案受害者朱令去世 终年50岁}{https://www.zaobao.com/news/china/story20231223-1457918}{朱令是清华大学1992年级化学系学生,1994年11月因铊中毒身体出现异常。(互联网) (北京综合讯)1994年怀疑被人蓄意投毒的清华大学校友朱令星期五(12月22日)在北京去世,终年50岁。这起历经了近30年的投毒案至今还没找到凶手。 清华大学星期六(12月23日)在官方微博发布朱令去世的消息,并写道,朱令多年来与病痛顽强抗争,``我们对朱令的去世表示深切哀悼,向朱令的家人致以诚挚慰问……}

\entryitemWithDescription{中国万亿增发国债已安排逾8000亿元}{https://www.zaobao.com/news/china/story20231223-1457917}{(北京综合讯)中国增发国债第二批项目清单下达,以防洪和救灾项目为主,涉资金逾5600亿元(人民币,下同,1040亿新元)。 中国10月宣布将在2023年四季度增发1万亿元特别国债,以支持经济、缓解地方政府压力。据中国国家发改委微信公众号星期六(12月23日)发布的消息,当前两批项目下达,已涉及安排增发国债金额逾8000亿元……}

\entryitemWithDescription{尼加拉瓜获中国4.3亿美元贷款}{https://www.zaobao.com/news/china/story20231223-1457914}{(马那瓜综合讯)中国和尼加拉瓜将双边关系提升为战略伙伴关系后,两国随即签署4.3亿美元(5.7亿新元)的贷款协议,用于资助尼加拉瓜的国际机场和液化天然气接收站项目。 法新社星期五(12月22日)引述尼加拉瓜官员报道,北京的这笔贷款将用于资助蓬塔韦特(Punta Huete)国际机场项目和在特雷斯埃斯费拉斯(Tres Esferas)建设天然气终端的项目……}

\entryitemWithDescription{学者:以哈战事以区域性冲突持续 台湾要避免台海走向类似状况}{https://www.zaobao.com/news/china/story20231223-1457909}{以色列和巴勒斯坦激进组织哈马斯之间的战事造成加沙地区逾2万人丧命。图为医务人员在遭以色列轰炸的加沙南部一处医院救治受伤男子。(法新社) 以哈战事爆发至今超过两个月,台湾学者分析,在中美两极化竞逐下,两强为避免陷入直接且全面的冲突,反而会允许区域性冲突持续;这是以哈冲突未歇的原因之一,而台海冲突也有可能以这种形式发生……}

\entryitemWithDescription{中国官方:将听取各方意见进一步完善网游新规}{https://www.zaobao.com/news/china/story20231223-1457908}{(北京综合讯)中国官方拟订管理办法,以强力整顿网游,引发市场巨震,龙头腾讯一天就跌掉接近小米公司的总市值,当局星期六(12月23日)表示将听取各方意见,进一步完善网游新规。 中国国家新闻出版署在星期五(12月22日)中午公布《网络游戏管理办法》(征求意见稿),向社会公开征求意见……}

\entryitemWithDescription{新闻人间:国安法首案犯人唐英杰狱中任升旗手}{https://www.zaobao.com/news/china/story20231223-1457748}{新闻人间:唐英杰 《香港国安法》实施三年以来,港府国安处一共拘捕了260人,大部分人涉嫌在港进行危害国家安全的行为。日前,首起违反《香港国安法》案件的犯人唐英杰现身电视台剖白悔悟之心,引起了社会热议。 这个名为《有法安国》的节目,由香港警务处协助、无线电视台拍摄,共有12集,每集两分钟,旨在加强港人对国安法认识。服刑中的唐英杰在最后一集出现,成为节目压轴``主角''……}

\entryitemWithDescription{黄小芳:此瑞幸非彼瑞幸}{https://www.zaobao.com/news/china/story20231223-1457778}{中国最大连锁咖啡店品牌瑞幸咖啡控告泰国同名咖啡店侵权并在本月初败诉后,12月19日反被泰国瑞幸起诉索赔100亿泰铢。(档案照片) 持续两年的``中泰瑞幸之争''本周又出现戏剧性反转,中国最大连锁咖啡店品牌瑞幸咖啡控告泰国同名咖啡店侵权并在本月初败诉后,星期二(12月19日)反被泰国瑞幸起诉索赔100亿泰铢(3.82亿新元)。 泰国瑞幸的反击引起中国网民热议,事件一度登上微博热搜……}

\entryitemWithDescription{赵少康指赖清德为``台湾和平恐怖分子'' 萧美琴辩称``赖萧配''不会让台海爆发战争}{https://www.zaobao.com/news/china/story20231223-1457797}{台湾民进党总统候选人赖清德老家违建争议持续发酵。(路透社) 台湾在野国民党副总统候选人赵少康星期五在政见发表会上,猛攻执政的民进党总统候选人赖清德老家违建争议,并称赖为``台湾和平的恐怖分子''。民进党副总统候选人萧美琴则强调,赖清德和她主张捍卫台海和平现状,不会让台海爆发战争……}

\entryitemWithDescription{中国拟遏制网络游戏诱导消费引发圈内巨震}{https://www.zaobao.com/news/china/story20231222-1457794}{中国星期五(12月22日)发布网络游戏管理新规草案,向社会公开征求意见,旨在遏制诱导消费等现象。(法新社) (北京/上海综合讯)中国星期五(12月22日)发布网络游戏管理新规草案,向社会公开征求意见,旨在遏制诱导消费等现象。此举引发市场担忧中国对网络游戏新一轮整改,以网易、腾讯为首的游戏公司股价星期五应声大幅下跌……}

\entryitemWithDescription{中国大陆恢复输入台湾石斑鱼 软硬两手策略推进两岸协商}{https://www.zaobao.com/news/china/story20231222-1457793}{中国大陆海关总署星期五(12月22日)宣布,即日起恢复输入台湾石斑鱼,各海关恢复受理台湾石斑鱼报关,并依法依规实施检验检疫。输大陆石斑鱼须来自获得注册登记的养殖场。(自由时报) 中国大陆继星期四中止《海峡两岸经济合作框架协议》(ECFA)下12项石化产品关税减让后,星期五又宣布即日起恢复台湾石斑鱼输入大陆,重申只要坚持``九二共识''、反对``台独''就好商量……}

\entryitemWithDescription{法官裁定黎智英案未超检控时限 明年1月再开庭}{https://www.zaobao.com/news/china/story20231222-1457760}{黎智英的妻子李韵琴(左)与女儿Claire(右)、幼子崇恩(后)星期五(12月22日)抵达香港西九龙裁判法院听审。(法新社) (香港综合讯)香港法官驳回了壹传媒创办人黎智英法律团队要求撤销其``串谋刊印及复制煽动刊物''控罪的申请,认定该项检控未超过时限。案件押后至2024年1月2日继续审理……}

\entryitemWithDescription{波音在华业务破冰 四年来首次向中国直接交付客机}{https://www.zaobao.com/news/china/story20231222-1457756}{吉祥航空的一架波音787型客机,星期五(12月22日)降落在上海浦东国际机场。(法新社) (北京 / 阿灵顿综合讯)美国波音公司星期四(12月21日)宣布,向吉祥航空交付一架787梦想客机。这是四年来波音首次直接向中国航空公司交付商用客机,也可能是重启波音737 MAX交付的前奏。 综合路透社、彭博社等报道,波音在一份声明中确认此事……}

\entryitemWithDescription{中国禁止出口多项稀土加工技术}{https://www.zaobao.com/news/china/story20231222-1457714}{(北京/华盛顿综合讯)中国星期四(12月21日)宣布,禁止稀土的提炼、加工和利用技术的出口,以维护其在稀土技术上的市场主导地位。 据中国商务部官网消息,中国商务部和科技部星期四联合发布《中国禁止出口限制出口技术目录》,其中禁止出口的类别共24个,限制出口的类别共110个。此次发布是对2020年目录的修订,自发布日起生效……}

\entryitemWithDescription{大陆中止ECFA下12项产品关税减让 台指责政治目的大于经济目的}{https://www.zaobao.com/news/china/story20231222-1457577}{基隆港位于北台湾,是台湾北部海运枢纽,以货柜为主、散货为辅,为台湾第二大集装箱码头港区。(法新社) 继中国商务部上星期认定台湾对大陆贸易限制措施构成贸易壁垒后,中国国务院关税税则委员会星期四(12月21日)再宣布中止《海峡两岸经济合作框架协议》(ECFA)下石化业12项税目产品的关税减让……}

\entryitemWithDescription{赖清德老家违建争议冲击选情 ``赖皮寮''成游客拍照打卡热点}{https://www.zaobao.com/news/china/story20231222-1457569}{12月20日,蓝营揭弊天王、前立委邱毅在互联网公布赖清德老家空拍图,让精心修整的院子和多名特勤人员曝光,再次引起热议。(互联网截图) 民调领先的台湾副总统、民进党总统候选人赖清德,老家违建争议持续发酵,被网民戏称为``赖皮寮'',成了游客拍照打卡热点。 《联合早报》星期四(12月21日)在赖家门外两小时期间,约200名来自各方的民众络绎不绝寻访新景点……}

\entryitemWithDescription{因天气寒冷 积石山地震灾民安置成难题}{https://www.zaobao.com/news/china/story20231221-1457560}{寒冷的天气给积石山地震的救灾和灾民安置工作带来严重影响,图为灾民在临时搭建的帐篷外烤火取暖。(路透社) (积石山综合讯)中国甘肃积石山地震已造成至少135人遇难,由于当地天气寒冷,无法快速开展灾后重建,灾民安置成为难题。 位于甘肃和青海两省交界处的临夏州积石山县,星期一(12月18日)深夜发生6.2级地震……}

\entryitemWithDescription{中菲紧张进一步升级 王毅:两国关系站在十字路口}{https://www.zaobao.com/news/china/story20231221-1457554}{一艘中国海警舰艇(左)12月10日在南中国海争议海域驶过一艘菲律宾运补船(右)。(法新社) 中菲紧张本周进一步升级,中国外长王毅星期三(12月20日)警告菲律宾若误判形势,一意孤行,甚至与不怀好意的外部势力相互勾联,继续生事生乱,中国必将维权并坚决回应。菲律宾总统小马可斯隔天则表示菲国不屈服于胁迫,并仍致力于加强武装部队及现有联盟……}

\entryitemWithDescription{中国各地遇寒潮 上海迎来40年最低气温}{https://www.zaobao.com/news/china/story20231221-1457545}{上海星期四(12月21日)迎来40年来最寒冷天气,民众穿上厚厚的羽绒服在外滩拍照。(路透社) (上海综合讯)中国各地寒潮来袭,上海星期四(12月21日)迎来40年来同期最寒冷天气。 据上海市气象局官方微博消息,寒潮叠加辐射降温作用,上海市区星期四早晨的最低气温跌至不足零下3摄氏度,郊区的气温更降至零下4摄氏度到零下6摄氏度……}