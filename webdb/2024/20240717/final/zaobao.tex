\entryitemWithDescription{戴庆成:香港年轻人为何躺平?}{https://www.zaobao.com/news/china/story20240717-4278076}{香港雇员再培训局一项调查显示,港青就业意欲偏低。一些斗志不强的香港年轻人则索性``躺平''。图为香港维多利亚港湾。(法新社) 近来香港楼价距离历史高位已经下跌了近三成,不过依然贵绝全球,大部分基层市民只能继续望``楼''兴叹。有调查显示,年轻人申请租住公屋蔚然成风,30岁以下拥有大专或以上学历申请者的比率在过去10年不断攀升,由33\%升至52\%……}

\entryitemWithDescription{深圳7月底开通首条自动驾驶公交车线路}{https://www.zaobao.com/news/china/story20240716-4279305}{图为深圳无人驾驶公交车。(互联网) (深圳/上海综合讯)继无人驾驶出租车(德士)在中国多个城市运行后,深圳市预计7月底正式开通首条自动驾驶公交车线路。 据深圳新闻网消息,深圳巴士集团计划于2024年内在深圳前海推广20台自动驾驶公交车,运营场景涵盖地铁站、商圈、居民住宅区、中央商务区、产业园区、文化旅游景区等,致力打造中国一线城市中心城区规模最大的自动驾驶公交车队……}

\entryitemWithDescription{哈马斯和法塔赫本月北京再会 分析:巴总统缺席和解可能性不大}{https://www.zaobao.com/news/china/story20240716-4279262}{以哈冲突持续了九个月,巴勒斯坦两股关系不睦的政治势力哈马斯和法塔赫,据报本月将在北京举行高层会晤,以结束巴勒斯坦内部分裂。中国外交部星期二(7月16日)并未否认这项消息。 受访学者分析,尽管哈马斯和法塔赫都提升此次赴北京会晤的代表层级,但领导法塔赫的巴勒斯坦总统阿巴斯预计缺席,因此与哈马斯正式和解的可能性不大……}

\entryitemWithDescription{据报中美高层军事加快恢复对话步伐 战区指挥官有望通话}{https://www.zaobao.com/news/china/story20240716-4279098}{据报中美高层军事对话恢复步伐加快,战区指挥官间有望通话。(路透社) (香港/华盛顿综合讯)据报中美高层军事对话恢复步伐加快,战区指挥官间的通话有望恢复,危机沟通工作组会议可能在年底前进行。有学者分析,两国领导人都希望寻求双边关系稳定,但双方立场仍存在根本性分歧。 据《南华早报》星期二(7月16日)报道,美国总统拜登上星期四(7月11日)在北约峰会结束后的新闻发布会上,透露了两军恢复对话的消息……}

\entryitemWithDescription{中国大陆和港台旅客免签入境泰国可停留60天}{https://www.zaobao.com/news/china/story20240716-4278474}{泰国政府宣布,中国在内的93个国家及地区护照持有人,可以免签入境单次停留最多60天。图为泰国曼谷素万那普机场。(法新社) (曼谷综合讯)为促进旅游业和提振经济,泰国政府宣布新签证措施,包括中国在内的93个国家及地区护照持有人,可以免签入境单次停留最多60天……}

\entryitemWithDescription{民调:民进党选后支持度首度下滑 但赖清德支持度同比略增}{https://www.zaobao.com/news/china/story20240716-4278986}{民调显示,台湾海基会前董事长郑文灿涉贪事件冲击当地民众对民进党政府的信任感,但不影响新任总统赖清德的声望。(法新社) (台北综合讯)台湾总统赖清德就职即将满两个月,一项民调显示,台湾海基会前董事长郑文灿涉贪事件冲击当地民众对民进党政府的信任感,但不影响新任总统的声望,赖清德的支持度同比略增 ……}

\entryitemWithDescription{瑞银:美国60\%关税将导致中国经济增长率减半}{https://www.zaobao.com/news/china/story20240716-4278810}{瑞银集团报告显示,美国若对所有中国输美商品征收60\%的新关税,将导致中国经济增长率减少一半以上。图为上海陆家嘴金融中心。(法新社) (华盛顿/北京综合讯)瑞银集团一份最新研究显示,美国若对所有中国输美商品征收60\%的新关税,将导致中国经济增长率减少一半以上。 据彭博社报道,美国前总统特朗普今年早些时候据报曾表示如果再次当选总统将考虑对中国进口商品征收60\%的关税……}

\entryitemWithDescription{彭博社:中国国有金融公司要求香港员工退还部分奖金}{https://www.zaobao.com/news/china/story20240716-4278712}{(华盛顿/上海综合讯)中国一些大型国有金融公司据报要求派驻香港的员工,退还已发放的部分奖金,显示在经济放缓之际,金融行业的紧缩政策进一步升级。 彭博社星期二(7月16日)引述知情人士报道,中国光大集团和华融国际控股有限公司近几个月来都要求驻香港的部分管理人员甚至前员工,退还部分过去发放的奖金……}

\entryitemWithDescription{北京市规定义务教育学校不得选用境外教材}{https://www.zaobao.com/news/china/story20240716-4277619}{(北京综合讯)中国首都北京市发布文件订明,中小学教材须在官方发布的教材目录中选用,义务教育学校不得选用境外教材,普通高中一般也不得选用境外教材。 综合北京市教育委员会官网和《北京日报》报道,北京市教委7月5日发布《中小学教材选用实施细则》,对该市中小学教材选用作出上述规定。 文件订明,教材版本选定后,应保持相对稳定。小学、初中、高中各学段教材使用周期内,除国家、市有统一要求外,一般不得中途更换……}

\entryitemWithDescription{批中国咄咄逼人 新西兰总理:非常关切仁爱礁局势}{https://www.zaobao.com/news/china/story20240715-4268441}{新西兰总理拉克森(右)今年6月13日在惠灵顿总督府同到访的中国总理李强举行会谈。(法新社) 新西兰总理拉克森星期天批评中国在南中国海问题上咄咄逼人,试图阻止菲律宾向仁爱礁(菲称阿云津礁)的坐滩军舰提供补给。他希望今年与马尼拉达成部队互访协议,允许新西兰在菲律宾部署军事资产……}

\entryitemWithDescription{约20名中国大陆官员据报获准赴金门参加活动}{https://www.zaobao.com/news/china/story20240715-4268349}{(台北综合讯)金门附近海域态势紧张之际,据报约150名中国大陆人士获准组团到金门参加金厦泳渡活动,包括20名大陆官员。 台湾《旺报》星期一(7月15日)引述知情人士报道,台湾陆委会为了让活动避免统战色彩,在审核大陆方面提报名单时,特意把几名台办和统战系统官员排除,只批准放行大陆体育事务官员入境……}

\entryitemWithDescription{台湾首艘自造潜舰``海鲲号'' 在高雄出坞拟9月海试}{https://www.zaobao.com/news/china/story20240715-4267927}{(台北综合讯)台湾首艘自行建造的潜舰``海鲲号''首艘原型舰,星期一在高雄出坞,预计9月会进入海试阶段。 综合台湾《自由时报》和《中国时报》报道,海鲲号潜舰在台船公司干船坞坐墩近五个月后,星期一(7月15日)从注满水的船坞移往已设置好``活动浮台''的91号码头……}

\entryitemWithDescription{中国第二季经济增速弱于预期 分析:须加码提振内需}{https://www.zaobao.com/news/china/story20240715-4267419}{内需复苏乏力之际,中国第二季经济增长主要来自强劲的制造业活动。图为江苏连云港码头等待出口的汽车。(法新社) 中国第二季经济增速放缓至五个季度来最低的4.7\%,比市场预期来得差。分析认为,要确保5%左右的全年增长目标,官方须加码提振内需,带领世界第二大经济体走出通货紧缩阴影……}

\entryitemWithDescription{港媒:大陆学生用假学历赴港报读延烧至本科}{https://www.zaobao.com/news/china/story20240715-4267533}{香港一名女子7月11日在维多利亚港前摆姿势拍照。(法新社) (香港综合讯)中国大陆学生涉嫌用假学历报读香港高等院校的事件继续延烧。继30名大陆生被曝使用海外院校假学历报读香港大学硕士班课程后,香港再有大学据报正调查大陆生用海外虚假高中学历报读本科专业……}

\entryitemWithDescription{中正纪念堂三军仪队交接展示 首度移出蒋介石铜像大厅}{https://www.zaobao.com/news/china/story20240715-4267516}{台湾三军仪队星期一上午首度在台北中正纪念堂主堂体前的民主大道,进行巡查及训练展示。(法新社) (台北综合讯)台湾三军仪队星期一起正式移出中正纪念堂蒋介石铜像大厅,结束历时44年在大厅内的站哨与交接展示。这也是1980年中正纪念堂开馆以来,三军仪队首次移出中正纪念堂。 综合《联合报》、《自由时报》等台媒报道,星期天(7月14日)下午5时是最后一场的礼兵交接,期间一度有人高喊``中华民国万岁''……}

\entryitemWithDescription{中国多地倡议节约用电:路灯间隔开暂停灯光秀}{https://www.zaobao.com/news/china/story20240715-4267086}{(北京综合讯)中国进入夏季用电负荷高峰,多个省市单日用电量创历史新高,电力供应出现缺口。江西、安徽、重庆、太原等省市发布节约用电的倡议,包括间隔开启路灯、暂停城市灯光秀等,以缓解电力供应紧张。 综合澎湃新闻与央视财经报道,6月以来,中国多省市出现持续性高温天气,加上工业生产恢复,多地用电负荷突破历史记录。在工业大省广东,用电负荷在7月9日与10日连续两天创历史新高,最高达到1.49亿千瓦……}

\entryitemWithDescription{中国144小时过境免签政策扩大至河南云南两省}{https://www.zaobao.com/news/china/story20240715-4266424}{外国旅客7月10日在中国北京首都机场边防检查站,申请过境免签入境。(新华社) (北京综合讯)中国宣布144小时过境免签政策实施范围,进一步扩大至河南全省和云南省部分地区……}

\entryitemWithDescription{庄慧良:``郑文灿模式''的殒落}{https://www.zaobao.com/news/china/story20240715-4258271}{历经法庭三次攻防,涉贪的台湾海峡交流基金会前董事长郑文灿上星期四(7月11日)被收押,执政的民进党立即对他停权三年。无论华亚科学园区收贿案最后判决为何,这位曾经权倾一时的``大阿哥''政治生命已告终结。 郑文灿7月6日被桃园地检署声押禁见的消息震惊全台。当天他尚能面带笑容走出桃园地方法院,其友人于30分钟筹措500万元(新台币,下同,21万新元)协助交保……}

\entryitemWithDescription{中国周二开始进入``七下八上''防汛关键期}{https://www.zaobao.com/news/china/story20240714-4257930}{为应对后续长江上游可能发生的大洪水,三峡水库上星期六(7月13日)加开两孔,增至六孔泄洪,加速腾出防洪库容。(中新社) (北京/重庆综合讯)中国将从星期二(7月16日)开始进入为期一个月的``七下八上''防汛关键期,全国洪水多发频发,容易发生流域性洪水。中国官方为此调度四川、安徽、湖北、河南、山东等15省份,研究部署防汛救灾工作……}