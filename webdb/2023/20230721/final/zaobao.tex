\entryitemWithDescription{李家超星期日率团访新印马 分析:有助港加入RCEP}{https://www.zaobao.com/news/china/story20230720-1415852}{香港特首李家超将于星期日(7月23日)率领代表团访问新加坡、印度尼西亚及马来西亚,加强香港与这三国在经贸和投资的合作。有分析认为,港府去年申请加入《区域全面经济伙伴关系协定》(RCEP),至今仍未有进展。李家超这次出访部分亚细安成员国,相信可对加入协定起到加速作用……}

\entryitemWithDescription{陈婧:中国GDP``保5''都悬?}{https://www.zaobao.com/news/china/story20230720-1415591}{中国第二季经济数据出炉后,多家投资银行纷纷下调对全年经济增长的预测,其中最低的降到了5%。 花旗、摩根大通和法国兴业银行均把对全年中国GDP增幅预测,从此前的5.5%下调至5%,摩根史丹利更将预测大幅下调0.7个百分点至5%。 这让人想到今年初中国刚走出冠病疫情影响时经济强势反弹,不少机构乐观预测全年经济增幅可突破6\%。中国政府3月份将官方增长目标定为5%左右时,还被认为过于保守……}

\entryitemWithDescription{早说}{https://www.zaobao.com/news/china/story20230719-1415589}{(互联网) 选举靠大家,民调不靠谱,胜利是自己打拼来的,只要我们有信心,一定会赢。创造卢秀燕的奇迹,侯友宜会当选总统! ------台湾总统大选最新民调星期三(7月19日)出炉,国民党总统参选人侯友宜再次垫底。台中市长卢秀燕当晚出席侯友宜在台中千人造势晚会,提到自己当初参选市长时,所有民调也说她会输。她强调民调怎么样没有关系,全力以赴认真拉票,就会创造奇迹……}

\entryitemWithDescription{AIT处长:北京没理由因赖清德过境美国而挑衅}{https://www.zaobao.com/news/china/story20230719-1415563}{美国在台协会(AIT)处长孙晓雅说,北京没有理由以台湾副总统赖清德过境美国为借口,进行任何挑衅行动。 综合路透社和《联合报》报道,孙晓雅星期三(7月19日)在台北举行的记者会上提到赖清德过境议题时说,这符合美国长期以来的作法,也符合美台之间非官方关系的本质,同时符合美国的一个中国政策。这次赖清德过境美国,是跟过去一样的惯例……}

\entryitemWithDescription{学者:中国料不会把气候议题独立于其他中美议题处理}{https://www.zaobao.com/news/china/story20230719-1415558}{中国国家副主席韩正(右)7月19日上午在北京会见美国总统气候问题特使克里(左)。(新华社) 美国总统气候问题特使克里和中国国家副主席韩正会面,呼吁中美把气候变化议题与更广泛的外交问题分开处理;韩正则敦促美国为中美各领域交流合作创造良好环境。 克里星期天(7月16日)起对中国展开四天访问,并在星期三(19日)访华最后一天与韩正会面……}

\entryitemWithDescription{欧盟高级官员博雷利称盼今年秋季实现访华之行}{https://www.zaobao.com/news/china/story20230719-1415535}{欧盟外交与安全政策高级代表博雷利表示,希望能在今年秋天开展推迟已久的访华行。 博雷利接受彭博电视访问时说,他得到的保证是,会晤将在中欧领导人峰会前举行。他也说,访华目的是在峰会筹备阶段进行战略对话。 博雷利此前曾表示,希望中欧能在今年年底前举行领导人峰会,但目前尚未敲定峰会具体日期……}

\entryitemWithDescription{中国海关拦截部分日本进口水产品}{https://www.zaobao.com/news/china/story20230719-1415532}{日本政府称,中国开始对进口的日本水产品进行全面辐射检测,部分水产品已被中国海关拦截。此举被视为中国对日核处理水排海问题施压。 综合路透社、《读卖新闻》以及《日本时报》报道,日本政府发言人松野博一星期三(7月19日)披露上述消息。 报道引述熟悉日中关系的消息人士称,中国此举被视为对日本计划将福岛核处理水排入大海的举措施压……}

\entryitemWithDescription{特稿:两岸学界恢复实体互动 台湾选战为交流添变数}{https://www.zaobao.com/news/china/story20230719-1415511}{防疫解封后的第一个暑假,台湾前总统马英九邀请陆生团访台,属疫后首见。图为马英九7月18日陪同陆生参访团在台北与文化大学师生进行交流。 (香港中通社) 防疫解封后的第一个暑假,两岸学界恢复实体互动,台湾前总统马英九也邀请陆生团访台,属疫后首见。不过,随着台湾选战揭幕,台湾官方加强对大陆赴台人员审核,也为交流增添变数……}

\entryitemWithDescription{知情人士:摩根士丹利将200多名技术人员调离中国大陆}{https://www.zaobao.com/news/china/story20230719-1415506}{美国金融机构摩根士丹利位于纽约的办公室。(路透社) 美国媒体引述知情人士称,在北京对获取境内存储数据收紧监管后,摩根士丹利将200多名技术开发人员从中国大陆调离。 彭博社星期三(7月19日)报道,知情人士称,被调离的员工相当于摩根士丹利在中国大陆技术人员总数的逾三分之一,他们主要将调往香港和新加坡,其中多数人员的调配已完成。 据报道,摩根士丹利此举是为应对中国政府对数据收集愈发严格的监管监控……}

\entryitemWithDescription{台湾向美国采购NASAMS防空导弹系统}{https://www.zaobao.com/news/china/story20230718-1415207}{两岸紧张关系未见缓和之际,台湾将向美国购买``国家先进地对空防空导弹系统''(NASAMS),以提升台军的防空能力。 综合《联合报》和路透社报道,台湾国防部长邱国正星期二(7月18日)受访时证实了这项采购案,并称赞NASAMS在俄乌战争中展现的性能等各方面都不错。 不过,他指出,台湾至今还没有收到美国方面有关出售NASAMS的正式通知……}

\entryitemWithDescription{郭台铭吁台湾在``一中各表''框架下 与中国大陆合作走向和平}{https://www.zaobao.com/news/china/story20230718-1415201}{鸿海集团创办人郭台铭投书《华盛顿邮报》批评民进党总统参选人赖清德放弃``一中各表''的九二共识,加剧台海战争威胁。图为郭台铭星期天(7月16日)在台北举行的716凯道游行上发表讲话。(彭博社) 2024年可能独立参选台湾总统的鸿海集团创办人郭台铭,在美国报章投书批评民进党总统参选人赖清德放弃``一中各表''的九二共识,加剧台海战争威胁……}

\entryitemWithDescription{百岁``老朋友''基辛格访华 北京与中国防长李尚福会面}{https://www.zaobao.com/news/china/story20230718-1415193}{美国前国务卿基辛格星期二在北京会见中国国务委员兼国防部长李尚福时指出,美中任何一方都承担不起把对方作为对手的代价。 综合中国国防部官网的新闻稿和彭博社报道,李尚福星期二(7月18日)会见到访北京的基辛格。这是刚满100岁的基辛格四年来首次到访中国首都。 李尚福向基辛格表示,美方一些人未同中方相向而行,致使中美关系徘徊在建交以来的低谷……}

\entryitemWithDescription{香港官方建议国安法节目豁免持平要求}{https://www.zaobao.com/news/china/story20230718-1415179}{香港通讯事务管理局最新建议,有关国民教育、国民身份认同及认识《香港国安法》的电视台和电台节目,可豁免确保节目正反观点需要保持平衡的持平规定。 综合《明报》和《星岛日报》报道,香港通讯事务管理局在星期一(7月17日)发表咨询文件,建议放宽部份规管。有关咨询为期一个月。目前持牌的香港电视台及电台播放节目时,须确保内容遵守持平的规定……}

\entryitemWithDescription{广州职校食堂吃出避孕套?校方否认}{https://www.zaobao.com/news/china/story20230718-1415176}{广东广州一所职业学校食堂的饭菜中,出现疑似避孕套的胶制异物。(互联网) 继上个月发生在江西以及重庆的鼠头事件后,广东省会广州一所职业学校也陷入食品安全问题漩涡。有学生在该校食堂饭菜中发现疑似避孕套的异物,但校方对此否认并强调该异物是鸭子的眼球膜。 综合中国多家媒体报道,一名学生星期一(7月17日)在广州市华世外语艺术职业学校食堂用餐时,发现饭菜中有疑似避孕套的胶制异物……}

\entryitemWithDescription{传台陆委会准陆配亲属自7月20日起 经小三通赴台澎金马探亲}{https://www.zaobao.com/news/china/story20230718-1415158}{金门县政府星期二(18日)称,台湾政府将自7月20日起开放中国大陆配偶的大陆亲戚,持探亲入出境许可证循``小三通''途径,至台湾本岛、澎湖、金门、马祖探亲。 据《中国时报》星期二报道,台湾政府大陆委员会自今年3月25日起开放金门和厦门之间的小三通,作为台湾人赴大陆经商或旅游的中转途径。但对于何时能重新开放陆客入境,却始终没有下文……}

\entryitemWithDescription{戴庆成:香港零售业为何反弹乏力?}{https://www.zaobao.com/news/china/story20230718-1414895}{香港政府为了刺激经济,在刚刚过去的星期日(7月16日)向全港市民派发本年度第二期消费券。虽然当天台风袭港,许多市民拿到价值2000港元(下同,338新元)的消费券后,仍然第一时间外出逛街购物,尽快享受消费乐趣。 从电视新闻画面所见,各大商场、食肆人头涌涌,多家店铺的收银台前皆排起结账的长龙。受访的零售商都兴高采烈地说,这次派发消费券的措施将为他们带来一至两成的营业增长……}

\entryitemWithDescription{赖清德8月过境美国赴巴拉圭 北京向华府提严正交涉}{https://www.zaobao.com/news/china/story20230717-1414902}{台湾副总统赖清德星期天(7月16日)在民进党全台党代表大会致词。(路透社) 台湾总统府星期一(7月17日)宣布,副总统赖清德将以总统蔡英文特使身份过境美国,参加南美洲友邦巴拉圭总统当选人佩纳(Santiago Pena)8月15日的就职典礼。中国大陆已向美国提出严正交涉,并表明将采取坚决有力措施捍卫国家主权和领土完整……}

\entryitemWithDescription{美媒:美日军方制定计划应对台海冲突 日本未承诺出兵}{https://www.zaobao.com/news/china/story20230717-1414888}{美国媒体引述知情人士称,美国和日本的军方官员一年多来一直在致力于制定一项针对台海冲突的计划,以应对中国大陆的武统威胁,但日本至今未承诺出兵。 《华尔街日报》上星期六(7月15日)报道,该计划是美国应对北京武力夺取台湾威胁的最重要组成部分之一。 报道称,日本距离台湾最近处仅70英里,且日本驻有约5万4000名美军,主要集中在冲绳岛……}