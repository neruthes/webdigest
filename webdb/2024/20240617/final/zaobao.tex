\entryitemWithDescription{百年黄埔校庆 赖清德称大陆意图``吞并消灭中华民国''是台军最大挑战}{https://www.zaobao.com/news/china/story20240617-3936095}{台湾总统赖清德星期天(6月16日)以三军统帅的身份,亲自到位于高雄凤山、承袭黄埔军校传统的陆军官校,出席黄埔百年校庆活动并阅兵。(路透社) 台湾总统赖清德星期天在高雄主持黄埔建校百年校庆活动,重申``中华民国和中华人民共和国互不隶属'',并强调台军面对的最大挑战,是中国大陆意图``并吞台湾、消灭中华民国''。 受访学者指出,赖清德在上台逾一月后首次对军队发表完整讲话……}

\entryitemWithDescription{访沪引发追星热潮 狮城门将哈山:在家乡也没获得过这么多支持}{https://www.zaobao.com/news/china/story20240616-3936373}{狮城门将哈山在上海出席商场活动时,受到守候在现场的大批中国民众欢迎。(陈婧摄) 狮城门将哈山在上海参观商场户外集市时,现场制作椰浆饭。(陈婧摄) 中国球迷争相和哈山合影留念,他都一一配合。(陈婧摄) 哈山到北外滩来福士商场参观时,与上足球课的孩童进行互动训练。(陈婧摄) 在中国一夜爆红的新加坡国家足球队门将哈山·桑尼,时隔20余年再度到访上海,抵沪第一天就引发追星热潮……}

\entryitemWithDescription{中国北方料继续高温 华南江南则强降雨连连}{https://www.zaobao.com/news/china/story20240616-3936289}{(北京综合讯)中国气象部门预测,中国北方地区下来两天将持续高温天气,而南方地区则将面对强降雨。 据中国天气网消息,中东部地区高温范围星期天(6月16日)缩减、强度也有所减弱。不过星期一(17日)至星期二(18日),北方高温再度发力,河北、河南、山东等地将再现大范围、集中性的高温天气,河北南部最高气温将达到40摄氏度及以上……}

\entryitemWithDescription{澳门以``危害公共安全''为由拒香港中大讲师入境}{https://www.zaobao.com/news/china/story20240616-3935906}{(香港综合讯)澳门治安警察局以``强烈迹象从事危害公共安全活动''为由,拒绝让香港中文大学新闻与传播学院一名讲师入境。 综合《明报》、网媒``香港01''等港媒报道,香港中大新闻与传播学院高级讲师谭蕙芸应澳门传媒工作者协会(简称澳门传协)邀请,6月15日到澳门参加工作坊,分享人物专访提问技巧、如何提升文章质感等采访与写作心得……}

\entryitemWithDescription{高铁动卧列车开行 京港沪港间夕发朝至}{https://www.zaobao.com/news/china/story20240616-3935678}{大批乘客乘坐从香港西九龙站前往北京西站的首发动卧列车。(中新社) (香港/北京综合讯)往来于北京与香港、上海与香港的高铁动卧列车,自星期六(6月15日)正式运行,京港、沪港之间实现夕发朝至,使中国大陆与香港人员往来更加便利。 香港铁路有限公司星期六在港举办起航仪式,首发高铁动卧列车D910次在星期六傍晚6时24分从香港西九龙站出发,隔天早晨6时53分抵达北京西站……}

\entryitemWithDescription{李强:中国愿同澳洲延续合作研究保护大熊猫}{https://www.zaobao.com/news/china/story20240616-3935063}{中国总理李强(右起)于澳洲当地时间星期天(6月16日)上午,在南澳州州长马思诺、澳洲外交部长黄英贤陪同下,到阿德莱德动物园考察中澳大熊猫保护合作研究工作。(路透社) (阿德莱德综合讯)正在澳大利亚访问的中国总理李强说,中国愿同澳洲延续合作研究保护大熊猫,希望澳洲一直是大熊猫的友好家园……}

\entryitemWithDescription{台外交部感谢G7首将挺台有意义参与国际组织纳入公报}{https://www.zaobao.com/news/china/story20240616-3935511}{(台北综合讯)台湾外交部感谢七国集团(G7)领导人在联合公报中,首次纳入支持台湾有意义参与国际组织的文字。 七国集团领导人峰会于6月13日至15日在意大利普利亚召开,会后发表联合公报重申,维护台海的和平稳定对国际的安全与繁荣不可或缺,呼吁以和平方式解决两岸议题。公报也说,支持台湾有意义参与国际组织,包括世界卫生大会(WHA)及世界卫生组织(WHO)的技术性会议……}

\entryitemWithDescription{特稿:深中通道能否拉动珠江两岸经济融合?}{https://www.zaobao.com/news/china/story20240616-3933525}{连接深圳和中山的深中通道,今年6月底通车在即,中国广东省政府外事办公室5月下旬组织新闻媒体采访团,邀请中外新闻记者前往实地采访。 全程长24公里的深中通道,是继港珠澳大桥之后、连接珠江东西两岸另一项超级工程。这条跨海通道,将把深圳和中山之间两个小时的车程缩短至半小时内,大大提高两地的通勤和物流效率。 珠江东西两岸城市长期发展不平衡现象突出……}

\entryitemWithDescription{中国特稿:《青绿》《咏春》登陆狮城 热辣滚烫中国IP出海不难出圈难}{https://www.zaobao.com/news/china/story20240616-3923184}{《只此青绿》今年3月在新加坡滨海艺术中心演出。图为演出时期周边产品销售现场。(受访者提供) 全球明星蜂拥而至之际,新加坡也成为中国高雅艺术走出国门的首选地。《咏春》《只此青绿》《红楼梦》等中国热门舞剧先后登上新加坡舞台,新加坡人对这些中国文化IP的接受度如何?中国舞台作品在本地演出市场是昙花一现,还是厚积薄发? 新加坡近期成了中国高雅艺术演出的海外首选地……}

\entryitemWithDescription{中国海警新规实施 菲律宾向联合国递交南中国海延长大陆架权利主张}{https://www.zaobao.com/news/china/story20240616-3932921}{赋予中国海警更大执法权的《中国海警法》星期六(6月15日)正式生效,菲律宾军方高层敦促该国渔民无视新规,继续在南中国海争议海域捕鱼。图为中国海警船5月16日在南中国海争议水域对一艘菲律宾渔船进行监控。(法新社) 赋予中国海警更大执法权的《中国海警法》星期六(6月15日)正式生效,菲律宾军方高层敦促该国渔民无视新规,继续在南中国海争议海域捕鱼……}

\entryitemWithDescription{美媒:世界反兴奋剂机构未处理中国三奥运泳手药检阳性}{https://www.zaobao.com/news/china/story20240615-3932784}{(纽约综合讯)美国媒体报道称,在2021年东京奥运会前卷入兴奋剂争议的23名中国游泳运动员中,有三人几年前就被检测出违禁药物呈阳性,但世界反兴奋剂机构(WADA)得知后并没有处理。 《纽约时报》今年4月报道,23名中国游泳运动员2021年在东京奥运会前药检时,被测出违禁药物曲美他嗪(TMZ)阳性……}

\entryitemWithDescription{李强抵澳 强调中澳关系重回正确发展轨道}{https://www.zaobao.com/news/china/story20240615-3932443}{中国总理李强(右一)星期六(6月15日)抵达阿德莱德机场,对澳洲进行正式访问,澳洲外长黄英贤(右二)到机场迎接。(新华社) (阿德莱德综合讯)中国总理李强星期六(6月15日)起访问澳大利亚,他在抵步后表示,中澳两国关系在经历波折后重回正确发展轨道。 李强星期六下午抵达南澳城市阿德莱德,澳洲外交部长黄英贤、南澳州州长马思诺,以及中国驻澳大使肖千等到机场迎接。 这是中国总理时隔七年后再次访澳……}

\entryitemWithDescription{重申两岸同属中华民族 王沪宁:背叛分裂国家必遭唾弃}{https://www.zaobao.com/news/china/story20240615-3931549}{中国大陆年度对台活动``海峡论坛''大会,6月15日上午在福建厦门开幕。(中新社) 中国大陆全国政协主席王沪宁星期六(6月15日)在海峡论坛上重申两岸同属中华民族,并强调``凡是数典忘祖,背叛分裂国家的,必为历史和人民所唾弃''。 受访学者分析,这番论述是针对台湾总统赖清德的不点名批评,也意味着北京软硬两手、区隔官民的对台政策将持续加剧……}

\entryitemWithDescription{新闻人间:建制派``班长''太太火拼港大校长}{https://www.zaobao.com/news/china/story20240615-3929369}{国际高等教育咨询机构QS本月初公布最新的全球大学排名,香港大学跃升至第17位,排名优于中国清华大学和美国耶鲁大学,成绩骄人。不过,这所百年学府近来也出现了一起严重的人事纠纷风波。 事源由王沛诗担任主席的港大校委会5月28日通过多位副校长的暂任安排,但港大校长张翔6月5日发电邮给全校师生,声称对有关任命``毫不知情'',批评校委会``动摇香港大学学术自主的百年基石''……}

\entryitemWithDescription{庄慧良:``宝总''快闪宝岛}{https://www.zaobao.com/news/china/story20240615-3929440}{中国大陆知名影星胡歌星期三(6月12日)抵台,当天仅出席在台北松山文创园区举办的``对话青年''活动,隔天一早便返回上海。 22小时旋风似的访问,让许多来不及目睹巨星风采的影迷直呼可惜。台湾的大陆委员会对胡歌此行之低调也``蛮讶异的''。 因《琅玡榜》、《繁花》等作品在台享有高知名度的胡歌,睽违六年再次来台,立即成为媒体焦点……}

\entryitemWithDescription{欧盟对中国电动车加征关税 中国或提高大排量汽油车进口关税}{https://www.zaobao.com/news/china/story20240614-3928063}{欧盟委员会6月12日发表声明,拟从7月4日起对从中国进口的电动汽车征收临时反补贴税。图为6月6日在比利时布鲁塞尔,一辆电动汽车在欧盟委员会附近一充电站充电。(新华社) (北京/华盛顿综合讯)欧盟委员会宣布对中国电动车加征关税后,中国国家发展和改革委员会称,中方将``采取一切必要手段'',坚决维护合法权益。中国官媒称,中方正在酝酿反制措施,或提高大排量汽油车进口和白兰地关税……}

\entryitemWithDescription{赖清德告诉美国周刊:提两岸互不隶属为团结台湾人民}{https://www.zaobao.com/news/china/story20240614-3928450}{美国《时代》周刊星期四(6月13日)以赖清德作为当期封面人物。(互联网) 台湾总统赖清德接受《时代》(TIME)周刊专访,提到就职演说时重申``中华民国与中华人民共和国互不隶属'',目的是为了团结台湾人民,而且台湾前总统马英九、蔡英文也提过相关论述。但马英九批评,赖清德``口出狂言,又设词狡赖'',刻意拉他背书,是``如假包换的`两国论'\,''……}

\entryitemWithDescription{在华外企高管对业务信心上升 但近六成拟将投资迁印度}{https://www.zaobao.com/news/china/story20240614-3929513}{(彭博讯)美国智库一项调查显示,尽管对中国经济和地缘政治紧张局势的担忧加剧,但与六个月前相比,在华外企高管目前对中国大陆的信心略有上升。与此同时,逾半受访企业拟将从投资从中国转往印度。 据彭博社报道,世界大型企业联合会(the Conference Board)星期五(6月14日)公布一项调查结果,显示与去年下半年相比,受访企业对中国当前和未来商业环境的看法略有改观……}

\entryitemWithDescription{中国恒大前总裁夏海钧半价出售香港豪宅}{https://www.zaobao.com/news/china/story20240614-3928893}{(香港综合讯)深陷债务泥沼的中国房企恒大集团前董事局副主席兼行政总裁夏海钧,最近抛售一处香港豪宅,出手价格与买入价相比几乎``腰斩''。 综合《明报》与每日经济新闻报道,夏海钧最近以8200万港元(1421万新元)出售一处香港豪宅,价格包含三个车位。这一住宅位于港岛天后柏傲山,夏海钧在2019年从新世界集团购入,持有约五年……}

\entryitemWithDescription{土耳其对中国汽车加征关税 中国商务部:得不偿失}{https://www.zaobao.com/news/china/story20240614-3929111}{(北京综合讯)土耳其宣布把所有从中国进口的汽车关税提高40\%后,中国商务部批评土耳其多变的政策既损害双方合作企业及当地消费者的利益,又加剧中国企业对土耳其营商环境的担忧、打击他们赴土投资的信心,最终必定得不偿失。 土耳其6月8日宣布上述措施时解释,这是为避免国家的经常账户余额出现更大赤字,并保护国内汽车制造业……}

\entryitemWithDescription{美澳关切六流亡港人特区护照被撤销}{https://www.zaobao.com/news/china/story20240614-3928618}{(华盛顿/北京综合讯)香港政府撤销六名流亡海外港人的特区护照,引起美国、澳大利亚等西方国家的关切,中国外交部则批评美澳等国干预香港法治。 据法新社报道,美国指香港官员试图恐吓异议人士并让他们噤声,呼吁港府立即停止这一举动。 美国国务院发言人星期五(6月14日)说:``美国仍对香港当局试图将最近颁布的第23条立延伸至境外,作为其持续进行跨国压制行动的一部分深表关切……}

\entryitemWithDescription{乌克兰和平峰会在即 中国据报推销自身的和平方案}{https://www.zaobao.com/news/china/story20240614-3927196}{(北京综合讯)乌克兰和平峰会即将召开之际,中国据报正致力向与会的多国政府推销早前提出的解决乌克兰危机替代方案。 据路透社报道,由乌克兰总统泽连斯基提议的这场峰会将于6月15日至16日在瑞士举行。乌克兰、美国等国极力邀请中国与会,但在俄罗斯没有受邀的情况下,中国已表态不会出席会议……}