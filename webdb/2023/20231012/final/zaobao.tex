\entryitemWithDescription{韩国瑜出山挺战斗蓝诸将 并肩侯友宜号召在野整合}{https://www.zaobao.com/news/china/story20231011-1441847}{高雄前市长韩国瑜(第二排中)星期三(10月11日)出席媒体人赵少康(第二排左一)发起的``战斗蓝''辅选活动,与国民党总统参选人侯友宜(第二排左二)并肩打气。(赵少康自媒体提供) 蓝军人气王、高雄前市长韩国瑜星期三(10月11日)出席媒体人赵少康发起的``战斗蓝''辅选活动,与国民党总统参选人侯友宜并肩打气。受访时评人指出,在野阵营蓝白合可能演变成蓝白斗,韩国瑜关键时刻出手有望促进整合……}

\entryitemWithDescription{香港区议会选举年底举行 学者:很可能将清一色由建制派角逐}{https://www.zaobao.com/news/china/story20231011-1441827}{人们步行在香港维多利亚港附近的一个广场。(法新社) 香港将于今年底举行新一届区议会选举,包括民主党、民协在内的多个非建制派政团近来纷纷表态有意参选,但称向港府索取提名人士的联络方法时遇到困难。受访学者认为,这意味着区议会选举很可能将清一色由建制派人士角逐。 香港新一届区议会选举下星期二(10月17日)开始接受报名……}

\entryitemWithDescription{大陆国台办回应蔡英文双十讲话:搞``和平分裂''是痴心妄想}{https://www.zaobao.com/news/china/story20231011-1441821}{对于台湾总统蔡英文在任内最后一次双十庆典演说,表达愿与中国大陆建立和平共存之道,大陆国台办回应说,``九二共识''是两岸对话协商的共同政治基础,民进党当局欲搞``和平分裂''是痴心妄想。 蔡英文2016年5月执政,虽主张``维持现状'',承诺会遵守《中华民国宪法》及《台湾地区与大陆地区人民关系条例》,但拒绝承认``九二共识'',两岸政府制度性沟通管道也完全中断……}

\entryitemWithDescription{台湾国防部:大陆军机今年扰台月均增百余架次}{https://www.zaobao.com/news/china/story20231011-1441819}{(台北综合讯)台湾国防部说,中国大陆军机今年扰台次数较过往大幅增加,去年每月平均280余架次,今年每月增至约380余架次。 综合《联合报》与中时新闻网报道 ,台湾国防部星期四(10月12日 )将到立法院外交国防委员会进行业务报告,书面报告星期三(10月11日 )送至立法院。报告称,解放军军事扩张已严重威胁台海及周边区域安全,军机扰台次数大增,台海紧张情势升高……}

\entryitemWithDescription{中国中东问题特使向埃及官员表明愿推动以巴停火 学者:北京预计不会实质性介入调停}{https://www.zaobao.com/news/china/story20231011-1441816}{中国政府中东问题特使翟隽10月10日就以巴局势同埃及外交部巴勒斯坦事务部长助理乌萨玛通电话。(互联网) 新一轮以巴冲突爆发后,中国政府中东问题特使翟隽与以巴邻国埃及外交官员通话,重申中国呼吁立即停火,并称中国愿同埃及保持沟通协调,推动冲突双方尽快停火止暴。 受访学者分析,中国寻求树立中立调停者形象,但料不会实质性介入调停以巴冲突……}

\entryitemWithDescription{澳籍记者成蕾被拘三年后获释 中国指她服刑期满被驱逐出境}{https://www.zaobao.com/news/china/story20231011-1441810}{澳大利亚华裔记者成蕾(右)星期三(10月11日)在华被拘三年后获释返回澳洲,澳洲外长黄英贤在墨尔本机场与成蕾会面。(路透社) (堪培拉/北京综合讯)澳大利亚华裔记者成蕾被中国以涉嫌国家安全罪拘留三年后,于星期三(10月11日)获释。中国官方指她非法将工作中掌握的国家秘密内容提供给境外机构,在服刑期满后被驱逐出境……}

\entryitemWithDescription{中国据报考虑提高2023年预算赤字}{https://www.zaobao.com/news/china/story20231011-1441782}{(北京彭博电)中国据报正在考虑提高2023年的预算赤字,并准备推出新一轮刺激措施,以助实现全年经济增长目标。 知情人士说,北京正在考虑至少发行1万亿元人民币(1867亿新元)的额外主权债务,用于水利工程等基础设施项目的支出,这可能会使今年的预算赤字远超3月中国全国两会上公布的3\%上限。 知情人士说,这一计划由中国财政部和国家发改委牵头,还需得到国务院和立法机构的最终批准……}

\entryitemWithDescription{杨丹旭:以巴冲突考验中国外交}{https://www.zaobao.com/news/china/story20231011-1441519}{10日晚,亲以色列的示威者聚集在匈牙利``多瑙河畔鞋履''雕塑前的大道,抗议哈马斯突袭以色列。(路透社) 巴勒斯坦激进组织哈马斯上周六(10月7日)向以色列发起多年来最大规模的袭击,震惊世界。以色列随即向哈马斯宣战,并形容这场袭击是以色列的``911''事件。 新一轮以巴冲突引爆后,中东地区再陷动荡,大量无辜的平民被屠杀、绑架,面临暴力对待,一场大规模的人道危机山雨欲来……}

\entryitemWithDescription{蔡英文双十节向大陆喊话:和平是两岸唯一选项}{https://www.zaobao.com/news/china/story20231010-1441512}{台湾总统蔡英文(左)与副总统赖清德星期二(10月10日)出席双十节庆典。蔡英文发表任内最后一次双十节演讲时强调,维持现状是确保和平的关键,愿与北京发展双方可接受的互动基础。(彭博社) 台湾总统蔡英文星期二(10月10日)发表任内最后一次双十节演讲时向中国大陆喊话,强调和平是两岸的唯一选项,维持现状是确保和平的关键,愿与北京发展双方可接受的互动基础……}

\entryitemWithDescription{中菲紧张升级 中国海警黄岩岛海域驱离菲海军炮艇}{https://www.zaobao.com/news/china/story20231010-1441507}{这张在今年8月23日所摄的照片显示,两艘中国海警船拦截前往仁爱礁的一艘菲律宾补给船。(法新社) 中菲两国在南中国海的紧张持续升级,中国外交部星期一(10月9日)敦促菲律宾停止在海上挑衅滋事,中国海警隔天在黄岩岛海域驱离菲国海军炮艇。受访学者分析,尽管中国本周加强外交军事威慑,但反应仍属克制,旨在遏制马尼拉继续挑战北京底线……}

\entryitemWithDescription{舒默访华期间 美国放宽韩在华工厂进口限制}{https://www.zaobao.com/news/china/story20231010-1441478}{(北京/首尔综合讯)美国参议院多数党领袖舒默访华期间,中美官员分别就中美经贸关系和气候问题进行会谈。同一天,美国放宽对韩国在中国的工厂进口美国半导体设备的限制。 据中国商务部官网消息,中国商务部部长王文涛星期一(10月9日)在北京会见了舒默率领的美国国会参议院两党代表团,双方就中美经贸关系和共同关心的经贸问题进行了理性、务实的讨论……}

\entryitemWithDescription{一带一路面对压力 中国官方强调保障供应链稳定畅通}{https://www.zaobao.com/news/china/story20231010-1441470}{中国官方强调将构建一带一路对外开放新格局,保障一带一路产业链供应链的稳定畅通。图为香港特区政府和香港贸易发展局9月13日合办的第八届``一带一路''高峰论坛。(路透社) 一带一路倡议面对政治与债务压力之际,中国官方强调将构建一带一路对外开放新格局,并促进政策沟通和战略对接,保障一带一路产业链供应链的稳定畅通。 中国国务院新闻办公室星期二(10月10日)举办发布会,介绍一带一路倡议十周年成果……}

\entryitemWithDescription{中沙海上反恐联训首次在中国举行}{https://www.zaobao.com/news/china/story20231010-1441461}{(广州综合讯)中国和沙特阿拉伯在广东湛江启动海外海上反恐军事演习。 综合新华社和《解放军报》报道,中沙``蓝剑-2023''海军特战联合训练开训仪式星期一(10月9日)在湛江举行,中沙双方100多名联训官兵参加了开训仪式。这是中沙联训首次在中国举办……}

\entryitemWithDescription{香港输入人才计划已吸引6万人赴港 远超原定目标}{https://www.zaobao.com/news/china/story20231010-1441439}{(香港综合讯)在香港输入人才计划下,至今已有6万人抵港,远超港府年初定下全年3万5000人的目标。 据星岛网报道,香港财政司司长陈茂波星期一(10月9日)在第三届香港国际人才高峰论坛上公布了上述数据。 陈茂波说,输入人才计划截至今年9月共收到16万份申请,其中超过10万份获批。其中,``高才通''计划收到超过5万份申请,超过3万9000份获批……}

\entryitemWithDescription{英媒:欧盟计划对中国钢铁展开反补贴调查}{https://www.zaobao.com/news/china/story20231010-1441411}{欧盟据报计划宣布对中国钢铁制造商展开反补贴调查。图为今年9月中国江苏淮安一家钢铁工厂的工人在进行作业。(法新社) (布鲁塞尔综合讯)英国媒体报道,欧盟计划宣布对中国钢铁制造商进行反补贴调查。 英国《金融时报》引述两名知情官员星期二(10月10日)报道称,欧盟已同意联手美国,保护钢铁行业免受来自中国的廉价竞争。知情官员称,美国要求欧盟对中国钢铁制造商采取行动,以避免美国重新对欧盟钢铁征收关税……}

\entryitemWithDescription{戴庆成:香港议员水平为何参差不齐?}{https://www.zaobao.com/news/china/story20231010-1441176}{香港立法会今年会期至今,一共通过了24项政府法案,其中有三分之二在表决时,参与表决议员人数都未达到法定要求的一半。(法新社) 香港近年开始实施``爱国者治港''制度,所有立法会议席由``爱国爱港''人士担任,议会气氛比以前平静许多。但在过去一星期,立法会以至整个政坛却突然掀起了一股``猜猜他是谁''的热潮……}

\entryitemWithDescription{韩国瑜出面促台湾在野总统参选人整合 获侯友宜柯文哲郭台铭正面回应}{https://www.zaobao.com/news/china/story20231009-1441189}{台湾高雄前市长韩国瑜星期一(10月9日)在脸书公开与民众党总统参选人柯文哲的合照,强调``整合不是个人意愿的抉择,而是国家社会的期待''。(韩国瑜脸书) 台湾高雄前市长韩国瑜星期一(10月9日)在脸书发出呼吁,希望国民党总统参选人侯友宜、民众党总统参选人柯文哲和独立参选的鸿海集团创办人郭台铭,齐心合力让台湾更好,获三人积极回应……}

\entryitemWithDescription{中国—南京服务业推介会在新举行}{https://www.zaobao.com/news/china/story20231009-1441168}{由江苏省南京市商务局等主办、南京生态科技岛开发区管委会等承办、新加坡中华总商会支持的中国---南京服务业扩大开放高质量发展推介会,星期一(10月9日)上午在新加坡举行,开展投资促进与经贸交流活动。 该活动也是新加坡---南京生态科技岛经贸合作交流会。生态科技岛是新苏会框架下的旗舰项目之一,也是江苏省第二大对外合作项目……}