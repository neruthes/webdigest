\entryitemWithDescription{戴庆成:香港告别``三权分立''?}{https://www.zaobao.com/news/china/story20240116-1462300}{香港律政司宪制及政策事务科上个月在《基本法简讯》一篇探讨香港政制的文章中,提及特区权力全属中央政府转授,使用``三权分立''描述政制并不妥当。(路透社档案图) 台湾民进党在刚刚过去的周六(1月13日)如预料赢得总统大选,不过在立委选举的表现却不理想,由国民党取而代之成为立法院最大党……}

\entryitemWithDescription{被誉为``台湾曼德拉''的民进党前主席施明德过世}{https://www.zaobao.com/news/china/story20240116-1462314}{1993年11月,施明德接任民进党主席,并于1994年5月蝉联党主席。1996年3月,因民进党在台湾选举中大败,施明德请辞党主席。(自由时报) 被誉为``台湾曼德拉''的民进党前主席施明德星期一(1月15日)与世长辞,享寿83岁。 这一天也是施明德的生日,施明德的女儿施笳和施蜜娜在脸书发文说,施明德没有忌日,只有生日,一生所捍卫的精神和价值会在世界诞生、长长久久地茁壮……}

\entryitemWithDescription{美代表团选后访台 大陆表示坚决反对干涉台事务}{https://www.zaobao.com/news/china/story20240116-1462313}{美国跨党派代表团星期一(1月15日)在台北的民进党中央党部与民进党籍的正副总统当选人赖清德(右二)和萧美琴(右一)会面。美国代表团成员包括前国安顾问哈德利(右三)。(法新社) 由美国前高官组成的跨党派代表团,在台湾大选隔天高调抵台,并在星期一(1月15日)一天内会见台湾总统蔡英文、民进党籍的正副总统当选人赖清德和萧美琴,以及落败的两名在野党候选人。对此,中国大陆表示坚决反对……}

\entryitemWithDescription{瑙鲁与台湾断交 学者:北京给赖清德下马威 释出坚决反台独信号}{https://www.zaobao.com/news/china/story20240115-1462310}{瑙鲁星期一(1月15日)毫无预警宣布与台湾断交后,位于台北天母的使馆特区外悬挂的瑙鲁国旗同一天被撤下。(路透社) 台湾民进党主席赖清德赢得总统选举后不到48小时,台湾友邦瑙鲁星期一(1月15日)毫无预警宣布与台湾断交,与中国大陆复交。 台海两岸之间的``邦交国战'',一向是北京压缩台湾国际空间的手段之一,以及北京释放警告的方法……}

\entryitemWithDescription{王毅首度就红海紧张局势公开发声 分析:不点名吁胡塞及美英保持克制}{https://www.zaobao.com/news/china/story20240115-1462301}{中国外长王毅星期天(1月14日)在埃及开罗首度就红海紧张局势公开发声,呼吁停止袭扰民船。(路透社) 也门胡塞武装持续袭击红海船只近两个月之际,正在非洲出访的中国外长王毅在埃及首度就红海紧张局势公开发声,呼吁停止袭扰民船,也表示安理会从未授权任何国家对也门使用武力,应避免给局势火上浇油,不点名呼吁胡塞武装及美国和英国在红海保持克制……}

\entryitemWithDescription{台湾选举落幕 台股小涨 新台币走弱}{https://www.zaobao.com/news/china/story20240115-1462283}{台湾股市星期一(1月15日)小幅上涨。图为台湾证券交易所屏幕上显示的股票数据。(彭博社) (台北综合讯)台湾选举结果尘埃落定,台湾股市星期一(1月15日)小涨,新台币兑美元则走弱。 综合彭博社和《经济日报》报道,台股开盘时上涨超过百点,一度重返1万7600点大关,但最终收盘时仅上涨33.99点。 至于新台币兑美元汇率,盘中一度重贬逾一角,最后收在31.215元、贬值8.6分,汇价创近一个月新低……}

\entryitemWithDescription{李强:中国开放大门只会越开越大}{https://www.zaobao.com/news/china/story20240115-1462267}{中国总理李强(右二)与瑞士联邦主席阿姆赫德(左二)1月14日共同乘专列自苏黎世前往瑞士首都伯尔尼,并在途中进行茶叙。(新华社) 中国总理李强星期天(1月14日)起在瑞士达沃斯出席世界经济论坛2024年年会并对瑞士、爱尔兰进行正式访问,他与瑞士联邦主席阿姆赫德会面时,强调中国开放的大门只会越开越大,欢迎更多瑞士企业赴华投资兴业……}

\entryitemWithDescription{​蓝绿争台立法院龙头 民众党称改革立场比人选重要}{https://www.zaobao.com/news/china/story20240115-1462248}{(台北综合讯)台湾立法院龙头之争继续发酵,扮演关键少数的民众党提出具体条件,要求民进党和国民党有意角逐立法院正副院长的人士先表达改革立场,并称``我们根本不在乎是谁''。 综合台媒报道,台湾2024年大选落幕,立法院席次朝野三党均不过半,拥有八个立委席次的民众党成关键少数。外界关注在2月1日开议后的立法院正副院长选举,是否出现国民党和民众党(蓝白)合作或民进党和民众党(绿白)合作……}

\entryitemWithDescription{外交部:将在我国的``一个中国''政策基础上继续发展对台友好关系}{https://www.zaobao.com/news/china/story20240115-1462058}{我国外交部祝贺赖清德和民进党胜选,并强调将在新加坡的``一个中国''政策基础上,继续发展对台友好关系。 根据我国外交部官网1月14日发布的文告,外交部发言人在回应媒体询问时说,我们乐见台湾选举圆满结束,祝贺赖清德和他的政党胜选。发言人说:``新加坡会在我们的`一个中国'政策基础上,继续发展与台湾和台湾人民密切和长久的友好关系。'' 外交部说,新加坡一贯支持两岸关系和平发展……}

\entryitemWithDescription{于泽远:赖清德上台将引发台海战争?}{https://www.zaobao.com/news/china/story20240115-1462077}{台湾大选结果出炉,民进党候选人赖清德、萧美琴当选台湾下届正副总统。中国大陆有不少人认为,素有``台独金孙''之称的赖清德上台,将使两岸关系继续恶化,甚至引发台海战争。 实际上,去年11月国民党与民众党``蓝白合''破局后,这次台湾大选就没有了多少悬念。虽然岛内不希望民进党继续执政的民众占据多数,但无论国民党还是民众党都很难集合不支持民进党的选票,民进党靠基本盘就能击败另外两党的对手……}

\entryitemWithDescription{民进党流失大量年轻选票受冲击最大 分析:年轻人不满没改善高房价切身问题}{https://www.zaobao.com/news/china/story20240114-1462119}{民进党支持者星期天(1月13日)在台北的竞选总部外,为赖清德当选总统而欢呼。不过,民进党流失大量年轻选票,赖清德接棒的是总统得票、立委席次``双不过半''的弱势政府。(彭博社) 台湾大选结果显示,蓝绿两大党都流失大量年轻选票。学者分析,绿营的民进党若不回应和改善年轻人在乎的切身课题,在两年后的地方选举将持续面对威胁;蓝营的国民党则须考虑修正路线、争取主流民意,才有望争取年轻选民并重返执政……}

\entryitemWithDescription{台大选隔天美国高级代表团抵台 王毅:任何台独行动都将受到严厉惩罚}{https://www.zaobao.com/news/china/story20240114-1462115}{美国前国家安全顾问海德利(右二)和美国在台协会主席罗森博格(左二)星期天(1月14日)晚抵达台北,台湾外交部北美司司长王良玉(左一)在机场迎接。美国在台协会台北办事处处长孙晓雅(Sandra Oudkirk)(右一)也前往迎接。(台湾外交部网站) (台北综合讯)台湾大选结束次日,由前高官组成的美国高级代表团旋即抵台;多个西方国家也就台湾选举表态……}

\entryitemWithDescription{胡锡进反驳大陆舆论对台政策质疑声 称大选谁赢绝非最重要}{https://www.zaobao.com/news/china/story20240114-1462114}{民进党赢得台湾2024总统大选后,中国大陆网络舆论出现质疑大陆对台政策的声音。图为民众1月9日从台北剑潭山眺望台北101大楼。(法新社) (北京综合讯)台湾民进党组合赢得总统大选后,中国大陆网络舆论出现质疑大陆对台政策的声音。大陆《环球时报》前总编辑胡锡批评这种观点是短视和幼稚,他认为选举结果对两岸整体格局影响有限,强调谁胜出绝非最重要的事情,台湾问题应交由专业团队来处理……}

\entryitemWithDescription{民众党关键地位左右立法院龙头之争}{https://www.zaobao.com/news/china/story20240114-1462112}{台湾2024年大选落幕,立法院席次出现朝野三党不过半,2月1日开议的立法院龙头之争备受瞩目。o国民党不分区立委第一名的韩国瑜能否成为立法院长仍有悬念。 在野国民党(蓝)于大选中跃升为第一大党,拥有52席立委,加上有案在身以无党籍参选的国民党籍立委员陈超明,以及亲蓝的无党籍原住民立委高金素梅,共有54席,距离过半的57席只差3席,但国民党不分区立委第一名的韩国瑜能否成为立法院长仍有悬念……}

\entryitemWithDescription{中国多地探索教师退出机制 教师职业不再是``铁饭碗''}{https://www.zaobao.com/news/china/story20240114-1462089}{中国北京丰台、贵州贵阳、浙江宁波等地正探索建立教师退出机制。图为河北邢台市平乡县幼儿园的老师和小朋友1月12日一起剪窗花。(新华社) (北京综合讯)中国北京丰台、贵州贵阳、浙江宁波等地正探索建立教师退出机制,以防止个别教师躺平,让教师职业不再是``铁饭碗''……}

\entryitemWithDescription{分析:大陆或将拿台湾对陆出口前30大产品开刀}{https://www.zaobao.com/news/china/story20240114-1462084}{民进党在台湾总统选举中胜出后,一般预测中国大陆将加大对台湾的经贸施压。图为台湾新北市1月13日人来人往的市场街道。(路透社) (台北综合讯)台湾民进党在总统选举中胜出后,一般预测中国大陆将加大对台湾的经贸施压。分析人士预测,大陆除了将扩大取消海峡两岸经济合作架构协议(ECFA)的优惠关税项目之外,未来也可能把目标延伸到台湾对陆出口的前30大产品,加大打击面……}

\entryitemWithDescription{美学者提醒:台湾选后应保持克制同时加强备战}{https://www.zaobao.com/news/china/story20240114-1462080}{美国政治学者戴雅门(Larry Diamond)认为,台湾在选后应该保持克制、降温、不挑衅,同时加强备战,建设台湾安全与韧性。(缪宗翰摄) 对于台湾选后可能产生的地缘政治风险,美国著名政治学者戴雅门(Larry Diamond)星期天(1月14日)指出,北京必然推动统一进程,台湾应该保持克制、降温、不挑衅,同时加强备战,建设安全与韧性……}

\entryitemWithDescription{港人盼大陆恢复``一签多行'' 吸引深圳居民南下消费}{https://www.zaobao.com/news/china/story20240114-1462066}{香港经济在疫情防控措施解除后表现未如理想,越来越多港人希望恢复``一签多行''政策,吸引深圳居民南下消费,激活香港经济。图为食客2023年12月20日在香港庙街夜市摊位找美食。(法新社) 香港去年解除疫情防控措施后,访港旅客人数一直未能恢复至疫前水平,当地舆论提出恢复深圳居民``一签多行''措施,吸引深圳居民南下消费,激活香港经济……}

\entryitemWithDescription{中国特稿:欧美进军路跌宕 中国电动车加速开往东南亚}{https://www.zaobao.com/news/china/story20240114-1461707}{随着中国电动车国内销量增长放缓,越来越多中国车企把目光投向海外市场。图为一辆比亚迪电动车在北京街道上行驶。(路透社) 随着中国新能源汽车国内销量增长放缓,越来越多汽车企业加速海外市场布局作为新的业务增长点。但欧美针对汽车产业的贸易保护主义近年有抬头的迹象,官方纷纷采取一系列措施遏制中国电动车产业的发展。这对中国新能源汽车出海的步伐有什么影响?中国车企在东南亚的部署会不会加快……}

\entryitemWithDescription{接棒``双不过半''弱势政府 赖清德:台湾政治须走向协商合作}{https://www.zaobao.com/news/china/story20240114-1461980}{当选台湾总统的赖清德与搭档萧美琴胜选后出席在台北民进党总部外举行的记者会。台湾星期六(1月13日)举行的总统选举全球瞩目,吸引了逾400名国际记者赴台采访。(法新社) 当选台湾总统的赖清德说,民进党在立法院席次未过半,说明人民期待``有效率的制衡'',台湾政治须走向协商与合作。 赖清德星期六(1月13日)在胜选后的国际记者会上致辞说,民进党无法在立法院席次过半,代表努力不够,必须虚心检讨……}

\entryitemWithDescription{拼政党轮替失败 侯友宜眼眶泛红为败选负责 学者:国民党中央须改变官僚老旧习气}{https://www.zaobao.com/news/china/story20240114-1461975}{国民党正副总统候选人侯友宜(左二)、赵少康(右二),以及党主席朱立伦(左一)、侯友宜竞选办公室执行长金溥聪(右一)星期六(1月13日)晚在竞选总部,为没能实现政党轮替,向支持者鞠躬致歉。(法新社) 台湾最大在野国民党力拼政党轮替再次失败。国民党总统候选人侯友宜星期六(1月13日)败选后,三度向支持者鞠躬致歉,眼眶泛红表示,``我尊重台湾人民作出的最后选择''……}

\entryitemWithDescription{韩国瑜出任立法院长?民众党意愿成关键}{https://www.zaobao.com/news/china/story20240114-1461973}{主张废除死刑的社民党台北市议员苗博雅,挟带青年族群的高声量,在台北市大安区立委选举拿下逼近45\%的得票率,虽竞选失利,卻是历来泛绿阵营在该区最佳表现。(档案照片) 韩国瑜可能出任立法院长 台湾立委选举落幕,国会朝小野大局面底定。外界推测,未来立法院长可能由国民党不分区立委排名第一的前高雄市长韩国瑜出任,但仍要看民众党所扮演的关键少数,是否同意……}

\entryitemWithDescription{赖清德当选总统 分析:两岸关系将受冲击}{https://www.zaobao.com/news/china/story20240114-1461971}{台湾民进党主席赖清德(左)在星期六举行的大选中,以逾558万得票数胜出。图为赖清德和副总统当选人萧美琴(右)在胜选后召开国际记者会。(法新社) 自称务实台独工作者的台湾民进党主席赖清德在星期六(1月13日)举行的大选中,以逾558万得票数胜出,在三角战中以40.05%得票率当选总统。民进党由此首次得以连续执政超过八年,两岸关系和台海稳定前景则料受冲击……}