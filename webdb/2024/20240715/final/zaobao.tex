\entryitemWithDescription{庄慧良:``郑文灿模式''的殒落}{https://www.zaobao.com/news/china/story20240715-4258271}{历经法庭三次攻防,涉贪的台湾海峡交流基金会前董事长郑文灿上星期四(7月11日)被收押,执政的民进党立即对他停权三年。无论华亚科学园区收贿案最后判决为何,这位曾经权倾一时的``大阿哥''政治生命已告终结。 郑文灿7月6日被桃园地检署声押禁见的消息震惊全台。当天他尚能面带笑容走出桃园地方法院,其友人于30分钟筹措500万元(新台币,下同,21万新元)协助交保……}

\entryitemWithDescription{中国周二开始进入``七下八上''防汛关键期}{https://www.zaobao.com/news/china/story20240714-4257930}{为应对后续长江上游可能发生的大洪水,三峡水库上星期六(7月13日)加开两孔,增至六孔泄洪,加速腾出防洪库容。(中新社) (北京/重庆综合讯)中国将从星期二(7月16日)开始进入为期一个月的``七下八上''防汛关键期,全国洪水多发频发,容易发生流域性洪水。中国官方为此调度四川、安徽、湖北、河南、山东等15省份,研究部署防汛救灾工作……}

\entryitemWithDescription{上台后首登军舰视导 赖清德期许台军守护海疆安全}{https://www.zaobao.com/news/china/story20240714-4257550}{台湾总统赖清德(左三)星期六(7月13日)与国防部长顾立雄(左二)等人前往基隆慰勉海军舰队官兵,并登上沱江级巡逻舰``旭江舰''视导。(赖清德脸书) (台北综合讯)台湾军方星期六(7月13日)首次通报中国大陆火箭军在内蒙古试射之际,台湾总统赖清德同日前往基隆慰勉海军舰队官兵,并登上军舰视导。这是他今年5月20日上台后首次这么做,以表明守护台湾的决心……}

\entryitemWithDescription{传在塔吉克斯坦建秘密军事基地 中国外交部间接否认}{https://www.zaobao.com/news/china/story20240714-4257417}{(北京/伦敦综合讯)中国据传正在中亚国家塔吉克斯坦建设秘密军事基地,以应对塔利班统治下的阿富汗构成的安全威胁。中国外交部上星期五(7月12日)间接否认传言,表示不掌握情况,并称中国在中亚没有任何军事基地。 据英国《每日电讯报》上星期三(10日)报道,卫星图片显示,北京在塔吉克斯坦建设秘密军事基地。该设施位于近4000米的高山上,有瞭望塔和两国驻军……}

\entryitemWithDescription{特稿:深圳失业者图书馆``伪装''上班}{https://www.zaobao.com/news/china/story20240714-4255146}{有深圳失业网民称,他们选择在图书馆``伪装''上班,以维持日常的生活节奏,避免家人担忧。本报记者日前走访位于红山区的深圳图书馆时发现,即便是工作日,馆内也是人满为患。许多年轻人带着一台电脑,有的在发简历,有的在看视频,有的则在抱头大睡。(林煇智摄) 35岁的郑敏(化名)是一名财务经理,去年底被一家咨询服务公司裁退后,一直处于失业状态……}

\entryitemWithDescription{中俄时隔一年在太平洋海域再展开海上联合航巡}{https://www.zaobao.com/news/china/story20240714-4257168}{俄罗斯护卫舰格罗姆基号在广东湛江举行的中俄``海上联合-2024''联合演习期间驶入湛江港,图片取自官方7月13日发布的视频。(路透社) (北京/莫斯科综合讯)北约上周将中国列为俄罗斯对乌克兰开战的``决定性赋能者''后,中国国防部星期天(7月14日)宣布中俄海军军舰近日在太平洋西部和北部海域开展第四次海上联合巡航,这是两国时隔近一年在相同海域再开展联合巡航……}

\entryitemWithDescription{中菲执法部门合作遣返三名绑架嫌犯}{https://www.zaobao.com/news/china/story20240714-4256802}{中国和菲律宾两国执法部门星期六(7月13日)合作遣返三名涉嫌在菲绑架中国公民的犯罪嫌疑人。(中国驻菲律宾大使馆网站) (北京/马尼拉综合讯)中国和菲律宾两国执法部门合作,遣返三名涉嫌在菲绑架中国公民的犯罪嫌疑人。这是两名中国企业高管在菲被撕票后,双方加强合作打击跨国犯罪的最新动作……}

\entryitemWithDescription{台湾特稿:党内挥扫贪大刀 赖清德要砍哪儿?}{https://www.zaobao.com/news/china/story20240714-4227774}{台湾总统赖清德(左)上任不久,海峡交流基金会前董事长郑文灿7月初突然因涉贪去职,这股肃杀气氛令执政的民进党内噤若寒蝉,在野党也胆颤心惊。图为赖清德4月公布新任内阁时,介绍郑文灿将出任海基会董事长一职……}

\entryitemWithDescription{中国海军万吨医院船赴南中国海执行首次任务}{https://www.zaobao.com/news/china/story20240713-4254882}{中国南部战区海军万吨新舰``丝路方舟''号医院船将奔赴西南沙及华南沿海岛礁进行医疗巡诊,开展海上伤员救治演练等任务。(央视新闻) (北京综合讯)中国南部战区海军万吨新舰``丝路方舟''号医院船星期三(7月10日)赴南中国海域,开启入列以来的首次医疗服务。 综合南部战区微信公号与央视新闻消息,``丝路方舟''号在广东湛江某军港解缆起航,奔赴西南沙及华南沿海各岛礁,目的地包括杆列岛、永暑礁、永兴岛等……}

\entryitemWithDescription{中国再爆水路食用油化工油混运}{https://www.zaobao.com/news/china/story20240713-4254662}{中国媒体爆出有煤油化工罐车运输食用油后,又爆出水路运输也有化工油和食用油混用问题。图为在长江水域行驶的货船、船舶。(新华网) (北京综合讯)中国以煤油化工罐车运输食用油引起舆论哗然,而最新的媒体报道揭露,不仅是公路货运,食用油经水路运输也存在混运现象。 财新网星期五(7月12日)报道,有食用油专用运输船船长对其意向客户称,他们是持有植物油运输证的食用油运输船,但没货时也接运白油、柴油等其他油品……}

\entryitemWithDescription{郑文灿将至少被关押两个月}{https://www.zaobao.com/news/china/story20240713-4254813}{台湾海峡交流基金会前董事长郑文灿(左二)涉嫌在华亚科学园区开发案中收贿,7月11日历经三次羁押庭后被裁定收押禁见。(互联网) (台北讯)台湾高等法院驳回海基会前董事长郑文灿代表律师的抗告,郑文灿将至少被关押两个月。 综合《联合报》和中时新闻网等报道,民进党籍桃园市前市长郑文灿在两度交保后,星期四(7月11日)在第三次羁押庭召开后被裁定收押禁见。郑文灿不服裁决,由代表律师提出抗告……}

\entryitemWithDescription{欧盟对华电动车关税案德国或投弃权票 分析:关税难逆转 但税率可谈判}{https://www.zaobao.com/news/china/story20240713-4254567}{路人走过意大利米兰的比亚迪汽车门店前。照片摄于3月20日。(路透社) 德国据报将在星期一(7月15日)就欧盟对中国电动汽车征收关税提案投下弃权票,从而推动欧盟与中国继续就此磋商。分析认为,欧盟对华加征关税的大势难以逆转,但具体税率还有谈判空间。 路透社星期五(7月12日)引述匿名知情人士说法称,德国在关税案第一阶段投票中选择弃权,是因为欧盟对华反补贴调查仍在继续,双方谈判也还在进行……}

\entryitemWithDescription{中国网络频现``历史的垃圾时间''论调 官媒反驳:伪学术概念}{https://www.zaobao.com/news/china/story20240713-4254420}{(北京综合讯)中国互联网世界频繁出现``历史的垃圾时间''论调,官媒发文反击,称这是一个伪学术概念,某些人借题发挥,对国家发展阴阳怪气,影射``无奈无望''。 所谓``垃圾时间''(Garbage~time),原是一个用在多种限时制体育赛事中的术语,意指一场比赛中,对垒两队的分数差距太大、胜负结果难以改变时,剩余的比赛时间就称为``垃圾时间''~……}

\entryitemWithDescription{江西基础教育大面积肃贪 至少14名校长、书记落马}{https://www.zaobao.com/news/china/story20240713-4254297}{(上海综合讯)中国江西省基础教育领域掀起大面积肃贪行动,四个月内至少有14名中学校长、书记落马。 据江西宜春市纪委监委微信公号``廉洁宜春''星期五(7月12日)通报,该市万载县第二中学原党总支书记、校长宋雄伟涉嫌严重违纪违法,主动向组织交代问题,目前正接受县纪委监委纪律审查和监察调查……}

\entryitemWithDescription{台国防部首次通报大陆火箭军多批试射 分析:向对岸展示台军侦察能力}{https://www.zaobao.com/news/china/story20240713-4253954}{中国人民解放军东部战区5月24日发布的这张未注明日期照片显示,中国军方举行``联合利剑-2024A''军演期间,在福建省的一座基地台发射火箭。(法新社) (台北/北京综合讯)台湾国防部首次通报中国人民解放军火箭军部队在内蒙古进行多波次试射,分析认为这是一种``侦查性吓阻'',旨在向中国大陆展示台湾侦察、掌握其军事动态的能力……}

\entryitemWithDescription{中国抗议拜登签署西藏法案}{https://www.zaobao.com/news/china/story20240713-4254011}{(北京/华盛顿综合讯)美国总统拜登签署了一项名为促进解决西藏争议的法案,被中国斥为粗暴干涉中国内政,并向美方提出严正交涉。 拜登是在当地时间星期五(7月12日)签署《促进解决藏中争议法案》(简称法案)成法。他通过白宫官网发表声明说,他与国会两党一样,致力于促进藏人人权与支持维护西藏独特的语言、文化和宗教遗产,并将继续呼吁北京恢复与达赖喇嘛或其代表进行不设前提的直接对话,以寻求解决分歧的办法……}

\entryitemWithDescription{于泽远:中美爆发电子战?}{https://www.zaobao.com/news/china/story20240713-4248971}{中美两军6月下旬在南中国海进行电子战的消息,近日在中国互联网上传得沸沸扬扬。消息称,中美这场12小时的电子对抗战,使菲律宾吕宋岛北部的全球定位系统(GPS)信号和通信信号一度出现大面积失灵,结果是解放军大获全胜。 一些港台媒体和名嘴也对中美电子战发表了评论,但中美双方都未证实这场电子战的存在。有中国通信专家指出,所谓中美电子战导致吕宋岛大面积断网,是假消息……}

\entryitemWithDescription{郑文灿从总统接班人到涉贪人}{https://www.zaobao.com/news/china/story20240713-4246557}{民进党``胖周瑜''郑文灿涉贪出事,撼动全台。 57岁的郑文灿属于野百合学运世代领袖人物,也是桃园升格直辖市后首任且连任的市长。在台湾前总统蔡英文执政时期,郑文灿一度晋升党内热门的接班梯队。``胖周瑜''这个绰号也是蔡英文给的,赞许郑文灿足智多谋且身段柔软。 从呼声极高的总统接班人,陨落成被收押禁见的涉贪人士,郑文灿短短几年经历仕途过山车……}

\entryitemWithDescription{王毅批华盛顿峰会无端指责中国 分析:中方反驳北约烈度不强}{https://www.zaobao.com/news/china/story20240712-4251450}{北大西洋公约组织本周指责中国是俄罗斯对乌克兰开战的``决定性赋能者''后,中国外长王毅星期四反驳北约无端指责中国。图为王毅5月8日旁观中国和塞尔维亚两国元首举行会谈。(路透社) 北大西洋公约组织本周指责中国是俄罗斯对乌克兰开战的``决定性赋能者''后,中国外长王毅星期四反驳北约无端指责中国,并敦促北约不要插手亚太事务、不要干涉中国内政、不要挑战中方正当权益……}

\entryitemWithDescription{中国上半年自然灾害致经济损失达931亿人民币}{https://www.zaobao.com/news/china/story20240712-4251716}{7月11日航拍照片显示,重庆市垫江市在暴雨过后面对洪水泛滥,当地居民正利用轮式装载机进行疏散。 (法新社) (北京/重庆综合讯)据中国官方通报,今年上半年全国因自然灾害造成的直接经济损失达931.6亿元(人民币,下同,173亿新元),造成3238.1万人次不同程度受灾。 路透社报道,根据官方数据,这是自2019年以来,上半年与灾害相关的最严重损失……}

\entryitemWithDescription{揭中国运油乱象记者微博显示注销 安危引关注}{https://www.zaobao.com/news/china/story20240712-4250995}{(北京综合讯)中国媒体《新京报》揭露罐车运输食用油的乱象后,参与这篇报道的记者韩福涛的个人微博账号显示已注销,原因不明,其个人安危引起中国网民关注。 《新京报》7月初报道,中国粮油企业使用的罐车卸完煤制油,不清洗就直接运输食用油,这种乱象引发轩然大波。中国国务院食安办星期二(7月9日)宣布成立调查组彻查……}

\entryitemWithDescription{传美在台协会处长提一中政策被删 台总统府要求媒体撤下报道}{https://www.zaobao.com/news/china/story20240712-4249869}{美国在台协会(AIT)台北办事处新任处长谷立言,在拜会台湾总统赖清德时提到一中政策,但这番言论据报被台湾总统府删除。图为谷立言7月10日在台湾总统府发表讲话。(法新社) (台北综合讯)美国在台协会(AIT)台北办事处新任处长谷立言,在拜会台湾总统赖清德时提到一中政策,但这番言论据报被台湾总统府删除。总统府否认删除的动机和可能性,并要求媒体撤下报道……}