\entryitemWithDescription{1月至2月 中国医院冠病死亡人数骤降97\%遭外界质疑}{https://www.zaobao.com/news/china/story20230216-1363565}{中国医院的冠病相关死亡人数骤降,再次引发外界质疑数据真实性。 《人民日报》健康客户端根据来自中国疾病预防控制中心公布的数据报道,在2022年12月8日至2023年2月9日,累计在医院感染冠病死亡病例8万3150例,其中基础疾病合并冠病感染导致病情加重死亡的占大部分,共有7万6519例,占比约92.02\%……}

\entryitemWithDescription{分析:中国足协主席陈戌源等四人先后落马 表明腐败几乎渗透各部门各行业}{https://www.zaobao.com/news/china/story20230216-1363566}{陈戌源(黑衣捧杯者)曾在上海港务系统工作长达近半世纪,2019年12月成为首位非体育系统出身的中国足协主席。(互联网) 有中国体育记者称,陈戌源因为在国家队征战期间,``让国脚陪着踢球,让保障人员陪着打麻将\ldots\ldots 每天歌舞升平,分奖金''而被调查……}

\entryitemWithDescription{陆委会同意以专案方式 让上海台办参加台湾灯会}{https://www.zaobao.com/news/china/story20230216-1363567}{台湾政府的大陆委员会星期三宣布,由于台北市政府表达充分配合民进党政府政策及法令规范,同意以专案方式许可上海市台湾事务办公室六名官员,赴台参加台湾灯会。 陆委会星期三(2月15日)表示,台北市政府日前提出专案申请,邀请上海市政府人员于2月18日至20日赴台参加``2023台湾灯会在台北''相关活动及市政交流……}

\entryitemWithDescription{中国恢复签发韩公民赴华短期签证}{https://www.zaobao.com/news/china/story20230216-1363568}{中国自2月18日起,恢复签发韩国公民赴华访问、商务、过境,以及一般私人事务类短期签证。 中国驻韩国大使馆微信公号在星期三(2月15日)早上9时,公布上述消息。此前,韩国宣布从上星期六(2月11日)起恢复向中国公民发放赴韩短期签证。 中国国家移民管理局也公告,鉴于韩国已恢复审发中国公民赴韩短期签证,国家移民管理机构自2月18日起对韩国公民恢复签发口岸签证,恢复实行72、144小时过境免签政策……}

\entryitemWithDescription{加拿大将停止资助有国安疑虑研究项目}{https://www.zaobao.com/news/china/story20230216-1363569}{加拿大政府周二宣布,将停止资助与外国国防或安全实体有关联的研究项目。 据法新社报道,加拿大政府周二(2月14日)宣布,如果研究专案中课题敏感,参与人员与敌对国的国防或安全实体有关连,政府将停止资助这些项目。 加拿大科学卫生和公共安全部长在声明中写道:``加拿大先进的研究生态系统处于世界前沿,但对于想对加拿大国家安全构成威胁的国家来说,它也可能是一个有吸引力的目标……}

\entryitemWithDescription{台海基会董事长李大维:恢复两岸直航很快有答案}{https://www.zaobao.com/news/china/story20230216-1363570}{对于中国大陆喊话恢复两岸直航航点,台湾海基会董事长李大维星期三(2月15日)受访时说,目前正在讨论,很快就会有答案。学者张五岳则分析,两岸人员交流及航点都可望逐步恢复,但双方仍难恢复官方授权对话谈判,因此唯一能做的仍是避免误判,做好危机管控。 大陆国务院台湾事务办公室本月初向台湾喊话,呼吁率先恢复16个两岸直航航点……}

\entryitemWithDescription{英议员强烈反对下 新疆自治区主席取消访英}{https://www.zaobao.com/news/china/story20230216-1363571}{在英国国会议员强烈反对之下,英国外交部星期二(2月15日)说,新疆维吾尔自治区党委副书记、区政府主席艾尔肯·吐尼亚孜,已取消本周访问英国的计划。 综合路透社、法新社报道,英国外交部发言人说:``据我们了解,新疆自治区主席已取消对英国的访问''。发言人说,英国政府将继续利用所有机会,对中国政府在新疆不可接受的侵犯人权行为采取行动……}

\entryitemWithDescription{早说}{https://www.zaobao.com/news/china/story20230216-1363572}{深圳和香港要发展,如果不融合,谁也出不了彩;如果融合了,大家一起出彩。 ------香港中文大学(深圳)全球与当代中国高等研究院院长郑永年教授接受《深圳特区报》访问时说,推动高质量发展需要三大平台:基础科研能力、将科研成果转换为应用技术的能力,以及开放的金融平台为企业发展提供支持,深圳就需要借助香港的发达的金融基础。这就是为什么中国中央政府反复强调粤港澳大湾区融合发展,融合了才能一起出彩……}

\entryitemWithDescription{南岳衡山雪初霁 雾凇隐现云海中}{https://www.zaobao.com/news/china/story20230216-1363573}{湖南省衡阳市南岳衡山风景区2月15日雪后初霁,晶莹剔透的雾凇在云海里若隐若现,景色如画,登山游客忙着赏雪拍照……}

\entryitemWithDescription{陈婧:扫黑剧《狂飙》反派太红引争议}{https://www.zaobao.com/news/china/story20230216-1363574}{``你飚到哪了?我已经快飚完了\ldots\ldots'' 最近和中国朋友们聊天,大家三句不离``飚''------说的是农历新年期间热播的电视剧《狂飙》。 这部由中国央视和网络影视平台爱奇艺联合出品的扫黑题材剧集,开播以来口碑和收视率都一路``狂飙'',热度甚至盖过春节档电影。该剧在2月1日播出大结局当天,相关词条霸屏微博热搜榜;影音口碑网站豆瓣上至今共有超过53万人打分,评分仍高达8.5……}

\entryitemWithDescription{去年封控使商业活动停滞不前 欧盟商会吁中国恢复外企信心}{https://www.zaobao.com/news/china/story20230216-1363575}{图为2月14日情人节的上海外滩,不少民众在外滩金融中心的商场露台,在高达6米的巨型玫瑰艺术装置(右下)前拍照留念。(新华社) 据彭博社报道,中国欧盟商会上海分会在公布的年度报告中建议,中国官方应消除外国企业面对的投资壁垒,促进人民币国际化,并加强对中小企业的金融支持。 (上海彭博电)随着中国结束持续三年的清零政策,在上海的欧洲企业呼吁中国政府设法恢复外国企业的信心,采取措施修复与国际之间的关系……}

\entryitemWithDescription{巴菲特减持台积电致股价跌5%}{https://www.zaobao.com/news/china/story20230216-1363576}{巴菲特几个月内买入又卖出台积电股,牵动其股价涨了又跌。图为台积电在台南厂房上的公司标志。(路透社) ``股神''巴菲特麾下的伯克希尔哈撒韦(Berkshire Hathaway),在2022年第四季度减持约5180万股台湾积体电路制造公司(台积电)的股票,占持股数量约86\%,导致台湾这家芯片制造商巨头盘后股价重跌5%……}

\entryitemWithDescription{报告:中国去年防疫开支至少688亿元}{https://www.zaobao.com/news/china/story20230216-1363577}{中国多个省区市政府的年度预算报告显示,地方政府在2022年至少花费了3520亿元(人民币,下同,688亿新元)遏制冠病传播。在经济增长放缓的年头,这笔开支使各地财政备受压力。 富裕省份支出较多 据路透社报道,中国31个省区市当中,目前有至少20个已公布当地去年的抗疫支出,其中富裕省份的支出最多……}

\entryitemWithDescription{爱尔兰终止``黄金签证''项目}{https://www.zaobao.com/news/china/story20230216-1363578}{爱尔兰经过评估,决定关闭受中国富人阶层青睐的``黄金签证''项目。 这项2012年推出的投资移民计划,适用于个人财富至少200万欧元(286万新元)的非欧盟公民。爱尔兰司法部星期二(2月14日)宣布,2月15日后将不再接受相关申请。 司法部长哈里斯说:``我们要审查所有签证项目,包括对更广泛公共政策的影响,例如签证项目在文化、社会和经济方面的持续适当性和适用性……}

\entryitemWithDescription{中国恢复采购俄罗斯打折原油}{https://www.zaobao.com/news/china/story20230216-1363579}{消息人士透露,中国石油和中国石化正在恢复采购俄罗斯的打折原油。 据路透社报道,知情人士透露,中国国家炼油商已获得许可,可以从贸易公司处以折扣价购买俄罗斯原油,这将大幅降低中国石油进口成本,并提高炼油利润率。 路孚特和船舶追踪网站Kpler数据显示,中石油将于本月晚些时候在广西钦州炼油厂接收约150万桶乌拉尔原油。该炼油厂去年10月和11月分别进口了73万桶乌拉尔原油……}

\entryitemWithDescription{台专家:气球侦察或成常态 为防擦枪走火各国须建军事互信}{https://www.zaobao.com/news/china/story20230215-1363232}{中国气球2月初被发现入侵美国领空,随后遭美军击落,气球碎片掉落在南卡罗来纳州外海。图为美国海军人员2月10日准备将捞起的气球残骸运送往弗吉尼亚州的联邦调查局实验室进行分析。 (美国海军/路透社) 台湾国防安全专家指出,中美气球事件反映出各国对探空气球被更广泛运用在军事探勘而感到的不安。这种军事科技的急速发展,破坏了自冷战以来的军事互信,加深地缘强权之间的猜忌……}

\entryitemWithDescription{台官方:台海未发现大陆侦察气球}{https://www.zaobao.com/news/china/story20230215-1363233}{台湾官方说,未在台海周边空域发现任何中国大陆的侦察气球,所出现的大多为用于气象探测的气球。 英国《金融时报》星期一(2月13日)引述台湾高官报道,中国大陆军用气球近年数十次出现在台湾空域,平均每月一次,令人担忧北京可能在为攻台做准备……}

\entryitemWithDescription{李家超:港府仍希望今或明年完成第23条立法}{https://www.zaobao.com/news/china/story20230215-1363234}{香港特首李家超重申特区政府的立场没变,仍希望在今年或明年完成《基本法》第23条立法。 根据香港政府新闻网站,李家超星期二(2月14日)在行政会议前回答媒体提问时说,为《基本法》第23条立法是港府的宪制责任。他认为,各方可能对第23条立法所针对的问题严重性未必全面掌握,并强调国家安全风险千变万化,目前国际关系复杂,国安风险仍然有潜伏在香港的可能性……}

\entryitemWithDescription{早说}{https://www.zaobao.com/news/china/story20230215-1363236}{面对强权争霸的地缘政治情势,中华民国和国民党都没有``选边站''的问题,只有与盟友及国际友人共同维系台海和平与稳定的强烈意愿。 ------国民党国际事务部主任兼驻美代表黄介正,据报在农历新年期间走访20几个美国国会议员办公室。他星期二(2月14日)分析说,美国根本不在意台湾的政党是不是玩表面的``亲美比赛'',坦诚分析观点角度,具体扩大共同利益,才是真正有信赖关系的盟友……}

\entryitemWithDescription{杨丹旭:中美气球事件能否在慕尼黑翻篇?}{https://www.zaobao.com/news/china/story20230215-1363237}{中国最高外交官员王毅星期二启程前往欧洲,此行将访问法国、意大利、匈牙利和俄罗斯,也会到德国出席慕尼黑安全会议。 除了中欧关系和俄乌战争进入一周年之际的中俄互动,王毅欧洲行的另一看点是:在气球事件后,中美外交高层是否会在慕尼黑实现原本计划在北京进行的面对面会晤。 一只飘移到美国上空的中国气球,2月以来成为中美交锋焦点,不仅搅黄美国国务卿布林肯访华行,还引发两国新一轮外交博弈……}

\entryitemWithDescription{台湾下周开放港澳居民赴台自由行 大陆有意磋商恢复台农渔产品进口}{https://www.zaobao.com/news/china/story20230215-1363238}{针对何时解禁让陆客访台,陆委会称大陆``突然完全解封'',让台方对大陆的疫情判断存在不确定,未来将按照中央流行疫情指挥中心的判断来处理。 两岸官方都释出春暖花开的和缓信号,台湾陆委会宣布下周一(2月20日)起开放港澳居民自由行赴台观光。中国大陆国台办也表态,愿与台湾共同努力,为恢复台湾农渔产品输入大陆提供帮助……}

\entryitemWithDescription{盼回国民党参加党内初选 郭台铭表态有意竞选2024年总统选举}{https://www.zaobao.com/news/china/story20230215-1363239}{鸿海集团创办人郭台铭星期二(2月14日)与台湾在野国民党前立法院长王金平见面后明确表示,希望回到国民党参加党内总统初选,代表国民党参选2024年总统。 郭台铭2019年曾以荣誉党员回归国民党,但参与党内总统初选落败,愤而宣布退选,未支持国民党提名的总统候选人韩国瑜。此次卷土重来,他不讳言称,四年前退党是年轻气盛、一时冲动,现在若能够为台湾来做事,他不拘任何形式,义不容辞……}

\entryitemWithDescription{甘肃山区梯田 大雪后如仙境}{https://www.zaobao.com/news/china/story20230215-1363240}{(新华社) 甘肃省东乡族自治县山乡在2月14日一场大雪后,银装素裹,云雾缭绕,宛如仙境。图为当天拍摄的东乡族自治县山区梯田雪景……}