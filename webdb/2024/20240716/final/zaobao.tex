\entryitemWithDescription{批中国咄咄逼人 新西兰总理:非常关切仁爱礁局势}{https://www.zaobao.com/news/china/story20240715-4268441}{新西兰总理拉克森(右)今年6月13日在惠灵顿总督府同到访的中国总理李强举行会谈。(法新社) 新西兰总理拉克森星期天批评中国在南中国海问题上咄咄逼人,试图阻止菲律宾向仁爱礁(菲称阿云津礁)的坐滩军舰提供补给。他希望今年与马尼拉达成部队互访协议,允许新西兰在菲律宾部署军事资产……}

\entryitemWithDescription{约20名中国大陆官员据报获准赴金门参加活动}{https://www.zaobao.com/news/china/story20240715-4268349}{(台北综合讯)金门附近海域态势紧张之际,据报约150名中国大陆人士获准组团到金门参加金厦泳渡活动,包括20名大陆官员。 台湾《旺报》星期一(7月15日)引述知情人士报道,台湾陆委会为了让活动避免统战色彩,在审核大陆方面提报名单时,特意把几名台办和统战系统官员排除,只批准放行大陆体育事务官员入境……}

\entryitemWithDescription{台湾首艘自造潜舰``海鲲号'' 在高雄出坞拟9月海试}{https://www.zaobao.com/news/china/story20240715-4267927}{(台北综合讯)台湾首艘自行建造的潜舰``海鲲号''首艘原型舰,星期一在高雄出坞,预计9月会进入海试阶段。 综合台湾《自由时报》和《中国时报》报道,海鲲号潜舰在台船公司干船坞坐墩近五个月后,星期一(7月15日)从注满水的船坞移往已设置好``活动浮台''的91号码头……}

\entryitemWithDescription{中国第二季经济增速弱于预期 分析:须加码提振内需}{https://www.zaobao.com/news/china/story20240715-4267419}{内需复苏乏力之际,中国第二季经济增长主要来自强劲的制造业活动。图为江苏连云港码头等待出口的汽车。(法新社) 中国第二季经济增速放缓至五个季度来最低的4.7\%,比市场预期来得差。分析认为,要确保5%左右的全年增长目标,官方须加码提振内需,带领世界第二大经济体走出通货紧缩阴影……}

\entryitemWithDescription{港媒:大陆学生用假学历赴港报读延烧至本科}{https://www.zaobao.com/news/china/story20240715-4267533}{香港一名女子7月11日在维多利亚港前摆姿势拍照。(法新社) (香港综合讯)中国大陆学生涉嫌用假学历报读香港高等院校的事件继续延烧。继30名大陆生被曝使用海外院校假学历报读香港大学硕士班课程后,香港再有大学据报正调查大陆生用海外虚假高中学历报读本科专业……}

\entryitemWithDescription{中正纪念堂三军仪队交接展示 首度移出蒋介石铜像大厅}{https://www.zaobao.com/news/china/story20240715-4267516}{台湾三军仪队星期一上午首度在台北中正纪念堂主堂体前的民主大道,进行巡查及训练展示。(法新社) (台北综合讯)台湾三军仪队星期一起正式移出中正纪念堂蒋介石铜像大厅,结束历时44年在大厅内的站哨与交接展示。这也是1980年中正纪念堂开馆以来,三军仪队首次移出中正纪念堂。 综合《联合报》、《自由时报》等台媒报道,星期天(7月14日)下午5时是最后一场的礼兵交接,期间一度有人高喊``中华民国万岁''……}

\entryitemWithDescription{中国多地倡议节约用电:路灯间隔开暂停灯光秀}{https://www.zaobao.com/news/china/story20240715-4267086}{(北京综合讯)中国进入夏季用电负荷高峰,多个省市单日用电量创历史新高,电力供应出现缺口。江西、安徽、重庆、太原等省市发布节约用电的倡议,包括间隔开启路灯、暂停城市灯光秀等,以缓解电力供应紧张。 综合澎湃新闻与央视财经报道,6月以来,中国多省市出现持续性高温天气,加上工业生产恢复,多地用电负荷突破历史记录。在工业大省广东,用电负荷在7月9日与10日连续两天创历史新高,最高达到1.49亿千瓦……}

\entryitemWithDescription{中国144小时过境免签政策扩大至河南云南两省}{https://www.zaobao.com/news/china/story20240715-4266424}{外国旅客7月10日在中国北京首都机场边防检查站,申请过境免签入境。(新华社) (北京综合讯)中国宣布144小时过境免签政策实施范围,进一步扩大至河南全省和云南省部分地区……}

\entryitemWithDescription{庄慧良:``郑文灿模式''的殒落}{https://www.zaobao.com/news/china/story20240715-4258271}{历经法庭三次攻防,涉贪的台湾海峡交流基金会前董事长郑文灿上星期四(7月11日)被收押,执政的民进党立即对他停权三年。无论华亚科学园区收贿案最后判决为何,这位曾经权倾一时的``大阿哥''政治生命已告终结。 郑文灿7月6日被桃园地检署声押禁见的消息震惊全台。当天他尚能面带笑容走出桃园地方法院,其友人于30分钟筹措500万元(新台币,下同,21万新元)协助交保……}

\entryitemWithDescription{中国周二开始进入``七下八上''防汛关键期}{https://www.zaobao.com/news/china/story20240714-4257930}{为应对后续长江上游可能发生的大洪水,三峡水库上星期六(7月13日)加开两孔,增至六孔泄洪,加速腾出防洪库容。(中新社) (北京/重庆综合讯)中国将从星期二(7月16日)开始进入为期一个月的``七下八上''防汛关键期,全国洪水多发频发,容易发生流域性洪水。中国官方为此调度四川、安徽、湖北、河南、山东等15省份,研究部署防汛救灾工作……}

\entryitemWithDescription{上台后首登军舰视导 赖清德期许台军守护海疆安全}{https://www.zaobao.com/news/china/story20240714-4257550}{台湾总统赖清德(左三)星期六(7月13日)与国防部长顾立雄(左二)等人前往基隆慰勉海军舰队官兵,并登上沱江级巡逻舰``旭江舰''视导。(赖清德脸书) (台北综合讯)台湾军方星期六(7月13日)首次通报中国大陆火箭军在内蒙古试射之际,台湾总统赖清德同日前往基隆慰勉海军舰队官兵,并登上军舰视导。这是他今年5月20日上台后首次这么做,以表明守护台湾的决心……}

\entryitemWithDescription{传在塔吉克斯坦建秘密军事基地 中国外交部间接否认}{https://www.zaobao.com/news/china/story20240714-4257417}{(北京/伦敦综合讯)中国据传正在中亚国家塔吉克斯坦建设秘密军事基地,以应对塔利班统治下的阿富汗构成的安全威胁。中国外交部上星期五(7月12日)间接否认传言,表示不掌握情况,并称中国在中亚没有任何军事基地。 据英国《每日电讯报》上星期三(10日)报道,卫星图片显示,北京在塔吉克斯坦建设秘密军事基地。该设施位于近4000米的高山上,有瞭望塔和两国驻军……}

\entryitemWithDescription{特稿:深圳失业者图书馆``伪装''上班}{https://www.zaobao.com/news/china/story20240714-4255146}{有深圳失业网民称,他们选择在图书馆``伪装''上班,以维持日常的生活节奏,避免家人担忧。本报记者日前走访位于红山区的深圳图书馆时发现,即便是工作日,馆内也是人满为患。许多年轻人带着一台电脑,有的在发简历,有的在看视频,有的则在抱头大睡。(林煇智摄) 35岁的郑敏(化名)是一名财务经理,去年底被一家咨询服务公司裁退后,一直处于失业状态……}

\entryitemWithDescription{中俄时隔一年在太平洋海域再展开海上联合航巡}{https://www.zaobao.com/news/china/story20240714-4257168}{俄罗斯护卫舰格罗姆基号在广东湛江举行的中俄``海上联合-2024''联合演习期间驶入湛江港,图片取自官方7月13日发布的视频。(路透社) (北京/莫斯科综合讯)北约上周将中国列为俄罗斯对乌克兰开战的``决定性赋能者''后,中国国防部星期天(7月14日)宣布中俄海军军舰近日在太平洋西部和北部海域开展第四次海上联合巡航,这是两国时隔近一年在相同海域再开展联合巡航……}

\entryitemWithDescription{中菲执法部门合作遣返三名绑架嫌犯}{https://www.zaobao.com/news/china/story20240714-4256802}{中国和菲律宾两国执法部门星期六(7月13日)合作遣返三名涉嫌在菲绑架中国公民的犯罪嫌疑人。(中国驻菲律宾大使馆网站) (北京/马尼拉综合讯)中国和菲律宾两国执法部门合作,遣返三名涉嫌在菲绑架中国公民的犯罪嫌疑人。这是两名中国企业高管在菲被撕票后,双方加强合作打击跨国犯罪的最新动作……}

\entryitemWithDescription{台湾特稿:党内挥扫贪大刀 赖清德要砍哪儿?}{https://www.zaobao.com/news/china/story20240714-4227774}{台湾总统赖清德(左)上任不久,海峡交流基金会前董事长郑文灿7月初突然因涉贪去职,这股肃杀气氛令执政的民进党内噤若寒蝉,在野党也胆颤心惊。图为赖清德4月公布新任内阁时,介绍郑文灿将出任海基会董事长一职……}

\entryitemWithDescription{中国海军万吨医院船赴南中国海执行首次任务}{https://www.zaobao.com/news/china/story20240713-4254882}{中国南部战区海军万吨新舰``丝路方舟''号医院船将奔赴西南沙及华南沿海岛礁进行医疗巡诊,开展海上伤员救治演练等任务。(央视新闻) (北京综合讯)中国南部战区海军万吨新舰``丝路方舟''号医院船星期三(7月10日)赴南中国海域,开启入列以来的首次医疗服务。 综合南部战区微信公号与央视新闻消息,``丝路方舟''号在广东湛江某军港解缆起航,奔赴西南沙及华南沿海各岛礁,目的地包括杆列岛、永暑礁、永兴岛等……}

\entryitemWithDescription{中国再爆水路食用油化工油混运}{https://www.zaobao.com/news/china/story20240713-4254662}{中国媒体爆出有煤油化工罐车运输食用油后,又爆出水路运输也有化工油和食用油混用问题。图为在长江水域行驶的货船、船舶。(新华网) (北京综合讯)中国以煤油化工罐车运输食用油引起舆论哗然,而最新的媒体报道揭露,不仅是公路货运,食用油经水路运输也存在混运现象。 财新网星期五(7月12日)报道,有食用油专用运输船船长对其意向客户称,他们是持有植物油运输证的食用油运输船,但没货时也接运白油、柴油等其他油品……}

\entryitemWithDescription{郑文灿将至少被关押两个月}{https://www.zaobao.com/news/china/story20240713-4254813}{台湾海峡交流基金会前董事长郑文灿(左二)涉嫌在华亚科学园区开发案中收贿,7月11日历经三次羁押庭后被裁定收押禁见。(互联网) (台北讯)台湾高等法院驳回海基会前董事长郑文灿代表律师的抗告,郑文灿将至少被关押两个月。 综合《联合报》和中时新闻网等报道,民进党籍桃园市前市长郑文灿在两度交保后,星期四(7月11日)在第三次羁押庭召开后被裁定收押禁见。郑文灿不服裁决,由代表律师提出抗告……}

\entryitemWithDescription{欧盟对华电动车关税案德国或投弃权票 分析:关税难逆转 但税率可谈判}{https://www.zaobao.com/news/china/story20240713-4254567}{路人走过意大利米兰的比亚迪汽车门店前。照片摄于3月20日。(路透社) 德国据报将在星期一(7月15日)就欧盟对中国电动汽车征收关税提案投下弃权票,从而推动欧盟与中国继续就此磋商。分析认为,欧盟对华加征关税的大势难以逆转,但具体税率还有谈判空间。 路透社星期五(7月12日)引述匿名知情人士说法称,德国在关税案第一阶段投票中选择弃权,是因为欧盟对华反补贴调查仍在继续,双方谈判也还在进行……}

\entryitemWithDescription{中国网络频现``历史的垃圾时间''论调 官媒反驳:伪学术概念}{https://www.zaobao.com/news/china/story20240713-4254420}{(北京综合讯)中国互联网世界频繁出现``历史的垃圾时间''论调,官媒发文反击,称这是一个伪学术概念,某些人借题发挥,对国家发展阴阳怪气,影射``无奈无望''。 所谓``垃圾时间''(Garbage~time),原是一个用在多种限时制体育赛事中的术语,意指一场比赛中,对垒两队的分数差距太大、胜负结果难以改变时,剩余的比赛时间就称为``垃圾时间''~……}

\entryitemWithDescription{江西基础教育大面积肃贪 至少14名校长、书记落马}{https://www.zaobao.com/news/china/story20240713-4254297}{(上海综合讯)中国江西省基础教育领域掀起大面积肃贪行动,四个月内至少有14名中学校长、书记落马。 据江西宜春市纪委监委微信公号``廉洁宜春''星期五(7月12日)通报,该市万载县第二中学原党总支书记、校长宋雄伟涉嫌严重违纪违法,主动向组织交代问题,目前正接受县纪委监委纪律审查和监察调查……}