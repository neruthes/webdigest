\entryitemWithDescription{陈婧:耶伦访华是为了卖美债?}{https://www.zaobao.com/news/china/story20230706-1411130}{美国财长耶伦访问中国的消息出炉后,资本市场反响热烈,外界也对她此行会谈些什么充满好奇。 耶伦是继美国国务卿布林肯之后,短短三周内第二位访问北京的白宫高官。和布林肯两天一夜的中国之行相比,耶伦的访华行程长达四天,外界预计她将与多名中国经济和财经领域官员会面,包括刚刚走马上任的中国央行党委书记潘功胜……}

\entryitemWithDescription{歌手李玟患抑郁症 轻生离世得年48岁}{https://www.zaobao.com/news/china/story20230705-1411112}{享誉大中华地区的美籍华语流行女歌手李玟,有无数的歌曲代表作,其中歌曲《A Love Before Time》曾入围奥斯卡最佳原创歌曲奖。(李玟微博) 天后李玟(CoCo)惊传过世!她的姐姐李思林星期三(7月5日)在脸书发布噩耗,透露李玟患上抑郁症,上星期天(7月2日)在家中轻生,星期三离世,得年48岁……}

\entryitemWithDescription{香港立法会料周四通过区议会改革方案 学者:区议会将由港府主导}{https://www.zaobao.com/news/china/story20230705-1411107}{香港立法会《2023年区议会(修订)条例草案》星期三(7月5日)恢复二读辩论。根据香港立法会程序,参加会议的议员若不足法定人数,主席须宣布休会。但由于本届立法会再没有民主派议员随时``突袭''提出点算人数,周三的会议大部分时间只有二三十名议员在场。(中通社) 香港立法会周三(7月5日)恢复二读辩论区议会改革方案,预料周四大概率通过,以便今年年底举行的新一届区议会选举可以全面落实``爱国者治港''……}

\entryitemWithDescription{赖清德投书美媒端出``和平保台''方案 学者指华府欢迎但对选情提振有限}{https://www.zaobao.com/news/china/story20230705-1411102}{民进党总统参选人、现任台湾副总统赖清德在美国媒体投书,提出加强国防、经济、民主和维持现状的``和平四大支柱政见''。(赖清德脸书) 民进党总统参选人、现任台湾副总统赖清德在美国媒体投书,提出加强国防、经济、民主和维持现状的``和平四大支柱政见'',为他的``和平保台''口号端出具体方案。 学者认为,这番表态会受到华府欢迎,但对缓和两岸关系、提振选情的效果有限……}

\entryitemWithDescription{国民党中央设法化解``换侯''危机}{https://www.zaobao.com/news/china/story20230705-1411087}{台湾在野国民党总统参选人侯友宜民调持续垫底,部分国民党中常委酝酿在7月23日国民党全台党代表大会前的一次中常会提案``换侯''。国民党主席朱立伦在星期三(7月5日)中常会强调,任何杂音、不当举措,只会长他人志气、灭自己威风,国民党要团结才能胜选。 侯友宜竞选办公室执行长金溥聪星期三召开记者会,呼吁国民党同志要有耐心,真正决战是最后三个月……}

\entryitemWithDescription{张忠谋:国安、经济与科技领先已凌驾全球化之上}{https://www.zaobao.com/news/china/story20230705-1411073}{全球最大晶片代工企业台积电创始人张忠谋说,全球化已被重新定义,国家安全、科技领先及经济领先的重要性,已凌驾于全球化之上……}

\entryitemWithDescription{欧盟对中国限制原材料出口表关切 分析:中欧紧张关系将加剧}{https://www.zaobao.com/news/china/story20230705-1411071}{欧盟外交与安全政策高级代表博雷利原定于7月10日访华,但欧盟发言人7月4日称,中国推迟了博雷利的访华行程。(路透社) 中国出台镓和锗相关出口管制措施后,中欧紧张关系再出现升级迹象;欧盟委员会星期二(7月4日)对中国新措施表达关切,同天传出中国单方面推迟欧盟外交与安全政策高级代表博雷利下周访华行程的消息……}

\entryitemWithDescription{重庆万州区洪涝灾害致15死四失踪}{https://www.zaobao.com/news/china/story20230705-1411059}{重庆万州区7月4日遭受暴雨袭击,救援人员用冲锋舟在受洪水淹没的街道协助民众疏散。(路透社) 中国西南、东北多地近日连遭暴雨,西部直辖市重庆万州区暴雨引发洪涝灾害,截至星期三(7月5日)上午10时,已造成当地15人死亡,四人失踪……}

\entryitemWithDescription{中国国防部:美军售加速把台变成``火药桶''}{https://www.zaobao.com/news/china/story20230705-1411056}{针对美国最新批准对台军售,中国国防部新闻发言人谭克非星期三(7月5日)批评,美国此举蓄意推升台海紧张局势,无异于加速把台湾变成``火药桶'',把台湾民众推向灾难深渊。 据中国国防部网站消息,谭克非说,中方坚决反对美方向台出售武器,已向美方提出严正交涉。他批评,美方罔顾中方核心关切,粗暴干涉中国内政,蓄意推升台海紧张局势……}

\entryitemWithDescription{浙江破获首起利用ChatGPT制作虚假视频案}{https://www.zaobao.com/news/china/story20230705-1411032}{中国再添一起利用人工智能技术制作传播假新闻的案件。浙江绍兴警方星期三(7月5日)通报,破获该省首起利用ChatGPT技术制作虚假视频的案件。 据绍兴公安官方微信公号``绍兴公安群蓝星''消息,绍兴上虞区警方6月2日在网上巡查时发现,有网民发布了关于上虞工业园区发生火灾的视频,并在短时间内浏览量迅速攀升。警方介入核实后,确认视频为不实信息……}

\entryitemWithDescription{澳总理:无法接受港府悬红通缉两澳居民}{https://www.zaobao.com/news/china/story20230705-1411020}{针对香港特区政府悬红通缉的潜逃犯中包括两名澳大利亚居民,澳洲总理阿尔巴尼斯抨击港府这一做法``实在令人无法接受''。 据路透社报道,阿尔巴尼斯星期三(7月5日)接受澳洲电视台九号电视台采访时作出上述表态。 阿尔巴尼斯称,澳洲将持续在能够合作的领域与中国合作,但两国存在分歧的问题上,澳洲政府必须提出反对,``而我们确实在人权问题上存在分歧''……}

\entryitemWithDescription{分析:中日虽因福岛核污水争论 仍保有沟通管道}{https://www.zaobao.com/news/china/story20230705-1411019}{日本传出计划最快8月起,将福岛核污水排入太平洋。中国生态环境部7月5日批评,日本在核污染水排海的正当性、净化装置可靠性、监测方案完善性等方面还存在诸多问题。图为福岛第一核电站后方。(彭博社) 在获得联合国国际原子能总署(IAEA)批准后,日本传出最快8月起将福岛核污水排放入海,中国官方星期三(7月5日)再度发出批评。就当两国陷入龃龉之际,日本前众院议长河野洋平、冲绳县知事却正访问北京……}

\entryitemWithDescription{杨丹旭:中国反向``卡脖子''}{https://www.zaobao.com/news/china/story20230705-1410748}{美国财政部长耶伦访华前夕,北京以国家安全为由出台最新的原材料出口管制令,让中美科技战加速升温。 中国商务部和海关总署星期一(7月3日)发布公告,从8月1日起,中国将对镓、锗相关物项实施出口管制,理由是``维护国家安全和利益''。 根据公告,这些物项要出口,必须办理出口许可手续,出口商得通过省级商务主管部门向商务部提出申请,还得上报物项的最终接收方和用途……}

\entryitemWithDescription{侯友宜改口称唯有台海和平才缩短兵役 台媒评论:忌惮美方反弹}{https://www.zaobao.com/news/china/story20230704-1410747}{侯友宜星期一刚宣示若当选总统,将把兵役由一年恢复四个月;不料隔天就改口,称重要前提是台海两岸必须先和平稳定。图为台军一辆架设美制拖式导弹射击系统的悍马军车,星期一在屏东准备进行实弹射击演习。(路透社) 台湾在野国民党总统参选人侯友宜,星期一(7月3日)首度表态接受``合乎中华民国宪法的九二共识'',并宣示若当选总统,``确定两岸稳定和平以后,我就(把兵役由一年)恢复四个月''……}

\entryitemWithDescription{英美强烈谴责香港通缉流亡海外民运人士}{https://www.zaobao.com/news/china/story20230704-1410741}{香港警方国安处首次悬红通缉八名涉嫌违反《香港国安法》且潜逃海外的民主派人士,引起美国强烈批评,认为是威胁全球人民人权和基本自由的危险先例。香港特首李家超回应时表示,支持警方行动,会用尽一切合法手段终身追究他们的法律责任。 李家超星期二(7月4日)出席行政会议前表示,这次行动是要发出明确信号,表明港府不会容忍有关八人的违法行为。他相信警方会拘捕到他们,无论相关人士走到天涯海角,特区政府都执法必严……}

\entryitemWithDescription{定格百年摄影展苏州启幕 新中历史照片最吸引中国读者}{https://www.zaobao.com/news/china/story20230704-1410730}{伴随着``定格百年''的背景灯亮起,《联合早报》中国摄影展星期二(7月4日)在苏州金鸡湖畔的苏州中心启幕,标志着早报下来两个月在中国的五城摄影巡展拉开序幕。 这是早报继今年4月在新加坡办摄影展后,在海外举办的首场百年报庆摄影展。 《联合早报》总编辑吴新迪在启幕礼上,讲述了早报与中国过去30多年的缘分……}

\entryitemWithDescription{蒋万安今起访新四天借镜城市治理}{https://www.zaobao.com/news/china/story20230704-1410727}{台北市长蒋万安表示,希望借镜新加坡的城市治理,提升市民生活环境。(台北市政府提供) 台北市长蒋万安星期三(7月5日)率团访问新加坡四天进行市政交流,拜访建屋发展局、陆路交通管理局、市区重建局和永续发展与环境部,以及参观组屋区、小贩中心、滨海湾花园和星耀樟宜。 44岁的蒋万安是国民党中生代领袖,也是台湾已故前总统蒋经国的孙子。这是他上任半年来第一次出访……}

\entryitemWithDescription{贺建奎再提争议性研究 用基因编辑预防阿尔茨海默病}{https://www.zaobao.com/news/china/story20230704-1410716}{曾因``基因编辑婴儿''获刑三年的中国科学家贺建奎发布一份研究提案,希望利用基因编辑技术来预防阿尔茨海默病,他的这一提议再次引发争议。 综合彭博社与CNN报道,贺建奎上周四(6月29日)在推特账号公布一项研究计划,他认为,阿尔茨海默病可能有基因特性,提议对某段特定的基因序列进行编辑,来测试这能对阿尔茨海默氏症产生的预防作用……}

\entryitemWithDescription{国台办回应侯友宜九二共识说 重申两岸同属一中}{https://www.zaobao.com/news/china/story20230704-1410695}{台湾最大在野党国民党总统参选人侯友宜星期一(7月3日)首度表态支持``合乎中华民国宪法''的``九二共识''后,中国大陆隔日回应,重申``九二共识''的核心意涵是两岸同属一中,北京愿在坚持``九二共识''、反对``台独''的共同政治基础上,与国民党保持良性互动……}

\entryitemWithDescription{【东谈西论】国民党``侯老三''放大招,总统选情能起死回生吗?}{https://www.zaobao.com/news/china/story20230704-1410659}{2024年台湾总统大选候选人(从左到右):侯友宜(中国国民党)、柯文哲(台湾民众党)、赖清德(民主进步党)。(互联网) 2024年台湾总统选举进入半年倒计时,各政党厉兵秣马加速冲刺。 民调落后的最大在野党国民党候选人侯友宜星期一(7月3日)晚上突然放大招,松口表示支持九二共识。侯友宜还承诺维护两岸和平,让台湾的兵役制度从一年恢复到四个月……}

\entryitemWithDescription{戴庆成:香港保得住``廉洁之都''地位吗?}{https://www.zaobao.com/news/china/story20230704-1410367}{上周六(7月1日)是香港回归中国26周年纪念日。连月来,建制派媒体持续不断地报道和宣传香港过去26年在经济、文化、教育等领域取得的一系列成就。有趣的是,竟然没有人指出香港九七年回归后在廉政方面的成绩。 ``香港,胜在有ICAC。''香港廉政公署(ICAC)自1974年成立以来一直独立运作,不惧不偏地奋力反贪……}