\entryitemWithDescription{新闻人间:岑浩辉——从``澳人治澳''过渡到``京人治澳''?}{https://www.zaobao.com/news/china/story20240914-4713964}{澳门特首选举提名期星期四(9月12日)结束,终审法院前院长岑浩辉星期二(9月10日)向选管会提交提名表时,一人独占约96%选委的提名,可以说已经预告:选举虽未举行,结果都知道了,很有中国特色。 对于岑浩辉出任下一任澳门特首,不少评论都提到``京人治澳'',甚至认为这是``澳人治澳''时代的终结……}

\entryitemWithDescription{陈婧:华为与苹果的隔空交手}{https://www.zaobao.com/news/china/story20240914-4724349}{位于上海的华为旗舰店。(彭博社) 一年前的此时,上海南京路步行街上的华为旗舰店几乎每天都门庭若市。开业前三小时,就有人在店外排队,等着抢购新发售的Mate 60 Pro手机。 这副热闹场景,反衬得马路对面的苹果旗舰店略显冷清。 一年过后,苹果和华为选在同一天发布新手机。从中国社媒平台的反响来看,华为声势再胜一筹……}

\entryitemWithDescription{中国防长董军:各国正当主权安全利益神圣不可侵犯}{https://www.zaobao.com/news/china/story20240914-4724612}{中国国防部长董军星期五(9月14日)在北京香山论坛开幕式上发表主旨演讲。(路透社) 中国国防部长董军在北京香山论坛上强调,各国正当的主权安全利益神圣不可侵犯,并呼吁国际社会就俄乌战争和以哈冲突劝和促谈,但他罕见没有在主旨演讲中提及台湾问题,整体调子相对过去缓和。 受访学者认为,董军讲话对外展现柔和姿态,旨在延续中美两国军方近期的互动态势,稳定地区局势……}

\entryitemWithDescription{中国70年来首次调高退休年龄 用15年时间上调三至五岁}{https://www.zaobao.com/news/china/story20240913-4724802}{中国官方9月13日宣布逐步提高法定退休年龄。图为在安徽阜阳一个公园里休息的中国老人。(法新社) 为应对人口老龄化问题 中国官方酝酿超过10年的延迟退休政策星期五(9月13日)落地,将在15年内把职工退休年龄分情况上调三至五岁,以缓解人口老龄化带来的劳动力不足,减轻政府的养老负担。 这是中国自上世纪50年代以来首次调高退休年龄……}

\entryitemWithDescription{黄永宏:中国在南中国海角色关键}{https://www.zaobao.com/news/china/story20240913-4724733}{国防部长黄永宏说,中国在南中国海发挥关键作用,希望中菲能继续就南中国海主权争议展开对话,并建议下一步扩大《海上意外相遇规则》。 黄永宏星期五(9月13日)在北京举行的第十一届香山论坛第一次全体会议上发言时,作出上述表示。他也说,中菲持续的沟通渠道如热线电话是有必要的,尤其是在危机发生的时候。 中菲过去一年在具主权争议的仁爱礁(菲称阿云津礁)海域频起冲突,紧张局势不断升级……}

\entryitemWithDescription{民调显示多数人认为柯文哲获法院公平审理 但500万人抱不平}{https://www.zaobao.com/news/china/story20240913-4724708}{(自由时报) 台湾在野的民众党主席柯文哲涉及京华城弊案被羁押禁见,台湾民意基金会星期五(9月13日)公布民调结果显示,40.3\%民众觉得柯文哲有获得法院公平审理,但26.4\%不作如是想。 台湾民意基金会董事长游盈隆指出,在台湾一个百分点约有19万5000人,26\%代表超过500万人为柯案抱不平,这是一个不可忽视的现象……}

\entryitemWithDescription{港府谴责英国发表的香港半年报告}{https://www.zaobao.com/news/china/story20240913-4724455}{游客9月5日在香港西九龙海滨长廊散步。(法新社) 英国政府星期四(9月12日)发表最新一份《香港半年报告》,批评港府持续针对任何被视作``威胁国安''的行为,正损害香港的国际声誉,敦促北京、港府恪守《中英联合声明》。北京和港府回应时批评英方粗暴干涉香港事务。 由英国外交部定期撰写的《香港半年报告》长达36页,是外长拉米7月上任后的首份……}

\entryitemWithDescription{恒大创始人许家印据报被移送至深圳特别拘留中心}{https://www.zaobao.com/news/china/story20240913-4722758}{继一年前因涉嫌违法犯罪被中国当局带走后,中国恒大集团创办人许家印据报已在数月前被移送到深圳一个特别拘留所。图为许家印2017年在香港出席房地产开发商年度业绩新闻发布会。(路透社档案照片) (香港路透电)继一年前因涉嫌违法犯罪被中国当局带走后,中国恒大集团创办人许家印据报已在数月前被移送到深圳一个特别拘留所……}

\entryitemWithDescription{特朗普政府驻联合国大使吁台湾尽自身责任 增购打击武器}{https://www.zaobao.com/news/china/story20240913-4708586}{特朗普政府时期的美国驻联合国大使克拉夫特(台上者)9月12日在``2024台北安全对话''发表专题演讲时提出,台湾面对中国大陆的威胁,必须成为太平洋的波兰。(缪宗翰摄) 美国前驻联合国大使克拉夫特(Kelly Craft)星期四(9月12日)在台北演讲时强调,如果特朗普重新执政,台湾应该要尽更多自身责任,成为``太平洋的波兰'',包括增加国防预算、采买更多打击型武器、扩展核能发电等……}

\entryitemWithDescription{于泽远:中国足球为什么落后?}{https://www.zaobao.com/news/china/story20240913-4708282}{继上周客场0比7惨败给日本队后,中国国家男子足球队(国足)本周二(9月10日)又在主场1比2输给了沙特队,在世界杯亚洲区18强预选赛中两战皆负,出线前景堪忧。 从场面上看,中国国足在与沙特比赛中的表现,明显强于对阵日本时的全面溃败:国足开场14分钟就先进一球,19分钟时沙特队又被红牌罚下一人。但即使11人打10人,中国队还是先被扳平,并在比赛快结束时被对手绝杀……}

\entryitemWithDescription{中国在香山论坛提高全球南方声量}{https://www.zaobao.com/news/china/story20240912-4710491}{中国通过今年的香山论坛,提升全球南方在防务问题上的声量,不仅首次将全球南方纳入这项防务论坛的全体会议议程,还在论坛伊始的高端访谈中,首次邀请多位亚非与中国学者探讨该议题。 不过学者指出,全球南方国家存在很多不同,也面对内部存在矛盾、深受大国关系紧张影响等挑战,只有管理好这些挑战,才能促进共同利益。 本届香山论坛于9月12日至14日在北京国际会议中心举行,主题为``共筑和平、共享未来''……}

\entryitemWithDescription{台湾自救宣言60周年 赖清德:共产党威胁依然存在}{https://www.zaobao.com/news/china/story20240912-4710365}{今年是1964年《台湾人民自救运动宣言》发表60周年,台湾总统赖清德呼吁民众,要有勇气揭穿共产主义要并吞台湾的骗局,继续为台湾的民主自由打拼,建立合理繁荣的社会。 彭明敏文教基金会星期四(9月12日)主办``世代流转、民主承启''《台湾人民自救运动宣言》60周年纪念研讨会。赖清德在致词时表示,宣言中很多理想虽然已经实现,但仍要继续推动这项运动中尚未完成的目标……}

\entryitemWithDescription{忧北京华盛顿台北均为``最糟情况''做准备 大陆学者:采三步骤扭转趋势}{https://www.zaobao.com/news/china/story20240912-4710220}{美国众议院本周又通过至少四项收紧出口管制议案,加强在新兴技术领域对中国的出口管制。(路透社) 中国大陆学者吴心伯认为,中美要扭转台海局势往负面趋势发展,就应采三步骤:一关上台湾独立大门、二创造两岸和平统一环境、三两岸坐下谈判。他担心北京、华盛顿和台北三方都在为``最糟情况''做准备,并称``这非常危险''……}

\entryitemWithDescription{美国前中情局人员为中国从事间谍活动 被判监禁10年}{https://www.zaobao.com/news/china/story20240912-4710206}{(华盛顿综合讯)一名前美国中央情报局人员承认为中国从事间谍活动后,在星期三(9月11日)被判处10年监禁。 综合路透社和法新社报道,美国司法部在声明中说,71岁的马玉清(音译,Alexander Yuk Ching Ma)5月与检察官达成认罪协议,承认串谋收集并向中国提供国防信息。 马玉清原籍香港,后入籍美国,曾于1982年至1989年在中情局工作,2020年8月被捕……}

\entryitemWithDescription{澳门特首提名结束 岑浩辉预计无悬念当选}{https://www.zaobao.com/news/china/story20240912-4709739}{澳门终审法院前院长岑浩辉(右)星期二(9月10日)到水坑尾澳门公共行政大楼地下递交2024澳门特首选举候选人提名表。(中新社) 澳门第六任行政长官选举提名星期四(9月12日)结束,只有前终审法院院长岑浩辉获得足够提名票报名,若无意外将只有他一人入闸参选并当选。 受访澳门民主派人士认为,岑浩辉上台后将会致力推动经济多元化,但要改变目前澳门博彩业一业独大的问题颇有难度……}

\entryitemWithDescription{涉贪被捕人设崩塌 柯文哲总统梦碎 | 世界大解说}{https://www.zaobao.com/news/china/story20240912-4709141}{``其实政治很简单,不要贪污就好。'' 这句话是台湾反对党民众党主席柯文哲的经典名言,言犹在耳,他却因京华城弊案涉嫌收贿图利,目前在台北看守所羁押禁见。 由他一手创办的民众党和所代表的第三势力,在台湾还有没有生存空间? 他掀起的``白色力量''式微,又怎么影响朝野权力消长? 本期《世界大解说》请来两位专家一一解析……}

\entryitemWithDescription{中国2023年结婚人数近十年来首次回升}{https://www.zaobao.com/news/china/story20240912-4708487}{呼和浩特市敕勒川草原在9月1日举行一场漫相亲会暨草原集体婚礼,六对新人在敕勒川、阴山下喜结良缘。图为新娘们向单身青年抛手捧花。(中新社) (北京/上海综合讯)中国官方数据显示,去年全国办理结婚登记的人数较前一年增长了12.4\%,是近十年来首次止跌回升。 中国民政部官网发布《2023年民政事业发展统计公报》显示,2023年办理结婚登记768.2 万对,结婚率为5.4‰,比上一年增长0.6个千分点……}

\entryitemWithDescription{中菲就仙宾礁问题进行磋商 仍坚持各自立场}{https://www.zaobao.com/news/china/story20240912-4708349}{中菲星期三(9月11日)在北京举行南中国海问题双边磋商机制团长会晤。中国外交部副部长陈晓东(右)与菲律宾外交部副部长拉扎罗(左)会面时握手致意。(取自拉扎罗的X账号) (北京/马尼拉综合讯)中国和菲律宾在南中国海水域争端持续延烧之际,两国就仙宾礁(菲国称萨比纳礁)等问题进行磋商。双方依旧坚持各自的立场,但愿意沟通缓解地区紧张局势……}

\entryitemWithDescription{沈泽玮:中国延迟退休或引入移民?}{https://www.zaobao.com/news/china/story20240912-4696456}{``年轻人找不到工作,老的退不下来,真是一生都不合时宜的人呐。'' 有关审议中国延迟法定退休年龄进入全国人大议程的消息星期二(9月10日)经媒体报道后,迅即在中国社媒上引发舆论热议,有网民如是感叹。 延迟法定退休年龄看似箭在弦上,中国老中青群体都心烦……}

\entryitemWithDescription{国台办批赖清德制造``绿色恐怖'' 追杀不支持民进党及不认同台独人士}{https://www.zaobao.com/news/china/story20240911-4697552}{台湾总统赖清德8月21日在台北举行的凯达格兰论坛上发表讲话。(路透社) 台湾在野民众党主席柯文哲因涉及京华城弊案被羁押禁见,中国国务院台湾事务办公室批评台湾总统赖清德在岛内大肆制造``绿色恐怖'',追杀不支持民进党、不认同台独的各界人士,谋求一人、一党之私利,并为其遂行台独排除障碍……}

\entryitemWithDescription{美众议院通过可致香港经贸办被强制关闭法案 中国称将坚决反制}{https://www.zaobao.com/news/china/story20240911-4697305}{香港一名男子9月5日在维多利亚港的海滨长廊钓鱼,当时超强台风``摩羯''正穿过南中国海向中国南部海岸移动。(法新社) 美国众议院星期二(9月10日)通过《香港经济贸易办事处认证法案》,如果参议院也表决通过并交总统签署成法,预料年底就会生效,导致香港三个驻美国的经贸办被强制关闭。 受访的香港学者认为,事件将对香港带来负面冲击,港府有必要加强与欧美商界人士交流,解释香港仍是良好投资的城市……}