\entryitemWithDescription{侧记:黄循财首次以总理接班人之姿率团访华}{https://www.zaobao.com/news/china/story20231207-1454627}{副总理兼财政部长黄循财(前排右二)12月6日在人民大会堂与中国总理李强(前排右一)会晤,参加会见的新加坡政要包括(前排右三起)贸工部长颜金勇、教育部长陈振声、国家发展部长李智陞、交通部代部长兼财政部高级政务部长徐芳达、交通部兼永续发展与环境部高级政务部长许连碹博士、国家发展部兼外交部高级政务部长沈颖;以及(后排右起)贸工部兼文化、社区及青年部政务部长刘燕玲,卫生部兼律政部高级政务次长拉哈尤玛赞……}

\entryitemWithDescription{萧美琴被指``战猫外交''背后是卑躬屈膝 台外交部:在野党扭曲台美关系}{https://www.zaobao.com/news/china/story20231206-1454620}{台湾在野国民党立委候选人星期三(12月6日)揭露一份台美谈判密件,指时任驻美代表的民进党副总统候选人萧美琴,去年4月便知台湾同意开放美国莱克多巴胺猪肉和日本福岛五县市食品的政治代价,仍无法被美国纳入印太经济框架(IPEF)的事实,政府吹嘘萧美琴的``战猫外交'',背地里却是卑躬屈膝……}

\entryitemWithDescription{中国部分地方健康码重新上线 受访民众不认为官方会收紧防疫}{https://www.zaobao.com/news/china/story20231206-1454608}{《联合早报》记者星期三(12月6日)登陆``天府健康通'',发现四川的健康码已重新上线。(手机截图) 中国今年入冬以来呼吸道感染患者持续增多,医院儿科就诊量居高不下,广东、四川等地近期重现健康码,引发部分民众担忧严格防疫措施时隔一年卷土重来。 但有迹象显示,重新上线的健康码并未投入使用,有分析相信官方不会在经济不佳之际收紧防疫,以免引起民众反弹……}

\entryitemWithDescription{全球首座第四代核电站在山东投入商业运行}{https://www.zaobao.com/news/china/story20231206-1454549}{华能石岛湾高温气冷堆核电站正式投入商业运行。图为摄于2016年的华能石岛湾高温气冷堆示范工程首台反应堆压力容器吊装。(新华社) (北京新华电)全球首座第四代核电站:位于山东省荣成市的华能石岛湾高温气冷堆核电站正式投入商业运行……}

\entryitemWithDescription{港反修例中枪抗争者:参加更生计划后 学会避免被人煽动}{https://www.zaobao.com/news/china/story20231206-1454539}{曾参与香港反修例抗争而中枪的曾志健,在香港无线电视推出的国安法资讯节目《有法安国》中现身说法但未露脸。(香港警察脸书截图) (香港综合讯)曾在反修例抗争期间中枪被捕的香港青年在国安法节目上说,他参加了更生计划,并从中学会如何管理情绪,``避免被人煽动和唆摆''。 香港警方星期二(12月5日)在脸书发布一段时长两分钟的视频,同时介绍该视频源自香港无线电视11月底开播的国安法资讯节目《有法安国》……}

\entryitemWithDescription{美海军巡逻机飞越台海 中国大陆战机跟监警戒}{https://www.zaobao.com/news/china/story20231206-1454537}{(北京/台北综合讯)美国海军一架巡逻机星期三(12月6日)飞越台湾海峡,中国大陆解放军东部战区称已组织战机跟监警戒。 据路透社报道,美国海军第七舰队星期三发布声明称,P-8A海神海上巡逻侦察机在国际空域飞越了台海。P-8A是美军长程反潜机,能执行海上巡逻、侦察与反潜任务。 声明称,此举表明了美国对自由开放的印太地区的承诺。``美国军队在国际法允许的任何地方飞行、航行和行动……}

\entryitemWithDescription{王纬温:中美两军不沟通 南中国海紧张难管控}{https://www.zaobao.com/news/china/story20231206-1454386}{中美元首11月15日在旧金山会晤,宣布重启两军高层沟通会议机制,但这项标志性成果,三周以来迟迟未落实。近期持续升级的南中国海紧张局势,本周则再度冲击中美关系。在菲律宾12月3日称逾135艘中国民兵船现踪菲国海域后,美国隔天派军舰驶入中菲有主权争议的仁爱礁(菲称阿云津礁)邻近海域,是小马可斯政府上台近一年半来首次,介入姿态明显上升……}

\entryitemWithDescription{``赖萧配''支持度明显领先 柯文哲:民调没有造假但可操纵}{https://www.zaobao.com/news/china/story20231205-1454382}{``蓝白合''破局后,相关民调显示,台湾执政的民进党总统候选人赖清德和副总统候选人萧美琴的民调明显领先,最大在野党国民党(蓝)总统候选人侯友宜和副总统候选人赵少康居次,而另一在野党民众党(白)总统候选人柯文哲和副总统候选人吴欣盈已见败象……}

\entryitemWithDescription{早知:民调的真伪奥秘}{https://www.zaobao.com/news/china/story20231205-1454378}{台湾政治民调的真实性如何确定? 每份民调有不同的调查方式, 结果会有很大的差别,看一份民调数据,最好审视该民调的原始资料,包括它的抽样方法、样本数、如何加权,访谈题目顺序,以及交叉分析等资料,从中可检验数据的合理性。 拒绝公开上述背景资料的民调必须存疑,台湾选举期间常冒出名不见经传的单位做的民调,得格外留意……}

\entryitemWithDescription{分析:周庭弃保流亡 可能令香港23条立法``加辣''}{https://www.zaobao.com/news/china/story20231205-1454359}{周庭2017年6月4日在香港街头演说的情景。(法新社档案照片) 涉嫌违反国安法被捕的前``香港众志''副秘书长周庭,保释期间获准离港赴加拿大读书,但在日前突然宣布弃保流亡加国,引起香港社会和多国关注。有分析认为,这次事件影响深远,可能会令当局明年就《基本法》23条立法时``加辣''……}

\entryitemWithDescription{涉违反中国宗教事务条例 喜茶``佛喜''茶拿铁下架}{https://www.zaobao.com/news/china/story20231205-1454339}{自2017年新茶饮市场出现``联名''概念以来,喜茶推出了超过百款跨界联名产品。图为已下架的``佛喜''茶拿铁。(互联网) (上海/深圳综合讯)中国著名茶饮品牌喜茶与景德镇中国陶瓷博物馆近日联名推出一款以佛教为主题的茶拿铁``佛喜'',因涉嫌违反宗教法规,在上市不到一周后即被下架……}

\entryitemWithDescription{民进党前主席施明德病危}{https://www.zaobao.com/news/china/story20231205-1454316}{82岁的施明德出生于台湾高雄,是推动台湾民主的``美丽岛事件''主要领导人物。(施明德脸书) (台北综合讯)台湾资深媒体人陈文茜透露,82岁的民进党前主席施明德因肝癌复发,住进加护病房,性命垂危。 陈文茜在星期一(12月4日)播出的政论节目《TVBS战情室》上,谈到民进党副总统候选人萧美琴是美籍还是``中华民国国籍''的争议时感叹,台湾当前政治环境一党独大,中选会根本不是独立机关,执政党说了算……}

\entryitemWithDescription{中国发布自动驾驶汽车运输安全服务试行指南}{https://www.zaobao.com/news/china/story20231205-1454298}{中国星期二发布了自动驾驶汽车运输安全服务试行指南。图为百度推出的阿波罗自动驾驶出租车服务。(路透社) (北京综合讯)中国发布自动驾驶汽车运输安全服务试行指南,要求从事城市公共汽电车客运和道路旅客运输经营的自动驾驶汽车,随车配备一名驾驶员或运行安全保障人员。 中国交通运输部星期二(12月5日)在官网发布上述指南,规范自动驾驶技术在运输服务领域的应用……}

\entryitemWithDescription{戴庆成:香港传媒寒冬期未过}{https://www.zaobao.com/news/china/story20231205-1454155}{``你看了《新闻女王》吗?''香港无线电视台(TVB)近日正在播映的台庆重头剧《新闻女王》,由于题材新颖,剧情节奏明快,开播后随即掀起追看热潮,也吸引了不少传媒工作者捧场观看,至今已连续两周位列收视榜首。 但《新闻女王》大受欢迎,并未能挽救TVB的颓势……}

\entryitemWithDescription{广汽本田据报首次在华裁员 公司回应是误读}{https://www.zaobao.com/news/china/story20231204-1454158}{(广州综合讯)日本本田汽车据报自1998年与广汽集团成立合资公司以来首次裁员,但广汽本田回应指相关报道不属实。 日经中文网星期一(12月4日)报道称,广汽本田计划裁减约占公司总员工数7\%的900名员工,主要涉及广州市工厂从事汽车组装的派遣员工。广汽本田已于11月下旬通知这些员工,并承诺提供合同规定的经济补偿。 据报道,广汽本田今年1至10月的销量同比下降18.5\%……}

\entryitemWithDescription{蔡英文:台湾不要香港式和平 要有尊严和平}{https://www.zaobao.com/news/china/story20231204-1454157}{台湾总统蔡英文(左五)星期天(12月3日)在台北为执政的民进党正副总统候选人赖清德(左四)和萧美琴(左六)助选时间接称,2024年总统大选是香港式和平与有尊严和平的选择……}