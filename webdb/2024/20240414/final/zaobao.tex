\entryitemWithDescription{中国特稿:中国青年遍寻疗愈 只为能好好睡一觉}{https://www.zaobao.com/news/china/story20240414-3346165}{疗愈师李艳在位于北京东三环的壹衍自然疗法工作室里用颂钵为客人做疗愈。(孟丹丹摄) 近年来,中国经济持续下行,各行各业内卷不断,处于激烈竞争中的年轻人在``996''加班文化、``躺平可耻''的主流舆论下,再加上手机、互联网对睡眠时间的侵占,不少人陷入焦虑抑郁以及失眠状态。都市年轻人纷纷追捧颂钵音疗、芳香疗愈、琴床疗愈以及禅修冥想疗愈等减压新方式……}

\entryitemWithDescription{萧美琴指赖清德有和平四大支柱 形塑台湾稳定的未来}{https://www.zaobao.com/news/china/story20240413-3408846}{台湾候任副总统萧美琴星期六(4月13日)出席福和会举办的台湾关系法45周年论坛。(庄慧良摄) 台湾候任副总统萧美琴指出,台海稳定备受国际关注,政府必须形塑台湾稳定、可信任的环境,故候任总统赖清德提出强化国防吓阻能力、建构更强经济韧性、深化与民主国家的伙伴关系,以及稳定而有原则的两岸领导能力等和平四大支柱的主张。 萧美琴星期六(4月13日)在``台湾关系法45周年福和会论坛''演说时发表上述讲话……}

\entryitemWithDescription{德总理朔尔茨访华 分析:料将就工业补贴和俄乌战争问题提要求}{https://www.zaobao.com/news/china/story20240413-3408446}{德国总理朔尔茨星期六(4月13日)启程前往中国访问。图为他星期五(4月12日)在德国柏林与格鲁吉亚总理科巴希泽会晤后出席记者会。(路透社) 中国和欧洲在德国总理朔尔茨访华前相互隔空喊话,中国商务部部长王文涛警告,欧委会近期针对中国电动汽车等产品的``消极动作''将动摇双边经贸互信的基础;朔尔茨和法国总统马克龙则强调,要重新平衡欧中之间的贸易关系……}

\entryitemWithDescription{中国机电商会称欧盟对华电动汽车调查扭曲和不透明}{https://www.zaobao.com/news/china/story20240413-3408422}{比亚迪欧洲汽车销售事业部总经理舒酉星2月26日在瑞士日内瓦举行的日内瓦国际车展开幕式上,展示比亚迪仰望U8运动型多用途车(SUV)。(路透社) (布鲁塞尔综合讯)虽然欧盟对电动汽车的反补贴调查结果仍未揭晓,中国机械和电子进出口领域的最大商会批评该调查不仅不利于中国制造商,而且过程不透明并违反全球贸易规则……}

\entryitemWithDescription{台外长:力积电在印度建厂 开创台印半导体合作里程碑}{https://www.zaobao.com/news/china/story20240413-3408373}{力积电是台湾继台积电和台联电之后第三大晶圆厂商。该公司将在印度兴建首座晶圆厂。(力积电官网) (台北综合讯)台湾外交部长吴钊燮说,台印关系越来越热络,台湾半导体制造商力积电在印度兴建首座晶圆厂,更开创了台印半导体产业合作的里程碑。 据台湾外交部官网消息,吴钊燮星期五(4月12日)接受印度寰宇一家(WION)电视新闻台视频访问,内容包括台印关系、台海与印太安全等议题,专访已于当晚播出……}

\entryitemWithDescription{两军工企业换高层 陈锡明获任命为中国航天科工董事长}{https://www.zaobao.com/news/china/story20240413-3408535}{中国航天科工集团有限公司星期五(4月12日)召开中层以上管理人员大会,会议上宣布了陈锡明任公司董事长、党组书记。(互联网) (北京综合讯)中国电子科技集团原总经理陈锡明获任命为中国军工巨头中国航天科工的董事长,这意味着此前消失在公众视野多时的航天科工原董事长、党组书记袁洁已被免职……}

\entryitemWithDescription{历时半年调查 港大校长张翔行为不当指控不成立}{https://www.zaobao.com/news/china/story20240413-3407281}{张翔2016年起担任香港大学校长,由去年开始第二任期。(香港大学网站) (香港综合讯)经过半年调查,香港大学校务委员会星期五(4月12日)深夜公布,对校长张翔此前管理大学财产不当的指控不成立……}

\entryitemWithDescription{新闻人间:张维为和同济大学生}{https://www.zaobao.com/news/china/story20240413-3376554}{新闻人间:张维为和同济大学生(联合早报) 自带热搜体质的上海复旦大学中国研究院院长张维为本周又出圈了,这次令他出圈的,不是夸张的言论,而是一群参加他讲座的大学生。 中国排名前列的同济大学两周前(3月29日)举办了一场讲座,主讲人张维为发表了题为《文明型国家视角下的中国崛起》的演讲,现场有300多名师生聆听……}

\entryitemWithDescription{美日菲领导人会晤引发中国强烈反应}{https://www.zaobao.com/news/china/story20240412-3394934}{中国外交部亚洲司司长刘劲松(右)星期五(4月12日)约见日本驻华使馆首席公使横地晃,提出严正交涉,表达严重关切和强烈不满。(取自中国外交部网站) 美国、日本和菲律宾首个三国峰会引发中国强烈反应,中国外交部发言人批评美日菲联合声明是针对中国的肆意抹黑攻击,并表示坚决反对有关国家操弄集团政治。受访学者预计中国不会对美国采取强硬手段,但会向邻国日本和菲律宾加码施压……}

\entryitemWithDescription{北京大力推动``爱国爱港者治港''香港公务员辞职率大增}{https://www.zaobao.com/news/china/story20240412-3394891}{香港自从2020年实施国安法后,近年不断有公务员辞职;截至去年11月底,公务员空缺率高达一成。 香港立法会下星期开始审议2024/25年度政府开支。公务员事务局在文件中回复议员查询时表示,2018/19年至2022/23年五个财政年度期间,共有1124名公务员因为干犯严重不当行为,或被裁定干犯刑事罪行而需接受正式纪律行动和惩处,其中有124名公务员被革职……}

\entryitemWithDescription{王瑞杰与李家超会晤 探讨深化新港多领域合作}{https://www.zaobao.com/news/china/story20240412-3394277}{我国副总理兼经济政策统筹部长王瑞杰(左)星期五(4月12日)与香港行政长官李家超会面。(通讯及新闻部提供) 我国副总理兼经济政策统筹部长王瑞杰星期五(4月12日)与香港特区行政长官李家超会晤,探讨新港两地如何在贸易、创新、研究等领域深化合作。 王瑞杰表示,对香港在``一国两制''框架下继续繁荣发展充满信心;李家超则指出,香港将发挥好在该框架下的``超级联系人''和``超级增值人''角色……}

\entryitemWithDescription{学者分析二次``习马会''后 中国大陆对台态度``先软后硬''但风险可控}{https://www.zaobao.com/news/china/story20240412-3393492}{台湾学者分析认为,在候任总统赖清德5月20日上任前,中国大陆对台态度会``先软后硬'',迫近520时再对台升高压力。 不过目前大陆和美国、美国与台湾的沟通都顺畅,台湾前总统马英九访问大陆有助北京对內交待,台湾内部政局朝小野大,故风险可控,海峡两岸应不会有任何意外。 国策研究院基金会星期五(4月12日)举行``台湾关系法45周年、马英九访陆与美日菲峰会''座谈会……}

\entryitemWithDescription{小野:犹豫蛮久才决定接任台湾文化部长}{https://www.zaobao.com/news/china/story20240412-3393679}{笔名``小野''的台湾知名作家、编剧李远透露,他被征询后犹豫了蛮久,才决定接任文化部长。(自由时报视频截图) (台北综合讯)笔名``小野''的台湾知名作家、编剧李远透露,他被征询后犹豫了蛮久,才决定接任文化部长。 综合台湾《自由时报》《联合报》等报道,台湾行政院侯任院长卓荣泰星期五(4月12日)在记者会上公布新一批内阁成员名单……}

\entryitemWithDescription{中国驻菲使馆:中方在菲设秘密潜伏小组是无稽之谈}{https://www.zaobao.com/news/china/story20240412-3392969}{(北京/马尼拉综合讯)中国驻菲律宾大使馆星期三(4月10日)在官网强调,中国在菲设立秘密潜伏小组是无稽之谈,并指有一只``无形之手'' 要毒化中菲关系氛围。 中国驻菲使馆发言人在官网说,菲律宾个别官员和媒体捕风捉影,对中国进行恶意揣测和无端指责,在菲煽动仇华情绪,中国对此坚决反对……}

\entryitemWithDescription{告别野蛮成长?中国出台微短剧备案新规}{https://www.zaobao.com/news/china/story20240412-3391338}{(北京/广州综合讯)针对中国网络微短剧存在的内容低俗、血腥暴力等问题,中国监管部门出台新规,自6月1日起,所有网络微短剧均须持有相应发行许可证或完成报备后方可播出。 综合财新网与21世纪经济报道,中国国家广电总局已出台针对微短剧的备案新规,要求所有播出、引流、推送的网络微短剧均须持有《网络剧片发行许可证》或完成相应上线报备登记程序,自6月起,未经审核且备案的微短剧不得上网传播……}

\entryitemWithDescription{王瑞杰访深圳 了解其创新生态系统}{https://www.zaobao.com/news/china/story20240411-3378985}{我国副总理兼经济政策统筹部长王瑞杰星期四(4月11日)在深圳参观了四家企业。(王瑞杰脸书) 正在中国访问的副总理兼经济政策统筹部长王瑞杰星期四(4月11日)在此行的第二站深圳,参观了四家企业和前海深港现代服务业合作区的展览,了解深圳经济及其创新生态系统。 王瑞杰当晚在脸书发文说,位于大湾区的深圳,拥有许多处于技术前沿的创新公司……}

\entryitemWithDescription{中国产能过剩争议不减 分析:与10年前因果不同}{https://www.zaobao.com/news/china/story20240411-3379120}{针对欧盟向中国风力涡轮机供应商发起补贴调查,中国商务部发言人何亚东回应说,中国风电等新能源企业依靠持续技术创新、完善产供链体系和充分市场竞争实现快速发展,赢得竞争优势,``这是任何补贴也补不出来的''。图为海南文昌木兰湾沿海的风电场。(中新社) 中国工业产能利用率走势(早报图表) 美国财政部长耶伦的访华行程落下帷幕,但围绕中国产能过剩的争议依然沸沸扬扬……}

\entryitemWithDescription{港媒:李强6月访问澳洲 中国将取消对澳龙虾进口禁令}{https://www.zaobao.com/news/china/story20240411-3378206}{中国总理李强1月16日在瑞士达沃斯世界经济论坛上发表讲话。(路透社) (北京综合讯)香港媒体报道,中国总理李强将在6月访问澳大利亚,以巩固两国正在改善的双边关系。这将是2017年以来中国总理再度访澳。 香港《南华早报》星期四(4月11日)引述消息人士报道,李强将在6月第三周访问澳洲。这将是他自去年3月就任总理以来,首次访问澳洲。 消息人士还说,中国针对澳洲龙虾已实施三年多的非正式进口禁令也将解除……}

\entryitemWithDescription{无国界记者组织代表在香港被拒入境}{https://www.zaobao.com/news/china/story20240411-3378204}{(伦敦/巴黎综合讯)无国界记者组织称,该组织的台北倡议专员星期三(4月10日)抵达香港国际机场时被拒绝入境,并被扣留了六个小时接受盘问。这是该组织成员首次在香港机场遭拒入境或被拘留。 综合路透社与法新社报道,无国界记者组织是在星期三发表声明,披露其台北倡议专员亚历山德拉(Aleksandra Bielakowska)在香港机场遭遇的上述对待……}