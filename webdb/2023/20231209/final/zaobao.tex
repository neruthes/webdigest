\entryitemWithDescription{新闻人间:公主选副总统 吴欣盈``无心赢''?}{https://www.zaobao.com/news/china/story20231209-1455012}{新闻人间------民众党副总统候选人吴欣盈(联合早报制图) 台湾在野阵营``蓝白合''破局后,民众党隔天(11月24日)推出新光集团``大公主''吴欣盈担任副总统候选人,与总统候选人、党主席柯文哲搭档。 从吴欣盈的政治资历、媒体应对来观察,许多评论员都认为,民众党和吴欣盈看来可能真的``无心赢''……}

\entryitemWithDescription{庄慧良:葛来仪等三位美国学者呼吁``冻结台独党纲''的警讯}{https://www.zaobao.com/news/china/story20231209-1455079}{台湾在野民众党总统候选人柯文哲星期四(12月7日)重申其``深绿''背景,并提及若当选,未来外交政策将遵循总统蔡英文路线,强调他没有执政的民进党台独党纲包袱,两岸可以更和善。随即被民进党总统候选人赖清德竞选总干事潘孟安奚落说,如果认为蔡英文路线是对的,``那支持正版赖清德就好了……}

\entryitemWithDescription{黄循财:新中互免签证有利人员交往 首先要设法增加两地往来航班}{https://www.zaobao.com/news/china/story20231209-1455081}{新加坡和中国将落实30天互免签证安排,副总理兼财政部长黄循财指出,这将便利两国人员往来,也符合新加坡利益,但首先要设法增加往来两地的航班。 黄循财星期五(12月8日)结束访华行程前在北京接受新加坡媒体采访。他在谈到新中未来的互免签证安排时强调,民间交流的加强,将为改善双方在广泛领域的合作提供更好基础……}

\entryitemWithDescription{黄循财:新中双方人员的接触``绝对是有效合作的基础''}{https://www.zaobao.com/news/china/story20231209-1455080}{新中官员多年来在不同双边合作机制下共事,副总理兼财政部长黄循财认为,双方建立的友谊和联系是可贵的,通过个人层面的接触形成的关系``绝对是有效合作的基础''。 黄循财星期二(12月5日)起率团访华四天,期间会见多名中国高层,并主持新中双边合作联合委员会(JCBC)会议……}

\entryitemWithDescription{黄循财:永远不要押注中国衰弱 中国仍能为双边合作提供巨大机会}{https://www.zaobao.com/news/china/story20231208-1455077}{冠病疫情后中国经济的表现,引发外界担忧中国增长前景,但副总理兼财政部长黄循财强调,``永远不要押注中国衰弱'',中国将继续是一个能给新中双边合作带来巨大机会的经济体。 他说,这是因为``中国经济规模巨大,在先进制造、绿色经济等领域有很多优势;此外(中国)还有如此巨大的市场''……}

\entryitemWithDescription{特稿:台湾选总统很烧钱 每组候选人估计打底10亿新台币}{https://www.zaobao.com/news/china/story20231208-1455075}{2024年1月13日的台湾总统与立委选举,牵动台海局势与两岸关系走向,引起海内外关注,各政党倾全力呼唤选民热情,背后是一场激烈的烧钱和烧脑竞赛。图为民进党12月2日在彰化县的一场大型夜间造势活动。(民进党提供) 打一场台湾总统选战,要烧多少钱?按照规定,每组候选人不能超过4.2749亿元(新台币,下同,1824万新元),但超过也不会受罚……}

\entryitemWithDescription{黄循财与中国新财长蓝佛安会面 讨论在双边与区域平台加强合作}{https://www.zaobao.com/news/china/story20231208-1455043}{副总理兼财政部长黄循财(左)星期五(12月8日)在北京钓鱼台国宾馆与10月底履新的中国财政部部长蓝佛安会面。(海峡时报) 副总理兼财政部长黄循财星期五(12月8日)在北京钓鱼台国宾馆,与中国财政部长蓝佛安会面,讨论两国财政部如何加强双边合作,以及在``亚细安+3''、亚太经济合作组织(APEC)等区域平台的合作。 根据总理公署文告,蓝佛安向黄循财介绍了中国经济和金融的优先事项……}

\entryitemWithDescription{中国调降外国公民赴华签证费用}{https://www.zaobao.com/news/china/story20231208-1455039}{(北京综合讯)中国对多国公民实施免签政策后,宣布驻外使领馆将下调办理赴华签证的费用,降幅达25\%。 中国外交部领事司星期五(12月8日)发布《关于阶段性减免来华签证费的通知》,中国驻外使领馆自2023年12月11日至2024年12月31日,按现行收费标准的75\%收取签证费……}

\entryitemWithDescription{中国拟规定运营者一小时内上报较大网安事件 迟瞒报造成重大危害从重处罚}{https://www.zaobao.com/news/china/story20231208-1455031}{(北综合京讯)中国强化网络安全监管,拟要求网络运营者对``较大、重大与特别重大''网安事件一小时内上报,因迟漏报造成重大危害后果,将从重处罚。 中国国家互联网信息办公室星期五(12月8日)发布《网络安全事件报告管理办法(征求意见稿)》,公开征求意见至明年1月7日……}

\entryitemWithDescription{港府拟设国家发展成就馆开展爱国教育引争议}{https://www.zaobao.com/news/china/story20231208-1455030}{近年积极推动爱国主义教育的港府建议设立一座以中国发展成就为主题的博物馆,地点初步考虑目前尖沙咀科学馆的位置,至于沙田的文化博物馆则会重建为科学馆。消息一出,引起社会哗然和不满,认为港府的考虑是以政治挂帅。 港府自2019年反修例风波后就积极在学校和社会开展爱国主义教育。特首李家超今年在10月发表的《施政报告》中提出设立两座博物馆,以介绍国家成就和抗战历史……}

\entryitemWithDescription{美国防授权法案拟助台湾培训军队 分析:或促美台联合作战}{https://www.zaobao.com/news/china/story20231208-1455028}{美国国会12月7日公布协商版《国防授权法案》内容,其中要求美国国防部协助台湾培训军队。图为台湾义务役士兵11月23日在台中一处军事基地展示战斗技能。(法新社) 美国国会星期四(12月7日)公布协商版《国防授权法案》内容,其中要求美国国防部协助台湾培训军队。受访学者分析,美国若真将培训台军入法,隐约透露出为美台联合作战做准备的意涵……}

\entryitemWithDescription{韩咏红:特朗普阴影笼罩2024年}{https://www.zaobao.com/news/china/story20231208-1454841}{英国《经济学人》杂志上个月推出2024年趋势的预测系列,封面设计留给了地球和阴影------美国前总统特朗普的侧面剪影,如日蚀般覆盖半个地球。《经济学人》称,他们的年度预测从未像2024年般被一个人的阴影笼罩,``特朗普构成2024年全世界最大的危险''。 《经济学人》一贯擅长设定议题和打造概念……}

\entryitemWithDescription{新中双边合作联委会会议达成24项成果 展现双边关系升级后的全方位高质量和前瞻性}{https://www.zaobao.com/news/china/story20231208-1454846}{新加坡和中国在两国最高层级的年度双边合作会议上达成24项成果,展现新中关系升级后的全方位、高质量和前瞻性;两国也计划通过互免签证协议,便利人员往来,目标在2024年早些时候落实免签安排……}

\entryitemWithDescription{王金平任侯友宜的全台后援会总会长}{https://www.zaobao.com/news/china/story20231207-1454828}{国民党总统候选人侯友宜(左二)和副手赵少康(左一),以及国民党主席朱立伦(右一)星期四(12月7日)上午拜会前立法院长王金平。(侯友宜竞选办提供) (台北综合讯)在国民党正副总统候选人侯友宜和赵少康,以及党主席朱立伦亲自力邀下,台湾前立法院长王金平点头答应担任侯友宜的全台后援会总会长……}

\entryitemWithDescription{台网曝情治机关监听政治人物 官方:境外认知作战}{https://www.zaobao.com/news/china/story20231207-1454826}{台湾社群网站出现一份疑似情治机关监听多名政治人物及外国驻台机构的匿名资料。内政部和法务部等相关单位星期四(12月7日)纷纷强调,该份错假资料来自境外,意图在大选期间对台进行认知作战,呼吁民众切勿轻信。 不过,在野国民党正副总统候选人侯友宜、赵少康等人呼吁检调应先厘清事实,若此事为真,监听是否依法请法官开监听票?为何资料会外泄?若此事为假,便不应以讹传讹……}