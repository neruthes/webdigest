\entryitemWithDescription{新闻人间:中国富商``天上人间''老板在美国认罪}{https://www.zaobao.com/news/china/story20240323-3193392}{1990年代在中国名噪一时的``天上人间''夜总会老板覃辉隐身海外沉寂多年后,本周意外地重返公众视野。 美国司法部官网星期一(3月18日)公告,旅美的中国富商覃辉在纽约被控进行非法政治捐献,指其曾在2021年至2022年纽约市长选举期间,以他人名义向纽约州和罗德岛的国会议员候选人提供总数达1万1600美元(1万6000新元)的政治捐款……}

\entryitemWithDescription{庄慧良:韩风草偃}{https://www.zaobao.com/news/china/story20240323-3203850}{台北市中华路一家张记韭菜水煎包,因立法院长韩国瑜上星期五(3月15日)在院会的``震撼教育''爆红,人潮络绎不绝,连搭乘游览车到立法院参观的民众,也慕名到该店排队,店家这几天都提早售罄打烊……}

\entryitemWithDescription{23条立法通过后 港官员据报做好被制裁准备}{https://www.zaobao.com/news/china/story20240322-3203719}{香港立法会通过俗称《基本法》23条立法草案的《维护国家安全条例草案》后,外界关注这会否引起欧美国家新一轮对香港的制裁。据悉,建制派政治人物已做好被制裁的准备,包括安排好资产,以及短期内避免踏足美国。 香港立法会星期二(3月19日)三读通过《基本法》23条立法,有关条例23日生效,引起国际社会广泛关注……}

\entryitemWithDescription{25家外来企业与香港特区政府引进办签约}{https://www.zaobao.com/news/china/story20240322-3203319}{(香港综合讯)香港《维护国家安全条例》通过后首日,25家外来企业正式签约落户香港。中国媒体称,其中六家企业来自美国,其余的均来自中国大陆。 香港立法会星期二(3月19日)全票通过《维护国家安全条例》(即《基本法》23条立法草案),引发英、美、欧盟官方及舆论批评如潮,包括指新法让一般人在香港生活、工作以及经商变得困难,也无法向国际组织提供确定性等……}

\entryitemWithDescription{中国智库:中美南中国海反侦察对抗扩展至全天候}{https://www.zaobao.com/news/china/story20240322-3203592}{中国智库``南海战略态势感知计划''主任胡波星期五(3月22日)在《2023年美军南海活动不完全报告》发布会上分析美国过去一年在南中国海的军事行动。(``南海战略态势感知计划''提供) 美国可能在总统大选年加大在南中国海的军事存在,当前美国无人机已全面替代了此前在该海域使用的电子侦察机,并且在晚间行动。这意味着中美在南中国海的反侦察对抗已扩展至全天候,双方危机管理的复杂性和难度将大幅提高……}

\entryitemWithDescription{涉恒大巨额欺诈案 普华永道在中国被查}{https://www.zaobao.com/news/china/story20240322-3202722}{普华永道会计师事务所在比利时布鲁塞尔办公楼顶部展示的公司标志。(路透社) (香港综合讯)中国正在调查普华永道在中国房地产巨头恒大集团的会计操作中所扮演的角色。恒大被指夸大超过5600亿元(人民币,下同,1045亿新元)的业务收入。 据彭博社星期五(3月22日)报道,在中国房地产业陷入困境之前,普华永道审计了多家开发商……}

\entryitemWithDescription{36架次大陆军机在台海周边活动 单日数量今年新高}{https://www.zaobao.com/news/china/story20240322-3201834}{台湾军方星期五(3月22日)通报,36架次中国大陆军机在台海周边活动,改写今年单日数量新高。 据台湾国防部通报,截至星期五清晨6时之前的24小时内,36架次大陆军机在台海周边活动,其中13架次军机逾越台海中线,配合六艘次大陆舰艇执行联合战备警巡。 在台海中线两侧、北部和西南空域持续活动的军机,包括歼击机和无人机。台军称已运用任务机、舰及岸置导弹系统,进行严密监控……}

\entryitemWithDescription{中国多家公募基金被异地证监局驻场检查}{https://www.zaobao.com/news/china/story20240322-3202736}{(上海综合讯)中国媒体报道,北京、上海、深圳等地的多家公募基金近日被异地证监局驻场检查。 据财新网星期五(3月22日)报道,3月19日起,北京、上海、深圳等地的多家的公募基金被地方证监局交叉现场检查。这波检查内容具体涉及业务监管、廉洁建设、合规风控等。现场检查的形式属于传统监管方式……}

\entryitemWithDescription{金门翻船事件后 中国大陆加大``三无''船只取缔力度}{https://www.zaobao.com/news/china/story20240322-3202995}{(北京综合讯)金门翻船案后,中国大陆加大取缔``三无''船只(无船名、无船舶证书、无船籍港登记)的力度。中国农业农村部称,将继续坚持零容忍的态度,坚决清理取缔涉渔``三无''船舶……}

\entryitemWithDescription{韩咏红:中国基层恶性事故增多会不会难制止}{https://www.zaobao.com/news/china/story20240322-3194269}{南京居民楼大火、燕郊商铺爆炸、邯郸初中生杀人\ldots\ldots 近一个月来,中国地方上的严重事故与骇人命案接二连三,让资深媒体人胡锡进都坐不住了。胡锡进上周发文警告,基层恶性事故最近``有点多'',干部躺平的多、忙于形式主义,恶性事故似乎增多的趋势可能制止不住。 话说中国这么大,发生这样那样的意外实不足为奇,但近期一系列事故的严重性与频密程度却让人不得不关注……}

\entryitemWithDescription{中国三地同一天发生汽车冲撞事故 胡锡进:公众很关心}{https://www.zaobao.com/news/china/story20240321-3193620}{(北京/沈阳/台州/德州综合讯)中国辽宁省沈阳市、浙江省台州市,以及北京市在星期二(3月19日)接连发生严重车辆冲撞事件,共造成七人死亡,超过30人受伤。其中,北京的事故地点位在知名旅游景点附近……}

\entryitemWithDescription{《苹果日报》前社长董桥新书在大陆下架 原因或与黎智英有关}{https://www.zaobao.com/news/china/story20240321-3194036}{香港作家、前《苹果日报》社长董桥在中国大陆出版的三本新书已下架。(香港大学出版社提供) (香港综合讯)香港作家、前《苹果日报》社长董桥在中国大陆出版的三本新书已下架。港媒揣测,这或许与他曾在反修例运动期间,就如何号召港人上街游行向壹传媒创办人黎智英出谋划策有关。 《星岛日报》星期四(3月21日)报道,董桥今年1月下旬在大陆出版三本简体字新书,这些书由北京商务印书馆出版……}

\entryitemWithDescription{经济学家:中国楼市萎缩趋势将至少持续到2025年底}{https://www.zaobao.com/news/china/story20240321-3193167}{恒生银行首席经济学家王丹(右下)星期三(3月20日)在国际金融论坛主办的对话会上预测,中国住房投资和销售量将在今年进一步萎缩,幅度将超过去年。出席对话会的包括国际货币基金组织驻中国首席代表巴奈特(右上)、瑞银投资银行首席中国经济学家汪涛(左下)。对话会主持人是国际金融论坛常务副理事长林建海(左上)……}

\entryitemWithDescription{台国安局长:蔡英文若登上太平岛会激化南中国海紧张局势}{https://www.zaobao.com/news/china/story20240321-3193663}{台湾国安局长蔡明彦指出,南中国海政治情势复杂,若总统蔡英文此时登上太平岛,会被国际社会解读为激化当地紧张情势,可能伤害台湾的外交利益,国安单位也必须考虑蔡英文的飞航维安问题。 台湾海巡署``南沙太平岛港侧浚深及码头整修工程''近日将举行完工启用仪式,朝野立委对蔡英文应否于5月20日卸任前登上太平岛争论不休……}

\entryitemWithDescription{东航坠机两周年 中国调查进展被批``等于没说''}{https://www.zaobao.com/news/china/story20240321-3193073}{中国东方航空MU5735客机2022年3月21日在起飞64分钟后,偏离巡航高度8900米快速下降,坠毁于广西梧州市藤县埌南镇莫埌村附近,机上123名旅客和九名机组成员全部遇难。图为其中较大的一件散落的飞机残骸。(中新社) (北京综合讯)在中国东方航空MU5735航班坠机两周年前夕,中国民航局发布调查进展情况通报,但没有公布黑匣子内容,也未提供坠机原因,被中国网民批评``等于没说''……}

\entryitemWithDescription{西藏招商引资挂钩高考引争议}{https://www.zaobao.com/news/china/story20240321-3192522}{西藏官方星期一(3月18日)发布通知称,满足相关条件投资者子女,可在当地参加高考。图为2021年6月1日身穿毕业袍的大学生,在拉萨布达拉宫广场前拍照。(路透社 ) (北京/拉萨综合讯)中国西藏为了吸引更多投资者,承诺让他们的孩子在当地参加高考,进一步凸显中国高考招生的地域不平等问题……}

\entryitemWithDescription{杨丹旭:新质生产力的是是非非}{https://www.zaobao.com/news/china/story20240321-3189138}{``新质生产力''最近在中国火起来后,朋友传来一则消息,白酒商五粮液的老总说,白酒产业是发展新质生产力的重要载体。 上网一搜,原来贵州茅台早在今年1月已先一步表态,称茅台通过科技创新不断培育新质生产力。朋友开玩笑说:``连白酒都要成为科技产品吗?'' 追溯起来,官方首次提出``新质生产力''的概念是在去年9月……}

\entryitemWithDescription{台外长称南中国海局势战云密布 蔡英文登太平岛``时间点不合适''}{https://www.zaobao.com/news/china/story20240320-3188741}{面对在野国民党强烈要求总统蔡英文应在卸任前登上太平岛彰显主权,台湾外交部长吴钊燮强调,太平岛主权属于台湾,但目前南中国海局势战云密布,美国学者甚至认为南中国海比台海更容易发生战争,蔡英文此时登太平岛``时间点不合适''……}