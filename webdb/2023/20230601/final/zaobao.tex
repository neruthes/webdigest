\entryitemWithDescription{中国早点:中美军事关系难解冻}{https://www.zaobao.com/news/china/story20230601-1400141}{美国总统拜登5月21日在日本广岛团结七国集团(G7)成员国并组成抗中联盟后,立即预测美中关系很快解冻,强调西方七大经济强国意在降低风险而非与中国脱钩。拜登的预言看似很快成真,中美外交商贸高层互动频率上周重新增多,提高外界对北京与华盛顿加速走出今年2月气球事件阴霾的期待。 不过,中美高层重启对话的势头很快又再度遇冷……}

\entryitemWithDescription{韩国女团称澳门粉丝``Macanese''在大陆引争议}{https://www.zaobao.com/news/china/story20230531-1400145}{韩国女团BLACKPINK因在官方微博以``Macanese(澳门人)''称澳门歌迷,被大陆网民指用词不当。图为该女团澳门演唱会现场。(取自BLACKPINK微博) 韩国女团BLACKPINK在社交媒体以``Macanese(澳门人)''称澳门歌迷,被大陆网民指用词不当、刻意不用Chinese(中国人)。后团队将``Macanese''更改为``Macau(澳门)'',但还是被批……}

\entryitemWithDescription{蓝绿互批令台湾选战升温 郭柯看海让蓝白合作成疑}{https://www.zaobao.com/news/china/story20230531-1400144}{郭台铭(左)星期二(5月30日)深夜在脸书贴出与柯文哲在金门烈屿看海聊天的照片,对岸不远处是中国大陆的厦门。(取自郭台铭脸书) 七个月后投票的台湾总统选战升温,无法获得国民党征召的鸿海集团创办人郭台铭,与民众党主席柯文哲在金门并肩看海后,从前一晚的``海誓山盟''改口,称酒喝太多忘了说过什么,让人对``郭柯配''或``蓝白合''继续雾里看花……}

\entryitemWithDescription{中国与印度驻派记者面临清零局面}{https://www.zaobao.com/news/china/story20230531-1400126}{中国与印度关系持续僵化之际,有消息人士称,新德里拒绝为最后两名驻印度中国官媒记者续签签证。今年以来中印围绕派驻记者的外交摩擦,让两国几乎驱逐了对方所有的记者。 美国《华尔街日报》星期二(5月30日)引述消息人士称,新德里方面本月拒绝为新华社和中国中央电视台的两位在印记者续签。报道称,这是最后两名仍在印度的中国官媒记者,但随着他们的签证过期,两人已经离境……}

\entryitemWithDescription{港府提交区议会改革草案 冀立法会7月中前通过}{https://www.zaobao.com/news/china/story20230531-1400123}{香港特区政府星期三(5月31日)向立法会提交改革区议会的修例草案,希望在7月中立法会休会前通过,以便下一届区议会全面落实``爱国者治港''。 受访学者认为,区议会在2019年反修例运动期间一度被民主派操控;当局这次改革,将可以从制度上确保区议会日后牢牢掌握在建制派手中……}

\entryitemWithDescription{中国海军将赴印尼参加多国海上联演}{https://www.zaobao.com/news/china/story20230531-1400120}{中国解放军海军将赴印度尼西亚参加6月上旬举行的多国海上联合演习。 中国国防部星期三(5月31日)发布消息,称应印尼海军邀请,中国解放军海军湛江舰、许昌舰将赴印尼参加6月上旬举行的``科莫多-2023''多国海上联合演习。 路透社报道,主办国也邀请朝鲜、俄国、韩国、美国等参加演习。报道说,此次演习是在中美于本区域加强推动军事外交策略的大背景下举行……}

\entryitemWithDescription{中国国防部:中美两军交流接触并未中断}{https://www.zaobao.com/news/china/story20230531-1400118}{针对中国拒绝中美两国防长在新加坡会晤,以及中方多次拒绝美方请求两军交流的说法,中国国防部回应称两军接触交流并未中断,当前交流困难的责任完全在美方。 根据中国国防部网站消息,国防部发言人谭克非星期三(5月31日)说,中国重视发展中美两军关系和开展各层级沟通,两军接触交流并未中断。 谭克非说,当前中美两军交流面临困难,责任完全在美方。他说,``对话不能没有原则,沟通不能没有底线……}

\entryitemWithDescription{多家跨国企业总裁访华 表态继续在中国开展业务}{https://www.zaobao.com/news/china/story20230531-1400113}{特斯拉创办人兼总裁马斯克(中)星期三在北京离开中国商务部大楼之前,与中国商务部长王文涛(左一)道别。(路透社) 在特斯拉创办人兼总裁马斯克(Elon Musk)拜会中国高官和企业家之际,多位跨国企业高级主管也在本周密集到访中国。 据中国工业与信息化部官方微信公众号的消息,工信部部长金壮龙星期三(5月31日)在北京会见马斯克,双方就新能源汽车和智能网联汽车发展等课题交换意见……}

\entryitemWithDescription{香港修订国歌指引 若主办方拒核对须拒参赛}{https://www.zaobao.com/news/china/story20230531-1400079}{香港体育协会暨奥林匹克委员会(简称港协)公布经修订的处理国歌和区旗指引,明订若主办机构拒绝让领队核对国歌和区旗,领队须拒绝队员参赛或参与颁奖礼,直至完成核对程序。 香港《明报》报道,港协星期二(30日)以实体会议及线上形式同步举行简介会,说明指引重点。 港协在新闻稿提到七项修订,包括要求各体育总会每次出发参赛前,须领取附有正确国歌硬件的工具包,并须由主办机构签署标准收条,书面确认收妥等……}

\entryitemWithDescription{于泽远:中国官媒再谈``颜色革命''}{https://www.zaobao.com/news/china/story20230531-1399747}{5月中旬以来,与中俄交好的欧洲国家塞尔维亚政局发生动荡,数千人走上首都贝尔格莱德街头示威,矛头指向总统武契奇领导的政府。武契奇5月27日辞去执政党主席职务,并公开指责有外国试图在塞尔维亚搞``颜色革命''。 塞尔维亚局势引起中国舆论关注……}

\entryitemWithDescription{曾在快速反应部队服役 胡中强出任中国南部战区副司令员}{https://www.zaobao.com/news/china/story20230530-1399787}{曾在中国人民解放军快速反应部队服役的胡中强中将,已出任解放军南部战区副司令员兼南部战区陆军司令员。 解放军南部战区公众号上星期六(5月27日)披露,代号为``友谊盾牌---2023''的中国和老挝联合演习,5月26日落下帷幕。消息称,老挝副总理兼国防部长占沙蒙上将、解放军南部战区副司令员兼南部战区陆军司令员胡中强中将,中国驻老挝大使、驻老挝使馆国防武官出席了结束仪式……}

\entryitemWithDescription{中国经济复苏不均衡 摊贩街边赚钱补贴收入}{https://www.zaobao.com/news/china/story20230530-1399782}{图为5月27日晚上,一名男子在上海街边的面摊煮面。(路透社) 中国社会活动在冠病疫情结束后复常,但因经济复苏不均衡,就业机会和薪资增长呆滞,摊贩到街边谋生补贴收入,43岁的卖米糕小贩王春香(译音)是其中一人。 据路透社星期二(5月30日)报道,王春香推着售卖蒸甜米糕小车穿梭于上海街道,边卖米糕,边躲闪监管人员的追查。 王春香受访时说,她赚取的薪资太低,不足以维持生计……}

\entryitemWithDescription{港府宣布成立特首政策组专家组 包括中国大陆著名政治学者郑永年}{https://www.zaobao.com/news/china/story20230530-1399780}{香港特区政府星期二宣布成立特首政策组专家组,委任56名成员,中国大陆著名政治学者郑永年也在列。 根据香港政府新闻网消息,港府星期二(5月30日)宣布成立特首政策组专家组,由56名来自商业和金融界别、专业人士、智库及学术界等不同背景的人士组成,就各项专题向特首政策组提供专业意见和崭新构思,任期一年,即时生效。 专家组分为三个组别:研究策略专家组、经济发展专家组、社会发展专家组……}

\entryitemWithDescription{中国神舟十六号航天员 顺利进驻``天宫''空间站}{https://www.zaobao.com/news/china/story20230530-1399777}{神舟十六号航天员乘组星期二(5月30日)从飞船返回舱进入轨道舱,与神舟十五号航天员乘组实现''太空会师``。图为两个乘组的航天员一起拍摄``全家福''。 (中国载人航天工程办公室 供图) 中国``神舟十六号''载人飞船星期二(5月30日)成功发射,并与``天宫''空间站的``天和''核心舱成功对接,三名航天员顺利从飞船返回舱进入轨道舱,标志着中国空间站全面转入常态化运营模式……}

\entryitemWithDescription{中国鼓励高校毕业生在社区内就业}{https://www.zaobao.com/news/china/story20230530-1399776}{在中国青年就业形势严峻的背景下,中国民政部要推动社区社会组织开发更多工作岗位,鼓励高校毕业生在社区内就业。 民政部办公厅上星期六(27日)发布《关于做好2023年社会组织助力高校毕业生就业工作的通知》,指出各地要推动社区社会组织充分挖掘社区服务需求,开发社区服务岗位,鼓励和支持高校毕业生在社区内就业……}

\entryitemWithDescription{特稿:中俄关系阻碍中国与国际航天合作?}{https://www.zaobao.com/news/china/story20230530-1399755}{图为5月30日在北京航天飞行控制中心拍摄的画面,显示神舟十六号载人飞船于空间站``天和''核心舱径向端口成功对接。(新华社) 中国航天发展30年内迅速完成``三步走'',但紧张的地缘政治形势,拖慢了中国航天国际合作步伐。受访学者分析,国际航天合作有助树立中国科技领导者的形象,但中俄的密切关系可能成为各国与中国航天合作的一道阻碍……}

\entryitemWithDescription{中国外长会见马斯克 :打造更好的法治化营商环境}{https://www.zaobao.com/news/china/story20230530-1399739}{中国国务委员兼外长秦刚会见特斯拉首席执行官马斯克,并表示中国将继续坚定不移推进高水平对外开放,为包括特斯拉在内的各国企业打造更好的营商环境。马斯克则表示,反对脱钩断链,愿继续拓展在华业务。 中国外交部网站星期二(5月30日)消息,秦刚在北京会见马斯克时表示,中国式现代化将创造前所未有的增长潜力和市场需求,中国新能源汽车产业发展前景广阔……}

\entryitemWithDescription{侯友宜支持度不升反降并落后柯文哲}{https://www.zaobao.com/news/china/story20230530-1399735}{民众党主席柯文哲(右)和鸿海集团创办人郭台铭(左)星期二在金门参加迎城隍活动时一起合力扶轿,引起台湾媒体有关``郭柯配''的选举联想。 (香港中通社) 新北市长侯友宜虽然获得国民党征召出战2024年台湾总统大选,但他的最新民意支持度不升反降……}

\entryitemWithDescription{【东谈西论】台湾能甩掉``行人地狱''的恶名吗?}{https://www.zaobao.com/news/china/story20230530-1399665}{台湾是许多人出国旅游时向往的休闲度假胜地,网民和国际媒体近年来把台湾交通形容为 ``行人地狱''。(法新社) 去年,台湾每10万人交通死亡数为13.2,这是日本、新加坡的六倍,而且死亡人数还在创新高。 台湾每年有超过3000人死于车祸,其中有10\%是行人。当地的民众把台湾交通成为 ``行人地狱''。台湾的交通状况还引起美国和日本媒体的注意。 台湾的交通怎么了……}