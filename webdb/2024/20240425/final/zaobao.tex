\entryitemWithDescription{陈婧:老外又来了}{https://www.zaobao.com/news/china/story20240425-3494614}{从新加坡飞回上海,一踏进浦东机场入境厅,映入眼帘的是一整面墙的支付宝广告。大屏幕上亮出和支付宝合作的多家外国付款机构标识,上方是醒目的中英双语广告词:``欢迎使用Alipay+钱包畅游中国''。 根据广告,外国游客可以扫码查看中国大陆支持的电子钱包列表;支付宝还在浦东机场里设立服务台,协助入境游客设置电子钱包。几名洋人旅客在广告牌停下脚步,掏出手机研究起来……}

\entryitemWithDescription{台湾宪法法庭激辩死刑的存废}{https://www.zaobao.com/news/china/story20240424-3493732}{台湾宪法法庭星期三(4月23日)举行死刑是否违宪的公开激烈辩论,三个月内将有判决。由于约八成主流民意反对废除死刑,一旦法院判决死刑违宪,将对社会和政治带来深远冲击。 推动废死、台独与非核家园都明列在民进党党纲,被视为民进党``神主牌''。虽然台湾法律还没有废死,却是几乎没有执行死刑的``实质废死''状态。 过去五年476起杀人案,只有一起判死刑……}

\entryitemWithDescription{美学者:美中之间最大风险是南中国海}{https://www.zaobao.com/news/china/story20240424-3494262}{美国学者一致认为,美中之间最大的风险不再是台湾,而是南中国海,长远则是科技之争。有美国学者指出,目前东南亚国家同时担忧中国在南中国海的行为会更强硬,以及美国对区域的投入减弱,这些都推动地区军费走高。 中美学者星期二(4月23日)在新加坡出席南洋理工大学拉惹勒南国际研究院举办的第五届``三边交流''闭门论坛后,分别在会场外接受本地媒体采访……}

\entryitemWithDescription{布林肯抵上海 星期五赴北京与王毅会谈}{https://www.zaobao.com/news/china/story20240424-3494204}{美国国务卿布林肯(右一)搭乘行政专机于4月24日下午抵达上海,上海市政府外办主任孔福安(左三)与美国驻华大使伯恩斯(左二),分别代表中方及美方在机场迎接。(路透社) 美国国务卿布林肯乘坐的车队(图中未显示)4月24日驶过上海外滩时,马路边挤满了民众。(路透社) 美国国务卿布林肯(前排左)和美国驻华大使伯恩斯(前排右)4月24日在上海豫园漫步……}

\entryitemWithDescription{中国将派遣三名宇航员飞赴``天宫''空间站}{https://www.zaobao.com/news/china/story20240424-3493954}{中国神舟十八号载人飞行任务航天员(左起)李广苏、叶光富、李聪4月24日在酒泉卫星发射中心问天阁与媒体记者集体见面。(法新社) 中国神舟十八号载人飞行任务航天员(左起)李广苏、叶光富、李聪4月24日在酒泉卫星发射中心问天阁与媒体记者集体见面……}

\entryitemWithDescription{【早知】布林肯访华 中美如何角力?}{https://www.zaobao.com/news/china/story20240424-3493301}{布林肯(右一)星期三下午飞抵上海后,他步下舷梯,受到上海市政府外事办公室主任孔福安(左三)、美国驻中国大使伯恩斯(左二)和美国驻上海总领事王汉(Scott Walker,左一)的迎接。(路透社) 美国国务卿布林肯星期三(4月24日)正式展开访华行程,但美国对中国各类制裁不仅未见松懈,反而有逐渐升温之势,而双方对会谈的目标不尽相同,也揭示着这位美国高级外交官此行将面临严峻挑战……}

\entryitemWithDescription{美官员:美国没有计划立即就俄罗斯问题制裁中资银行}{https://www.zaobao.com/news/china/story20240424-3491137}{(华盛顿综合讯)美国国务卿布林肯访华前夕,美国官员透露,美国已初步讨论对部分中资银行实施制裁,但尚未制定有关制裁实施的计划。 匿名的美国官员星期二(4月23日)向路透社说,美国在短期内没有计划对中国的银行实施制裁,并希望通过外交手段来避免采取制裁行动。布林肯星期三(4月24日)起访问中国,这是他时隔不到一年再度访华……}

\entryitemWithDescription{杨丹旭:布林肯访华能谈出些什么?}{https://www.zaobao.com/news/china/story20240424-3487734}{美国国务卿布林肯星期三(4月24日)将开启任内第二次访华行程。眼下中美关系表面上看似趋稳,实际上却暗流涌动,双边关系在科技、经贸等多方面都在经受考验。 美国在科技战上继续对中国步步紧逼,最新的动作是众议院在上周六(20日)通过一项强制字节跳动出售TikTok的``不卖就禁''法案。参议院将在本周表决,这项法案大概率将通过,并由美国总统拜登签署成为法律……}

\entryitemWithDescription{广东强降雨持续 深圳进入暴雨防御状态}{https://www.zaobao.com/news/china/story20240423-3487557}{深圳市气象台星期二(4月23日)发布暴雨红色预警信号,全市进入暴雨紧急防御状态。深圳部分低洼路段积水,出行市民以及城市交通受到一定影响。(中新社) (广州/北京/巴黎综合讯)中国广东省连日遭遇暴雨袭击,多个地区灾情险情严峻,深圳星期二(4月23日)也进入暴雨紧急防御状态,全市多区发布了最高级别的暴雨红色预警信号……}

\entryitemWithDescription{民进党内阁布局完成 在野党无入阁}{https://www.zaobao.com/news/china/story20240423-3486043}{台湾候任总统赖清德的新内阁人事完成布局,星期二(4月23日)由候任行政院长卓荣泰宣布第五波内阁名单,星期四(25日)由赖清德压轴主持最后第六波名单,宣布国防、外交、国安和政务委员等首长人事安排。 民进党籍的赖清德与副总统萧美琴,以及卓荣泰组建的新内阁团队,将在5月20日宣誓就职。赖清德先在4月10日宣布卓荣泰将出任行政院长,卓荣泰在随后的两周内分五批宣布内阁新人事……}

\entryitemWithDescription{《维护国家安全条例》已禁传播虚假陈述 港府不会就假新闻立法}{https://www.zaobao.com/news/china/story20240423-3487434}{香港律政司司长林定国说,香港基本法23条立法已禁止传播虚假陈述,因此港府无需就假新闻立法。图为林定国在今年4月15日全民国家安全教育日前接受港媒访问。(香港中通社) 香港2019年发生反修例风波后,``假新闻''议题一度成为特区政府关注的重点。但港府周二(4月23日)首次明确表明,新订立的《维护国家安全条例》(即俗称的基本法23条立法)已涵盖煽动罪行,目前没计划再制定``假新闻法''……}

\entryitemWithDescription{海船碰擦广东九江大桥后沉没 四人失联}{https://www.zaobao.com/news/china/story20240423-3486986}{桥墩被撞的广东佛山九江大桥。(新华社) (佛山综合讯)广东遭遇暴雨洪灾之际,佛山发生一起海船碰擦大桥事故,造成四人失联,初步判定是洪水导致船员操作失当造成事故……}

\entryitemWithDescription{布林肯访华前夕 传美国拟制裁中国金融机构图阻对俄支援}{https://www.zaobao.com/news/china/story20240423-3486657}{美国国务卿布林肯将于星期三(4月24日)开启访华行程。(路透社档案照) 就在美国国务卿布林肯访华前,美国媒体传出华盛顿正研拟制裁一些中国金融机构,以阻止它们对俄罗斯的商业支援。中国外交部星期二(4月23日)回应称,中国将坚定捍卫自身合法权益。受访学者分析,两国立场泾渭分明,布林肯此行恐难有实质进展……}

\entryitemWithDescription{台湾花莲余震不断 不到一天逾200次}{https://www.zaobao.com/news/china/story20240423-3485654}{富凯大饭店早在4月3日强震后,已列入存在风险的受损建筑物红单;星期二(4月23日)余震之后更进一步倾斜。(法新社) (台北综合讯)台湾花莲海域4月3日发生里氏规模7.2的强震后,余震不断。星期一(4月22日)晚至隔天上午余震次数更是猛增,不到一天时间发生了200多次地震,所幸损失轻微,没有人伤亡……}

\entryitemWithDescription{老挝将250名电诈犯罪嫌疑人移交给中国}{https://www.zaobao.com/news/china/story20240423-3486698}{(北京综合讯)中国与老挝警方继续开展国际警务执法合作,最新捣毁三个位于老挝金三角经济特区的犯罪窝点,抓获250名跨境裸聊敲诈和电诈犯罪嫌疑人,并已全部移交中国警方。 据中国公安部网站星期二(4月23日)的消息,中国公安机关近日通过国际警务执法合作,将相关线索通报老挝警方。老挝警方迅速组织执法力量,对金三角经济特区内的犯罪窝点开展统一清查行动,抓获目标犯罪嫌疑人160名,网上在逃人员90名……}

\entryitemWithDescription{日本成中国``五一''出境游第一目的地}{https://www.zaobao.com/news/china/story20240423-3486922}{樱花盛开景象,与白雪皑皑的富士山遥相呼应。(路透社) (北京/东京综合讯)在线旅游平台信息显示,即将来临的``五一''假期里,日本是中国游客出境游的第一目的地,其次为泰国和韩国。 据财新网报道,多个平台数据显示,中国赴日旅游热度暴增。在携程网上,五一假期出境游目的地中,日本的热度居首,民宿平台Airbnb数据也显示,日本是今年春季热门搜索目的地的榜首……}

\entryitemWithDescription{中国海军成立75周年 对外展示装备和战力进步}{https://www.zaobao.com/news/china/story20240423-3481807}{中国官方上周六(4月20日)邀请多国媒体访问海防重镇青岛,并开放贵阳舰等海军舰艇供外媒参观。图为开放日活动中,一名孩童坐在贵阳舰的甲板上。(路透社) 中国海军成立75周年之际,官方邀请多国媒体访问海防重镇青岛,并向境外媒体开放多艘现役主力战舰,让外界有难得的机会实地感受中国海军在区域防空能力上的升级。 本周二(4月23日)是中国海军成立75周年……}

\entryitemWithDescription{赖清德5月20日宣誓就职前 学者:北京对台施展两手策略}{https://www.zaobao.com/news/china/story20240422-3481526}{民进党籍的台湾候任总统赖清德将在5月20日宣誓就职。(赖清德脸书) 台湾5月20日新总统与执政团队就职倒数四周,北京加大对台军事与经济施压力道。受访学者分析,北京通过经营中美对话来管控台海风险,无论对待中美或两岸关系,都更关注长远发展而非短期争霸。 民进党籍的候任总统赖清德与副总统萧美琴,以及候任行政院长卓荣泰组建的新内阁,将在5月20日宣誓就职……}

\entryitemWithDescription{中国驻美大使批美国对华搞霸凌 分析:中国政治圈子对美失望不满升高}{https://www.zaobao.com/news/china/story20240422-3482329}{中国驻美国大使谢锋4月20日在哈佛肯尼迪中国论坛开幕式上发表演讲。(新华社) 美国国务卿布林肯本周即将访华,中国驻美大使谢锋批评美国对华搞霸凌,并警告若任由竞争主导中美关系,就如同在危险的悬崖边上高速飙车,自以为车技高超,但稍有不慎就会跌落悬崖,即使设再多护栏也无济于事……}

\entryitemWithDescription{中国南方强降雨天气持续 造成广东四死10失联}{https://www.zaobao.com/news/china/story20240422-3482468}{中国南方地区近期遭遇极端降雨,多个地方累计降水量破4月历史纪录。图为星期一(4月22日)无人机拍摄到广东省清远地区的一处公路被洪水淹没。(路透社) (北京/广州综合讯)中国南方地区近期遭遇极端降雨,多个地方累计降水量破4月历史纪录。连日来的持续强降雨,造成广东多地受灾严重,目前已造成全省四人死亡、十人失联。 据中国天气网消息,进入4月以来,中国南方地区降雨频繁……}