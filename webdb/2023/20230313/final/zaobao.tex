\entryitemWithDescription{学者:全员无国务院领导经验 李强内阁角色料着重执行非制定政策}{https://www.zaobao.com/news/china/story20230313-1371927}{由排名第一的国务院副总理丁薛祥(前排中)星期天领誓,国务院副总理以及国务委员星期天(3月12日)集体进行宪法宣誓就职。 后排左起为秦刚、吴政隆、李尚福、张国清、何立峰、刘国中、王小洪和谌贻琴。(法新社) 新加坡国立大学李光耀公共政策学院助理教授陆曦分析,此次内阁名单没有任何意外,李强若想增强外界的信心,他的内阁团队必须进行制度性建设,并要营造长期的制度环境,不能只出台短期临时的政策……}

\entryitemWithDescription{易纲连任央行行长 分析:显示北京要保持政策连贯性}{https://www.zaobao.com/news/china/story20230313-1371930}{分析认为,中国央行行长易纲连任,释放金融改革将稳扎稳打的积极信号,稳固市场信心。图为易纲在3月3日的新闻发布会上讲话。(路透社) 已届退休年龄的中国央行行长易纲连任职务,出乎外界意料。分析认为,易纲连任显示中国官方有意保持金融经济领域政策连贯性,释放金融改革将稳扎稳打的积极信号,稳固市场信心……}

\entryitemWithDescription{虽连任卫健委主任 清零推手马晓伟反对票最多}{https://www.zaobao.com/news/china/story20230313-1371931}{马晓伟以2917票赞成,21票反对,八票弃权,连任中国国家卫生健康委员会主任。(互联网) 中国国家卫生健康委员会主任马晓伟续任职务。他是国务院组成部门负责人当中,获最多反对票的人选。 中国全国人大全体会议星期天(3月12日)上午表决通过由国务院总理李强提名的各部部长、各委员会主任、中国央行行长、审计长、秘书长等人选。外交、国防、公安等``一把手''均获2946张赞成票,无反对票……}

\entryitemWithDescription{早 说}{https://www.zaobao.com/news/china/story20230313-1371932}{台湾的政论节目很多,为了让名嘴每天都有话题,所以就像八点档连续剧,一条基本的主线外,要加入很多的插曲情节,且剧本的走法要符合大部分观众的认知模式,这样大家才不用花太多脑力往下看下去。 ------国民党智库执行长柯志恩星期天(3月12日)在脸书贴文如是说。她也说,目前主线发展到``民进党已定于一尊,国民党还在乱'',但她不觉得国民党有乱,只是这不符合外界对国民党的期待……}

\entryitemWithDescription{过去仰赖赴陆田野调查方式风险大增 两岸情势改变在台大陆研究系统转型}{https://www.zaobao.com/news/china/story20230313-1371934}{政治大学国际关系研究中心是台湾最早成立,从事中国大陆研究的机构。 (缪宗翰摄) ▲台湾已故总统蒋中正在1953年批示,要求成立对中、苏共研究的机构,开启台湾从事中国大陆研究的源头。图为蒋中正批示手稿影本……}

\entryitemWithDescription{多次播错国歌 港府咎责谷歌}{https://www.zaobao.com/news/china/story20230313-1371935}{香港特首李家超对香港队多场国际赛事接连出现错播中国国歌事件,表示不能接受,并指相关事件涉及搜索平台谷歌提供的不正确结果。 综合``香港01''、香港电台网报道,李家超星期天(12日)出发去北京、准备参加星期一举行中国全国人大会议闭幕式之前,重申中国国歌代表民族、国家尊严,港府会竭尽一切所能,确保国歌在不同场合能正确播放……}

\entryitemWithDescription{百度赶在16日推出类似ChatGPT聊天机器人}{https://www.zaobao.com/news/china/story20230313-1371936}{百度将在3月16日推出类似ChatGPT的聊天机器人,其员工赶在最后期限前完成相关工作,但机器人目前仍难以完成一些基本功能。 据《华尔街日报》报道,知情人士透露,为研发这款名为``文心一言''(Ernie Bot)的人工智慧聊天机器人,数百名员工夜以继日地工作。公司已从其他团队借调人手,和功能强大的计算机芯片来支援研发工作。美国去年底实施制裁,禁止中国公司购买新芯片……}

\entryitemWithDescription{港媒:若麦卡锡访台 北京提议哈里斯访华}{https://www.zaobao.com/news/china/story20230313-1371937}{香港媒体报道,北京通过``二轨''向美国提议,如果美国众议院议长麦卡锡访问台湾,美国要派副总统哈里斯访问中国大陆,以显示北京获尊重,到访的美国官员层级比访台的更高。 《南华早报》上星期五(3月10日)报道,虽然麦卡锡和台湾总统蔡英文见面的地点从台北改成加州,但麦卡锡并没有排除未来可能访问台湾……}

\entryitemWithDescription{港民建联主席李慧琼 当选全国人大常委}{https://www.zaobao.com/news/china/story20230312-1371612}{新任港区全国人大代表、香港立法会最大党民建联主席李慧琼星期六(3月11日)当选第14届中国全国人大常委。 综合《明报》《星岛日报》和网媒``香港01''等报道,全国人大会议星期六上午投票选出159名全国人大常委。 香港唯一的候选人李慧琼在2939张有效选票中,以2901张赞成、35张反对和三张弃权当选,接替民建联元老谭耀宗,成为香港历来最年轻的全国人大常委……}

\entryitemWithDescription{台湾称不与大陆进行金援竞赛}{https://www.zaobao.com/news/china/story20230312-1371613}{西太平洋岛国密克罗尼西亚联邦据传希望台湾提供5000万美元(6745万新元)援助,换取密台建交。有台湾立委质疑这是``凯子外交''。台湾外交部回应说,外交援助绝非``凯子外交'',也不会与中国大陆进行金援竞赛……}

\entryitemWithDescription{拒国安处要求交资料 港支联会三前成员判监四个半月}{https://www.zaobao.com/news/china/story20230312-1371614}{曾在香港主办天安门事件纪念活动多年的支联会,三名前成员拒绝应香港警方国安处要求,交出该会运作资料,星期六(3月11日)被判监四个半月。 综合香港01、星岛网、《明报》和法新社报道,被判监的三名成员分别是支联会前副主席邹幸彤、前常委邓岳君与徐汉光。香港国安法指定法官、西九龙法院裁判官罗德泉在判词中说,维护国家安全至关重要,判刑须反映维护国家安全的决心……}

\entryitemWithDescription{中国特稿:两岸关系趋缓春暖未必花开}{https://www.zaobao.com/news/china/story20230312-1371615}{新任中国全国政协主席王沪宁(面向镜头,中)3月9日参加全国人大会议台湾代表团的审议时指出,大陆进一步掌握了实现完全统一的战略主动。(新华社) 今年中国两会(全国人大、全国政协会议)期间透露的信息显示,大陆今年度对台政策,将以主导恢复民间交流、促进统一为主轴。不过,分析认为,北京``硬的一手''会对民间交流产生制约;两岸官方若进一步将交流与协商``两制''方案划上等号,反而令局势再添变数……}