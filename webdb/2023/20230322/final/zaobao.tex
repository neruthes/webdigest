\entryitemWithDescription{一个过境访美 一个``登陆''祭祖 蔡英文马英九 反方向出访在台引热议}{https://www.zaobao.com/news/china/story20230322-1374934}{民进党立委王美惠星期二在立法院质询行政院长陈建仁时,赞美台湾总统蔡英文月底出访中美洲友邦、过境美国,但质疑前总统马英九去中国大陆祭祖的动机。(法新社) 路透社引述一位美国高官的话说,蔡英文未来两周过境美国是标准做法,北京没有理由错误解读此次符合美国长期政策的过境,更不应以此为借口加强在台海周边地区的侵略活动……}

\entryitemWithDescription{26年来首位德国部长来访 欧美政治人物掀访台热潮}{https://www.zaobao.com/news/china/story20230322-1374935}{德国教育与研究部长瓦特辛格(右)与台湾国科会主委吴政忠会谈后表示,希望台湾和德国共同为经济创新和高科技研究创造更多附加价值。(取自台湾国科会网站) 欧美政治人物访问台湾本周迎来热潮,包括26年来首位访台的德国部长瓦特辛格见证签署``台德科技合作协议'',引起中国外交部反弹,向德国提出外交申诉……}

\entryitemWithDescription{台总统选举民调:赖清德支持度上升8.5个百分点}{https://www.zaobao.com/news/china/story20230322-1374936}{亲绿的台湾民意基金会星期二(3月21日)发布2024年总统选举的最新民调,现任副总统赖清德的支持度较2月上升了8.5个百分点,以36.2%支持度领先新北市长侯友宜与民众党主席柯文哲。 台湾民意基金会在网站公布,赖清德的支持度从2月的27.7%上升至36.2%,侯友宜则从2月的32.4%滑落至24.8%,柯文哲从19.5%微跌至19.0%……}

\entryitemWithDescription{港支联会前主席何俊仁 保释期间涉干扰证人被捕}{https://www.zaobao.com/news/china/story20230322-1374938}{香港国安处警员星期二(3月21日)押走何俊仁(右二)的情景。(路透社) 已解散的香港支联会前主席何俊仁涉嫌在保释期间干扰证人,再次被港警国安处拘捕,罪名是``妨碍司法公正''。 综合《星岛日报》与香港01报道,香港国安处警员星期二(3月21日)上午11时许在香港天后庙道一寓所带走了何俊仁……}

\entryitemWithDescription{中国东航空难一周年前夕 民航局:还在持续深入调查}{https://www.zaobao.com/news/china/story20230322-1374939}{中国东方航空MU5735航班空难事故一周年前夕,中国民用航空局星期一(3月20日)晚间通报,由于本起事故非常复杂、极为罕见,调查还在持续深入进行中。 中国民航局在官网发布事故调查进展情况。通报称,一年来技术调查组对飞机残骸进行详细检查,确定坠地前飞机关键操纵部件可能的工作状态,包括对100余件重要残骸进行实验,分析损坏原因……}

\entryitemWithDescription{早 说}{https://www.zaobao.com/news/china/story20230322-1374940}{我不会猜测所采取的反制措施,但中国不会轻易咽下这口气。 ------针对荷兰政府在华盛顿施压下,3月8日决定将扩大限制对华出口先进芯片制造技术,中国驻荷兰大使谈践接受荷兰媒体访问时如是说。他说,荷兰政府决定对向中国出口先进芯片制造技术实施新限制,将引发不明确的后果……}

\entryitemWithDescription{称被中国民航局改为``红眼航班''难以经营 3月26日起新航暂停飞重庆}{https://www.zaobao.com/news/china/story20230322-1374942}{新航原本向中国民航局申请今年夏秋季每周往返新加坡和重庆10趟。不过,当局仅批准新航继续每周一从新加坡飞重庆一次,航班时间更被改为凌晨2时15分抵达重庆江北机场,逾一小时后(即凌晨3时30分)离港。新航评估``红眼航班''难以经营,选择取消飞重庆的计划……}

\entryitemWithDescription{能源需求加速增长 中国大量租用超级油轮}{https://www.zaobao.com/news/china/story20230322-1374943}{中国正大量租用超级油轮,显示这个全球第二大经济体的能源需求,在解除冠病疫情封控措施后已加速增长。 每艘可装载200万桶原油据《华尔街日报》星期一(3月20日)报道,石油贸易商用超大型油轮(VLCC)将原油运往中国这个全球最大的石油进口国。这种油轮的体积堪比埃菲尔铁塔,每艘可装载200万桶原油……}

\entryitemWithDescription{业者:港陆解除防疫限制 艺术市场前景看涨}{https://www.zaobao.com/news/china/story20230322-1374945}{由洛杉矶概念艺术家安沃·艾力克创作,名为``重力''的埃及图坦卡蒙国王作品,上星期五(3月17日)起在金钟太古广场展出。(路透社) 香港巴塞尔艺术展主办者说,中国大陆和香港解除了所有防疫限制措施,令他们看涨该区域的艺术市场前景……}

\entryitemWithDescription{卜睿哲:万一发生台海战争 美国就算要驰援也得花上数星期}{https://www.zaobao.com/news/china/story20230321-1374535}{台湾前参谋总长李喜明上将(背向镜头)向卜睿哲(右二)提问,一旦台海发生冲突,美国是否会驰援、是否已在准备方案?台上还有远见·天下文化事业群创办人高希均教授(右一)和苏起。(温伟中摄) 美国在台协会前理事主席卜睿哲认同台湾国安会前秘书长苏起的评估,万一发生台海战争,美国就算要驰援也得花上数星期……}

\entryitemWithDescription{港巴冲撞分界堤 44人受伤送院}{https://www.zaobao.com/news/china/story20230321-1374536}{香港九龙巴士公司的一辆双层巴士,星期一(20日)上午9时42分从九龙半岛的将军澳前往荃湾西站,驶至呈祥道时突然失控猛撞路中分界堤,巴士左边车头被削开毁烂,挡风玻璃碎裂,左边车身搁在石堤上。该辆巴士失事时连同巴士车长,全车约有77人,当中上层55名乘客一度被困,部份人撞损受伤流血。消防人员耗时约1小时将乘客逐一救出,44人受伤送院……}

\entryitemWithDescription{新西兰外长今天访华}{https://www.zaobao.com/news/china/story20230321-1374538}{新西兰外交部长马胡塔将于星期二(3月21日)访华,在北京与中国外交部长秦刚会面。这将是自2019年以来,新西兰部长首次访问中国。 中国外交部发言人汪文斌星期一(20日)宣布,应中国国务委员兼外长秦刚邀请,新西兰外长马胡塔将于3月22日至25日访问中国。 据路透社报道,马胡塔在声明中说,她将向秦刚提出新西兰对关键安全挑战的关切,如俄罗斯非法入侵乌克兰,同时也为人权议题发声,以反映新西兰的价值观……}

\entryitemWithDescription{中国在黄海实弹演习}{https://www.zaobao.com/news/china/story20230321-1374539}{美韩、美日在日本海附近展开联合军演之际,中国军方自星期一(3月20日)开始在黄海北部海域进行为期四天的实弹射击,禁止船只驶入。 据中国海事局网站消息,大连海事局上星期四(3月16日)发布航行警告,自3月20日零时至24日24时,黄海北部部分海域进行实弹射击,禁止船只驶入。中国军方此次是在一个月内第二次在黄海海域进行实弹射击……}

\entryitemWithDescription{早说}{https://www.zaobao.com/news/china/story20230321-1374540}{台海危机风险愈演愈烈,一旦国家(中国政府)因形势所逼不得不出手解决台湾问题时,美国势必对中国采取更疯狂的制裁行动。美国官方已多次在这方面放出狠话,到时香港极有可能成为美国金融制裁打击的目标。 ------中国全国侨联副主席、中国和平统一促进会香港总会理事长卢文端,在《明报》撰文谈及香港须应对美国制裁封锁……}

\entryitemWithDescription{戴庆成:北京组建``中央港澳办''的意义}{https://www.zaobao.com/news/china/story20230321-1374541}{维护国家安全位居前列位置,显示中央港澳办成立后,北京在香港将会更主动应对各种风险挑战,以维护国安……}

\entryitemWithDescription{港口空集装箱堆积 海关总署长称国际看好中国出口能力}{https://www.zaobao.com/news/china/story20230321-1374542}{中国港口据报出现大量集装箱空箱堆积情况,引发外界对中国外贸形势可持续能力的担忧。中国海关总署署长俞建华星期一(3月20日)回应称,大量空箱在中国港口蓄势待发,某种程度上反映了国际市场依然看好中国下一阶段的出口能力。 据新华社报道,俞建华在发布会上说,空箱增多有前一个时期新集装箱投放量过大、国内堆存成本较低、国外疫情缓解后空箱短期大量回流的原因,也有季节性规律作用……}

\entryitemWithDescription{因疫情防控及中美关系降温 美赴华留学生人数跌至新低}{https://www.zaobao.com/news/china/story20230321-1374543}{受中国严格的疫情防控和日渐紧张的中美关系影响,美国赴华留学生骤降至20多年来的最低点,在2020至2021学年仅382名。 据美国国务院下属的美国教育中心(EducationUSA)公布的数据,2020至2021学年的赴华留学生人数不及前一学年2481人的六分之一,较2018年至2019学年的1万1639人更是锐减了近97\%……}