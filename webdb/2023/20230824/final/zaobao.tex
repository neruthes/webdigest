\entryitemWithDescription{华为据报秘密建芯片厂规避美国制裁}{https://www.zaobao.com/news/china/story20230823-1426573}{中国通讯巨头华为的品牌标志,在6月28日举行的上海世界移动通信大会上展现。(路透社) 美国半导体协会称,中国通讯巨头华为正在中国各地以其他公司名义秘密建造一系列半导体制造设施,以规避美国禁止其采购美国芯片制造设备的制裁。 综合彭博社、路透社报道,总部位于华盛顿的美国半导体协会指,华为自去年开始投入芯片生产领域,并获得中国政府约300亿美元(约407亿新元)的补助……}

\entryitemWithDescription{中国房地产危机加剧 官媒称``房住不炒''原则不应也不会改变}{https://www.zaobao.com/news/china/story20230823-1426570}{中国房地产市场下行压力持续加大,多家房企面临资金链断裂风险,令市场对下半年调整房地产调控政策的预期增加。不过,官媒星期三(8月23日)发文强调,``房住不炒''是促进房地产市场平稳健康发展的一项总原则,不应也不会改变,必须长期坚持。 中国国务院主办的《经济日报》星期三发表评论,批驳了外界认为当前形势下再提``房住不炒''不合时宜的观点……}

\entryitemWithDescription{云南警方密集反诈宣传 警惕``境外高薪招聘''藏凶险}{https://www.zaobao.com/news/china/story20230823-1426567}{中国云南省多地警察密集展开新一轮的反诈骗宣传,警惕民众``境外高薪招聘''的陷阱,提醒不要被诱骗非法出境;一旦上钩,将被当成``猪仔''随意买卖,受尽折磨……}

\entryitemWithDescription{``八二三''金门炮战65周年 蔡英文宣示维护和平须壮大国防}{https://www.zaobao.com/news/china/story20230823-1426565}{台湾总统蔡英文(前排右一)任内第三次到金门主持八二三炮战纪念活动,与国民党总统参选人侯友宜(左一)一同向捐躯保台的阵亡军人默哀致敬。(自由时报) 台湾总统蔡英文星期三(8月23日)在金门主持``八二三''炮战65周年祭悼典礼时表示,如果没有1958年八二三的胜利,就没有今日的台湾。民进党政府致力维护台海和平稳定的立场非常坚定,而要维护和平,必须先要有强大的国防……}

\entryitemWithDescription{港府不排除扩大日本水产品进口禁令 学者:视民意反应而定}{https://www.zaobao.com/news/china/story20230823-1426548}{日本首相岸田文雄8月21日宣布,福岛第一核电站的核废水最快可在星期四(24日)开始排入海中。隔天,香港渔民团体在日本驻香港领事馆外高举抗议标语。(法新社) 日本政府宣布由星期四(8月24日)起启动核废水排海,香港特区政府同日起禁止来自日本10个都县的水产品进口,并表示一旦其他都县的水产品出现明显问题,不排除会扩大禁令……}

\entryitemWithDescription{美商务部长下周访华 中国驻美大使:希望美中相向而行}{https://www.zaobao.com/news/china/story20230823-1426545}{美国商务部长雷蒙多下周将访问中国,是今年6月以来第四位访华的美国内阁级高官。白宫国家安全顾问沙利文表示,不期望雷蒙多之行会取得根本改观的重大成果。 在雷蒙多即将访问北京和上海之前,于美东时间星期二(8月22日)在华盛顿同中国驻美国大使谢锋见面。美国商务部表示:``他们进行了富有成效的讨论……}

\entryitemWithDescription{中国就核废水排海召见日本驻华大使 声称将采取必要措施}{https://www.zaobao.com/news/china/story20230823-1426528}{中国外交部副部长孙卫东星期二(8月22日)召见日本驻华大使垂秀夫,就日本启动福岛核废水排海提出严正交涉,并称若日本一意孤行,中国将采取必要措施。 据中国外交部网站发布的消息,孙卫东在召见垂秀夫时,指责日本``公然向包括中国在内的周边国家和国际社会转嫁核污染风险、将一己私利凌驾于地区和世界各国民众长远福祉之上'',``极其自私自利,极不负责任'',中方表示严重关切、强烈反对……}

\entryitemWithDescription{专家:台湾民众党``成也虹安,败也虹安''}{https://www.zaobao.com/news/china/story20230823-1426519}{台湾大学医学教授出身的柯文哲(左)和拥有机械工程博士学位的高虹安,是民众党两大人气明星。如今高虹安在贪污罪名下被起诉,力挺她的柯文哲也面对民调下滑的压力。(自由时报) 台湾民众党籍新竹市长高虹安上周被控诈领助理费,若罪名成立可能下台。民众党主席、总统参选人柯文哲力挺高虹安,支持她捍卫清白;此事发酵后被指是导致柯文哲民调下滑,被国民党总统参选人侯友宜追上的转折点之一……}

\entryitemWithDescription{美国以``强制同化''西藏儿童为由限制部分中国官员签证}{https://www.zaobao.com/news/china/story20230823-1426514}{美国指一些中国官员涉嫌将100万名西藏儿童送入寄宿学校,进行``强制同化'',宣布将对这些官员实施签证限制。对此,中国外交部表示,``强制同化''纯属子虚乌有,呼吁美国立即撤销错误决定。 据法新社报道,美国国务卿布林肯星期二(8月22日)在一份声明中,指中国政府的``强制性政策''旨在消除西藏年轻一代中独特的语言、文化和宗教传统……}

\entryitemWithDescription{哈佛中国慈善趋势报告:房地产行业2020年捐赠者数量数额双双领先}{https://www.zaobao.com/news/china/story20230823-1426472}{在中国多家房地产企业深陷债务困境之际,美国哈佛大学新出炉的一份报告发现,房地产行业是2020年中国慈善捐赠者数量最多的行业,捐赠数额同样遥遥领先。 哈佛大学肯尼迪政府学院拉贾瓦利基金会亚洲研究所(Rajawali Foundation Institute for Asia)本月发布这份报告,报告对2020年中国的上千笔捐赠进行分析……}

\entryitemWithDescription{【视频】直击洗钱``福建帮''成员户籍地——茶都安溪}{https://www.zaobao.com/news/china/story20230823-1426209}{一排残破低矮的砖房旁,矗立着崭新的高层别墅;坑洼不平的乡间小路上,接连驶过宝马、马赛地、玛莎拉蒂等豪车。这样割裂的场景,在福建安溪并不罕见。 这个泉州市下辖的小县城原本以乌龙茶``铁观音''发源地而闻名,但近年来更为人熟知的是另一个别称:诈骗之乡。中国民间还流传一句顺口溜:``十个安溪九个骗,还有一个在锻炼(学习)。'' 本月轰动新加坡的10亿元洗钱案,10名涉案人中至少一半来自安溪……}

\entryitemWithDescription{杨丹旭:全民反谍进行时}{https://www.zaobao.com/news/china/story20230823-1426226}{中国国家安全部星期一(8月21日)又高调曝光了一起牵涉美国中央情报局的间谍案。 据官方披露的案情,一名``80后''中国国家部委干部郝某在日本留学期间因为办理赴美签证,结识了美国驻日本使馆官员,被策反后成为美国中情局间谍。 案情显示,郝某在对方的要求下,回国进入``核心要害单位''工作,还与美方签署了间谍协议,接受考核和培训,并在境内多次同中情局人员秘密接触,提供情报、收取报酬……}

\entryitemWithDescription{港教育局要求新聘教师须通过国安法测试}{https://www.zaobao.com/news/china/story20230822-1426213}{从今年9月起,香港所有学校的新聘教师,须在基本法和香港国安法测试中取得及格成绩。 据香港中通社报道,香港教育局星期一(8月21日)向全港学校发出通告,提醒包括私立学校在内的所有学校,聘任教学及非教学人员的注意事项,以及有关加强把关工作的措施。 根据通告,须在香港基本法及香港国安法测试取得及格成绩的教师,包括公营学校、直接资助计划学校及参加幼稚园教育计划的幼稚园的新聘教师,含新入职教师及转校教师……}

\entryitemWithDescription{乘摩托艇投奔韩国 男子被指是中国的维权分子}{https://www.zaobao.com/news/china/story20230822-1426211}{一名被指是维权分子的中国朝鲜族男子乘坐摩托艇从山东省偷渡到韩国仁川后被捕。图为韩国海岸警卫队工作人员在仁川检查摩托艇。(法新社) 日前从中国乘坐摩托艇投奔韩国被捕男子,是35岁的中国朝鲜族维权人士权平。 韩国海岸警卫队星期天(8月20日)通报,他们逮捕了一名试图从中国乘坐摩托艇进入韩国的中国公民……}

\entryitemWithDescription{北京斥日排放核废水``极端自私'' 香港禁进口十都县水产}{https://www.zaobao.com/news/china/story20230822-1426206}{香港星期四(8月24日)禁日本十都县水产品入口,有香港日式餐厅业者预计,餐厅生意可能因此暴跌七成。(路透社) (东京/北京/首尔/香港/台北综合讯)在邻国的反对声中,日本仍决定从星期四(8月24日)起,将处理过的福岛核电站放射性污水排入太平洋,预计将在30年内持续排放。 日本宣布有关决定后,中国外交部发言人汪文斌指责日本政府此举是``公然向全世界转嫁核污染风险,将一己私利凌驾于全人类长远福祉之上……}

\entryitemWithDescription{迹象显示平壤逐步开放边境 朝鲜首架民航客机抵达北京}{https://www.zaobao.com/news/china/story20230822-1426183}{高丽航空一架客机星期二(8月22日)抵达北京首都机场,这是三年来朝鲜首架降落在北京的国际商业航班。(法新社) (北京综合电)中朝在冠病疫情过后恢复往来于两国之间的国际航班,首架朝鲜客机星期二(8月22日)飞抵北京……}

\entryitemWithDescription{中美洲议会投票决定 中国大陆取代台湾成为常驻观察员}{https://www.zaobao.com/news/china/story20230822-1426180}{中美洲议会(Parlacen)在当地时间星期一投票决定,取消台湾20多年来担任常驻观察员的地位,转而由中国大陆取代。台湾政府提出抗议,并宣布退出中美洲议会。 综合路透社、法新社、新华社的报道,中美洲议会星期一(8月21日)在尼加拉瓜首都马那瓜召开会议,批准中国大陆成为常驻观察员……}

\entryitemWithDescription{蔡侯郭今天齐赴金门纪念``八二三'' 郭台铭会否宣布选总统受关注}{https://www.zaobao.com/news/china/story20230822-1426176}{鸿海创办人郭台铭(中)星期二(8月22日)在主流民意大联盟活动上发表《金门和平倡议》。他会否在星期三(23日)正式宣布参选总统,备受各方关注。(彭博社) 台湾总统蔡英文、国民党总统参选人侯友宜,以及酝酿参选总统的鸿海创办人郭台铭,星期三(8月23日)将在金门参与``八二三''炮战纪念活动。郭台铭会否正式宣布参选总统,他与侯友宜如何互动,备受关注……}

\entryitemWithDescription{【东谈西论】中国房地产陷入长痛僵局}{https://www.zaobao.com/news/china/story20230822-1426131}{中国大型房企为何会陷入债务危机?(中新社) 中国房地产巨头碧桂园,在本月初未能按时支付 2250万美元的债息,震惊四座,陷入违约边缘。此外,恒大集团也在美国申请破产保护,负债率是 2.4万亿人民币。 中国大型房地产企业接二连三陷入债务危机,这到底是怎么回事? 谁会成为下个倒下的房地产公司? 中国的房地产是否在酝酿一场中国版的 ``雷曼兄弟'' 危机……}

\entryitemWithDescription{戴庆成:香港中文大学再陷政治风波}{https://www.zaobao.com/news/china/story20230822-1425862}{趁着8月暑假还没过去,我上星期回了一趟母校,和当年的老师和同窗聚旧。香港中文大学依山而建,我们相约在山上的新亚书院餐厅吃午饭。 当天我在山下准时上了校巴,岂料校巴行驶了一两分钟就停止前进。原来,大学的马路是单线路,前面有一辆不熟悉交通情况的私人轿车突然迎面驶来,结果两个方向的汽车对望着都动弹不了,后面的车子也被迫停了下来。六七名学校保安在场手忙脚乱地维持秩序,但一时间也不知道怎样处理……}