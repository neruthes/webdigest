\entryitemWithDescription{港民建联主席李慧琼 当选全国人大常委}{https://www.zaobao.com/news/china/story20230312-1371612}{新任港区全国人大代表、香港立法会最大党民建联主席李慧琼星期六(3月11日)当选第14届中国全国人大常委。 综合《明报》《星岛日报》和网媒``香港01''等报道,全国人大会议星期六上午投票选出159名全国人大常委。 香港唯一的候选人李慧琼在2939张有效选票中,以2901张赞成、35张反对和三张弃权当选,接替民建联元老谭耀宗,成为香港历来最年轻的全国人大常委……}

\entryitemWithDescription{台湾称不与大陆进行金援竞赛}{https://www.zaobao.com/news/china/story20230312-1371613}{西太平洋岛国密克罗尼西亚联邦据传希望台湾提供5000万美元(6745万新元)援助,换取密台建交。有台湾立委质疑这是``凯子外交''。台湾外交部回应说,外交援助绝非``凯子外交'',也不会与中国大陆进行金援竞赛……}

\entryitemWithDescription{拒国安处要求交资料 港支联会三前成员判监四个半月}{https://www.zaobao.com/news/china/story20230312-1371614}{曾在香港主办天安门事件纪念活动多年的支联会,三名前成员拒绝应香港警方国安处要求,交出该会运作资料,星期六(3月11日)被判监四个半月。 综合香港01、星岛网、《明报》和法新社报道,被判监的三名成员分别是支联会前副主席邹幸彤、前常委邓岳君与徐汉光。香港国安法指定法官、西九龙法院裁判官罗德泉在判词中说,维护国家安全至关重要,判刑须反映维护国家安全的决心……}

\entryitemWithDescription{两岸关系趋缓春暖未必花开}{https://www.zaobao.com/news/china/story20230312-1371615}{新任中国全国政协主席王沪宁(面向镜头,中)3月9日参加全国人大会议台湾代表团的审议时指出,大陆进一步掌握了实现完全统一的战略主动。(新华社) 台湾工商、民间团体近期接连赴陆访问。图为2月24日台湾高雄里长暨基层社团参访团赴北京参访。(中新社)  台湾军队1月11日在高雄基地举行战备演习。(路透社)  中国大陆国台办主任宋涛2月中旬曾表示欢迎台湾绿营人士到大陆谈谈,走走看看……}

\entryitemWithDescription{新闻人间:马祖士兵 没肉吃}{https://www.zaobao.com/news/china/story20230311-1371308}{驻守马祖列岛的台湾军人在沙滩上划出喊饿文字,引发外界质疑台军``还没打仗,就已断粮''。 初到马祖服役的台湾男子,在西莒岛坤坵沙滩上划下几行字:``不要罐头''\,``马防部(马祖防卫指挥部)伙房,主菜是白饭''\,``马防部伙房都没肉,肚子饿都吃罐头泡面''。在当地经营旅游业的网红陈竑任将留字拍照上网,引起各界哗然……}

\entryitemWithDescription{温伟中:台湾能否当和战与统独棋手?}{https://www.zaobao.com/news/china/story20230311-1371310}{今年台海局势的两颗潜在炸弹------美国众议院议长麦卡锡访台以及台湾总统蔡英文访美,忽然调整为蔡英文过境美国时与麦卡锡在加州会面,初步拆除了可能导致军事冲突的引信。 有消息称麦卡锡被蔡政府说服,同意不在此时访台激怒北京,避免重演类似去年8月前议长佩洛西访台引发围台军演,甚至升级为台湾国防部长邱国正评估的严峻情况:解放军战斗机和军舰逼近12海里的台湾空域和海域,迫使台军必须采取武力反击……}

\entryitemWithDescription{台恢复两岸部分直航点 行政院长否认为蔡英文过境美国灭火}{https://www.zaobao.com/news/china/story20230311-1371311}{台湾自3月10日起,恢复开放深圳、广州、南京、重庆、杭州、福州、青岛、武汉、宁波、郑州10个两岸空运客运直航点,并预告将开放大陆13个包机航点,以及在清明节前恢复``小三通''客运中转。 (香港中通社) 台湾自星期五(3月10日)起恢复两岸部分直航航点,行政院长陈建仁否认这是为总统蔡英文4月过境美国先行灭火,强调是考量两岸疫情缓解,且清明节将届,民众有返台祭祖需求所作的安排……}

\entryitemWithDescription{传蔡英文月底过境美国 会见众议院议长麦卡锡}{https://www.zaobao.com/news/china/story20230311-1371312}{据报台湾总统蔡英文将在本月底过境美国期间,在纽约发表演讲,并在加利福尼亚州会见美国众议院议长麦卡锡。 英国《金融时报》星期四(3月9日)引述知情人士报道,蔡英文3月30日将在纽约一场由美国智库哈德逊研究所(Hudson Institute)赞助的活动上发表演讲,并获该研究所颁发``全球领导力奖''(global leadership award)……}

\entryitemWithDescription{网红``潮汕阿秋''抵港脚踏车被偷}{https://www.zaobao.com/news/china/story20230311-1371313}{中国大陆网红阿秋骑脚踏车到过32个省份,没想到脚踏车竟在香港被盗 。(互联网) 中国大陆一名分享骑行中国游历的短视频博主,抵达香港后脚踏车被盗。他声称,这是自己骑行逾三年以来第一次脚踏车被偷。香港警方星期五(3月10日)回应说,正追缉一名黑衣裤涉案男子。 据《星岛日报》报道,在抖音平台拥有超百万粉丝的骑行博主``潮汕阿秋'',以骑脚踏车游历中国、分享不同地区景色和风土人情知名……}

\entryitemWithDescription{台湾今起恢复10个两岸定期航班航点 清明节前开放``小三通''客运中转}{https://www.zaobao.com/news/china/story20230310-1371023}{台湾的大陆委员会发言人詹志宏星期四(3月9日)宣布,在现行五个航点外,自3月10日起恢复包括广州等10个两岸定期航班航点,另有13个航点得申请包机,清明节前会开放``小三通''客运中转……}

\entryitemWithDescription{提出必要时依法停工停业停市停课 西安流感应急预案引发再封控担忧}{https://www.zaobao.com/news/china/story20230310-1371024}{西安2021年12月23日封控期间的空荡荡街头。(新华社) 西安发布的流感高发期应急预案引发国内舆论反弹,学者认为,中国不会再出现大规模封控,政府在出台类似应急政策时应更注意措辞,以免引发社会恐慌情绪。 中国陕西省会西安在流感高发期发布流感大流行应急预案,提出要适时采取临时社会面管控措施,必要时依法采取停工、停业、停市、停课。预案随即引发国内舆论反弹,不少网民担忧疫情封控卷土重来……}

\entryitemWithDescription{特稿:应勇料接任最高检检察长}{https://www.zaobao.com/news/china/story20230310-1371025}{应勇预计将在星期六接替张军,出任最高检察院检察长。(互联网) 张军3月7日在全国人大会议上作最高检察院工作报告。(新华社) 中国两高(最高人民法院、最高人民检察院)本周末即将迎来高层人事调整。去年9月被任命为正部级的最高检副检察长、由此重返政坛一线的应勇,预计将在星期六(3月11日)举行的全国人大全体会议上,接替张军出任最高检检察长,跻身``党和国家领导人''行列……}

\entryitemWithDescription{本报和通商中国联办``解读两会''论坛 三专家剖析中国政经环境与走向}{https://www.zaobao.com/news/china/story20230310-1371026}{南洋理工大学经济学教授陈光炎(左起);华侨银行大中华地区研究主管谢栋铭;吉宝企业中国首席代表兼吉宝资本总裁吴来顺,将解读中国经济趋势。(档案照片) 走过三年多的冠病疫情,中国政府将今年的经济增长目标设在5%左右,属于市场预期的低端。保守的经济增速目标意味着什么?新一届中国政府领导班子上任后,中国的经济环境、政策会有怎样的变化……}

\entryitemWithDescription{蔡英文过境拟会麦卡锡 美国务院:程序符合一中政策}{https://www.zaobao.com/news/china/story20230310-1371027}{台湾总统蔡英文计划在4月初过境美国期间,与美国众议院议长麦卡锡会面。(路透社档案照) 针对台湾总统蔡英文计划在4月初过境美国期间与美国众议院议长麦卡锡会面,美国国务院称,台湾高阶官员过境美国行之有年,所有程序都符合美国的一中政策。 根据美国国务院官网发出的文字实录,国务院发言人普莱斯星期三(3月8日)在媒体简报会上应询时做出上述表述……}