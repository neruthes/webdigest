\entryitemWithDescription{新闻人间:崔建春——积极面向港人的中国外交部驻港一把手}{https://www.zaobao.com/news/china/story20240803-4418499}{北京在2019年香港修例风波后,下定决心处理香港的深层次政治问题,过去几年陆续撤换一批驻香港的官员,最近的一项变动是在今年4月任命驻尼日利亚前大使崔建春为外交部驻港特派员。 崔建春今年60岁,中国外交部驻港特派员公署网页未见他的详细履历,仅提及他为工商管理硕士。不过据港媒报道,崔建春早年曾在广东大亚湾核电站实习,于国企任职超过20年,长期任职财务及审计部门,2010年担任国际合作开发部主任……}

\entryitemWithDescription{中国央行顾问吁给民众发钱 学者:官方担忧引发通胀冲击中低阶层}{https://www.zaobao.com/news/china/story20240802-4418679}{中国知名经济学家、央行货币政策委员黄益平发文,呼吁官方改变重投资、轻消费的政策理念,采取支持消费增长的财政手段,``直接给老百姓发钱''。 黄益平也是北京大学国家发展研究院院长。北大国发院微信公众号星期四(8月1日)发出这篇题为《中国特色的宏观经济政策框架》的文章说,这是黄益平在长安讲坛第415期受邀发表演讲时的内容……}

\entryitemWithDescription{因被要求签署``一中原则''承诺书 台湾派往澳门官员无法获签证}{https://www.zaobao.com/news/china/story20240802-4417226}{(台北综合讯)台湾官方称,由于澳门特区政府要求派驻当地办事处的人员签署遵守``一中原则''承诺书才能获得签证,因此很难向办事处派遣员工。 综合台湾《自由时报》、彭博社等报道,台湾陆委会副主委梁文杰星期四(8月1日)在例行记者会上说,台湾驻澳门办事处一名外交部派驻人员7月底任满返台,澳门特区政府要求接任者签署遵守``一中原则''承诺书才发放签证……}

\entryitemWithDescription{台湾自主生产装甲车抗弹钢板出现裂痕}{https://www.zaobao.com/news/china/story20240802-4417156}{台湾自主生产的装甲车车体以高张力钢板焊接而成,具有抗弹功能。图为台湾军方今年6月底在南投向媒体展示台产装甲车。(法新社) (台北讯)台湾军方自主生产的云豹八轮装甲车据报抗弹钢板出现裂纹,台湾国防部证实这一消息,并已完成更换与修复,防弹功能正常……}

\entryitemWithDescription{韩咏红:中国军力发展速度赶超}{https://www.zaobao.com/news/china/story20240802-4413182}{美国国会参众两院军事委员会跨党派领导层组成的国防战略委员会,星期一(7月29日)发布的最新报告,毫不掩饰地表达了对美国面对蜡烛多头烧等处境的焦虑,以及美军面对的挑战。这份报告在中国国内没有引起多少关注,但中国民众或许会觉得,这是美国难得的、对中国实力的肯定评价……}

\entryitemWithDescription{卓荣泰:未来的核能 有任何的可能都可以讨论}{https://www.zaobao.com/news/china/story20240801-4412688}{台湾行政院长卓荣泰接受日媒专访时据称透露,为因应人工智慧(AI)和半导体的用电需求,台湾可能在2030年后重新考虑使用核电。图为卓荣泰5月31日在立法院进行任内首次施政报告。(法新社) 台湾行政院长卓荣泰接受日媒专访时据称透露,为因应人工智慧(AI)和半导体的用电需求,台湾可能在2030年后重新考虑使用核电。但行政院否认此说,强调现阶段重点要如期完成兴建中的电厂,未来将持续开发多元绿能……}

\entryitemWithDescription{中国公布首个基本医保参保长效机制}{https://www.zaobao.com/news/china/story20240801-4412603}{中国医保退保潮引发社会关注之际,国务院星期四公布首个基本医保参保长效机制,在放宽放开参保户籍限制、扩大医保共济范围的同时,也削弱断保人员享受的福利。 受访学者认为,此次医保改革有利于提升服务的合理性与居民的获得感,可以部分消除社会对医保的质疑,有利于促进社会公平正义。 据新华社报道,中国国务院星期四(8月1日)公布《关于健全基本医疗保险参保长效机制的指导意见》(简称《意见》)……}

\entryitemWithDescription{林毅夫:中国不会受误导放弃产业政策}{https://www.zaobao.com/news/china/story20240801-4412714}{知名经济学家、北京大学国家发展研究院名誉院长林毅夫认为,中国不会像当年的日本那样在美国误导下放弃产业政策,中国经济将继续发展。(中新社) (香港讯)知名经济学家、北京大学国家发展研究院名誉院长林毅夫认为,中国不会像当年的日本那样在美国误导下放弃产业政策,中国经济将继续发展。 《南华早报》星期四(8月1日)刊登对林毅夫的专访……}

\entryitemWithDescription{康福德高与小马智行合作 在中国试运营自动驾驶德士}{https://www.zaobao.com/news/china/story20240801-4411591}{卫生部长王乙康星期四(8月1日)在广州中新智慧园乘坐小马智行全无人自动驾驶车。(林煇智摄) 随着我国人口快速老龄化,更多企业在布局构建未来的能力。新加坡德士公司康福德高宣布与中国自动驾驶公司小马智行合作,推动自动驾驶德士的大规模商业运营,应对未来可能出现的运力短缺。 两家企业星期三(7月31日)在广州举行的新粤合作理事会上签署了合作备忘录,建立战略合作伙伴关系……}

\entryitemWithDescription{陆客赴港澳购物免税额提高 扩大至全部口岸}{https://www.zaobao.com/news/china/story20240801-4411930}{中国大陆提高自港澳入境的居民旅客携带行李物品免税额度的措施,从8月1日起推广至全部入境口岸,以进一步刺激香港和澳门的零售业。图为香港铜锣湾罗素街。(彭博社) 中国大陆提高自港澳入境的居民旅客携带行李物品免税额度的措施,从8月1日起推广至全部入境口岸,以进一步刺激香港和澳门的零售业。不过,受访学者认为,近来大陆民众消费意欲低,措施的作用不大……}

\entryitemWithDescription{中国禁止民用无人机出口用于军事目的}{https://www.zaobao.com/news/china/story20240801-4411900}{(北京综合讯)中国将从9月1日起调整无人机出口管制措施,禁止未列入管制的民用无人机出口用于军事目的、恐怖主义活动等。 据中国商务部官网消息,中国商务部、海关总署、中央军委装备发展部星期三(7月31日)联合发布关于优化调整无人机出口管制措施的公告,宣布调整特定无人驾驶航空飞行器及其相关物项的出口管制措施,从9月1日起实施……}

\entryitemWithDescription{大陆重申金门翻船案为恶性撞船 台陆委会:外界定调不必要}{https://www.zaobao.com/news/china/story20240801-4411363}{金门翻船事件的善后协商7月30日暂告一段落,中国大陆国务院台湾事务办公室隔天重申此案为``恶性撞船事件'',并强调善后沟通与两岸协商谈判无关。图为一艘船只在台湾金门渔港附近航行。(法新社档案照片) 金门翻船事件的善后协商星期二(7月30日)暂告一段落,中国大陆国务院台湾事务办公室隔天重申此案为``恶性撞船事件'',并强调善后沟通与两岸协商谈判无关……}

\entryitemWithDescription{中国制裁参与撰写涉藏法案美众议员麦戈文}{https://www.zaobao.com/news/china/story20240801-4410950}{中国宣布对参与撰写涉藏法案的美国众议员吉姆·麦戈文实施制裁,麦戈文回应称制裁是``荣誉勋章''。图为麦戈文6月3日在美国众议院规则委员会听证会上发言。(法新社) (北京综合讯)中国宣布对参与撰写涉藏法案的美国众议员吉姆·麦戈文实施制裁,麦戈文回应称制裁是``荣誉勋章''。 中国外交部星期三(7月31日)在官网公告,麦戈文近年来频繁采取``干涉中国内政、损害中国主权安全发展利益''的言行……}

\entryitemWithDescription{加拿大军舰航经台海 中国批滋扰搅局}{https://www.zaobao.com/news/china/story20240801-4410560}{(渥太华/北京/台北综合讯)中国人民解放军庆祝八一建军节前夕,加拿大军舰星期三(7月31日)过航台湾海峡,引发中国不满并批评加拿大此举是滋扰搅局。 综合路透社和法新社报道,加拿大国防部长布莱尔说,护卫舰蒙特利尔号最近在台湾海峡进行例行过航,并重申加拿大对自由、开放和包容的印太地区的承诺。他指出,正如加拿大2022年宣布的印太战略计划所述,加拿大正在增加加拿大皇家海军在印太地区的存在感……}

\entryitemWithDescription{沈泽玮:在奥运金牌与那英冠军之外}{https://www.zaobao.com/news/china/story20240801-4406745}{湖南卫视芒果台的《歌手2024》总决赛上周五(7月26日)晚上播出后,巴黎奥运会开幕式接着登场,中国和全球各国一起进入奥运会时间。 人脉强大的中国女歌手那英打败表现稳定的老外歌手夺冠,引发黑箱作业争议。数天后,中国男子体操队在奥运会团体赛因选手两次掉杆与金牌失之交臂,也引发各种议论。 前者是以比赛形式包装成的娱乐节目,歌手们都领了出场费,当然得配合剧本演出……}