\entryitemWithDescription{韩咏红:特朗普关税从``解放日''走到``妥协日''了?}{https://www.zaobao.com/news/china/story20250425-6240846}{美国总统特朗普在4月2日宣布对等关税政策,短短20天后,特朗普的高姿态已难以为继。特朗普在4月9日就已调整过一次战术,暂缓对其他国家课征对等关税,集中火力单挑中国;而今,特朗普恐怕又要眨眼了,对中国商品课征的145\%关税也可能显著下调。 特朗普近日罕见地对中国伸出橄榄枝,表示考虑大幅度降低对华关税,显露出急于与中国达成协议。但中国偏是表现得不着急……}

\entryitemWithDescription{学者:中国以2+2机制拉拢东南亚国家抗衡美国}{https://www.zaobao.com/news/china/story20250424-6240690}{受访学者分析,在中美关系因贸易战而全面恶化的大背景下,中国正以``外交、国防2+2对话机制''为新抓手,积极拉拢亚细安成员国,以抗衡美国在本区域的影响力,预计中国接下来将推动与更多区域国家建立2+2机制。 中国和印度尼西亚星期一(4月21日)在北京举行外长、防长对话机制下的首次部长级会议,两国在2023年就建立2+2对话机制。中国官方称,这是中国在全球建立的首个部长级2+2机制……}

\entryitemWithDescription{中埃空军首次联训 学者:中国初步具备向中东快速战略投送能力}{https://www.zaobao.com/news/china/story20250424-6240357}{中国空军上周派出多架战斗机、预警机、运输机与空中加油机前往埃及,进行两军首次联训。这是中国空军首次以完整作战体系进行跨洲机动。 受访学者认为,这是中国空军现代化进程的里程碑,有助于发展长途奔袭的技战术;这也标志着中国初步具备向中东进行快速战略投送的能力。 据中国国防部消息,中国与埃及两国空军于4月中旬至5月上旬,在埃及空军基地组织代号为``文明之鹰-2025''的联合训练……}

\entryitemWithDescription{台湾收紧民众申领大陆证照规定 持大陆定居证也将被撤销台湾身份}{https://www.zaobao.com/news/china/story20250424-6239789}{(台北综合讯)台湾再度收紧民众申领中国大陆证照的规定,持有大陆定居证的台湾民众也触犯相关法令,将丧失台湾身份。 综合《旺报》与《联合报》报道,台湾行政院公报显示,陆委会近日对《两岸条例》当中规定发布解释函令称,为确保两岸人员单一身份制度,两岸条例规定台湾人民不得在大陆地区``设有户籍'',或领用大陆地区护照,否则将丧失台湾身份……}

\entryitemWithDescription{神舟二十号成功发射 航天员将开展中国首次涡虫空间再生实验}{https://www.zaobao.com/news/china/story20250424-6239950}{(北京综合讯)中国星期四(4月24日)成功发射神舟二十号载人飞船。三名航天员将在空间站驻留约六个月,并开展中国首次涡虫空间再生实验。 综合新华社、央视新闻和《中国青年报》报道,搭载神舟二十号的长征二号F遥二十运载火箭,星期四下午5时17分在甘肃酒泉卫星发射中心点火发射。约10分钟后,神舟二十号与火箭成功分离,进入预定轨道……}

\entryitemWithDescription{台大罢免升温 朝野对抗加剧}{https://www.zaobao.com/news/china/story20250424-6240062}{台湾两大在野党将携手举办大型造势活动,抗议政府利用司法整肃异己,进行大罢免。兼任民进党主席的总统赖清德则公开肯定公民团体罢免在野党立委的行动影片,检调也持续搜索在野国民党宜兰县党部,朝野对抗情势正逐步升温……}

\entryitemWithDescription{美国国会罕见动用传唤权 调查中国三电信巨头}{https://www.zaobao.com/news/china/story20250424-6239456}{(纽约路透电)美国众议院中国问题特别委员会星期三(4月23日)罕见地动用传唤权,调查中国三大电信公司涉嫌支持中国军方和政府的行为。 据路透社报道,中国移动、中国电信和中国联通三家中国电信巨头收到该委员会的传唤通知,须回答他们是否可通过在美开展的云服务和互联网业务获取美国数据的问题。 美国两党议员持续对被指由中国主导的网络攻击事件表示担忧……}

\entryitemWithDescription{浙江一小学门外发生汽车冲撞人群事件}{https://www.zaobao.com/news/china/story20250424-6238514}{(香港综合讯)中国再发生校园伤人事件,浙江金华一所小学门外星期二(4月22日)放学时有汽车冲撞人群,伤亡人数不明。官方在事发两日后仍未发布通报,中国社媒上的相关信息均被删除。 综合《明报》《南华早报》和网媒``香港01''报道,这起事件发生在金华市苏孟乡中心小学门口,时间是星期二傍晚5时45分左右,正值放学时间。附近商户披露,有多名学生被撞……}

\entryitemWithDescription{学者:关税战伤敌一千自损八百 中美终将找到共处之道}{https://www.zaobao.com/news/china/story20250424-6238821}{美国总统特朗普祭出对等关税将届满一个月,如今传出可能降低对华关税,定居美国的资深华人学者赵全胜指出,中美领导阶层有一批人长期相信对方马上要垮台,但关税战让双方意识到,这无疑是``伤敌一千,自损八百'',因此迟早会找到共处之道。 特朗普4月2日宣布全面实施对等关税,随后在9日紧急暂缓90天,让各国争取与美国谈判的时间与空间,唯独对中国不断加码;北京也不甘示弱,提出相应反制……}

\entryitemWithDescription{打击电诈犯罪扩至纵深地带 缅甸向中国移交920余名嫌犯}{https://www.zaobao.com/news/china/story20250424-6238307}{(北京综合讯)中国与缅甸加大打击电信网络诈骗犯罪合作力度,两国最近一个月的合作执法行动,已从缅北地区扩大到当阳、勐休等缅甸纵深地带。 中国公安部官网星期三(4月23日)通报,缅甸执法部门近日将在当地抓获的920余名中国籍涉诈犯罪嫌疑人,通过云南西双版纳打洛口岸全部移交中国警方。 通报称,缅北电诈犯罪集团遭受重创,但部分涉诈人员为逃避打击,向当阳、勐休等纵深地带转移藏匿,继续实施跨境电诈……}

\entryitemWithDescription{沈泽玮:魔幻山城的中国式现代化演绎}{https://www.zaobao.com/news/china/story20250424-6234868}{时隔12年因工作再访重庆,春夏交替之际迎来烈日当空,与当年冬季旅游行走于山城迷雾的印象截然不同。 5000架无人机灯光秀,既展现科技力量也打造视觉盛宴;科企创始人讲述营商环境不断优化;村委会主任手捧涪陵榨菜分享东方酱腌菜``走出去''成果;火车司机传递通关速度如何带动互联互通跨境贸易;公安局交巡警总队科研处综合科科长讲述数字化赋能超大城市治理;社区党委书记分享中国特色``民主村''的前身今世……}

\entryitemWithDescription{中国网购平台将全面取消``仅退款''}{https://www.zaobao.com/news/china/story20250423-6234242}{在中国官方整治``内卷式竞争''的背景下,中国电商平台将全面取消``仅退款''选项。受访学者认为,此举有助行业回归良性竞争,是对电商市场的一次纠偏。 据《北京商报》报道,拼多多、淘宝、抖音、快手、京东等多个中国电商平台,星期二(4月22日)修改有关``仅退款''的相关条款,消费者申请``退款不退货'',将由商家自主处理……}

\entryitemWithDescription{3月人民币跨境收付占比刷新历史纪录}{https://www.zaobao.com/news/china/story20250423-6233857}{(华盛顿彭博电)随着美元的全球吸引力减弱、中美贸易紧张局势上升,3月中国投资者和贸易公司在国际结算中对人民币的使用大幅增加,创下历史纪录。 彭博社基于中国国家外汇管理局星期二(4月22日)公布的数据计算,3月中国大陆境内个人和机构的跨境业务中,人民币使用占比达 54.3\%,总额7249亿美元(9502亿新元)……}

\entryitemWithDescription{中国首次选拔训练外国航天员 巴基斯坦载荷专家有望进中国空间站短期工作}{https://www.zaobao.com/news/china/story20250423-6233118}{(北京综合讯)中国和巴基斯坦正在巴国开展航天员选拔工作,未来几年内将有一名巴基斯坦航天员有机会进入中国空间站执行短期飞行任务。 综合新华社与《中国航天报》消息,中国载人航天工程新闻发言人林西强星期三(23日)宣布,中巴正在巴基斯坦开展航天员的选拔工作,先在巴基斯坦进行初选,然后在中国进行复选和定选,将选拔两名巴基斯坦航天员到中国参加训练……}

\entryitemWithDescription{李家超:上任三年仅休两天假 为港付出``值得''}{https://www.zaobao.com/news/china/story20250423-6234174}{香港特首李家超近日透露,自己上任即将三年,至今只放过两天假,但他认为担任此职是使命,应把所有时间都投入工作,每当看到市民脸上有笑容,他就觉得一切的付出都``值得''。 李家超平日对个人生活极为低调,他星期二(4月22日)接受香港``now''电视台专访时罕有地谈及个人生活,忆述上任至今只试过两天完整休假,一天用来配眼镜,另一天是陪伴家人到医院……}

\entryitemWithDescription{中国多省公布首季GDP:浙江高技术产业增长快 广东低于全国平均水平}{https://www.zaobao.com/news/china/story20250423-6233417}{(杭州/广州综合讯)中国多个省市陆续公布今年一季度经济数据,坐拥``杭州六小龙''等科创企业的浙江地区生产总值(GDP)同比增长6\%,而传统外贸大省广东同比增长4.1\%,低于全国5.4\%的增幅。 浙江省星期一(4月21日)公布一季度GDP为2万2300亿元(人民币,下同,约4000亿新元),同比增长6%,增幅高于浙江年初设定的5.5\%左右的全年增长目标……}

\entryitemWithDescription{赖清德委任前副总统陈建仁 代表台湾出席教宗方济各丧礼}{https://www.zaobao.com/news/china/story20250423-6232538}{(台北综合讯)台湾外交部与梵蒂冈全力交涉,希望促成台湾总统赖清德参加已故教宗方济各的丧礼弥撒。但在双方会商后,改由台湾前副总统陈建仁代表出席。 台湾外交部星期三(4月23日)在官网发表声明称,经过台梵双方讨论,由于陈健仁曾六度晋见教宗方济各,并获教宗亲自颁授教廷宗座科学院院士证章,与教宗关系深厚,决定由陈建仁担任总统特使,代表台湾出席教宗丧礼……}

\entryitemWithDescription{中国限制韩企向美国国防承包商出口含重稀土产品}{https://www.zaobao.com/news/china/story20250423-6232023}{(伦敦 / 首尔综合讯)中国政府告诫韩国企业不要向美国国防承包商出口含有重稀土矿物的产品,违者可能会受其制裁。这项出口管制或将冲击高度依赖出口的韩国经济。 《韩国经济日报》引述当地生产电力变压器的企业消息人士称,目前已有至少两家生产商接获上述通知。其他收到通知的企业包括生产电池、显示器、电动汽车、航空航天和医疗设备的企业,这些产品都都以稀土为关键原料……}

\entryitemWithDescription{中国扩大试点范围 推动服务业开放提速}{https://www.zaobao.com/news/china/story20250423-6228260}{在中美贸易战升级引发全球不安的背景下,中国政府宣布加快推进服务业扩大开放试点,新增九个试点城市与155项试点任务,并突出对接高标准国际经贸规则。 受访学者认为,中国此举履行了继续推动全球化的承诺,有利于树立与美国贸易保护主义不同的形象,参与相关国际规则制定,并吸引更多外资进入其服务业……}

\entryitemWithDescription{台湾蓝白两党将携手上凯道向赖清德发出怒吼}{https://www.zaobao.com/news/china/story20250422-6228009}{台湾在野的国民党(蓝)主席朱立伦与民众党(白)主席黄国昌,星期二(4月22日)举行在野领袖峰会,双方都认为要抑制总统赖清德的专制独裁,但对于如何``倒赖''尚无具体共识。 朱立伦号召群众星期六(4月26日)到总统府前的凯达格兰大道``战独裁'',也邀黄国昌与会。黄国昌当面应允,称将代表民众党参加这场人民的主场,也期盼民众一起站出来向赖清德发出怒吼……}