\entryitemWithDescription{中国央行意外宣布降准 释放1万亿元人民币流动性提振经济}{https://www.zaobao.com/news/china/story20240124-1464179}{中国央行行长潘功胜星期三(1月24日)预告从2月5日起降低存款准备金率0.5个百分点。消息拉动陆港股市,恒生指数当日涨3.56\%,创两个月来最大单日涨幅。(香港中通社) 担忧股市开年持续低迷打击市场信心,中国官方密集释放积极消息,中国央行星期三(1月24日)意外宣布将在2月5日下调银行存款准备金率0.5个百分点,向市场提供流动性1万亿元(人民币,下同,约1892亿新元)……}

\entryitemWithDescription{日本经济代表团访华 将向中国高层提出撤销日本水产品禁令问题}{https://www.zaobao.com/news/china/story20240124-1464159}{(东京讯)日本经济界一个近200人的代表团本周访问中国,这是时隔逾四年后日本经济代表团首次访华,他们将向中国领导层提出撤销对日本水产品暂停进口的措施,并力争改善两国关系。 据日本共同社报道,这个由日中经济协会、经济团体联合会(经团联)、日本商工会议所等日本经济界团体组成的日中经济协会联合访华代表团共有约180人……}

\entryitemWithDescription{傅聪:欧盟对华电动车反补贴调查``不公平''}{https://www.zaobao.com/news/china/story20240124-1464142}{中国驻欧盟大使傅聪接受彭博社访问时,罕见地谈到去年7月被免职的中国前外长秦刚。(中国驻欧盟使团官网) 中国总理李强结束欧洲之行后,中国驻欧盟大使傅聪批评欧盟对中国电动车的反补贴调查``不公平'',并含蓄警告称,可能会有更多欧洲产品面对贸易调查。 受访学者分析,中国今年正加大力度寻求稳定中欧关系,以应对美国总统选举带来的不确定因素和连带冲击……}

\entryitemWithDescription{学者:赖清德两岸政策尚未调整 欧美访台潮徒增台海引爆点}{https://www.zaobao.com/news/china/story20240124-1464130}{台湾候任总统、现任副总统赖清德(右二)星期三(1月24日)在总统府接见美国前国务次卿克拉奇(Keith Krach,右三)率领的``美台商业协会访问团''。(台湾总统府提供) 台湾大选后迎来欧美政商人士访台潮,包括美国和立陶宛的国会议员团,以及美国前国务次卿克拉奇率领的``美台商业协会访问团''。受访学者研判,在台湾候任总统赖清德尚未厘清台独党纲和两岸路线前,海外议员访台或徒增台海局势的引爆点……}

\entryitemWithDescription{先进工作者举报局长后被判刑?官方:已成立调查组}{https://www.zaobao.com/news/china/story20240124-1464117}{(北京/砀山综合讯)安徽砀山县农机局一名退休职工七年前举报该局局长后被判刑,被举报的局长去年则按正常程序办理了退休。此事经媒体曝光后引发关注,当地星期三(1月24日)通报,已成立联合调查组调查此事。 据中国网媒赤焰新闻报道,今年64岁的李平是砀山县农机局退休职工,他在服役结束后,即转业到砀山县农机局,工作期间连年被评为``先进个人''\,``先进工作者''等荣誉称号……}

\entryitemWithDescription{美台智库调查:未来五年 大陆对台隔离封锁可能性比入侵大}{https://www.zaobao.com/news/china/story20240124-1464107}{1月13日,台湾总统选举当天,一艘中国大陆旅游船在台湾海峡航行后,返回大陆距离台湾本岛最近的福建平潭岛。(法新社) 美台智库合作的最新调查显示,多数专家认为,未来五年内,北京透过隔离或封锁手段对台施压的可能性,远高于侵略。参与执行调查的台湾学者进一步研判,在中美都有意缓和关系的情况下,北京尚不至于有过度强烈的举动……}

\entryitemWithDescription{台湾前友邦瑙鲁和中国大陆恢复外交关系}{https://www.zaobao.com/news/china/story20240124-1464091}{图为中国外长王毅(右)1月24日在北京同瑙鲁外长安格明(左)签署《中华人民共和国和瑙鲁共和国关于恢复外交关系的联合公报》后握手。(法新社) (北京/台北综合讯)太平洋岛国瑙鲁上周宣布与台湾断交,星期三(1月24日)和中国大陆恢复外交关系……}

\entryitemWithDescription{杨丹旭:中国股市保卫战}{https://www.zaobao.com/news/china/story20240124-1463953}{``白天炒A股,晚上看国足,酸爽。''中国股市星期一(1月23日)跌破2800点,社交媒体上一片哀嚎,一些股民在朋友圈这样自嘲。 中国股市的悲观情绪去年底以来持续发酵,短短一个多月A股先后失守3000点和2900点关口,上周四(1月18日)盘中一度跌破2800点,在多只挂牌基金托底下,当天闭市前总算惊险守住2800点。 不过,该来的还是逃不过……}

\entryitemWithDescription{赖清德亲赴立法院 勉励新科绿委多念书少应酬}{https://www.zaobao.com/news/china/story20240123-1463947}{台湾候任总统赖清德星期二(1月23日)以民进党主席身份,出席闭门的共识营。他勉励民进党新科立委,要多念书、倾听民意,少应酬。(赖清德脸书) 台湾候任总统赖清德勉励民进党新科立委,要多念书、倾听民意,少应酬。他透露在立法院12年从不应酬,连唱卡拉OK都没去。 赖清德以民进党主席身份出席闭门的共识营,据台媒引述不具名的党政人士形容,当选总统到立法会出席共识营属``史上首次''……}

\entryitemWithDescription{北京传考虑设2万亿元人民币平准基金救市 港资深投资人:未必发挥作用}{https://www.zaobao.com/news/china/story20240123-1463936}{北京可能出招救市的消息传出后,中国A股星期二(1月23日)扭转近来持续下滑的趋势,上证综合指数上升14点。香港股市更一度大涨逾550点,收市报15353点,上升392点,升幅2.6\%。图为香港交易所。(路透社) 中国大陆和香港股市开年后跌不停,市场信心低迷。有消息指中国政府正考虑设立2万亿元(人民币,下同,3815亿新元)的平准基金稳定股市……}

\entryitemWithDescription{传图瓦卢大选后断交 台湾外交部:图政府澄清并坚定与台邦谊}{https://www.zaobao.com/news/china/story20240123-1463919}{太平洋岛国图瓦卢驻台湾大使帕埃纽称,图瓦卢可能与台湾断交。图为拍摄于2004年2月19日的图瓦卢首都和主要岛屿纳富提岛。(法新社档案照) (台北/巴黎/华盛顿综合讯)太平洋岛国图瓦卢驻台湾大使帕埃纽称,图瓦卢可能与台湾断交,台湾外交部星期二(1月23日)称,图瓦卢政府第一时间向台湾澄清,此言属帕埃纽个人评论,非政府立场……}

\entryitemWithDescription{中国发布反恐白皮书 称反恐工作依法保障人权}{https://www.zaobao.com/news/china/story20240123-1463910}{(北京综合电)中国国务院新闻办公室发布的反恐白皮书强调,中国在反恐怖主义工作中依法保障人权。 据新华社报道,星期二(1月23日)发布的《中国的反恐怖主义法律制度体系与实践》白皮书称,中国长期面临着恐怖主义的现实威胁,对此逐步探索出``符合本国实际的反恐怖主义法治道路'',有力维护了国家安全、公共安全和人民生命财产安全……}

\entryitemWithDescription{中国将扩大转基因大豆玉米种植}{https://www.zaobao.com/news/china/story20240123-1463894}{(北京综合讯)中国已经完成转基因玉米和大豆的产业化应用试点工作,下一步将规范有序扩大其应用范围。 综合中国网和澎湃新闻报道,中国农业农村部种植业管理司司长潘文博星期二(1月23日)在国新办记者会上说,该部门审定通过了部分转基因玉米大豆品种,并向26家企业发放了转基因玉米大豆种子生产经营许可证。 同时,潘文博明确说明,这些品种实际种植区域还要符合中国国家生物育种产业化有关安排……}

\entryitemWithDescription{中国经济步日本后尘?分析:举国体制能避免}{https://www.zaobao.com/news/china/story20240123-1463887}{过去一年,中国面临通货紧缩、房地产泡沫化、失业率攀升,经济复苏不如预期。图为北京民众1月15日在建筑工地的起重机附近穿过十字路口。(路透社) 中国经济去年(2023年)遭遇通货紧缩与房产泡沫化等风险,是否重演日本失落的30年也成为话题。分析认为,中国高度集权的举国体制,能有效调动资源因应风险,接下来会持续推动产业重心转移到新能源等高科技领域,不至于步上日本老路……}

\entryitemWithDescription{中国留学生涉嫌威胁倡议民主人士在美受审 若定罪可判处五年监禁}{https://www.zaobao.com/news/china/story20240123-1463884}{在美国波士顿伯克利音乐学院留学的吴啸雷,2022年12月14日在马萨诸塞州切尔西的联邦调查局办公室接受讯问。(路透社) (华盛顿综合讯)一名中国留学生当地时间星期一(1月22日)在美国受审,他被指控恐吓一名倡议民主的旅美中国活动人士,并威胁要向中国执法部门举报她。 综合路透社与美国之音报道,在波士顿伯克利音乐学院留学的吴啸雷,星期一在波士顿面审……}

\entryitemWithDescription{联合国人权专家吁港府撤销对黎智英的所有指控}{https://www.zaobao.com/news/china/story20240123-1463868}{壹传媒集团创办人黎智英(右)在2020年涉嫌违反《香港国安法》被捕。图为黎智英2021年2月1日被警方押进香港终审法院外的一辆惩教车。(法新社) (香港综合电)联合国人权专家呼吁香港特区政府撤销对壹传媒集团创办人黎智英的所有指控,并立即释放他。黎智英目前正在香港接受串谋勾结外国或境外势力等三项罪名的审讯……}

\entryitemWithDescription{境外``猎艳''反被围猎 中国国安部公布一起被策反案件}{https://www.zaobao.com/news/china/story20240123-1463865}{(北京讯)中国国家安全部披露,一名国营企业关键岗位人员出国时,在情色场所被境外间谍情报机关拍下照片以此要挟,被迫加入间谍组织。 中国国家安全部微信公号星期二(1月23日)以《猎艳?猎物!》为题发文,公布上述案件。化名``李四''的某国企关键岗位人员在出国考察过程中,被导游邀请至当地色情娱乐场所消费……}

\entryitemWithDescription{戴庆成:港股何时见曙光?}{https://www.zaobao.com/news/china/story20240123-1463691}{恒生指数在2024年的跌势似乎尚未停止,在1月22日进一步下跌2.3\%,是2022年10月以来首次低于15000点收市。图为摄于去年11月30日的香港夜景。(路透社) 冠病疫情结束后,香港人北上消费逐渐成为潮流。我过去一星期在香港出席数场饭局,发现大部分餐厅的生意都很冷清,最夸张的一次竟然有三分之二座位没有客人,餐馆的员工比食客还要多……}

\entryitemWithDescription{蓝营化解内斗 ``花莲王''态度逆转挺``韩江配''}{https://www.zaobao.com/news/china/story20240122-1463706}{花莲县立委傅崐萁(右一)星期一(1月22日)在国民党新科立委座谈会中,表态支持``韩江配'',并与国民党主席朱立伦(左二)、不分区立委韩国瑜(右二)和江启臣(左一)牵手,营造团结气势。(自由时报) 台湾在野的国民党星期一(1月22日)戏剧性化解内斗危机。一天前宣布选立法院领袖的``花莲王''傅崐萁,态度逆转力挺``韩江配''(韩国瑜与江启臣)选立法院正副院长……}

\entryitemWithDescription{``AI教父''访华展示对中国市场重视}{https://www.zaobao.com/news/china/story20240122-1463704}{黄仁勋近日在上海参加英伟达中国区年会时,罕见脱下他最标志的黑色皮衣,改穿传统的东北大花袄,手持二人转手帕,同员工齐跳秧歌舞。(互联网) 美国政府去年10月收紧对中国出售尖端半导体产品的限制后,晶片巨头英伟达(Nvidia)联合创办人兼总裁黄仁勋时隔四年再次访华,上周视察北京、上海、深圳的办公室,并与员工欢庆春节……}

\entryitemWithDescription{云南镇雄山体滑坡埋18户 八人遇难47人失联}{https://www.zaobao.com/news/china/story20240122-1463687}{中国云南省镇雄县凉水村1月22日发生山体滑坡。截至发稿,救援人员救出四名被困人员,找到八具遇难者遗体。(新华社) (北京/武汉综合讯)中国云南省昭通市镇雄县星期一(1月22日)发生山体滑坡,灾难造成18户房屋被掩埋,共47人失联,目前已致八人遇难。 综合中国央视新闻与《新京报》消息,镇雄县塘房镇凉水村星期一清晨六时许发生山体滑坡。当时不少村民还在睡梦中,瞬间被从天而降的泥石掩埋……}

\entryitemWithDescription{港警推出``护港漂''计划 防止在港大陆人士被骗}{https://www.zaobao.com/news/china/story20240122-1463681}{为防止在港升学或工作的中国大陆居民被骗,香港警察推出``护港漂''计划,将防骗信息推广至线上港漂群体。(彭博社) (香港综合讯)香港警察为了防止在港升学或工作的中国大陆居民被骗,推出了``护港漂''计划,包括与网红合作,把防骗信息推广至线上港漂群体。 综合无线新闻、《星岛日报》、网媒``香港01''等港媒报道,过去一年不少从大陆赴港升学或工作的港漂,纷纷坠入电话或网上诈骗案……}