\entryitemWithDescription{杨丹旭:又是信心问题}{https://www.zaobao.com/news/china/story20231227-1458351}{中国市场在低迷的情绪下来到2023年尾,年关之前又遭遇重磅一击。 上周五(12月22日),国家新闻出版总署发布一份名为《网络游戏管理办法(草案征求意见稿)》的文件,就网络游戏管理公开向全社会征求意见。 草拟的新规中,限制游戏过度使用和高额消费等条款引发关注,其中包括网络游戏不得设置每日登录、首次充值、连续充值等诱导性奖励;不得以炒作、拍卖等形式提供或纵容虚拟道具高价交易行为等……}

\entryitemWithDescription{挺郭屏东议长周典论被收押 蓝营批执政党用司法威胁郭台铭不可支持``侯康配''}{https://www.zaobao.com/news/china/story20231226-1458348}{台湾鸿海集团创办人郭台铭已退出总统大选,但在野国民党屏东县议长周典论因涉嫌交付资金协助郭台铭参选连署,星期二(12月26日)被地方法庭裁定声押禁见。 周典论近日公开为国民党正副总统候选人侯友宜和赵少康助选。在周典论被收押后,赵少康直指检方在大选投票前不到三周之际采取此行动,是为了压制周在屏东为``侯康配''辅选,更重要是威胁郭台铭莫出面挺``侯康配''……}

\entryitemWithDescription{台总统候选人第二场政见会 蓝批绿爱财色成桃花党 绿批蓝紧抱一中和黑金}{https://www.zaobao.com/news/china/story20231226-1458345}{台湾总统候选人第二场电视政见会,蓝绿加码互攻。绿营的民进党总统候选人赖清德狠批国民党紧抱``一个中国''和黑金体质不改,蓝营的国民党总统候选人侯友宜则炮轰民进党充满贪腐和财色案件,形容对方是``桃花党''。 台湾将在明年1月13日举行总统与立委选举,星期二(12月26日)进行第二场电视政见会,最后一场政见会将在星期四(28日)举行……}

\entryitemWithDescription{台湾反渗透法介选首案 泛蓝协会成员被起诉}{https://www.zaobao.com/news/china/story20231226-1458302}{(高雄综合讯)台湾中华泛蓝协会成员郑志成涉嫌组织130多人赴中国大陆旅游,返台后支持特定候选人或政党,被台湾检方起诉。这是台湾首起以反渗透法起诉的介选案。 综合联合报、三立新闻网等报道,在高雄的桥头地方检察署星期一(12月25日)以违反反渗透法、总统副总统选举罢免法等起诉郑志成……}

\entryitemWithDescription{澜湄合作领导人会议 李强吁严打网赌电诈}{https://www.zaobao.com/news/china/story20231226-1458260}{(北京 / 内比都综合讯)中国加强大湄公河地区安全合作。中国国务院总理李强在与柬埔寨、老挝等国领导人会谈中,强调要严厉打击网络赌博、电信诈骗等犯罪活动。 新华社报道,李强星期一(12月25日)以视频方式出席澜沧江---湄公河合作第四次领导人会议,并与缅甸领导人敏昂莱共同主持这次会议。柬埔寨首相洪玛奈、老挝总理宋赛、泰国总理赛塔、越南总理范明政也出席了会议……}

\entryitemWithDescription{戴庆成:如果选香港年度汉字}{https://www.zaobao.com/news/china/story20231226-1458133}{临近年底,许多地方一如以往各自选出了一个代表2023年度的汉字,如新加坡选``诈''、马来西亚选``贵''、日本选``税''、台湾选``缺''。但在香港,一直有举行这个活动的当地最大政党民建联至今仍不见有任何动静。 翻看资料,民建联自从2013年开始就通过网络举办``香港年度汉字评选''活动,希望通过活动让市民观察和反思社会现况……}

\entryitemWithDescription{韩咏红:台湾大选进入紧张微妙时刻}{https://www.zaobao.com/news/china/story20231225-1458065}{作者说,台湾的政治文化十分感性,当事人眉宇间的些许变化、表情与肢体语言都可能产生效应,北京若有大动作也会直接冲击选情。图为台北原名中正广场的自由广场。(彭博社) 在震惊两岸的君悦酒店``蓝白合分手大戏''落幕一个月后,2024年台湾大选的新局面已稳定下来,此前被认定几乎无可能当选的蓝营,出现了一丝微弱的胜选希望。 台湾大选是否会上演最后一个月急逆转的奇迹……}

\entryitemWithDescription{圣诞节数日前 王沪宁强调全面从严治教}{https://www.zaobao.com/news/china/story20231224-1458063}{1949年中华人民共和国成立后,对中国大陆的基督教和天主教组织体系推行``三自(自治、自养、自传)爱国运动''政策,反对中国教会受西方教会的管辖及领导。图为中国天主教徒12月24日参加北京西什库天主教堂举行的圣诞弥撒……}

\entryitemWithDescription{台湾大选倒数三周 蓝绿白连续两天造势拼场}{https://www.zaobao.com/news/china/story20231224-1458062}{民众党总统候选人柯文哲星期六(12月23日)在高雄冈山造势,狭长街道挤满支持者。主办单位声称现场来了4万人,直播同时在线网民6万人。(民众党提供) 台湾大选倒数三周,蓝绿白三党连续两天造势拼场……}

\entryitemWithDescription{中国国航C919涨价 价格仍略低于波音与空客}{https://www.zaobao.com/news/china/story20231224-1458043}{5月28日在上海虹桥国际机场停机坪拍摄的C919飞机。当天,由C919大型客机执飞的东方航空MU9191航班平稳降落在北京首都国际机场,标志着该机型完成首个商业航班飞行,正式进入民航市场。(新华社) (北京综合讯)中国国际航空公司发布最新公告,披露其购买中国国产大型客机C919的计划和价格,其中每架C919报价已上调至1.08亿美元(1.43亿新元),但该价格仍低于波音和空客的同机型飞机……}

\entryitemWithDescription{香港公民党明年3月正式解散 曾被外界称为``大状党''}{https://www.zaobao.com/news/china/story20231224-1458029}{香港公民党成立于2006年3月,核心成员多为法律界人士,曾被外界称为``大状党''(大律师党)。(公民党脸书) (香港综合讯)香港公民党的清盘程序已步入尾声,这个成立17年的泛民主派政党将于明年3月正式解散。 综合《星岛日报》和``香港01''报道,公民党临时执委会主席梁家杰星期六(12月23日)说,临时执委会将于星期天(24日)总辞,按法律公民党将在明年3月正式解散……}

\entryitemWithDescription{中国前财长楼继伟:明年预算赤字最好维持在3.8\%左右}{https://www.zaobao.com/news/china/story20231224-1458021}{中国前财政部长楼继伟建议2024年赤字率保持在大约3.8\%。图为在中国国庆假期前夕,上海主要购物区南京路步行街涌现大批人潮。(路透社) (深圳综合讯)全球财富管理论坛理事长、中国前财政部长楼继伟近日认为,中国货币政策仍有较大降息和降准空间,建议2024年赤字率保持在大约3.8\%……}

\entryitemWithDescription{大陆空飘气球本月第五度逾越台湾海峡中线}{https://www.zaobao.com/news/china/story20231224-1458011}{(台北综合讯)台湾军方又发现一枚中国大陆空飘气球逾越海峡中线,是本月第五度侦获大陆空飘气球。 台湾国防部星期天(12月24日)在官网通报,星期六(23日)上午10时20分侦获一枚大陆空飘气球,逾越台湾海峡中线,位于基隆西北方约97海里(180公里),高度约2万英尺(6096公尺)以下,并在1小时15分钟后消失……}

\entryitemWithDescription{早知:台湾立法委员选制是怎么回事?}{https://www.zaobao.com/news/china/story20231224-1458010}{台湾将在2024年1月13日举行总统与立法委员选举,其中将选出113席立法委员,分别为区域立委73席、不分区立委34席,以及山地及平地原住民立委各三席。图为台湾执政党民进党的支持者12月21日在宜兰参加竞选活动。(法新社) 台湾立委怎么选,任期有多长? 立法院是台湾最高立法机构,立法委员是由人民直选、代表人民行使立法权的民意代表……}

\entryitemWithDescription{特稿:台湾立法院选情激烈 席位分布成影响两岸关键}{https://www.zaobao.com/news/china/story20231224-1458006}{民进党目前的立委选情暂时落后国民党。图为一名民进党支持者12月21日在宜兰参加该党造势活动,高举印有总统候选人赖清德和立委候选人陈俊宇人像的应援旗。(法新社) 台湾选战进入倒数三周,各主要政党无不力拼立法院席位。受访专家分析,若绿营继续执政,很可能面临朝小野大局面,这将对两岸等敏感政策产生牵制;倘若政党轮替,且蓝营成为国会最大党,则可能推进两岸谈判进程……}

\entryitemWithDescription{中国未成年网民近2亿人 网络游戏短视频日益普及}{https://www.zaobao.com/news/china/story20231224-1458002}{报告显示,中国未成年网民近2亿人。图为北京一名孩童11月22日放学回家时,坐在电动摩多车的后座上。(法新社) (北京综合讯)中国未成年网民近2亿人,网络游戏、短视频在这个群体中日益普及……}

\entryitemWithDescription{中国特稿:中国青年下乡寻找``市外''桃源}{https://www.zaobao.com/news/china/story20231224-1457662}{在城市打拼10年的刘斌放弃科技公司的工作,今年3月来到中国文创网红村---福建屏南县墘头村,成为``新村民'',开启向往已久的乡村生活。(受访者提供) 面对高压力的大城市生活,以及经济放缓下的就业压力,新一代中国青年掀起``下乡''风潮,重返农村追求慢节奏生活。但大部分农村地区仍然凋敝``空心'',缺乏就业机会,年轻人回农村是无奈的选择,还是更好的出路……}

\entryitemWithDescription{夏宝龙:把``爱国者治港''制度优势转化为治理效能}{https://www.zaobao.com/news/china/story20231223-1457931}{中国国务院港澳办主管的全国港澳研究会星期五(12月22日)在北京举行成立十周年庆祝大会,港澳办主任夏宝龙出席并致辞。(中新社) (香港/北京综合讯)中国国务院港澳办主任夏宝龙指出,如何把爱国者治港的制度优势转化为治理效能,是一国两制实践行稳致远必须要回答好的时代课题……}

\entryitemWithDescription{中国考研人数首次下滑 疫后考研潮结束}{https://www.zaobao.com/news/china/story20231223-1457922}{中国黑龙江省哈尔滨市考生在哈尔滨理工大学考点排队进入考场。(新华社) 中国硕士研究生招生考试初试星期六(12月23日)开考,今年考试报名人数为438万人,是在连续八年递增后首次下降。 2020至2022年三年疫情期间,中国高校生通过考研延迟就业的情况尤为突出。受访学者指出,这显示疫情后中国考研潮已结束,低迷的就业形势戳破了多年来形成的学历泡沫……}

\entryitemWithDescription{清华铊投毒案受害者朱令去世 终年50岁}{https://www.zaobao.com/news/china/story20231223-1457918}{朱令是清华大学1992年级化学系学生,1994年11月因铊中毒身体出现异常。(互联网) (北京综合讯)1994年怀疑被人蓄意投毒的清华大学校友朱令星期五(12月22日)在北京去世,终年50岁。这起历经了近30年的投毒案至今还没找到凶手。 清华大学星期六(12月23日)在官方微博发布朱令去世的消息,并写道,朱令多年来与病痛顽强抗争,``我们对朱令的去世表示深切哀悼,向朱令的家人致以诚挚慰问……}

\entryitemWithDescription{中国万亿增发国债已安排逾8000亿元}{https://www.zaobao.com/news/china/story20231223-1457917}{(北京综合讯)中国增发国债第二批项目清单下达,以防洪和救灾项目为主,涉资金逾5600亿元(人民币,下同,1040亿新元)。 中国10月宣布将在2023年四季度增发1万亿元特别国债,以支持经济、缓解地方政府压力。据中国国家发改委微信公众号星期六(12月23日)发布的消息,当前两批项目下达,已涉及安排增发国债金额逾8000亿元……}