\entryitemWithDescription{韩咏红:李强政府的第一份经济成绩单}{https://www.zaobao.com/news/china/story20240119-1462997}{在一系列不利因素铸成的阴霾气氛下,中国2023年经济增速达到5.2\%,实现了官方年初定下的5\%左右目标,可见中央政府去年第三季以来紧急推出的各项救市措施,勉强撑住了大局。 这是李强接任国务院总理后,中国第一次公布全年经济增长数据。虽然他的任期去年3月才正式开始,但也算是他的第一份年度成绩单。在当前环境下,堪称不易。不过,细看经济数据下的具体分项,可见形势依然严峻、不容乐观……}

\entryitemWithDescription{中菲紧张关系因台湾问题升级 两国同意继续管控涉海分歧}{https://www.zaobao.com/news/china/story20240118-1462986}{菲律宾总统小马可斯本周祝贺赖清德当选台湾总统,引发北京和马尼拉新一轮交锋后,两国外交官员召开会议,就南中国海和台湾问题交流,同意继续通过友好协商妥善管控涉海矛盾分歧,处理好海上紧急事态,特别是管控好有主权争议的仁爱礁(菲律宾称阿云津礁)现地局势……}

\entryitemWithDescription{中国大陆在台湾选后筑外交领域``一中''防火墙}{https://www.zaobao.com/news/china/story20240118-1462983}{中国总理李强1月17日同爱尔兰总理瓦拉德卡举行会谈,北京在随后发出的通稿中称,瓦拉德卡表示爱尔兰``一贯恪守一个中国原则''。(新华社) 台湾大选过后,中国大陆在外交领域加筑围堵,北京在领导人会晤、高层出访的多份官方文告中,都引述外国领导人坚持``一个中国原则''等表态。受访学者分析,北京的做法除了要在国际场域促进反台独外,同时也通过外交宣示说服国内鹰派舆论……}

\entryitemWithDescription{韩国瑜搭江启臣参选立法院正副院长}{https://www.zaobao.com/news/china/story20240118-1462965}{2020年选过总统的国民党新科立委韩国瑜(右)星期四(1月18日)宣布与国民党前主席江启臣(左)搭档,竞选立法院正副院长。(取自韩国瑜脸书) 台湾在野的国民党候任立委韩国瑜星期四(1月18日)宣布搭档前主席江启臣,参选立法院正副院长;行政院长陈建仁则率内阁总辞,获总统蔡英文慰留,将主持看守内阁至5月20日候任总统赖清德就职为止。 台湾1月13日完成总统与立委选举之后,立法院和内阁人事布局备受关注……}

\entryitemWithDescription{张家界``湘西赶尸''视频引争议 当地文旅局长否认出镜拍摄}{https://www.zaobao.com/news/china/story20240118-1462957}{网传视频显示,多名身穿黑色长袍的``僵尸''额头上贴着黄符,在``赶尸人''的手势挥动下,跟随音乐律动。(互联网) (成都/北京综合讯)中国各地近期纷纷使出浑身解数宣传旅游业,湖南张家界景区一段``湘西赶尸''的宣传视频引发争议,网传当地文旅局长亲自出镜拍摄,但相关局长迅速否认……}

\entryitemWithDescription{美前官员:拜登会强硬对待赖清德刺激两岸局势言行 展示管控美中关系能力}{https://www.zaobao.com/news/china/story20240118-1462955}{美国前国家安全委员会总统特别助理和东亚事务高级主任韦德宁(左上)、北京大学国际关系学院教授贾庆国(左下),以及南京大学国际关系学院执行院长朱锋(右下)星期三(1月17日)在香港全球化中心举办的线上论坛讨论台湾选举后的两岸局势走向。论坛由香港全球化中心主席邱震海(右上)主持……}

\entryitemWithDescription{英航将中国航线的普通话空服员增倍}{https://www.zaobao.com/news/china/story20240118-1462951}{英国航空将把中国航线上讲普通话的空服人员数量增加一倍。(路透社) (北京综合讯)英国航空(British Airways)将把往返中国航班讲普通话的空服人员数量增加一倍,以寻求在这个全球第二大航空市场扩展业务。 综合路透社和彭博社报道,英航首席客户官拉明(Calum Laming)说,英国航空预计飞往北京和上海的航班,到今年7月时将增加50名会讲普通话的空服员……}

\entryitemWithDescription{与瑙鲁断交后 台湾三太平洋友邦发声挺台}{https://www.zaobao.com/news/china/story20240118-1462947}{(台北/悉尼综合讯)太平洋岛国瑙鲁与台湾断交后,台湾在太平洋地区剩下的三个邦交国马绍尔群岛、帕劳和图瓦卢,都表达对台湾的支持。 马绍尔群岛外交与贸易部星期四(1月18日)在脸书发文形容,该国与台湾的友谊坚如磐石,``珍视与中华民国(台湾)的牢固关系''。 图瓦卢总理纳塔诺致函台湾驻图瓦卢代表处,重申图瓦卢为台湾坚定好友,将持续支持台湾民主发展……}

\entryitemWithDescription{香港艺发局停止资助舞台剧奖颁奖礼 指上届活动``损害声誉''}{https://www.zaobao.com/news/china/story20240118-1462937}{香港艺发局称,去年香港舞台剧奖颁奖礼后收到不少意见,指颁奖礼内容与安排有不妥之处。 图为第三十一届香港舞台剧奖颁奖礼合照。 (香港戏剧协会脸书) (香港综合讯)香港艺术发展局曾连续24年资助举行香港舞台剧奖颁奖礼,但艺发局指去年颁奖礼内容对局方声誉造成损害,决定扣减最后一期资助。去年这一颁奖活动上,主办单位曾邀请屡被港府批评的政治漫画家等人担任颁奖嘉宾……}

\entryitemWithDescription{王纬温:``双海''加紧联动下印太博弈再升级?}{https://www.zaobao.com/news/china/story20240118-1462762}{美国和菲律宾去年4月大幅升级军事关系,并在南中国海和台海拉开``双海''联动帷幕后,菲律宾总统小马可斯的对华政策不仅日益强硬,与中国在南中国海摩擦也不断加剧。 小马可斯本周更罕见向赖清德发文,祝贺这位中国大陆眼里的``顽固台独工作者''当选台湾总统。此举在为马尼拉与台北关系加温的同时,也加码挑动北京敏感神经,还进一步坐实双海联动对华围堵的看法,中西方在印太的博弈大局今年预计将比去年更紧张……}

\entryitemWithDescription{解放军台湾大选后首次执行联合战备警巡 18架次军机在台海活动}{https://www.zaobao.com/news/china/story20240118-1462763}{(台北/北京综合讯)台湾国防部星期三晚间公告,中国大陆出动空军战机18架次在台湾空域配合海军执行联合战备警巡。这是自今年台湾大选结束后,解放军在台海执行的大规模军事活动。 台湾国防部星期三(1月17日)晚间9时50分发出新闻稿,指出台军自当天晚上7时50时起,陆续侦获解放军苏恺-30、运-8等各型主、辅战机计18架次出海活动……}

\entryitemWithDescription{澳洲驻美大使陆克文:中国需减少强调意识形态 以恢复企业信心}{https://www.zaobao.com/news/china/story20240117-1462760}{中国国际经济交流中心副理事长朱民星期三出席世界经济论坛时表示,中国经济有韧性,但挑战也是巨大的。(路透社) 澳大利亚驻美国大使陆克文说,中国当局须减少对意识形态的强调,并重新强调市场经济中正常的企业效益问题,才能从根本上恢复民营企业信心……}

\entryitemWithDescription{赖清德退出民进党最大派系新潮流 学者:为朝野政党创造和谐协商的弹性空间}{https://www.zaobao.com/news/china/story20240117-1462745}{台湾民进党主席赖清德在1月13日举行的大选中,以逾558万得票数胜出,当选新一任总统。图为他在胜选后的国际记者会上发表讲话。(路透社) 台湾民进党主席、候任总统赖清德星期三(1月17日)宣布退出党内最大派系新潮流,理由是为了客观推动国政、团结领导党内。受访学者分析,赖清德此举或可为太偏重自身派系解套,也为朝野政党创造议题合作、和谐协商的弹性空间……}

\entryitemWithDescription{中驻澳大使:中国与太平洋岛国建交非军事安全战略}{https://www.zaobao.com/news/china/story20240117-1462733}{中国驻澳大利亚大使肖千1月17日在中澳媒体新年酒会上致辞说,中澳双方在相互尊重基础上妥善解决各自合理贸易关切。自澳洲总理阿尔巴尼斯上任以来,中澳关系有回暖迹象。(中新社) (堪培拉综合讯)中国驻澳大利亚大使肖千在太平洋岛国瑙鲁与中国建交后说,中国与该地区国家建立政治关系非关军事安全战略,澳大利亚无须为此感到焦虑……}

\entryitemWithDescription{台湾前外长:邦交不至于归零 但两岸不缓和挑战必增}{https://www.zaobao.com/news/china/story20240117-1462730}{对于台湾与诺鲁断交,台湾前外长林永乐1月17日表示, 他不认为台湾的邦交国会归零,但只要两岸关系不缓和下来,台湾新任政府想拓展国际空间势必会面临更多挑战。(缪宗翰摄) 南太平洋岛国瑙鲁决定与台湾断交后,台湾前外长林永乐星期三(1月17日)出席座谈会时指出,台湾的邦交国不至于归零;不过两岸关系不缓和,台湾国际空间挑战势必增加,而目前也还看不出新任总统当选人赖清德能与北京对话的基础……}

\entryitemWithDescription{中欧贸易紧张升级 李强重申希望欧盟审慎出台限制性经贸政策}{https://www.zaobao.com/news/china/story20240117-1462726}{中国总理李强(右)1月16日在瑞士达沃斯出席世界经济论坛2024年年会期间,与欧盟委员会主席冯德莱恩(左)会面。(新华社) (达沃斯/北京综合讯)在中欧贸易紧张局势升级之际,中国总理李强重申希望欧方在经贸问题上公正、透明对待中国企业,审慎出台限制性经贸政策。 据新华社报道,李强于当地时间星期二(1月16日)在瑞士达沃斯出席世界经济论坛2024年年会期间,与欧盟委员会主席冯德莱恩会面……}

\entryitemWithDescription{中国人口连续两年萎缩 长期风险上升冲击市场信心}{https://www.zaobao.com/news/china/story20240117-1462722}{2023年是中国全面解除冠病防控后首年,死亡人数同比增至1960年以来最高水平,新生儿人口则降至1949年以来最低水平。图为1月12日在北京公园里散步的祖孙老少。(路透社) 中国人口总数连续两年下降,且降幅进一步扩大,凸显出这个传统人口大国面对的长期风险持续上升,进一步加剧外界对中国经济前景的担忧……}

\entryitemWithDescription{清华大学再否认80\%毕业生出国留学}{https://www.zaobao.com/news/china/story20240117-1462713}{(北京综合讯)曾因毕业生出国留学问题被推上舆论风口浪尖的清华大学再次发文,否认80\%毕业生出国。 清华大学星期三(1月17日)在微信公众号发文,介绍毕业生去向……}

\entryitemWithDescription{大陆国台办回应``武统''声音:愿尽最大努力争取和平统一}{https://www.zaobao.com/news/china/story20240117-1462696}{中国大陆福建省平潭岛的一名男子,1月15日眺望台湾海峡波涛汹涌的海面。平潭岛是大陆距离台湾本岛最近的地方。(法新社) (北京/台湾综合讯)赖清德当选为新一任台湾总统后,中国大陆网络再次出现``武统''声音。中国国务院台湾事务办公室星期三称,这反映了台湾各界担忧民进党可能加剧台海形势紧张动荡……}

\entryitemWithDescription{杨丹旭:孙任泽的非正常死亡}{https://www.zaobao.com/news/china/story20240117-1462534}{刚看到同事传来的这篇报道,我以为我穿越了。这样的事情在十几二十年前时有耳闻,没想到还在上演。 中国媒体《财新》上周六发表一篇题为《嫌疑人孙任泽之死》的特稿。这名当时不满31岁的年轻人在2018年3月因为涉嫌寻衅滋事,被新疆伊犁的公安机关刑事拘留。同年9月,孙任泽在看守所接受审讯时昏迷,辗转在多家医院接受救治后,最终在当年11月不治身亡……}

\entryitemWithDescription{AIT主席罗森伯格对瑙鲁与台断交表示``遗憾''}{https://www.zaobao.com/news/china/story20240116-1462531}{美国在台协会(AIT)主席罗森伯格(Laura Rosenberger,右)星期二(1月16日)由发言人邓艾德(Ed Dunn)陪同,在台北开记者会回应瑙鲁与台湾断交议题。(路透社) 美国在台协会(AIT)主席罗森伯格(Laura Rosenberger)星期二(1月16日)对南太平洋岛国瑙鲁决定与台湾断交表示``遗憾''(unfortunate)……}

\entryitemWithDescription{国民党评估拿下立院龙头可能性高 赵少康建议立院副院长让给民众党}{https://www.zaobao.com/news/china/story20240116-1462517}{台湾立法院龙头改选在即,以关键少数自居的在野民众党立委开出改革立法院四条件,但执政的民进党立院党团以立法院正、副院长是否适任为重,最大在野国民党也以尊重立院党团意见为由,均给民众党软钉子碰。 依立法院长选举办法,第一轮选举若未过半,可以有第二轮选举,故除非民众党八席立委集体支持民进党,否则作为第一大党的国民党不分区首席当选立委韩国瑜,笃定当选立法院长……}

\entryitemWithDescription{小马可斯罕见祝贺赖清德当选 学者:应审慎看待不过度解读}{https://www.zaobao.com/news/china/story20240116-1462512}{菲律宾总统小马可斯罕见发文祝贺赖清德当选台湾总统后,菲律宾外交部和总统府隔天相继重申其``一个中国''政策。不过,北京仍对小马可斯言论表达强烈不满,并召见菲律宾驻华大使提出严正交涉。 受访台湾学者认为,应审慎看待小马可斯此次发文,不应过度解读,菲律宾对台议题的政策大方针未变,也未见台菲关系有重大突破迹象。 台湾民进党主席赖清德星期六(1月13日)在总统选举中胜出……}