\entryitemWithDescription{中国学者指现状不可持续 暗示北京可能采取措施恢复仁爱礁无人居住状态}{https://www.zaobao.com/news/china/story20240711-4242041}{图为6月10日中国海警赤瓜舰执法员~,将布设在仁爱礁坐滩菲律宾军舰附近的渔网收回。(新华社(中国海警供图)) 中菲近期在仁爱礁频繁对峙,长期研究南中国海问题的中国学者认为,中国允许菲律宾给坐滩军舰提供人道补给、但阻止运输建筑材料的管控模式不可持续,暗示北京可能采取措施,恢复仁爱礁无人居住状态……}

\entryitemWithDescription{又有中国民企的火箭在任务中发生故障}{https://www.zaobao.com/news/china/story20240711-4241864}{(酒泉综合讯)中国又有一家民营企业研制的火箭在任务过程中发生故障。 据新华社报道,双曲线一号遥八民营商业运载火箭星期四(7月11日)上午在酒泉卫星发射中心点火升空,但火箭飞行异常,最终试验任务失利。 研制该火箭的星际荣耀在官方微信公号上发声明说,发射任务失利的具体原因,将于详细调查和审议后第一时间公布……}

\entryitemWithDescription{66架次大陆军机出现台海 次数今年最多}{https://www.zaobao.com/news/china/story20240711-4241412}{(台北综合讯)台湾国防部星期四公布侦获66架次中国大陆军机在台海周边活动,次数为今年新高。此前一天,台湾军方也侦测到大陆航母现身距离台湾本岛最南部不远的菲律宾北部海域。 台湾国防部星期四(7月11日)在官网通报,从星期三(10日)清晨6时至星期四清晨6时,共侦获66架次大陆军机和七艘次大陆军舰在台海周边活动,其中56架次军机飞越台海中线,进入台湾防空识别区北部、西南及东南空域……}

\entryitemWithDescription{郑文灿涉贪收押禁见 民进党开铡停权三年}{https://www.zaobao.com/news/china/story20240711-4241358}{桃园地方法院星期四(7月11日)下午三度重开羁押庭,郑文灿(中)抵达法院时,全程面色凝重、不发一语。(互联网) 台湾前桃园市长、已请辞台湾海峡交流基金会董事长的郑文灿涉贪两度获交保,在星期四(7月11日)第三度召开的羁押庭,被裁定收押禁见;郑文灿委任律师表示将依法抗告。执政的民进党方面也证实,郑文灿自星期五(7月12日)起停权三年……}

\entryitemWithDescription{中国向美遣返一名涉性侵儿童美籍红通逃犯}{https://www.zaobao.com/news/china/story20240711-4239775}{(北京/华盛顿综合讯)中国公安部通报,已向美国遣返一名涉嫌侵犯儿童的美籍红色通缉令逃犯。 新华社星期四(7月11日)引述中国公安部消息报道,应美国执法部门请求,中国警方星期三(7月10日)在上海浦东国际机场,将在美涉嫌性侵儿童的美籍红通逃犯斯科特,移交给美国国务院外交安全局押解回国。 这是继今年6月美国向中国遣返两名涉嫌严重刑事犯罪的逃犯后,中美执法部门又一次合作成果……}

\entryitemWithDescription{港警逐步采用大陆制手枪 取代现有外国制手枪}{https://www.zaobao.com/news/china/story20240711-4239556}{(香港综合讯)香港警方宣布逐步采用中国大陆制半自动手枪,取代现有外国制左轮手枪。 综合香港电台网、无线新闻网和网媒``香港01''报道,香港警方星期四(7月11日)在记者会上公布更换手枪计划,称为配合警队的长远行动需要,已购置两款大陆制九毫米半自动手枪,型号分别是CF98-A和CS/LP5……}

\entryitemWithDescription{澳门政府保安系统多个网站及手机应用遭海外网攻}{https://www.zaobao.com/news/china/story20240711-4236707}{(澳门综合讯)澳门政府保安系统多个机构官网及手机应用星期三(7月10日)晚上遭到海外网络攻击,导致服务一度受阻。 据澳门特区政府新闻局官网,澳门保安司司长办公室星期三深夜发消息说,因受到来自海外的分散式拒绝服务攻击,导致保安司司长办公室、治安警察局等网站服务从当晚8时起受阻。司法警察局已就此启动刑事调查工作,追查犯罪源头。 当局没有透露该攻击是来自海外哪个国家或单位……}

\entryitemWithDescription{陈婧:三中全会前夕的资本市场}{https://www.zaobao.com/news/china/story20240711-4201091}{三周前在上海举办的陆家嘴论坛上,中国证监会主席吴清为他的主旨演讲做了一段有趣的开场白,这并未出现在多数媒体报道中。 吴清在演讲开篇自谦道,今年论坛的主题是``以金融高质量发展推动世界经济增长'',但以他的了解,如果谈世界经济和金融,会被世界听众批评;讲中国经济,会被国内听众批评;讲上海则会被上海和世界批评,因为上海是国际金融中心,上海的事全世界都关注,``所以我还是老老实实地讲资本市场''……}

\entryitemWithDescription{调查:美国在高收入国家的形象好过中国}{https://www.zaobao.com/news/china/story20240710-4200834}{(华盛顿综合讯)美国皮尤研究中心对全球35个高收入和中等收入国家展开调查,结果显示美国的形象在多数国家好于中国。这表明北京要在全球范围内赢得人心的努力,仍还有很长的路要走。 据彭博社报道,皮尤研究中心星期二(7月9日)公布调查结果显示,在18个高收入国家中,对美国看法正面的人数是对中国看法正面人数的两倍多,其中波兰、日本、美国和韩国的差距最为明显。17个中等收入国家中,对中美两国的看法不相上下……}

\entryitemWithDescription{台防长:解放军``山东''号航母经过巴林塘海峡}{https://www.zaobao.com/news/china/story20240710-4200989}{日本防卫省7月10日公布的照片显示,中国海军``山东''号航空母舰7月9日在日本冲绳县宫古岛东南方的太平洋上,进行舰载机起降训练。(法新社) 台湾海峡与南中国海局势令人不安之际,台湾军方星期三声称侦测到中国大陆海军山东号航空母舰编队现身菲律宾北部附近海域,穿越巴林塘海峡前往西太平洋……}

\entryitemWithDescription{非中国籍港澳永久居民 7月10日起可申请``回乡证''}{https://www.zaobao.com/news/china/story20240710-4200294}{图为7月10日,有外籍人士到位于香港中旅证件服务港岛中心,申领``港澳居民来往内地通行证(非中国籍)''。(中新社) 香港和澳门两地所有非中国籍的永久居民,星期三(7月10日)起可以申请``港澳居民来往内地通行证(非中国籍)''。持有这通行证者,日后前往中国大陆将无须另行申请签证,往返两地更加便利。 负责办理证件的香港中国旅行社,专门在港岛上环的证件服务中心,设立了两个柜枱处理相关申请……}

\entryitemWithDescription{中国本周两度发布南中国海生态调查报告 分析:中菲岛礁争议转向法律战}{https://www.zaobao.com/news/china/story20240710-4200573}{中国多家机构在今年5月至6月,对黄岩岛海域进行生态环境状况调查评估。图为调研人员在黄岩岛海域水下作业。(新华社) 中菲继续围绕南中国海争议海域的生态环境问题隔空交锋。在菲律宾指责中国是珊瑚礁破坏者后,中国官方再发调查报告,称目前由中国实际控制的黄岩岛(菲称斯卡伯勒浅滩)海域,生态环境质量优,珊瑚礁生态系统健康。 这是中国本周发布的第二份关于南中国海争议海域生态环境的调查报告……}

\entryitemWithDescription{中国对欧盟启动贸易投资壁垒调查}{https://www.zaobao.com/news/china/story20240710-4200004}{针对欧盟调查中国企业的做法,中国商务部星期三也对欧盟的铁路机车、光伏、风电和安检设备等产品,启动贸易投资壁垒调查。(路透社) (北京综合讯)针对欧盟调查中国企业的做法,中国商务部星期三也对欧盟的铁路机车、光伏、风电和安检设备等产品,启动贸易投资壁垒调查。 中国商务部星期三(7月10日)在官网公告,称在6月17日收到中国机电产品进出口商会正式提出申请书……}

\entryitemWithDescription{中国多家食用油企业回应罐车运输风波 鲁花主播喝油自证品质}{https://www.zaobao.com/news/china/story20240710-4200383}{(北京综合讯)近期引发热议的``罐车运输乱象''波及多家中国食用油企业。鲁花食用油主播星期三(7月10日)直播喝油,自证品质。金龙鱼食品公司则在同日回应交叉运输罐车到过旗下工厂时说,工厂在装运前进行了清罐验罐。 据蓝鲸财经报道,在千名观众的关注下,鲁花食用油旗舰店主播星期三在直播间里,现场喝油自证品质,且在评论区弹幕催促下,短时间内多次喝油……}

\entryitemWithDescription{北京拟支持自动驾驶汽车跑网约车}{https://www.zaobao.com/news/china/story20240710-4200048}{(北京综合讯)北京市经信局发文,拟支持自动驾驶汽车用于城市公共电汽车客运、网约车等城市出行服务。中国多省市已相继制定自动驾驶相关地方立法,加速拓展应用场景。 据《21世纪经济报道》星期三(7月10日)报道,北京市经信局6月30日发布《北京市自动驾驶汽车条例(征求意见稿)》,对外征求意见。 征求意见稿提出,利用自动驾驶汽车开展创新活动,需配备驾驶人或安全员。安全员可以是远程监控……}

\entryitemWithDescription{AIT新处长:美国持续支持台湾自我防御能力}{https://www.zaobao.com/news/china/story20240710-4199877}{美国在台协会(AIT)新任处长谷立言(左)星期三(7月10日)与台湾总统赖清德(右)见面并进行交谈。(法新社) (台北综合讯)美国在台协会(AIT)新任处长谷立言(Raymond Greene)在与台湾总统赖清德见面时说,美国会持续支持台湾自我防御能力,并进一步强化已坚如磐石的美台关系。 根据台湾总统府新闻稿,赖清德星期三(7月10日)在总统府接见谷立言……}

\entryitemWithDescription{杨丹旭:罐车运输乱象背后的调查力量}{https://www.zaobao.com/news/china/story20240710-4156004}{一则关于罐车运输乱象的报道,这几天在中国舆论掀起轩然大波,有关调查报道的力量和中国调查记者的处境也引发关注。 事缘上周二(7月2日),《新京报》发表了一篇长篇调查报道,揭露一些装液体的罐车在刚卸下煤制油等化工产品后,就装上大豆油等食用液体,为了节省成本,连罐体都不曾清洗的乱象。 有罐车司机透露,卸完煤制油若不清洗罐体,罐内会残留几千克到十几千克不等的煤制油……}

\entryitemWithDescription{瓦努阿图总理访问深圳 参观华为警务监控技术}{https://www.zaobao.com/news/china/story20240710-4151804}{(深圳综合讯)瓦努阿图总理萨尔维星期二在深圳走访了华为公司,并参观用于加强警务工作和减少犯罪活动的监控技术。中国近年持续推进与南太平洋岛国间的警务合作并扩大影响力,再次引起关注。 据路透社星期二(7月9日)报道,瓦努阿图政府发布声明称,华为向瓦努阿图首都维拉港等城市提供数码系统,以``减少犯罪活动''。 声明还提到,警务监控系统需要在瓦努阿图建立一个数据中心……}

\entryitemWithDescription{比亚迪斥资10亿美元赴土耳其建厂 分析:高关税迫使中国车企加速出海}{https://www.zaobao.com/news/china/story20240709-4155645}{土耳其总统埃尔多安(中)在伊斯坦布尔签字仪式前,会见比亚迪董事长兼总裁王传福(左三)。(法新社) 土耳其总统埃尔多安(右)在伊斯坦布尔会见比亚迪董事长兼总裁王传福。(法新社) 欧盟上周宣布对中国电动汽车加征临时关税后,中国电动车巨头比亚迪星期一(7月8日)与土耳其政府在伊斯坦布尔签署价值10亿美元(13.5亿新元)的投资协议,在土耳其建设年产15万辆电动车的工厂,预计2026年底投产……}

\entryitemWithDescription{新沪贸易总额去年增7.8\% 未来开启数码经济合作}{https://www.zaobao.com/news/china/story20240709-4155524}{新沪全面合作理事会第五次会议星期二(7月9日)在上海举行,我国文化、社区及青年部长兼律政部第二部长唐振辉(后排左二)和上海市长龚正(后排右二)共同见证签署15项合作协议。我国贸工部兼文化、社区及青年部高级政务部长刘燕玲(后排左一)和上海副市长华源(后排右二)也出席会议。(黎康摄) 新加坡和上海之间的经济联系持续加强,去年两地贸易总额同比增长7.8\%……}

\entryitemWithDescription{李家超:力争大熊猫10月1日抵港}{https://www.zaobao.com/news/china/story20240709-4154892}{香港特首李家超(前排左二)与夫人林丽婵(右一)7月8日在四川成都的中国大熊猫保护研究中心都江堰基地观看大熊猫(图中未显示)。(新华社) 继上周宣布北京将再赠港一对大熊猫后,香港特首李家超星期二(7月9日)表示会力争大熊猫在10月1日抵港,旅游业界人士则呼吁当局把握机会发展``熊猫经济'',提振香港旅游业……}