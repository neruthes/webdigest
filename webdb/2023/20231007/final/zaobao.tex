\entryitemWithDescription{新闻人间:再次卷入争议的香港大学校长张翔}{https://www.zaobao.com/news/china/story20231007-1440335}{新闻人间---港大校长张翔(Lianhe Zaobao) 香港继早前中文大学校董会改组风波闹得沸沸扬扬后,最高学府香港大学近来也爆出管治争议。港大校长张翔疑与校务委员会闹不和,遭``吹哨人''指控多起罪名。 首先是上个月,有港媒披露港大计划为校长座驾换车,张翔最后指定选购价格高昂的宝马,可过程中被质疑没有申报相关利益和招标……}

\entryitemWithDescription{温伟中:殉职消防员等不到的蜜月旅行}{https://www.zaobao.com/news/china/story20231007-1440353}{10月2日,是帅气阳光的33岁消防员赖俊儒期待已久的一天。 那一天,他原本准备搭长途班机,带相恋五年的英国籍新婚妻萨洱玛(Salma)到瑞士蜜月旅行,过后再到英国伯明翰探望岳父岳母。原本,他还将在旅途中的10月11日庆祝生日。 令人遗憾的是,他等不到这一天……}

\entryitemWithDescription{连署跨过门槛 郭台铭宣布拿到2024参选门票}{https://www.zaobao.com/news/china/story20231006-1440337}{(新北综合讯)台湾鸿海集团创办人郭台铭宣布,他的总统参选连署已越过门槛,并表示要``继续努力奋战到底''。 综合TVBS新闻网、《联合报》等多家台媒报道,郭台铭星期五(10月6日)晚在新北市板桥受访时说,他与副手赖佩霞的连署在星期三(10月4日)已经达标。 郭台铭竞选办公室发言人陈家颐当日受访时补充,自9月20日宣布连署以来,份数已经超过30万份,团队的目标是``越多越好''……}

\entryitemWithDescription{中国布设巅峰气象观测网 追踪青藏高原气候脉动}{https://www.zaobao.com/news/china/story20231006-1440321}{中国科考队员10月1日在卓奥友峰峰顶架设自动气象站。(新华社) (拉萨综合电)中国科学考察队10月1日在世界第六高峰卓奥友峰峰顶成功架设自动气象站,扩大了喜马拉雅山脉气象观测网络,以追踪气候变化对青藏高原的影响。 青藏高原是除了南北极以外,地球上冰川分布最广的地区,故有``亚洲水塔''之称。该地区脆弱的生态环境日益成为科学界关注的重点……}

\entryitemWithDescription{赖清德:台湾已主权独立 不必另行宣布}{https://www.zaobao.com/news/china/story20231006-1440310}{台湾副总统、民进党总统参选人赖清德8月25日在位于台北的民进党总部召开的记者会上,回应外媒提问。(法新社) (台北综合讯)台湾现任副总统、民进党总统参选人赖清德受访时称``台湾已经主权独立'',不必再另行宣布独立,并表示自己所有努力都在避免战争、稳定台海……}

\entryitemWithDescription{政经因素冲击下 在华日本人数跌至近20年来新低}{https://www.zaobao.com/news/china/story20231006-1440277}{(广州综合讯)在中国生活的日本人持续减少,从2012年顶峰的15万人,减至目前10万人以下近20年来的最低水平。受中日政经关系紧张等因素影响,人数今后也可能继续呈下跌趋势。 据日经中文网星期四(10月5日)报道,日本外交部的《海外日籍人数调查统计》显示,2022年在中国生活的日本人比2021年减少5\%,降至10万2066人……}

\entryitemWithDescription{分析:大陆不至于中止ECFA 或切香肠反制台贸易壁垒}{https://www.zaobao.com/news/china/story20231006-1440274}{《海峡两岸经济合作架构协议》(ECFA)是大陆海峡两岸关系协会、台湾海峡交流基金会,2010年6月29日在重庆代表两岸官方签订的双边经济协议。其中依据WTO规定,达成分阶段免关税默契,纳入早期收获清单的产品,依照双方约定的税率调降关税。图为两岸签署ECFA档案照片……}

\entryitemWithDescription{于泽远:许家印的见识与胆识}{https://www.zaobao.com/news/china/story20231006-1439974}{9月28日,中国国庆长假的前一天,恒大集团董事局主席许家印被官宣采取强制措施。(路透社) 9月28日,中国国庆长假的前一天,恒大集团董事局主席许家印被官宣采取强制措施。曾被媒体人誉为``恒而不倒,大而愈强''的恒大集团终于塌台了。 许家印显然要在看守所里度过他即将到来的65周岁生日,但他曾经的辉煌不会很快被人遗忘,尤其是他留给中国经济和社会的一堆烂摊子,更不会随着他的入狱就能轻易收拾干净……}

\entryitemWithDescription{台湾查获15万公斤美猪``洗产地'' 政府强调没检出``莱猪''}{https://www.zaobao.com/news/china/story20231005-1439999}{台湾桃园市卫生局和卫生福利部食品药物管理署查获两家公司将来自美国的15万公斤猪肉制成火锅肉片等贩售,产地则标示为加拿大和英国。这批猪肉8月下旬几乎都已售罄,被质疑在``洗产地''。 行政院发言人林子伦星期四(10月5日)在行政院会后强调,自2021年进口美猪两年半以来,无论是边境或市场均未查获检出含莱克多巴胺的猪肉,且此次稽查是食药署主动出击的成果……}

\entryitemWithDescription{台风``小犬''导致全台逾30万户家庭停电 一死300多伤}{https://www.zaobao.com/news/china/story20231005-1439991}{台风``小犬''星期四(10月5日)吹倒台湾南部屏东县整排的电线杆。(法新社) (台北/香港综合讯)台风``小犬''横扫了台湾南部地区,不仅给台湾带来有记录以来风力最强的阵风,也导致当地30多万户家庭停电,并造成一人死亡以及300多人受伤。 综合台媒报道,据台湾中央气象署通报,``小犬''在星期四(10月5日)上午登陆台湾最南端的岬角鹅銮鼻……}

\entryitemWithDescription{柯文哲访美遇台侨吁蓝白合 侯柯配之外的整合方案浮出台面}{https://www.zaobao.com/news/china/story20231005-1439972}{台湾民众党主席、总统参选人柯文哲星期四(10月5日)结束二度访美行程,返台前有台侨当面呼吁``蓝白合''。多名评论员认为,蓝白合未必是侯柯配,具体可行的在野整合方案,也包括让柯文哲当行政院长或立法院长。 台湾2024年1月13日将举行总统与立委选举,从10月5日开始倒数百日,喧腾大半年的在野整合议题,也进入成局或破局的摊牌时刻……}

\entryitemWithDescription{法新社:中国涉向瓦格纳集团提供卫星情报支持}{https://www.zaobao.com/news/china/story20231005-1439968}{俄罗斯雇佣兵组织瓦格纳集团已故首脑普里戈任掌控的尼卡-弗鲁特公司2022年11月15日与北京云泽科技有限公司签订一份英俄文书写的合约。(法新社) (巴黎法新电)法新社看到的一份文件显示,俄罗斯雇佣兵组织瓦格纳集团在2022年与一家中国公司签订合同,购买了两颗卫星并使用其图像,以辅助其在乌克兰和非洲的情报工作。 法新社星期四(10月5日)的报道说,这两颗卫星甚至在瓦格纳今年6月兵变中也发挥了作用……}

\entryitemWithDescription{中菲又在南中国海发生摩擦}{https://www.zaobao.com/news/china/story20231005-1439963}{(北京/马尼拉综合讯)中国和菲律宾又在有主权争议的南中国海海域发生摩擦。中国谴责四艘菲律宾船只未经中国政府允许,进入南沙群岛仁爱礁海域。菲律宾则指中国尝试``阻挡、骚扰和干扰''菲方执行的补给任务。 根据中国海警局在微信公号发布的消息,该局新闻发言人甘羽说,菲律宾两艘运补船和两艘海警船在星期三(10月4日)未经中国政府允许,擅自进入仁爱礁(菲律宾称阿云津礁)邻近海域……}

\entryitemWithDescription{``天宫''升级计划:中国打造``太空母港'' 挑战国际空间站}{https://www.zaobao.com/news/china/story20231005-1439962}{阿塞拜疆首都巴库星期三(10月4日)的国际宇航大会上,关于``天宫''空间站的小组讨论现场。(美国太空新闻网) (北京/巴库综合讯)中国推出替代国际空间站方案,计划将中国空间站``天宫''太空舱数量扩大一倍……}

\entryitemWithDescription{美媒:王毅本月将访华盛顿 为习拜会铺路}{https://www.zaobao.com/news/china/story20231005-1439907}{中国外长王毅据报本月将访美,为习拜会铺路。图为王毅9月26日出席在北京举行的《携手构建人类命运共同体:中国的倡议与行动》白皮书新闻发布会。(路透社) (华盛顿综合讯)美国媒体引述知情人士透露,中国外长王毅将在10月晚些时候访问华盛顿,为中美两国元首在11月亚太经济合作组织(APEC)峰会上的会面铺路……}

\entryitemWithDescription{陈婧:中国游客去哪儿?}{https://www.zaobao.com/news/china/story20231005-1439689}{前两天从新加坡飞回上海,樟宜机场随处可见面向中国消费者的银联卡和微信支付优惠广告,好几个柜台也推出``黄金周''特价折扣。 相比热火朝天的促销活动,机场里的中国游客却没有预期中多。柜员小姐坦言,虽然今年访新的中国旅客有所增加,但和疫情前还是相差甚远,``不懂他们去哪了?'' 中国疫情管控全面放开后的首个``十一''黄金周长假,出行数据迎来井喷式增长……}

\entryitemWithDescription{特稿:总统选战倒数100天 台湾在野整合有利形势流失中}{https://www.zaobao.com/news/china/story20231004-1439691}{台湾两大在野党总统参选人,国民党的新北市长侯友宜(左)和民众党主席、卸任台北市长柯文哲,在担任双北首长期间互动良好。(互联网) 台湾总统选战从星期四(10月5日)开始倒数100天,在野整合的有利形势正在流失中,``蓝白合''可能在10月中下旬成局或破局。 台湾将在2024年1月13日举行总统与立委选举,多份民调显示过半民众希望政党轮替……}

\entryitemWithDescription{【亚洲前瞻峰会】陆港学者:北京不急着武统台湾}{https://www.zaobao.com/news/china/story20231004-1439687}{亚洲前瞻峰会星期三(10月4日)下午举行分场,聚焦``在战略竞争时代下的中美合作''。参会专家包括亚洲协会美中关系中心阿瑟罗斯主任夏伟(中)、香港城市大学法学院教授王江雨(左),以及南京大学国际关系学院执行院长朱锋……}

\entryitemWithDescription{四家台企被指为华为建厂 台高官澄清有通过投资审查}{https://www.zaobao.com/news/china/story20231004-1439678}{彭博社点名台湾四家科技公司协助华为在深圳建设晶片厂基础设施,躲避美国的制裁。图为华为位于上海的门店。(路透社) 台湾经济部长王美花和国安局长蔡明彦星期三(10月4日)澄清说,被外媒点名协助中国大陆华为公司建厂的四家台湾公司,皆经台湾政府投资审查通过才去投资,而且所建工厂属于较低阶的废水或环保工程,并非科技管制出口的项目。 华为Mate 60系列采用中芯7奈米制程,成功突破美国禁令……}

\entryitemWithDescription{台风``小犬''逼近 台湾多县市停班停课}{https://www.zaobao.com/news/china/story20231004-1439667}{台湾台东县一名女子星期三(10月4日)坐在岸边,看着台风``小犬''登陆前,在沿海地区掀起的巨浪。(路透社) (台北综合讯)在台风``小犬''登陆之前,台湾取消了超过100趟境内外航班,并宣布多个县市停班停课。这是台湾在短短一个月内第二次面对大型风暴……}

\entryitemWithDescription{港府总部外发生伤人案 17岁青年持水果刀袭击保安}{https://www.zaobao.com/news/china/story20231004-1439645}{一名患有自闭症的17岁青年星期三上午持水果刀在港府总部大楼外袭击两名保安人员。(路透社) (香港综合讯)香港政府总部大楼外发生伤人案,一名患有自闭症的17岁青年,持水果刀袭击两名保安人员,导致一人手部被刺伤,另一人则扭伤脚部。 综合港媒报道,事发在星期三(10月4日)上午10时许,地点是在连接政府总部大楼的行人天桥。这名青年当时突然持刀袭击大楼外的保安人员,之后被其他在场保安人员和行人制伏……}