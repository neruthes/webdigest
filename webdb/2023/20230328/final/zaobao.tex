\entryitemWithDescription{马云现身杭州推翻拒回国传言 阿里股价闻声大涨}{https://www.zaobao.com/news/china/story20230328-1376973}{马云(右二)星期一(3月27日)在他创办的杭州云谷学校,与校园长们讨论ChatGPT、人工智能对教育带来的挑战与机遇。(取自云谷教育微信) 在海外旅行逾一年的阿里巴巴创始人马云,据报星期一(3月27日)已回到中国国内,不仅现身由他创办的杭州云谷学校,也在杭州一处隧道里被网民看到,稍早有关他拒绝中国政府游说回国的传言因而被否定。受利好消息提振,在香港挂牌的阿里股价一度大涨5.5%……}

\entryitemWithDescription{刚与洪都拉斯断交 被指不合时宜 马英九``登陆''引爆台湾政坛争议}{https://www.zaobao.com/news/china/story20230328-1376974}{一名独派人士用麦克风向马英九一行表达强烈抗议,被在场维持秩序的警员架走。(法新社) 中华爱国同心会人士在桃园机场外拉着``英九祭祖春暖花开我们都是一家人''的红布条,为马英九送行。(路透社) 中美洲国家洪都拉斯此前一天(3月26日)宣布与台湾断交,执政的民进党与独派人士指责马英九不宜在此时出访大陆……}

\entryitemWithDescription{台湾总统选举剩九个多月 ``蓝军''料最快下月确定候选人}{https://www.zaobao.com/news/china/story20230328-1376975}{鸿海创办人郭台铭星期一(3月27日)通过办公室宣布,当晚赴美进行12天科技经济开拓之旅,展现参选台湾总统的企图心。(取自郭台铭脸书) 台湾将在2024年1月13日举行总统与立委选举,距今只剩九个多月。蓝绿白三党竞逐总统大位,执政的``绿军''民进党、打着``白色力量''旗号的第三大党民众党都已敲定候选人,最大在野党``蓝军''国民党也终于启动征召程序,估计最快4月、最迟6月就能确定人选……}

\entryitemWithDescription{香港下月中举办大规模教育活动 加强港人国安意识}{https://www.zaobao.com/news/china/story20230328-1376976}{香港国安法实施逾两年后,香港特区政府计划下月中旬大规模举办国家安全教育日活动,以加强港人的国家安全意识。 《星岛日报》引述香港律政司司长林定国报道,4月15日是一年一度的``全民国家安全教育日'',今年主题为``国家安全.稳定繁荣基石'',将举办的活动类型包罗万象,包括开幕典礼、纪律部队综合升旗礼、专题讲座、学校工作坊、社区巡回展览、全港18区活动及嘉年华等……}

\entryitemWithDescription{新增两副部级高官 中国公安部领导班子基本定型}{https://www.zaobao.com/news/china/story20230328-1376977}{中国公安部新增两名副部级高官。公安部官网显示,陕西省原副省长、公安厅长徐大彤已出任公安部党委委员、副部长,成为公安部首位70后(1970年代出生)的副部长;广西壮族自治区原副主席、公安厅长凌志峰出任公安部党委委员、政治部主任。 公安部原副部长杜航伟、公安部政治部原主任冯延的简历,已从公安部官网中撤下。杜、冯两人都已超过60周岁,依照惯例,副部级高官一般在年满60周岁后退休……}

\entryitemWithDescription{巴西总统染冠病 推迟访华 日期将再确定}{https://www.zaobao.com/news/china/story20230328-1376978}{巴西总统卢拉因感染冠病推迟访华行程,巴西农业部长法瓦罗(Carlos Favaro)星期天(3月26日)说,中国政府将再确认卢拉访华的新日期,中国与巴西也将推迟签署合作协议。 据路透社报道,法瓦罗星期天抵达中国,与中方商讨解除向中国出口牛肉的禁令。他说,巴西政府包括农业部的所有政府工作都推迟进行。当中国政府确定空档,将重新安排巴西总统访华日期,届时双方再签署所有备忘录和协议……}

\entryitemWithDescription{早 说}{https://www.zaobao.com/news/china/story20230328-1376979}{如果中美之间发生``第二次冷战'',将比第一次冷战更加危险。中美之间的战争即使不会摧毁文明,也会让文明倒退,美国和中国在防止灾难性的冲突方面有着最低限度的共同义务。 ------美国前国务卿基辛格日前接受西班牙《世界报》采访时谈及中美关系……}

\entryitemWithDescription{香港为何多老年德士司机?}{https://www.zaobao.com/news/china/story20230328-1376980}{过去三年,香港因疫情几乎和外界断了联系。随着今年初港府宣布与海外陆续通关,访港旅客人数回升,不少海外和中国大陆的朋友纷纷赴港旅游或者公干。 在闲聊中,许多朋友都称赞香港经历前所未有的疫情肆虐后,社会活力并没有明显减弱。但一些朋友也不约而同地指出,在香港发现一个很奇怪的现象,就是德士司机多是老人家。 朋友举例说,他从香港机场坐德士去酒店,一上车才看到司机是一位白发苍苍的长者,估计有70岁……}

\entryitemWithDescription{港``黄色经济圈''退潮龙头表态割席}{https://www.zaobao.com/news/china/story20230328-1376981}{香港民主派商人周小龙旗下的Chickeeduck儿童服装店(左图),一度拥有十多家分店,目前只剩下两家分店,连周小龙本人也已移居海外。阿布泰国生活百货公司创办人林景楠(右图),在他自己的脸书专页宣布与``黄色经济圈''划清界限。(互联网/林景楠脸书) 香港2019年爆发反修例运动,期间民主派发起``黄色经济圈'',呼吁支持者光顾政见相似的商家……}

\entryitemWithDescription{美副助理国务卿华自强低调访中 分析:或为布林肯访华铺路}{https://www.zaobao.com/news/china/story20230328-1376982}{美国副助理国务卿华自强(Rick Waters)在过去一周低调访问中国,与学术界、商界人士密集会面交流。分析认为,华自强此行或为美国国务卿布林肯重新安排访华铺路。 据澎湃新闻报道,也是美国国务院中国协调办公室主管的华自强,在此次访华期间分别到访香港、上海与北京……}

\entryitemWithDescription{青年宿舍让港青圆独居梦}{https://www.zaobao.com/news/china/story20230328-1376983}{3月27日,香港特区政府民政及青年事务局向香港青年联会批出首个``将酒店和旅馆转作青年宿舍用途的资助计划''项目正式开幕,该项目位于铜锣湾摩理臣山道,以BeLIVING Youth Hub的名义营运,共提供97间房,最多194个宿位,月租约3800至4800港元,协助青年储蓄及扩阔人脉,解决就业、创业及置业难题。 上图为青年宿舍外观,下图为宿舍的房间设施及景观……}

\entryitemWithDescription{蔡英文上台七年第九个断交国 洪都拉斯宣布弃台湾与大陆建交}{https://www.zaobao.com/news/china/story20230327-1376646}{中国大陆外交部长秦刚(右)与洪都拉斯外长雷纳星期天在北京钓鱼台国宾馆,完成建交仪式后彼此祝贺。(路透社) 台湾外交部长吴钊燮说,台湾秉持最大诚意提出协助洪都拉斯方案,但洪方仍``恣意作态需索'',罔顾台湾多年情谊和大陆建交,令台湾痛心与遗憾……}

\entryitemWithDescription{最新民调: 台支持统独民意降低 要维持现状者增加}{https://www.zaobao.com/news/china/story20230327-1376647}{台湾最新民调显示,相较去年10月,最终支持统一、独立的民意都有所降低,但以支持独立者较明显;支持永远维持现状及以后再决定的,则增加了6.1个百分点。学者解读,在两岸关系不稳定时,倾向独立的民意会转化到维持现状,这是基于安全需求的权变。 台湾政府的大陆委员会长期委托政治大学选举研究中心,就民众当下对统独立场及两岸关系看法进行民意调查……}

\entryitemWithDescription{香港解封后首场获批准游行 反对将军澳填海}{https://www.zaobao.com/news/china/story20230327-1376648}{约80名示威者星期天(3月26日)参加``反对将军澳填海''和兴建垃圾处理场的游行,他们的颈部均挂上印有独立编号的卡牌,没有佩戴口罩。香港警方派出大批警员在场监视。(法新社) (香港综合讯)香港约有80名示威者星期天参加``反对将军澳填海''和兴建垃圾处理场游行。这是香港在2020年实行《香港国安法》以来,首个获警方批准的抗议游行……}

\entryitemWithDescription{日男疑涉间谍活动北京被捕}{https://www.zaobao.com/news/china/story20230327-1376650}{据日本媒体报道,一名50多岁日本男子疑似因从事间谍活动,本月在中国首都北京被拘捕。 据日本放送协会(NHK)星期六(3月25日)报道,这名男子是一家日本公司的高级主管。日本政府消息人士称,他本月较早前因涉嫌违反中国法律,而被中国安全机关拘留。据报,该名男子可能因涉嫌从事间谍活动而被捕。 日本驻中国大使馆已提出与男子谈话的请求,并正在收集关于男子被捕原因的信息……}

\entryitemWithDescription{早说}{https://www.zaobao.com/news/china/story20230327-1376651}{可以预见的是,美国一定会使用各种手段为台湾巩固``邦交''。但美国自己就与中国有外交关系,却阻挠其他国家与中国建交,令人匪夷所思。 ------上海大学特聘教授、拉美研究中心主任江时学3月26日接受香港中通社采访时表示,洪都拉斯决定和中国大陆寻求建交后,美国政府和台湾政府都对该国施加了不小压力……}

\entryitemWithDescription{于泽远:中国国务院工作规则变化}{https://www.zaobao.com/news/china/story20230327-1376652}{从规则变化看,李强领衔的新一届国务院在指导思想、工作原则和流程上都明显加强了国务院的政治属性……}

\entryitemWithDescription{中央财办副主任韩文秀: 中国须化解外压通过技术创新突破}{https://www.zaobao.com/news/china/story20230327-1376653}{韩文秀在北京举行的中国发展高层论坛上说,当前世界经济存在滞胀风险,全球产业链供应链面临重构,中国经济恢复的基础还不够稳固。他指出,中国要努力克服新的外部压力,以及人口负增长、老龄化对中国经济中长期发展的影响,其中要有效应对外部的遏制打压……}

\entryitemWithDescription{``鼎泰丰''创办人杨秉彝逝世 享年96岁}{https://www.zaobao.com/news/china/story20230327-1376655}{``鼎泰丰''创办人杨秉彝,日前安详辞世,享年96岁。(互联网) 台湾具代表性的连锁餐饮品牌``鼎泰丰''的创办人杨秉彝,日前安详辞世,享年96岁。 综合中时新闻网、《联合报》报道,鼎泰丰星期六(3月25日)对外发布消息,创办人杨秉彝日前安详辞世,家属盼低调办理治丧事宜,感谢各界关心。 根据鼎泰丰网站公布的信息,杨秉彝1927年出生于中国大陆山西省,21岁赴台湾闯荡……}

\entryitemWithDescription{中国国际时装秀 ``妇好鸮尊''绕``战马''}{https://www.zaobao.com/news/china/story20230327-1376656}{(中新社) 北京冬奥会服装设计师丁洁3月26日携``视界.边界''系列时装,亮相中国国际时装周(2023秋冬系列)。丁洁以``妇好鸮尊''为设计灵感,模特佩戴3D打印发饰,绕行秀场特别设置的``战马'',上演一场历史与未来的``时空对话''。``妇好鸮尊''是一件商代后期的青铜器,1976年出土于河南省安阳市殷墟妇好墓,共两件,分别藏于中国国家博物馆和河南博物院。妇好为商王武丁的配偶,曾多次出征……}

\entryitemWithDescription{劣质大陆旅团被指扰民 港府:解决措施研究中}{https://www.zaobao.com/news/china/story20230327-1376657}{有低素质的中国大陆旅游团被指在香港扰民,出现强迫购物等乱象。香港特区政府表示,已开始研究解决措施。 据《星岛日报》报道,有议员反映,素质较低的大陆旅行团近日重现在香港旧区街头。香港工联会的立法会九龙东议员邓家彪说,土瓜湾等旧区道路较狭窄,随着未来旅客增多,旅游巴士路边停泊可能造成堵塞,影响居民生活。 邓家彪说,劣质旅游团往往出现``强迫购物''的情况,损害香港旅游业的形象……}