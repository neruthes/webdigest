\entryitemWithDescription{于泽远:《罗刹海市》讽刺了谁?}{https://www.zaobao.com/news/china/story20230731-1419004}{在沉寂10多年后,52岁的歌手刀郎最近又火遍中国。7月19日,刀郎新歌《罗刹海市》一经发布便冲上热搜,此后10天,刀郎新专辑词条的相关搜索量同比暴涨6000多倍,7月27日,``刀郎专辑''登上淘宝热搜第二名。 刀郎本名罗林,年轻时曾在酒吧驻唱多年,生活一度十分窘迫。2004年初,刀郎专辑《2002年的第一场雪》在没有任何宣传的情况下,从新疆红遍大江南北,还被香港著名歌手谭咏麟翻唱……}

\entryitemWithDescription{国台办批民进党乞求美国军援 学者指向台各总统候选人划红线}{https://www.zaobao.com/news/china/story20230730-1419050}{美国政府宣布对台湾提供新一轮军事援助后,大陆国台办指民进党政府不断乞求美国提供军援,正在把台湾变成火药桶、弹药库。图为7月27日,新北市``汉光''演习中的一辆两栖突击车和两架黑鹰直升机。(彭博社) 针对美国政府宣布对台湾提供新一轮军事援助,中国国务院台湾事务办公室指民进党政府不断乞求美国提供军援,正在把台湾变成火药桶、弹药库……}

\entryitemWithDescription{德教育部长指中国政府奖学金是战略工具}{https://www.zaobao.com/news/china/story20230730-1418994}{德国教育部长瓦钦格认为,中国政府提供给青年学者的政府奖学金是战略工具,增加了科学间谍的风险,并呼吁修订与中国的学生交换活动。 综合法新社和德国之音报道,瓦钦格(Bettina Stark-Watzinger)在巴伐利亚媒体集团(Mediengruppe Bayern)星期六(7月29日)发表的采访中说:``中国正日益成为科学研究领域的竞争者和系统性竞争对手……}

\entryitemWithDescription{避免区域课题被西媒构建 学者建议中国和亚细安智库分享知识与智慧}{https://www.zaobao.com/news/china/story20230730-1418979}{深圳星期六(7月29日)举行2023年大湾区---亚细安经济合作论坛,中国和亚细安10国的智库代表出席,与会代表结成了中国-亚细安智库合作伙伴。(林煇智摄) 面对中美在东南亚的地缘政治竞争,印度尼西亚战略与国际问题研究中心高级研究员萨拉斯瓦蒂认为,亚细安和中国智库必须更积极分享学术知识,通过加强交流与合作分享在地智慧,避免区域课题由西方媒体构建……}

\entryitemWithDescription{法国经济部长:盼更好地进入中国市场而非脱钩}{https://www.zaobao.com/news/china/story20230730-1418977}{法国经济部长勒梅尔说,法国希望与中国建立更平衡的贸易关系。图为勒梅尔星期六在北京参加第九次中法高级别经济财金对话。(法新社) 正在中国访问的法国经济、财政及工业、数字主权部部长勒梅尔说,法国希望与中国建立更平衡的贸易关系,并强调与中国脱钩是``一种幻想''……}

\entryitemWithDescription{女性情趣用品店入驻深圳商场 专家:社会性观念解放跨进一步}{https://www.zaobao.com/news/china/story20230730-1418975}{女性情趣品牌``大人糖''7月中旬在深圳宝安区壹方城购物中心正式开业,成为全市首家入驻商场的情趣用品专卖店。(林煇智摄) 专卖成人情趣用品的女性情趣品牌在深圳大型商场开设首家实体店,大方地走入大众视野,引来不少好奇的公众进店驻足围观。 受访专家认为,这项尝试折射了中国社会的性解放和包容观念已跨进一步,女性也能正视自己的性需求,不需要遮遮掩掩……}

\entryitemWithDescription{《愿荣光归香港》禁制令申请驳回后 相关贴文被大量转发}{https://www.zaobao.com/news/china/story20230730-1418973}{香港法院驳回港府针对反修例运动歌曲《愿荣光归香港》的禁制令申请后,与这首歌有关的贴文被大量转发,但当地律师提醒,转发仍可能违法。 据香港《明报》星期天(7月30日)报道,禁制令申请被驳回后,有网民随即在社交平台分享《愿荣光归香港》的音乐短片,并加上``并非禁歌''字眼,相关帖子至星期六(7月29日)已有超过200次转发……}

\entryitemWithDescription{台湾特稿:国民党全代会热闹拉不动团结 泛蓝貌合神离}{https://www.zaobao.com/news/china/story20230730-1418462}{国民党7月23日举行全台党代表大会营造团结支持该党总统参选人侯友宜(前排中)的气氛。(国民党提供) ~ 民众党总统参选人柯文哲(中)和民众党前立委蔡壁如(左二)到台中大甲镇澜宫参拜,国民党前立委颜宽恒(右二)陪同。(民众党提供) 鸿海集团创办人郭台铭(中)对参选台湾总统之事多次强调,``民意大于党意''。图为他7月16日参加在台北凯达格兰大道举行、抗议民进党政府施政的活动……}

\entryitemWithDescription{中法财金对话 何立峰望法国能缓和中欧关系}{https://www.zaobao.com/news/china/story20230729-1418791}{何立峰(左)星期六在北京钓鱼台国宾馆与勒梅尔会面。(法新社) 中国国务院副总理何立峰星期六(7月29日)在北京举行的第九次中法高级别经济财金对话上说,希望法国能够``稳定中欧关系基调''。 据路透社报道,何立峰在中法对话的开场白中发表了上述言论,他还告诉法国经济、财政及工业、数字主权部部长勒梅尔,中国愿意深化与法国在金融和科技创新等传统领域的合作,同时补充说,中国相信中法双边关系拥有良好的基础……}

\entryitemWithDescription{【视频】贵州村超热度不减 民众:能倒逼中国足坛改革}{https://www.zaobao.com/news/china/story20230729-1418784}{贵州``村超''今年5月开赛以来,榕江县吸引了中国各地大量游客前往观赛游玩,带火了当地旅游、餐饮、住宿、文创、农特产品等行业,被称为``超经济''现象。图为人们在``村超''足球赛场旁的集市游玩。(新华社) ``我们国足真该向村超学一学!''现场评述员在星期五(7月28日)晚举行的贵州``村超''半决赛点球大战中,大赞参赛球员高度冷静的踢点球状态,引发球场内观众球迷当晚最高昂的一波欢呼声……}

\entryitemWithDescription{台风``杜苏芮''残余环流北上 北京等多地降大暴雨}{https://www.zaobao.com/news/china/story20230729-1418760}{受台风``杜苏芮''外围环流影响,山东省青岛市沿海7月29日掀起大浪。(新华社) 今年第5号台风``杜苏芮''登陆中国福建晋江沿海后,其残余环流继续北上,造成中国北方多地降大暴雨,气象当局星期六(7月29日)傍晚发布了最高级别的暴雨红色预警。 据环球网报道,这是中国中央气象台2010年正式启用预警发布机制以来发布的史上第二个暴雨红色预警,上一次是在2011年9月29日……}

\entryitemWithDescription{欧洲领导人可能缺席一带一路高峰论坛}{https://www.zaobao.com/news/china/story20230729-1418752}{中国将在今年10月举行第三届一带一路国际合作高峰论坛。图为一名参观者在阅览香港参与一带一路倡议的资讯板。(互联网) 在欧盟寻求对华去风险之际,有消息称,不少欧洲领导人料将缺席今年10月的一带一路国际合作高峰论坛,以寻求和中俄保持距离。 俄罗斯官员本周透露俄罗斯总统普京计划出席论坛;受访学者分析,欧洲领导人正在重新思考是否继续参加一带一路倡议,回避普京只是不参加论坛的表面原因……}

\entryitemWithDescription{中国国防部批日方刻意渲染中国军事威胁}{https://www.zaobao.com/news/china/story20230729-1418751}{针对日本新版国防白皮书将中国称为日本``前所未有的最大战略挑战'',中国国防部指责日方固守错误对华认知,刻意渲染中国军事威胁,对此表达坚决反对并已提出严正交涉。 据中国国防部微信公众号星期六(7月29日)消息,国防部新闻发言人谭克非就白皮书答记者问时也说,日本抹黑炒作中国军队正常的建设发展和军事活动,粗暴干涉中国内政,挑动地区紧张局势……}

\entryitemWithDescription{美国对台湾提供3.45亿美元军事援助}{https://www.zaobao.com/news/china/story20230729-1418734}{台湾士兵7月20日在新北市进行演练。一些士兵站在M60A3坦克上。(路透社) 美国宣布将向台湾提供价值3.45亿美元(4.59亿新元)的军事援助,这是拜登政府首次以从美国现有库存提取方式对台湾提供军援。 综合路透社、法新社、彭博社、美国之音和台湾《联合报》等报道,美国白宫当地时间星期五(7月28日)作出上述宣布,但没有公布军援清单……}

\entryitemWithDescription{新闻人间:刀郎的新歌}{https://www.zaobao.com/news/china/story20230729-1418487}{沉寂多年的中国大陆歌手刀郎最近又火了起来,一首新歌《罗刹海市》红遍全网。 引起广泛讨论的,并非音乐本身,而是对于歌词中影射、讽刺意涵的解读,以及刀郎与那英等中国音乐圈名人之间的恩怨。 刀郎原名罗林,今年52岁。20年前,他以《2002年的第一场雪》成名,又创作出《情人》、《冲动的惩罚》、《喀什噶尔胡杨》等脍炙人口的歌曲。刀郎的歌,曲调多取自民间,歌词也通俗直白,播放量一度在中国乐坛名列前茅……}

\entryitemWithDescription{温伟中:台湾选战的和平愿景与刀光剑影}{https://www.zaobao.com/news/china/story20230729-1418529}{迎向五个半月后的2024年1月13日台湾总统大选,主要竞争者的最大共同点,就是都强调和平。 各方说法虽不同,但核心信息都是``选我,就会带给你和平''。不管论述和方法如何,大家都许下了和平愿景,承诺当选后会努力维护台海和平稳定。 以酝酿参选的鸿海集团创办人郭台铭为例,他星期天(7月30日)要访问美国,出发前强调这是一趟和平之旅……}

\entryitemWithDescription{中国遇今年最强台风 ``杜苏芮''登陆福建}{https://www.zaobao.com/news/china/story20230728-1418514}{(小)台风``杜苏芮''在福建晋江沿海登陆后,福建省厦门市出现特大暴雨。图为一名警察走在被大雨淹没的街道上,水位高至大腿处。(法新社) 今年第五号台风``杜苏芮''星期五(7月28日)登陆中国福建晋江沿海,是今年以来登陆中国的最强台风,也是登陆福建第二强的台风……}

\entryitemWithDescription{知情人士:中国望金砖扩大阵容 遭巴西印度反对}{https://www.zaobao.com/news/china/story20230728-1418489}{知情官员透露,中国希望金砖国家组织快速扩大,通过壮大自身政治影响力来对抗美国,但遭到成员国印度和巴西的反对。 据彭博社星期五(7月28日)报道,金砖五国------巴西、俄罗斯、印度、中国和南非下个月将在约翰内斯堡召开金砖国家组织领导人峰会,讨论扩大金砖、接纳印度尼西亚和沙特阿拉伯的议题。 据知情官员称,中国在领导人峰会预备会中一再游说金砖扩大阵容,但印度和巴西提出反对意见……}

\entryitemWithDescription{传美国拟禁港特首出席APEC峰会 北京强烈不满提出严正交涉}{https://www.zaobao.com/news/china/story20230728-1418478}{香港特首李家超:任何国家或经济体主办APEC时,都应按指引、习惯及规则,邀请成员经济体领导人参与。(法新社) 《华盛顿邮报》引述消息指白宫决定不让香港特首李家超出席11月在旧金山举行的亚太经济合作组织(APEC)领导人非正式会议。中国表示强烈不满和坚决反对,并称或将对美实施制裁。 受访学者说,将李家超列入APEC的黑名单,显示中美关系基本上并没有缓和,事件将加剧华盛顿和北京之间的紧张……}

\entryitemWithDescription{香港高院驳回港府申请 拒禁《愿荣光归香港》}{https://www.zaobao.com/news/china/story20230728-1418466}{香港高等法院驳回了港府针对反修例运动歌曲《愿荣光归香港》提出的禁制令申请,称这可能会损害言论自由并造成潜在的``寒蝉效应''。 综合路透社、香港01报道,在多项国际体育赛事播出香港得奖选手的国歌时,误播了《愿荣光归香港》而不是中国国歌《义勇军进行曲》后,港府向法院申请禁止以任何方式传播《愿荣光归香港》……}