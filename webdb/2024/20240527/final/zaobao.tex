\entryitemWithDescription{于泽远:大陆围台军演效果如何?}{https://www.zaobao.com/news/china/story20240527-3726515}{在中国大陆解放军围台军演落下帷幕后,中国总理李强5月26日(星期天)抵达首尔,出席中日韩领导人会议。美国国防部5月24日宣布,美中两国防长将在下周举行的新加坡香格里拉对话期间会晤。中日韩领导人重启会议以及中美防长会晤,显示由台湾新任总统赖清德的``两国论''引发的台海军事危机,暂时告一段落。 对这次解放军围台军演,美国总统拜登5月25日重申美国坚定维护台海和平稳定……}

\entryitemWithDescription{中国国家烟草局副局长徐㼆落马}{https://www.zaobao.com/news/china/story20240526-3727072}{(北京综合讯)中国国家烟草专卖局副局长徐㼆,星期六(5月25日)被通报任上落马。 徐㼆落马前最后一次出席公开活动,是5月14日在河北省烟草专卖局(公司)调研,走访石家庄市部分零售客户,并听取河北省局(公司)工作汇报。 公开资料显示,今年60岁的徐㼆先后在云南、浙江、湖北、湖南、广东、广西等多地的烟草部门任职,也曾任江西省和北京市烟草专卖局(公司)局长、总经理……}

\entryitemWithDescription{全球首例 安徽医院移植猪肝给肝癌患者}{https://www.zaobao.com/news/china/story20240526-3726723}{(合肥讯)中国安徽省一家医院成功将猪肝移植到71岁的肝癌患者体内,完成全球首例活体人的异种肝移植手术,也是全球第五例活体人的异种器官移植手术。 安徽医科大学第一附属医院星期五(5月24日)在官方微信公号通报称,该院孙倍成教授团队和云南农业大学魏红江教授团队合作,5月17日在医院成功将10基因编辑供体猪的肝脏,移植到一名71岁右叶巨大肝癌的男性病人身上……}

\entryitemWithDescription{意大利部长吁欧盟仿效美国 对中国产品征税保护欧洲工业}{https://www.zaobao.com/news/china/story20240526-3726506}{意大利工业部长乌尔索于4月8日在巴黎郊外举行的欧洲产业政策三方会议后,向媒体发表讲话。(法新社) (北京/特伦托综合讯)中欧经贸关系持续紧张,意大利工业部长乌尔索星期六(5月25日)呼吁欧盟效仿美国,通过对中国产品征收关税,以保护自身工业。中国媒体同日则引述业内人士披露,中国可能针对从欧盟进口猪肉展开反倾销调查。 彭博社认为,这是中国通过官媒暗示北京准备对欧盟采取报复性措施……}

\entryitemWithDescription{台艺人接连表态支持一中 赖清德盼台湾人谅解}{https://www.zaobao.com/news/china/story20240526-3726349}{民进党星期天(5月26日)举行党代表及党职选举,台湾新任总统赖清德以民进党党员身份回到台南投票。(民进党提供) (台北综合讯)随着台湾艺人接二连三表态支持一个中国的立场,台湾新任总统赖清德表示,台湾的文化工作者到中国大陆被迫政治表态,讲什么内容固然是一回事,但更重要是他们内心的想法,``我们应该给予谅解,给予体谅''……}

\entryitemWithDescription{特稿:两岸关系恶化 陆配成夹心人}{https://www.zaobao.com/news/china/story20240526-3650663}{来自哈尔滨的姜凤(左)最近与丈夫廖登镇和女儿廖嫣然结伴旅游,到中国大陆内蒙古欣赏呼伦贝尔大草原的风光。(受访者提供) 台湾总统赖清德上台后两岸关系再恶化,40多万中国大陆配偶(简称陆配)成夹心人。受访陆配感叹两岸关系前景难测,最担心爆发台海冲突,下一代被迫上战场。 赖清德5月20日的就职演说阐述``中华民国与中华人民共和国互不隶属''、高举台湾``事实独立'',引起大陆强烈反弹,令台海局势升温……}

\entryitemWithDescription{分析:赖清德就职演说使两岸敌意螺旋升高}{https://www.zaobao.com/news/china/story20240526-3720270}{台湾总统赖清德的就职演说引发北京强烈不满,对台发动一系列文攻武吓。图为北京民众5月24日阅读一份报道解放军在台湾周边进行军事演习的报纸。(法新社) 台湾总统赖清德的就职演说引发北京强烈不满,对台发动一系列文攻武吓;赖清德政府事后则重申两岸政策并未改变,且富有善意。台湾学者分析,这次事件凸显台湾新政府与北京之间存在严重认知落差,未来四年两岸敌意恐进一步螺旋升高……}

\entryitemWithDescription{中国特稿:美禁令虽拉响警报 TikTok中国电商不畏触礁破浪求生}{https://www.zaobao.com/news/china/story20240526-3715711}{(插图/何汉聪) 广州电商培训机构美迪电商教育的讲师4月7日在给学员讲解社交平台上的营销技巧。(法新社) 随着社交电商这一商业模式的崛起,中国互联网巨头字节跳动旗下的TikTok近年在全球快速扩大电商地盘。大批中国卖家涌入这一平台,通过短视频与直播带货等路径,将产品直接销往全球。但发展势头强劲的TikTok面临将被美国全面下架的危机,借助该平台出海的企业,发展前景因此出现诸多不确定因素……}

\entryitemWithDescription{中美防长下周香会会晤 学者:坦诚对话有助划清不可逾越红线}{https://www.zaobao.com/news/china/story20240525-3723963}{中美防长将于下周在新加坡出席香格里拉对话期间举行会晤。(路透社) 台海局势升温之际,中美防长将于下周在新加坡出席香格里拉对话期间举行会晤。受访学者认为,中美在国防安全领域的对话正逐步走向稳定和常态化,坦诚对话有助于双方划清不可逾越的红线。 美国五角大楼星期五(5月24日)发文告宣布,美国国防部长奥斯汀将在下周访问新加坡,并出席香格里拉对话(简称香会)……}

\entryitemWithDescription{``香港间谍案''两男子明年2月受审 英国警方:另一英籍被告死因无可疑}{https://www.zaobao.com/news/china/story20240525-3723720}{63岁的香港驻伦敦经贸办行政经理袁松彪星期五(5月24日)离开伦敦中央刑事法院,他被指是间谍案主谋。(法新社) 38岁的移英港人卫志良星期五(5月24日)离开伦敦中央刑事法院,他被指涉嫌帮助香港当局在英国收集情报。(法新社) (伦敦/香港综合讯)被控在英国为香港当局收集情报的两名被告,将于明年2月受审;控方终止对开庭前身亡的另一名英国籍被告的诉讼,英国警方称他的死因并无可疑之处……}

\entryitemWithDescription{台湾邦交国危地马拉夏威夷果等产品被中国大陆拒绝入境}{https://www.zaobao.com/news/china/story20240525-3723838}{(危地马拉城综合讯)危地马拉总统阿雷瓦洛星期五(5月24日)称,中国大陆拒绝了来自该国的货物入境,推测原因可能与危地马拉和台湾的邦交关系有关。 据路透社报道,危地马拉出口商联盟星期四通报,目前至少有七个集装箱的夏威夷果被扣留,无法进入中国大陆。危地马拉出口商也接到中国大陆客户通知,称近期危地马拉咖啡和夏威夷果将被拒绝入境……}

\entryitemWithDescription{《愿荣光》发行商将从全球各音乐平台下架该歌曲}{https://www.zaobao.com/news/china/story20240525-3723578}{香港法院对反修例期间广为流传的歌曲《愿荣光归香港》批出禁制令后,歌曲创作团队接到发行商通知称,歌曲将从全球各大音乐平台下架。图为香港反修例抗争者2020年5月13日在香港一家商场内聚集并高唱《愿荣光归香港》。(法新社) (香港综合讯)香港法院对反修例期间广为流传的歌曲《愿荣光归香港》批出禁制令后,歌曲创作团队接到发行商通知称,歌曲将从全球各大音乐平台下架……}

\entryitemWithDescription{中国大陆海警首度公开在台东外海演练 官媒:对台本岛执法权的开始}{https://www.zaobao.com/news/china/story20240525-3723372}{中国大陆海警首次公布在台东外海演练的消息,其中一张配图可见台湾中央山脉。大陆官媒称,这说明演练距离台湾本岛非常近,传递了阻断台独人士外逃的信号。(``中国海警''微信公众号) (北京/台北综合讯)中国大陆海警首次公开在台东外海的演练,大陆官媒称,此举意味着大陆海警对台湾本岛明确执法权的开始……}

\entryitemWithDescription{台湾艺人阿信称``我们中国人''登热搜 两岸舆论两极}{https://www.zaobao.com/news/china/story20240525-3723190}{台湾乐队五月天主唱阿信星期五(5月24日)晚在北京鸟巢举行的演唱会上,与台下观众互动时脱口说出``我们中国人''。(网络视频截图) (北京/台北综合讯)两岸关系在台湾新总统赖清德宣誓就职后趋紧,众多台湾艺人的政治立场受到关注。台湾乐队五月天主唱阿信在北京开唱时,脱口说出``我们中国人'',引发两岸网民热议……}

\entryitemWithDescription{中国商人郭文贵被指欺骗10亿美元 购买豪宅游艇豪车}{https://www.zaobao.com/news/china/story20240525-3723385}{流亡美国的中国商人郭文贵被控欺诈一案开庭,美国检方指郭文贵骗取投资者超过10亿美元(13.5亿新元)资金,用于购买豪宅、游艇、豪车等。图为郭文贵2018年11月在纽约举行的新闻发布会上讲话。(路透社档案照) (纽约综合讯)流亡美国的中国商人郭文贵被控欺诈一案开庭,美国检方指郭文贵骗取投资者超过10亿美元(13.5亿新元)资金,用于购买豪宅、游艇、豪车等……}

\entryitemWithDescription{港媒:汪文斌将卸任中国外交部新闻发言人}{https://www.zaobao.com/news/china/story20240525-3723132}{据香港媒体报道,53岁的汪文斌将卸任中国外交部新闻发言人。(互联网) (北京综合讯)中国外交部新闻司副司长、发言人汪文斌据报将卸任,可能外派担任大使或转岗其他要职。 汪文斌星期五(5月24日)主持外交部例行记者会,他在记者会最后说,``感谢大家出席外交部例行记者会,我们再见!'' 网上流传的视频显示,散会后,汪文斌走向记者席,与现场一些记者握手……}

\entryitemWithDescription{新闻人间:何伯何太:忘年恋成全港人气王}{https://www.zaobao.com/news/china/story20240525-3720621}{最近在香港社会闹得沸沸扬扬的新闻,既不是英国港人间谍案,也不是港府力推盛事活动,而是一位何姓老人家与小30岁的女子``忘年恋''的故事。从半个月前开始,整个香港网络上都被这件事刷屏了。 何伯何太的故事是这样的:76岁的香港何伯与一名40多岁的中国大陆女子一见钟情。两人认识数月后,迅速发展成为亲密关系,不到一个月就闪婚……}

\entryitemWithDescription{韩咏红:台海成中美角力场}{https://www.zaobao.com/news/china/story20240525-3721408}{5月24日中国大陆报章报道解放军围台军演。(法新社) 中国大陆围台军演``联合利剑---2024A''在进行中,台湾立法院继续混乱如战场。星期五(5月24日)一早,对立阵营的立法委员就在会场内爆发肢体冲突,还有女立委扭成一团;会场外,傍晚时分已聚集了数以万计执政民进党的支持者,民进党因立委人数不够,就发起民众力量反对在野党力推的国会改革法案,称这是``黑箱''……}

\entryitemWithDescription{台立法院二读通过听证权修正案 官员如虚伪陈述将被追责}{https://www.zaobao.com/news/china/story20240525-3721434}{民进党立委(前三排)抗议在野的国民党和民众党(后排)推动立法院改革法案扩大权力。(彭博社) 立法院会审查立法改革法案时,民进党支持者在立法院外举着``不讨论,就没有民主''的标语……}

\entryitemWithDescription{解放军在台东外海进行挂实弹模拟打击 学者:北京准备打常态化消耗战}{https://www.zaobao.com/news/china/story20240524-3719986}{台湾海巡署星期五(5月24日)发布的照片显示,中国大陆军舰在台东绿岛东南方不远处航行。解放军东部战区宣称,当天空军与海军在台东外海,对重要目标实施模拟打击。(法新社) 解放军东部战区公布攻台模拟动画视频,显示台北市、台中市、高雄市和花莲县等地被锁定。(东部战区微博视频截图) 中国人民解放军针对台湾总统赖清德``新两国论''展开第二天围台军演,包括在台东外海进行挂实弹模拟打击……}

\entryitemWithDescription{英防长指中国提供俄罗斯致命援助 中使馆:无中生有}{https://www.zaobao.com/news/china/story20240524-3720335}{英国国防部长沙普斯指中国提供俄罗斯``致命援助'',中国驻英国大使馆发言人回应称,有关言论``无中生有''。图为5月15日行走在伦敦市中心的沙普斯。(法新社) (伦敦综合讯)针对英国国防部长沙普斯指中国提供俄罗斯``致命援助'',中国驻英国大使馆发言人回应称,有关言论``无中生有''……}

\entryitemWithDescription{香港一名女警据报入选中国第四批预备航天员}{https://www.zaobao.com/news/china/story20240524-3720624}{(香港综合讯)中国正在推进第四批预备航天员选拨工作,并首次在香港与澳门选拔载荷专家。港媒报道称,代表香港入选的是来自香港警队的总督察黎家盈。 综合《星岛日报》与am730报道,黎家盈将是香港首名女航天载荷专家,她已从警队借调香港保安局,去年开始在北京接受训练,具备航天员资格,目前正等候北京相关部门的正式公布……}