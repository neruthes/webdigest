\entryitemWithDescription{台专家:为防擦枪走火 气球侦察或成常态各国须建军事互信}{https://www.zaobao.com/news/china/story20230215-1363232}{台湾国防安全专家指出,中美气球事件反映出各国对探空气球被更广泛运用在军事探勘而感到的不安。这种军事科技的急速发展,破坏了自冷战以来的军事互信,加深地缘强权之间的猜忌。 中美气球事件近期再度挑动地缘政治局势,也显示随着技术的成熟,军事气球实践洲际飞行已非问题,并能大幅降低情搜成本,或许将成为新的高空侦察常态……}

\entryitemWithDescription{台官方: 台海未发现大陆侦察气球}{https://www.zaobao.com/news/china/story20230215-1363233}{中国气球2月初被发现入侵美国领空,随后遭美军击落,气球碎片掉落在南卡罗来纳州外海。图为美国海军人员2月10日准备将捞起的气球残骸运送往弗吉尼亚州的联邦调查局实验室进行分析。 (美国海军/路透社) 台湾官方说,未在台海周边空域发现任何中国大陆的侦察气球,所出现的大多为用于气象探测的气球……}

\entryitemWithDescription{李家超:港府仍希望 今或明年完成第23条立法}{https://www.zaobao.com/news/china/story20230215-1363234}{香港特首李家超重申特区政府的立场没变,仍希望在今年或明年完成《基本法》第23条立法。 根据香港政府新闻网站,李家超星期二(2月14日)在行政会议前回答媒体提问时说,为《基本法》第23条立法是港府的宪制责任。他认为,各方可能对第23条立法所针对的问题严重性未必全面掌握,并强调国家安全风险千变万化,目前国际关系复杂,国安风险仍然有潜伏在香港的可能性……}

\entryitemWithDescription{早 说}{https://www.zaobao.com/news/china/story20230215-1363236}{面对强权争霸的地缘政治情势,中华民国和国民党都没有``选边站'' 的问题,只有与盟友及国际友人共同维系台海和平与稳定的强烈意愿。 ------国民党国际事务部主任兼驻美代表黄介正,据报在农历新年期间走访20几个美国国会议员办公室。他星期二(2月14日)分析说,美国根本不在意台湾的政党是不是玩表面的``亲美比赛'',坦诚分析观点角度,具体扩大共同利益,才是真正有信赖关系的盟友……}

\entryitemWithDescription{中美气球事件能否在慕尼黑翻篇?}{https://www.zaobao.com/news/china/story20230215-1363237}{中国最高外交官员王毅星期二启程前往欧洲,此行将访问法国、意大利、匈牙利和俄罗斯,也会到德国出席慕尼黑安全会议。 除了中欧关系和俄乌战争进入一周年之际的中俄互动,王毅欧洲行的另一看点是:在气球事件后,中美外交高层是否会在慕尼黑实现原本计划在北京进行的面对面会晤。 一只飘移到美国上空的中国气球,2月以来成为中美交锋焦点,不仅搅黄美国国务卿布林肯访华行,还引发两国新一轮外交博弈……}

\entryitemWithDescription{台湾下周开放港澳居民赴台自由行 大陆有意磋商恢复台农渔产品进口}{https://www.zaobao.com/news/china/story20230215-1363238}{针对何时解禁让陆客访台,陆委会称大陆``突然完全解封'',让台方对大陆的疫情判断存在不确定,未来将按照中央流行疫情指挥中心的判断来处理。 两岸官方都释出春暖花开的和缓信号,台湾陆委会宣布下周一(2月20日)起开放港澳居民自由行赴台观光。中国大陆国台办也表态,愿与台湾共同努力,为恢复台湾农渔产品输入大陆提供帮助……}

\entryitemWithDescription{盼回国民党参加党内初选 郭台铭表态有意竞选2024年总统选举}{https://www.zaobao.com/news/china/story20230215-1363239}{鸿海集团创办人郭台铭星期二(2月14日)与台湾在野国民党前立法院长王金平见面后明确表示,希望回到国民党参加党内总统初选,代表国民党参选2024年总统。 郭台铭2019年曾以荣誉党员回归国民党,但参与党内总统初选落败,愤而宣布退选,未支持国民党提名的总统候选人韩国瑜。此次卷土重来,他不讳言称,四年前退党是年轻气盛、一时冲动,现在若能够为台湾来做事,他不拘任何形式,义不容辞……}

\entryitemWithDescription{甘肃山区梯田 大雪后如仙境}{https://www.zaobao.com/news/china/story20230215-1363240}{甘肃省东乡族自治县山乡在2月14日一场大雪后,银装素裹,云雾缭绕,宛如仙境。图为当天拍摄的东乡族自治县山区梯田雪景……}

\entryitemWithDescription{星云法师遗体火化后 留下舍利子不计其数}{https://www.zaobao.com/news/china/story20230215-1363241}{据台湾佛光山表示,星云法师在2月13日荼毘圆满后,遗骨烧出宛如珍珠的结晶舍利子,色泽乳白光滑,数量众多。(香港中通社) 台湾佛光山开山宗长星云法师生前预告``我没有舍利子'',但其遗体星期一(2月13日)火化后,留下的舍利子多到难以计算。 综合《联合报》《中国时报》和壹苹新闻网等台媒报道,星云法师遗体荼毘(火化)仪式13日在台南大仙寺举行……}

\entryitemWithDescription{中国足协主席陈戌源 涉严重违纪违法被查}{https://www.zaobao.com/news/china/story20230215-1363242}{中国足球协会主席、党委副书记陈戌源涉嫌严重违纪违法而被调查,是自去年11月以来第四名被查的中国足坛重要人物。 据湖北省纪委监委星期二(2月14日)消息,陈戌源涉嫌严重违纪违法,目前正接受中央纪委国家监委驻中国体育总局纪检监察组和湖北省监委审查调查……}

\entryitemWithDescription{青岛海域所发现飞行物 中国隔天军演料已击落}{https://www.zaobao.com/news/china/story20230214-1362814}{中国军事评论员研判,山东省青岛市官方通报发现不明飞行物,并声称准备击落,与解放军在黄海北部海域进行实弹射击有一定关联,并相信不明飞行物大概已被击落。 中国外交部星期一(2月13日)指责单是去年以来,美国高空气球就10余次非法飞越中国领空,白宫则否认北京的指控……}

\entryitemWithDescription{蔡英文总统到场颁褒扬令 5­万人含泪跪送星云法师最后一程}{https://www.zaobao.com/news/china/story20230214-1362815}{台湾佛光山开山宗长星云法师2月13日圆寂赞颂典礼后,遗体移往台南白河大仙寺火化。数万民众沿途含泪跪送,送他最后一程。(人间社提供) 逾5­万人在昨天台湾佛光山开山宗长星云法师的圆寂赞颂典礼后,送他最后一程。遗体火化后将永久安奉在佛光山万寿园。 台湾佛光山星期一(2月13日)上午为开山宗长星云法师举行圆寂(离世)赞颂典礼,逾5­万人送他最后一程,总统蔡英文也到场颁发褒扬令……}

\entryitemWithDescription{萧美琴:蔡英文出访邦交国是常态 未来会适时规划}{https://www.zaobao.com/news/china/story20230214-1362818}{台湾驻美代表萧美琴表示,台湾总统蔡英文在冠病疫情之前出访邦交国是常态,未来在适当时间点也会规划。 萧美琴星期一与立法院外交及国防委员会立委闭门茶叙前,被媒体问到蔡英文会否在下半年出访时过境美国。她说,蔡英文出访邦交国是常态,过去都有惯例,``未来适当时间点也会规划,但目前并没有很细部、确切的方案''……}

\entryitemWithDescription{称在港说普通话被歧视 大陆网红被批``为流量抹黑香港''}{https://www.zaobao.com/news/china/story20230214-1362819}{(香港综合讯)中国大陆网红发视频展示在香港只说普通话受到不公正待遇后,有不少大陆旅客反驳歧视论,也有媒体人批评网红为了流量抹黑香港,并呼吁港府要对大陆民众``讲好香港故事''。 拥有逾98万粉丝的抖音博主``沪漂女孩艺轩'',在2月7日的短视频中展示她屡次因只说普通话,在香港旺角、尖沙咀等旅游热门地受到服务人员不公正对待后,香港《星岛日报》2月10日到深圳采访市民对香港旅游的印象……}

\entryitemWithDescription{早说}{https://www.zaobao.com/news/china/story20230214-1362820}{本年度不止是23条立法不列入立法议程,连相关公众咨询亦不宜展开。 ------中国全国侨联副主席卢文端星期一(2月13日)在《明报》撰文,列举四个不宜在今年推进《基本法》23条立法的原因,包括应当先拼经济、加强联通世界等。他还提醒,台湾2024总统大选临近,香港如果在此时展开23条立法引发争议,极有可能又让民进党``捡到枪'',借机抹黑``一国两制'',称这是绝不能容许出现的局面……}

\entryitemWithDescription{戴庆成:简约公屋揭示港府管治危机}{https://www.zaobao.com/news/china/story20230214-1362821}{香港新一届特区政府在中央政府的大力支持下,上任以来不断触碰社会``老大难''问题,并提出一系列针对性措施,获得了社会舆论的普遍认可。然而,近来当局未顾及细节问题就匆促推出某些政策,也暴露出本届政府在管治方面存在一些问题。 这里说的是``简约公屋''。九七后住屋成为香港社会的头等问题……}

\entryitemWithDescription{调查:终身无孩率快速上升 中国育龄女性生育意愿不断降低}{https://www.zaobao.com/news/china/story20230214-1362823}{中国有相当比率的经济独立女性摒弃传统观念,认为结婚与生育并非人生必经之路,婚姻和生育是对女性的某种剥削。图为2月10日在北京王府井商业大街消费购物的女性。(彭博社) 中国人民大学新闻学院教授周小普接受《联合早报》采访时说,中国实施计划生育政策以来,独生子女现象改变了中国的家庭结构与关系,也影响了新生代的家庭与婚恋观念,他们的自我意识强,更关注自我价值。 中国年轻一代婚育观念正发生改变……}

\entryitemWithDescription{赞比亚反对中国要求让世行参与债务重组}{https://www.zaobao.com/news/china/story20230214-1362824}{中国要求让世界银行和其他多边贷款机构参与赞比亚的债务重组,遭到赞比亚的反对。赞比亚财长警告,这样的要求干扰债务减免进程,正阻碍该国的经济复苏。 据英国《金融时报》星期一(2月13日)报道,赞比亚财长穆索科图瓦内(Situmbeko Musokotwane)受访时说,今年是完成赞比亚约130亿美元(约173亿新元)外债重组的关键时刻,但中国政府提出的要求已对债务重组的谈判形成了一种干扰……}

\entryitemWithDescription{美媒:跨国企业高管陆续重返中国寻商机}{https://www.zaobao.com/news/china/story20230214-1362825}{(北京综合讯)随着中国重新开放,跨国公司的高层管理人员正陆续重返中国,以寻找重新开放所带来的商机。 据《华尔街日报》报道,大众汽车首席执行官奥博穆在1月底至2月初访问了中国。他也是1月初中国取消大部分防疫限制以来,首批访华的大型跨国公司高管之一。 报道引述知情人士透露,预计苹果公司首席执行官库克和辉瑞公司首席执行官艾伯乐,将在下个月访问中国。马赛地---奔驰集团说,集团董事长康林松也计划访问中国……}

\entryitemWithDescription{在日诞生中国大熊猫``香香''将回国}{https://www.zaobao.com/news/china/story20230214-1362826}{在日本东京上野动物园出生的5岁雌性大熊猫``香香'',将在下星期二(2月21日)归还中国。香香是上野动物园近29年来首次诞生的熊猫宝宝。 据日本共同社报道,上野动物园园长福田丰对香香的离开表示祝福,``虽然很不舍,但希望香香能努力尽快适应环境,找到好伴侣、留下后代''。 香香2017年6月出生,父母是中国旅日大熊貓``力力''和``真真''。它在2017年12月首次对公众亮相便大受欢迎……}