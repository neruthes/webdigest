\entryitemWithDescription{于泽远:普京给中国送大礼}{https://www.zaobao.com/news/china/story20240520-3688158}{俄罗斯总统普京上周访华期间,中俄签署了新的联合声明。与近年中俄签署的几份联合声明相比,新的联合声明显示中俄关系更加紧密,俄罗斯更加配合中国的战略需求,在经济、安全和政治等领域更加依赖中国。 这份中俄关于深化新时代全面战略协作伙伴关系的联合声明称,中俄关系正处于历史最好水平,反对任何阻挠两国关系正常发展,干涉两国内部事务,限制两国经济、技术、国际空间的企图……}

\entryitemWithDescription{针对欧美台日产共聚聚甲醛 北京宣布反倾销调查}{https://www.zaobao.com/news/china/story20240520-3688402}{中国大陆商务部星期天(5月19日)宣布对原产于欧盟、美国、台湾和日本的进口化学品共聚聚甲醛,展开反倾销调查。图为一名解放军士兵3月3日在北京天安门城楼前站岗。(法新社) 中国大陆商务部星期天(5月19日)宣布对原产于欧盟、美国、台湾和日本的进口化学品共聚聚甲醛,展开最长可达一年半的反倾销调查……}

\entryitemWithDescription{民众党号召民众上街 批民进党开八年``空头支票''}{https://www.zaobao.com/news/china/story20240519-3688548}{数千名示威者在民进党台北市总部外,手举番石榴(台湾称``芭乐''),抗议民进党成了只会开``芭乐票''(空头支票)的``芭乐党'',未兑现包括媒体、司法、立法院和宪政等领域的改革。(路透社) (台北综合讯)台湾民进党籍候任总统赖清德就职前一天,数千名示威者在民进党台北市总部外,抗议民进党执政八年跳票,多项改革原地踏步……}

\entryitemWithDescription{迟了近百年 美国宾大向林徽因追授建筑学学位}{https://www.zaobao.com/news/china/story20240519-3688415}{美国宾夕法尼亚大学星期六(5月18日)向林徽因的外孙女于葵(左)颁发林徽因的建筑学学士学位证书。(互联网) (费城/北京综合讯)美国宾夕法尼亚大学正式向中国建筑师、文学作家林徽因追授建筑学学士学位,以表彰她作为中国现代建筑先驱的卓越贡献……}

\entryitemWithDescription{两艘中国军舰抵达柬埔寨 展开两国有史以来最大规模的联合军演}{https://www.zaobao.com/news/china/story20240519-3688470}{中国解放军两艘海军军舰戚继光舰(右)和井冈山舰(左)星期天(5月19日)停靠在柬埔寨的西哈努克港。(法新社) (金边综合讯)中国两艘军舰抵达柬埔寨,参与两国间有史以来最大规模的联合军演。 法新社报道,中国解放军两艘海军军舰戚继光舰、井冈山舰星期天(5月19日)停靠在柬埔寨西哈努克港,这是中柬之间``金龙''军事演习的一部分。 戚继光号悬挂着``重洋远渡万里邻,志同友邦情义金''的布条……}

\entryitemWithDescription{广西多地遭暴雨袭击 南宁部分城区内涝严重多车被淹}{https://www.zaobao.com/news/china/story20240519-3688154}{中国广西自治区受高空槽、低涡切变和地面冷空气共同影响,星期六(18日)上午8时至星期天(19日)上午10时,出现一次大范围降雨过程。图为工作人员星期天在南宁东葛路开展排水作业。(新华社) (南宁综合讯)中国广西自治区不少地区在周末遭遇暴雨袭击,其中首府南宁经历一夜暴雨后,部分城区内涝严重,多辆车子被淹……}

\entryitemWithDescription{赖清德重申维持台海现状 推创意外交邀邦交国元首钓虾}{https://www.zaobao.com/news/china/story20240519-3687618}{台湾候任总统赖清德(左三)和台湾候任副总统萧美琴(右一)星期天(5月19日)在台北芝山钓虾场,邀请已经抵台的邦交国元首与眷属、官员等超过20位外宾体验台湾流行的钓虾活动。伊斯瓦蒂尼王国国王姆斯瓦蒂三世(左一)获得钓虾竞赛第一名。(法新社) (台北综合讯)台湾候任总统赖清德、副总统萧美琴在就职典礼前一天施展创意外交,邀请已经抵台的邦交国元首和官员体验台湾流行的钓虾活动……}

\entryitemWithDescription{特稿:降薪裁员不断 投行对中国大学生吸引力减弱}{https://www.zaobao.com/news/china/story20240519-3670719}{彭博社引述知情人士报道称,中金公司计划让一些高级职员降职、减薪。图为北京白领在中金公司总部所在的国贸写字楼二座外聊天。(彭博社) 今年4月,24岁的王轩(化名)在一家北京的投行实习了半年后,人力资源部明确告诉他没有留用机会。听到这个消息后,他比自己想得要平静:``早就料到是这样。'' 不祥的预感从去年就开始了……}

\entryitemWithDescription{蔡英文:北京若发动战争 经济发展或迟滞数十年}{https://www.zaobao.com/news/china/story20240519-3687437}{台湾总统府官网星期六(5月18日)发布总统蔡英文接受英国广播公司(BBC)专访的内容。(台湾总统府官网) (台北综合讯)台湾总统蔡英文卸任前接受英国媒体采访时说,不能排除两岸可能发生军事冲突,并警告如果北京武力攻打台湾,可能导致中国大陆的经济发展迟滞数十年。 台湾总统府官网星期六(5月18日)发布蔡英文接受英国广播公司(BBC)专访的内容……}

\entryitemWithDescription{台湾特稿:台湾停电还是缺电?非核绿能梦难圆}{https://www.zaobao.com/news/china/story20240519-3676198}{民众说:``从疫情恢复后,物价就一直涨,现在电价又要涨,物价只会跟着再往上涨。''图为台北市南门市场。(庄慧良摄) 4月15日,台湾北部差点大停电,再次暴露台湾供电不稳乃至缺电的危机,蔡英文政府坚称是电线老旧等问题引起的停电,而非缺电;企业界则非常忧心能源结构调整及其落后的进度,无法因应耗电量极大的半导体和人工智能产业的需求……}

\entryitemWithDescription{中国海警舰艇编队在黄岩岛海域训练}{https://www.zaobao.com/news/china/story20240519-3685668}{中国海警编队星期五(5月17日)在南中国海黄岩岛海域进行舰艇编队运动训练。(无人机照片)(新华社) 继菲律宾民间组织上百艘渔船前往黄岩岛水域后,中国披露海警编队在该海域维持的存在,官媒面向全球播出有关南中国海的纪录片,重申中国有关主权主张。 新华社星期五(5月17日)夜间发布一组照片,称中国海警3502编队在南中国海黄岩岛(菲律宾称斯卡伯勒浅滩)海域进行舰艇编队运动训练……}

\entryitemWithDescription{上海编制无需安全评估的数据清单 便于企业速传海外}{https://www.zaobao.com/news/china/story20240518-3684756}{上海编制了一份无需安全评估的数据清单,便于企业快速传输海外。图为民众游览上海外滩滨江。(法新社) (上海综合讯)上海编制了一份无需安全评估的数据清单,便于企业快速传输海外。此举被视为配合中国官方吸引外资的努力,以提振乏力的经济。 路透社星期五(5月17日)引述上海政府文件,指当局已编制了首批涵盖智能网联汽车、公募基金、生物医药三个领域的``一般数据''。这些数据将在传输过程中面对最少的监管……}

\entryitemWithDescription{台新任官员:赖清德520演说将承诺维持两岸现状}{https://www.zaobao.com/news/china/story20240518-3685002}{(台北综合讯)台湾即将上任的新一届政府官员透露,候任总统赖清德将在就职演说中,承诺维持两岸现状,``确保稳定现状不受到侵蚀'',并聚焦稳健、自信、负责、团结四大精神。 综合《自由时报》、中央社和路透社报道,这名不愿透露姓名的赖政府国安高层在一个内部简报会上说,赖清德将延续及维持胜选当天的谈话主轴,以及选举期间阐述的``民主和平繁荣''路线,新政府未来重大施政也将突出上述四大精神……}

\entryitemWithDescription{台立院冲突后 赖清德盼回归理性讨论}{https://www.zaobao.com/news/china/story20240518-3685196}{台湾立法院内星期五(5月17日)爆发一系列肢体冲突后,绿营支持者当晚聚集在立法院外举牌喊口号,支持民进党立委的抗争。(路透社) (台北综合讯)台湾朝野就立法院改革相关修正草案爆发激烈肢体冲突后,民进党主席兼候任总统赖清德强调,自己将恪守宪法,并期盼立法院朝野党团能回归理性讨论。 赖清德星期六(5月18日)凌晨在脸书发文说,他星期五(5月17日)晚和大家一样担心立法院,心系台湾的未来……}

\entryitemWithDescription{中国太空旅游飞行器2027年首飞}{https://www.zaobao.com/news/china/story20240518-3684662}{(北京讯)中国一家商业航天发射企业宣布,该公司研制的太空旅游飞行器将于2027年首飞,并于2028年开启载人太空边缘旅游计划。 ``国际火箭发射''官方微信公众号星期四(5月16日)发布了中科宇航探索技术有限公司(简称中科宇航)上述宣布。 据介绍,中科宇航的太空旅行飞行器采用单级火箭和旅游舱的组合形式,旅游舱配备了四扇全景舷窗,每次飞行可搭乘七名乘客……}

\entryitemWithDescription{新闻人间:北京管得住台湾名嘴的嘴?}{https://www.zaobao.com/news/china/story20240518-3677809}{新闻人间:北京管得住台湾名嘴的嘴?(联合早报制图) 表演能力不输专业演员的台湾名嘴摊上大事了。 中国大陆国台办发言人陈斌华5月15日在例行新闻发布会上回应媒体提问时,点名了北京要惩戒的五个台湾名嘴和其家属。罪名是``罔顾大陆发展进步的事实,蓄意编造有关大陆的虚假、负面信息'',大肆传播错误言论,蒙蔽部分台湾民众,挑动两岸敌意对立。 黑名单五人是:黄世聪、李正皓、王义川、于北辰、刘宝杰……}

\entryitemWithDescription{庄慧良:赖清德的瓶中信}{https://www.zaobao.com/news/china/story20240518-3682238}{台湾候任总统赖清德即将在下星期一(5月20日)正式就任;在``登基''大典前,他透过官方LINE推出``写一封瓶中信''活动,邀请民众写下对台湾未来的期许,希望 大家``写瓶中信,和赖清德、萧美琴一起共织台湾,民主前行''……}

\entryitemWithDescription{学者:中美防长若月底都在新加坡 极可能在香会正式对话}{https://www.zaobao.com/news/china/story20240517-3681824}{中国防长董军与美国防长奥斯汀据传本月底将在新加坡举行的香格里拉对话会见面;如若属实,这将是中美防长首次面对面接触。受访学者分析,中美目前已形成政治共识,不因有争议冲突而中断官方对话沟通。中美防长如果月底都在新加坡,很有可能举行正式对话。 奥斯汀与去年底出任防长的董军,上月中已举行视频会议并首次通话。美国众议院前议长佩洛西前年8月访台,中美防长正式对话因此中断长达近一年半……}

\entryitemWithDescription{中国买家据报对北极最后一块战略土地表兴趣}{https://www.zaobao.com/news/china/story20240517-3681699}{斯瓦尔巴最后一幅私有土地待售,拥有该地主权的挪威不希望地皮落入中国人之手。(联合早报制图) (奥斯陆综合电)中国买家据报有意购取北极群岛最后一幅私有战略土地,但拥有该地主权的挪威并不希望地皮落入外国之手。 综合彭博社和法新社报道,这块土地面积达60平方公里,大小跟曼哈顿差不多,离人烟稀少的朗伊尔城约64公里,有5公里长的海岸线,遍地高山冰川,是北极野生动物的家园,但没有基础设施……}

\entryitemWithDescription{指涉新疆强迫劳动 美国禁止26家中企进口产品}{https://www.zaobao.com/news/china/story20240517-3681943}{此照片摄于2021年4月1日,显示新疆维吾尔自治区一家棉质布料厂的工人在生产线上工作。(路透社档案照) (伦敦/巴黎综合讯)美国星期四(5月16日)宣布将26家中国企业纳入禁止进口名单,理由是这些企业生产产品的工厂涉嫌强迫新疆维吾尔人劳动。 综合路透社与法新社报道,26家中国公司被列入《防止维吾尔人强迫劳动法》实体清单,自星期五(17日)开始生效……}

\entryitemWithDescription{台湾立法院爆发肢体冲突 女立委遭扑倒抱腰袭臀 男立委被掌掴}{https://www.zaobao.com/news/china/story20240517-3678667}{民进党立委锺佳滨(右二)爬上立法院主席台,把阻挡他的国民党女立委陈菁徽扑倒在地,令旁观的立法院长韩国瑜(左二)大吃一惊。(法新社) 民进党立委郭国文(左)被指掌掴民众党团总召黄国昌。(取自Instagram) 台湾立法院星期五(5月17日)爆发一系列肢体冲突,国民党女立委陈菁徽遭民进党男立委锺佳滨扑倒抱腰袭臀,民众党团总召黄国昌遭民进党立委郭国文掌掴……}