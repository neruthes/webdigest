\entryitemWithDescription{陈婧:当``胖猫之死''占据热搜}{https://www.zaobao.com/news/china/story20240509-3620158}{上周和朋友聚餐时,一个中国女生刷着手机,疑惑地发问。她向我们展示手机屏幕,高居热搜榜首的是``胖猫事件''。 接下来几天,微博热搜榜更被``胖猫''全面占据。从``胖猫姐姐发声'',到``胖猫聊天记录'',再到``胖猫前女友回应''\ldots\ldots 一时间铺天盖地。即便在点进热搜,了解了胖猫事件的来龙去脉后,也很难不为这起事件的超高热度感到费解……}

\entryitemWithDescription{黄循财:``小院高墙''院子不断变大将伤及世界}{https://www.zaobao.com/news/china/story20240509-3620225}{副总理兼财政部长黄循财警告,在美国对中国采取``小院高墙''的出口限制下,院子若不断变大而导致技术分岔,将给世界带来伤害,因此将经济工具用于实现地缘政治目的时,须慎重。 美国对中国的科技制裁和出口限制,引发全球科技体系一分为二的担忧。黄循财星期一(5月6日)在接受英国《经济学人》专访,被问到新加坡如何应对时说:``我们希望出口限制的出台会得到谨慎权衡,因为如果存在国家安全问题,这是可以理解的……}

\entryitemWithDescription{中国官员:数日内或公布与菲律宾达成仁爱礁管控``新模式''协议证据}{https://www.zaobao.com/news/china/story20240508-3620105}{中菲本周持续就是否达成仁爱礁(菲律宾称阿云津礁)管控``新模式''展开舆论攻防。中国官员表示,北京可能在数日内公布与马尼拉达成协议的证据,即一段据说是今年初与菲律宾军方将领通话的录音。 受访学者研判,中菲之间应有达成仁爱礁管控新模式,但这并非两国官方正式协议,因此让菲律宾高层有底气加以否认……}

\entryitemWithDescription{台国防部:520前后秉持战备要求 严密关切对岸状况}{https://www.zaobao.com/news/china/story20240508-3619818}{(台北综合讯)台湾候任总统赖清德还有不到两周即上任,台湾国防部副部长柏鸿辉称,会秉持战备要求严密关切对岸状况。 综合路透社、台湾军事新闻通讯社和《联合报》报道,柏鸿辉星期三(5月8日)说,台军在联合情监侦上都有所掌握,会严密关切对岸可能做出的破坏地区和平举动,并与理念相同国家共同呼吁维护地区和平与稳定……}

\entryitemWithDescription{港屠龙小队队长曾向捐款的台湾人索要军火}{https://www.zaobao.com/news/china/story20240508-3619983}{(香港综合讯)香港反修例运动激进组织``屠龙小队'',涉嫌串谋在游行中设置炸弹案正在审讯中。据报,该组织队长、同时是控方证人的黄振强说,曾联络捐款的台湾人索要军火。 综合无线新闻网、文汇网、《英文虎报》等港媒报道,去年解散的香港民间人权阵线(民阵)在2019年12月10日举办国际人权日游行之前,香港警方在多个地区搜出手枪和子弹等武器……}

\entryitemWithDescription{福建舰完成首次海试 学者:动力系统合格,但还需检测}{https://www.zaobao.com/news/china/story20240508-3619901}{中国海军福建舰星期三(5月8日)完成为期八天的首次海试。(新华社) 中国海军航空母舰``福建号''星期三下午完成历时八天的首次海试。中国军事专家认为,首次海试的顺利完成表明福建舰动力系统合格,这是中国造船业的一次飞跃。 福建舰是中国海军的第三艘航母,首次采用平直甲板与电磁弹射系统,满载排水量8万余吨,是中国海军迄今最大的作战舰艇……}

\entryitemWithDescription{驻立陶宛台湾代表处改名?台湾外交部:名称是双方共识}{https://www.zaobao.com/news/china/story20240508-3619531}{(维尔纽斯/台北综合讯)针对立陶宛总统瑙塞达提出改变驻立陶宛台湾代表处的名称,以便化解与北京之间紧张关系的说法,台湾外交部回应表示,尊重立陶宛总统大选期间该国国内各方意见的表达,并希望各界了解有关名称是双方的共识……}

\entryitemWithDescription{TikTok曝光字节跳动创始人张一鸣长住新加坡}{https://www.zaobao.com/news/china/story20240508-3619051}{(华盛顿 / 新加坡综合讯)居住在新加坡的中国亿万富豪再添一人,中国短视频应用TikTok和母公司字节跳动对美国政府发出的诉状曝光,字节跳动创始人张一鸣长住新加坡,同时保留中国公民身份。 据彭博社报道,字节跳动在挑战华盛顿TikTok``不卖就禁''法令的诉讼中指明,今年41岁的张一鸣是居住在新加坡的中国公民,他目前拥有TikTok母公司约21\%的股份,估值超过400亿美元(542亿新元)……}

\entryitemWithDescription{上诉庭批禁制令 《愿荣光归香港》成香港回归以来第一首禁歌}{https://www.zaobao.com/news/china/story20240508-3618539}{香港反修例抗争者2020年5月13日在当地一家商场内聚集并高唱《愿荣光归香港》。(法新社档案照) (香港综合讯)香港上诉庭星期三(5月8日)裁定港府上诉得直,对反修例运动期间广为传唱的歌曲《愿荣光归香港》批出禁制令,使其成为香港九七回归以来第一首被法院裁定为禁歌的歌曲。 《愿荣光归香港》近年来在多场香港参与的国际或地区体育赛事上被主办方误播为香港``国歌'',引发港府强烈不满与批评……}

\entryitemWithDescription{中国或支持特斯拉测试``无人驾驶德士''}{https://www.zaobao.com/news/china/story20240508-3617865}{中国官媒称,中国政府或将先支持特斯拉``无人驾驶出租车''在中国国内进行测试和示范,但暂未完全批准其FSD在华全面落地。图为群众2023年9月4日在北京举办的中国国际服务贸易交易会上参观特斯拉汽车。(法新社) (上海综合讯)中国官媒称,特斯拉首席执行官马斯克访华期间提议,在中国测试无人驾驶德士(robotaxi)……}

\entryitemWithDescription{云南一医院发生持刀行凶案致两死21伤 中国网民高度关注}{https://www.zaobao.com/news/china/story20240507-3607408}{中国云南省镇雄县一所医院星期二(5月7日)发生持刀行凶案,造成两人死亡,21人受伤。警方说,落网嫌犯是当地村民。 新华社报道,云南省镇雄县公安局7日发布警情通报称,镇雄县城南医院当天早上11时37分许发生持刀行凶案,嫌犯是40岁的镇雄县泼机镇男村民,他在傍晚5时许于县内被捕。通报没有列明嫌犯的行凶动机……}

\entryitemWithDescription{中国官媒报道重现前防长魏凤和名字 分析:可能已安全着陆}{https://www.zaobao.com/news/china/story20240507-3607740}{中国央视《新闻联播》星期一(5月6日)在报道全国人大常委会前副委员长乌云其木格遗体在北京火化的消息时,画面中出现写有中国前防长魏凤和名字(左二花圈下排)的花圈。(央视官网视频截图) 曾任解放军战略军种火箭军首任司令员的中国前防长魏凤和,本周中国官媒新闻报道重现他的名字。学者研判,这意味着魏凤和可能已安全着陆,不会因为过去10个月军队整肃风暴而受到重罚……}

\entryitemWithDescription{蔡英文拟``赦扁'' 为赖清德拆弹或埋弹?}{https://www.zaobao.com/news/china/story20240507-3607283}{台湾前总统陈水扁因贪污洗钱等案,2008年11月被羁押,2015年获准保外就医。(自由时报) 台湾总统蔡英文将于5月20日卸任,台媒报道蔡英文有意在卸任前特赦涉贪的前总统陈水扁。总统府星期二(5月7日)重申依法办理的态度。外界则关注,蔡英文若真的特赦陈水扁,是为候任总统赖清德拆弹或埋弹? 蔡英文2016年就任以来一直避谈特赦陈水扁之事,而赖清德特赦陈水扁的态度非常明确……}

\entryitemWithDescription{中国驻俄大使:支持各方平等参与的乌克兰和平会议}{https://www.zaobao.com/news/china/story20240507-3605858}{(莫斯科综合讯)俄罗斯未获邀出席下月于瑞士举行的乌克兰相关会议后,中国驻俄罗斯大使张汉晖说,北京支持举行各方平等参与的乌克兰战争和平会议。 路透社报道,张汉晖星期二(5月7日)接受俄罗斯国家通讯社采访时说,中国支持适时召开俄乌双方认可、各方平等参与、公平讨论所有和平方案的国际会议。 瑞士将在下月15日至16日主持为期两天的高层会议,旨在实现乌克兰和平,但莫斯科没有受邀参加……}

\entryitemWithDescription{中国气候特使率团赴美开展气候变化会谈}{https://www.zaobao.com/news/china/story20240507-3605940}{中国气候变化事务特使刘振民将与美国总统国际气候政策高级顾问波德斯塔(John Podesta)进行会谈。这是两人上任以来的首次正式会议。(路透社档案照) (北京/华盛顿综合讯)中国气候变化事务特使刘振民5月7日至16日率团赴美,与美国总统国际气候政策高级顾问波德斯塔(John Podesta)进行会谈。这是中美新任气候特使上任以来的首次正式会议……}

\entryitemWithDescription{香飘飘包装讽日排海 直播间销售额暴增400倍}{https://www.zaobao.com/news/china/story20240506-3594777}{香飘飘旗下MECO果汁茶在日本销售的产品包装上,印制讽刺日本排海事件的标语,甚至直言:``请日本政客把核污水喝了''。(互联网) (北京 / 东京综合讯)中国速溶奶茶品牌香飘飘旗下MECO果汁茶在日本销售的产品包装上,被网民发现有嘲讽日本核污水排海事件的标语,引爆互联网热议。不少网民对此表示支持,香飘飘直播间销售额随后也大涨400倍……}

\entryitemWithDescription{中国拟强化对低收入人口的动态监测}{https://www.zaobao.com/news/china/story20240506-3595946}{(北京综合讯)中国正在加大对低收入群体的监测工作。中国民政部最新发文,要求各地要以``防风险''为目标,强化对低收入人口的动态监测工作。 据央视新闻报道,中国民政部办公厅印发通知,要求各地加强低收入人口认定和动态监测工作,进一步加强低收入人口动态监测和救助帮扶,发挥社会救助``保基本、防风险、促发展''功能作用……}

\entryitemWithDescription{台外长指徐巧芯爆料踩底线要提告}{https://www.zaobao.com/news/china/story20240506-3596028}{针对台湾外交部与捷克签订1000万美元援助乌克兰密约,在野的国民党立委徐巧芯质疑民进党政府图利特定厂商,并间接介入捷克当地政治。台湾外交部长吴钊燮痛批,徐巧芯踩到道德和外交底线,将对她提出告诉。 徐巧芯上周指台湾外交部与捷克签订一笔1000万美元(1350万新元)援助乌克兰的密约,指定``捷克卫生科技院''(CHTI)执行,其中300至400万美元要采购台湾医材,质疑究竟``谁能吃到这块肥肉……}