\entryitemWithDescription{港立法会通过爱国主义教育议案 学者:师资不足恐弄巧反拙}{https://www.zaobao.com/news/china/story20231102-1447511}{继中国全国人大常委会早前通过《爱国主义教育法》后,香港立法会星期四(11月2日)也通过一项无约束力的议员议案,要求港府完善推动爱国主义教育的政策。有受访学者认为,目前香港熟悉国情的教师不多,若太匆促推行爱国主义教育,恐怕会弄巧反拙。 这份名为``更深入和广泛地开展爱国主义教育''的议案,由中国全国人大常委、民建联议员李慧琼提出……}

\entryitemWithDescription{中国担任联合国安理会轮值主席 将聚焦推动以哈停火}{https://www.zaobao.com/news/china/story20231102-1447505}{中国从11月1日起担任联合国安理会本月轮值主席,中国常驻联合国代表张军在记者会上强调,应对以哈冲突仍是安理会本月头等要务。(香港中通社) (纽约/北京综合讯)中国在11月担任联合国安全理事会轮值主席,中国常驻联合国代表张军强调,应对以哈冲突仍是安理会本月头等要务,并指安理会必须采取``有意义的行动''……}

\entryitemWithDescription{柯侯郭同台活动``乔位置''八次 侯柯并坐变柯郭紧挨}{https://www.zaobao.com/news/china/story20231102-1447492}{经过八次座位安排调整,最终柯文哲(右一)与郭台铭(右二)比肩而坐,与侯友宜(左一)中间隔着葫芦寺董事长郭江源(左二)。(法新社) (台北综合讯)台湾民众党总统参选人柯文哲、国民党总统参选人侯友宜、独立参选台湾总统的鸿海集团创始人郭台铭同台出席一场宗教活动。三人抵达前,现场上演``乔位置大战'',座位安排被来回调整八次,更有蓝营人员亲手挪动位置……}

\entryitemWithDescription{中国生育为何难提升?分析:一胎化加剧性别不平等是主因}{https://www.zaobao.com/news/china/story20231102-1447480}{台湾中央研究院政治学研究所博士后研究员蔡仪侬(图)认为,中国大陆目前面临的人口问题,是1980年代实施的一胎化政策所造成。(缪宗翰摄) 中国大陆近年放宽生育政策,试图改善出生率下滑问题。但台湾学者认为,大陆1980年代为了实施一胎化政策,在农村广泛推行胎儿性别鉴定,以致男女失衡。这种政策长期实施,加剧了性别不平等、婚配不易等困境,是出生率难以提升的主因之一……}

\entryitemWithDescription{梅西中国行因李克强逝世取消 英媒:以示尊重}{https://www.zaobao.com/news/china/story20231102-1447449}{阿根廷``球王''梅西10月底在美国参加一场足球比赛。(路透社) (迈阿密/伦敦综合讯)阿根廷``球王''梅西原定本月到中国参加的足球巡回赛取消。英国媒体报道,巡回赛推广方NSN说,取消巡迴赛是为了对上星期逝世的中国前总理李克强表示尊重。 梅西目前效力于美国大联盟球会迈阿密国际。该球会星期三(11月1日)在官网发公告称,由于出现不可预见的情况,梅西此次的中国巡回赛取消……}

\entryitemWithDescription{美加军舰半年来第三次驶经台海 解放军海空兵力全程跟监警戒}{https://www.zaobao.com/news/china/story20231102-1447436}{(华盛顿/北京/台北综合讯)美国和加拿大军舰半年来第三次驶经台湾海峡,中国人民解放军东部战区再批美加公开炒作,并组织海空兵力全程跟监警戒。 美国海军第七舰队星期三(11月1日)在官网发布新闻稿,称美国驱逐舰``佩拉尔塔''号(USS Rafael Peralta)和加拿大皇家海军护卫舰``渥太华''号(HMCS Ottawa),当地时间星期三在台湾海峡开展例行驶经任务……}

\entryitemWithDescription{陈婧:有中国特色的万圣节}{https://www.zaobao.com/news/china/story20231102-1447286}{在上海这个中国最国际化的城市,过万圣节并不是什么新鲜事。但今年的万圣节庆祝活动格外``出圈'',不仅是因为参与人数多于往年,也因为涌现出太多令人拍案叫绝的造型。(法新社) 刚过去的这个周末,上海朋友圈里流传这么一句问候语,言下之意是:你参加万圣节巡游吗? 被称为``网红路''的巨鹿路是今年万圣节巡游的中心,这几天都是水泄不通。我也尝试过去凑热闹,但很快就被汹涌的人潮挤出圈外……}

\entryitemWithDescription{中国民众献花 李克强合肥故居外鲜花成山}{https://www.zaobao.com/news/china/story20231101-1447285}{中国原总理李克强10月27日凌晨在上海逝世,中国民众星期二(11月1日)继续到李克强位于安徽合肥的故居外献花,现场鲜花堆积如山。(彭博社) (合肥/北京综合讯)已故中国前总理李克强的遗体星期四(11月2日)将在北京火化。大批民众连日来到李克强位于安徽合肥的故居外献花。 李克强少年时曾居住在安徽省文史研究馆大院,也就是现在的合肥市红星路80号……}

\entryitemWithDescription{郭台铭低调递交连署书 蓝白对选总统仍无共识}{https://www.zaobao.com/news/china/story20231101-1447272}{鸿海集团创办人郭台铭(中)与副手赖佩霞(右三)11月1日下午前往台北市选委会递交台湾总统、副总统选举连署书。不过,郭台铭并未公布连署数量,仅表示会继续向前。(香港中通社) 台湾在野``蓝白合''前景依然不明朗,准备独立参选台湾总统的鸿海集团创办人郭台铭星期三(11月1日)下午递交连署书,但低调不受访。蓝白三主帅密会后,仍无法对如何决定正副总统人选达成共识,蓝营主张协商决定,白营仍坚持公开比赛……}

\entryitemWithDescription{黄永宏呼吁南中国海声索国达成渔业协定}{https://www.zaobao.com/news/china/story20231101-1447264}{南中国海局势近期升温,我国国防部长黄永宏医生呼吁声索国达成渔业协定,并加速《南中国海行为准则》磋商。 中国和菲律宾过去几个月在南中国海摩擦不断,两国船只上月还在阿云津礁(中国称仁爱礁)附近发生碰撞。 黄永宏星期三(11月1日)在北京接受媒体访问时,呼吁涉事各方为南中国海局势降温,并远离危险线和采取防范措施。 他说,各国应尽一切所能避免亚洲发生冲突……}

\entryitemWithDescription{中国互联网平台实施前台实名制}{https://www.zaobao.com/news/china/story20231101-1447253}{中国多家互联网平台宣布,将对有50万粉丝以上的社交媒体账号实施前台实名制,在个人账户的首页展示真实姓名。 中国主要的互联网平台抖音、微信、微博、快手、小红书、B站、百度与知乎,星期二(10月31日)在各自平台发布声明,将分批次引导粉丝量在50万以上和100万以上的自媒体账号,在个人主页展示真实姓名……}

\entryitemWithDescription{中科技部副部长出席英人工智能安全峰会}{https://www.zaobao.com/news/china/story20231101-1447246}{中国科技部副部长吴朝晖将率团出席在英国举行的人工智能安全峰会,并向参会各方介绍中国提出的《全球人工智能治理倡议》。(互联网) (北京/伦敦综合讯)中国科技部副部长率团出席在英国举行的人工智能安全峰会,被视为两国紧张关系进一步缓和。不过英国也称,中国不适合参加部分讨论……}

\entryitemWithDescription{中国北方雾霾将持续到11月中旬}{https://www.zaobao.com/news/china/story20231101-1447238}{北京商务中心区星期三(11月1日)被雾霾笼罩。(路透社) (北京综合讯)中国空气质量近日转差,京津冀等北方城市的雾霾情况预计将持续至11月中旬。 法新社引述瑞士空气净化信息科技公司IQAir称,北京星期三(11月1日)的PM2.5浓度,比世界卫生组织的标准高出二十多倍。IQAir也说,北京目前是世界上污染第三严重的城市……}

\entryitemWithDescription{中国通报多艘日本船只驶入钓鱼岛周围海域}{https://www.zaobao.com/news/china/story20231101-1447221}{(北京/东京综合讯)中国海警局通报,多艘日本船只驶入有领土争议的钓鱼岛(日本称``尖阁诸岛'')周围海域,并称已采取必要管控措施。 根据``中国海警''微信公众号消息,中国海警局新闻发言人甘羽说,日本``惠丸''\,``鹤丸''\,``第八泰生丸''三艘船只和数艘巡视船星期三(11月1日)非法进入钓鱼岛周围海域,中国海警舰艇依法采取必要管控措施……}

\entryitemWithDescription{台国防部:配合山东舰海空联训 37架次解放军军机越过海峡中线}{https://www.zaobao.com/news/china/story20231101-1447219}{(台北综合讯)台湾国防部说,侦察到中国大陆解放军``山东号''航母编队在台湾东南应变区外实施海空联训,并侦测到多架次解放军军机越过台海中线。 根据台湾国防部官网消息,自星期二(10月31日)上午6时至星期三(11月1日)上午6时,侦获中国大陆派出43架次军机、七艘次军舰持续在台海周边活动,其中37架次跨越海峡中线及其延伸线进入西南及东南空域……}

\entryitemWithDescription{在华被拘四年 澳籍作家杨恒均之子吁澳总理助父获释}{https://www.zaobao.com/news/china/story20231101-1447206}{在中国被拘留四年多的杨恒均又名杨军,是澳大利亚华裔时事评论家,网络作家。(互联网) (堪培拉综合讯)澳大利亚华裔作家杨恒均被中国以涉嫌间谍罪拘留四年多后,他的儿子写信请求澳洲总理阿尔巴尼斯,在访华期间敦促北京释放他们的父亲……}

\entryitemWithDescription{赖清德副手考虑六人 萧美琴是优先之上}{https://www.zaobao.com/news/china/story20231101-1447176}{台湾驻美代表萧美琴(右)今年8月在赖清德过境美国期间,在脸书上贴出两人看职棒时,分别穿上1号跟11号球衣的背影照,曾一度被解读为赖萧配已成形。(萧美琴脸书) (台北综合讯)台湾民进党总统参选人赖清德透露,考虑搭档的副总统人选目前有六人,外界热传的台湾驻美国代表萧美琴、文化部前部长郑丽君均在列,且萧美琴在名单中是``优先之上''……}

\entryitemWithDescription{杨丹旭:中国民间如何哀悼李克强}{https://www.zaobao.com/news/china/story20231101-1447056}{中国官方10月31日通报,李克强的遗体已在他逝世当天由专机从上海护送至北京,将于11月2日在北京火化。(路透社) 七个多月前才从中国第二把交椅上退下来的中国前总理李克强,上周五(10月27日)因突发心脏病猝然离世,引发各界震惊。 中国官方星期二(10月31日)通报,李克强的遗体已在他逝世当天由专机从上海护送至北京,将于11月2日在北京火化……}

\entryitemWithDescription{台湾前副防长:希望对岸深刻体会到武力解决是``最笨方法''}{https://www.zaobao.com/news/china/story20231031-1447052}{台湾前海军司令陈永康上将(左)和战略学者、国民党驻美代表黄介正教授,在台湾学术界推动台海兵棋推演多年。(温伟中摄) 台湾安全研究中心星期一(10月30日)发表区域安全兵推报告,设想中国大陆2027年攻台情境,推演大陆如何施展各种围困手法,并全方位加强备战与避战之道……}