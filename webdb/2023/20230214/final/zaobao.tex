\entryitemWithDescription{青岛海域所发现飞行物 中国隔天军演料已击落}{https://www.zaobao.com/news/china/story20230214-1362814}{中国军事评论员研判,山东省青岛市官方通报发现不明飞行物,并声称准备击落,与解放军在黄海北部海域进行实弹射击有一定关联,并相信不明飞行物大概已被击落。 中国外交部星期一(2月13日)指责单是去年以来,美国高空气球就10余次非法飞越中国领空,白宫则否认北京的指控……}

\entryitemWithDescription{蔡英文总统到场颁褒扬令 5­万人含泪跪送星云法师最后一程}{https://www.zaobao.com/news/china/story20230214-1362815}{台湾佛光山开山宗长星云法师2月13日圆寂赞颂典礼后,遗体移往台南白河大仙寺火化。数万民众沿途含泪跪送,送他最后一程。(人间社提供) 逾5­万人在昨天台湾佛光山开山宗长星云法师的圆寂赞颂典礼后,送他最后一程。遗体火化后将永久安奉在佛光山万寿园。 台湾佛光山星期一(2月13日)上午为开山宗长星云法师举行圆寂(离世)赞颂典礼,逾5­万人送他最后一程,总统蔡英文也到场颁发褒扬令……}

\entryitemWithDescription{萧美琴:蔡英文出访邦交国是常态 未来会适时规划}{https://www.zaobao.com/news/china/story20230214-1362818}{台湾驻美代表萧美琴表示,台湾总统蔡英文在冠病疫情之前出访邦交国是常态,未来在适当时间点也会规划。 萧美琴星期一与立法院外交及国防委员会立委闭门茶叙前,被媒体问到蔡英文会否在下半年出访时过境美国。她说,蔡英文出访邦交国是常态,过去都有惯例,``未来适当时间点也会规划,但目前并没有很细部、确切的方案''……}

\entryitemWithDescription{称在港说普通话被歧视 大陆网红被批``为流量抹黑香港''}{https://www.zaobao.com/news/china/story20230214-1362819}{中国大陆网红发视频展示在香港只说普通话受到不公正待遇后,有不少大陆旅客反驳歧视论,也有媒体人批评网红为了流量抹黑香港,并呼吁港府要对大陆民众``讲好香港故事''。 拥有逾98万粉丝的抖音博主``沪漂女孩艺轩'',在2月7日的短视频中展示她屡次因只说普通话,在香港旺角、尖沙咀等旅游热门地受到服务人员不公正对待后,香港《星岛日报》2月10日到深圳采访市民对香港旅游的印象……}

\entryitemWithDescription{早 说}{https://www.zaobao.com/news/china/story20230214-1362820}{本年度不止是23条立法不列入立法议程,连相关公众咨询亦不宜展开。 ------中国全国侨联副主席卢文端星期一(2月13日)在《明报》撰文,列举四个不宜在今年推进《基本法》23条立法的原因,包括应当先拼经济、加强联通世界等。他还提醒,台湾2024总统大选临近,香港如果在此时展开23条立法引发争议,极有可能又让民进党``捡到枪'',借机抹黑``一国两制'',称这是绝不能容许出现的局面……}

\entryitemWithDescription{简约公屋揭示港府管治危机}{https://www.zaobao.com/news/china/story20230214-1362821}{香港新一届特区政府在中央政府的大力支持下,上任以来不断触碰社会``老大难''问题,并提出一系列针对性措施,获得了社会舆论的普遍认可。然而,近来当局未顾及细节问题就匆促推出某些政策,也暴露出本届政府在管治方面存在一些问题。 这里说的是``简约公屋''。九七后住屋成为香港社会的头等问题……}

\entryitemWithDescription{调查:终身无孩率快速上升 中国育龄女性生育意愿不断降低}{https://www.zaobao.com/news/china/story20230214-1362823}{中国有相当比率的经济独立女性摒弃传统观念,认为结婚与生育并非人生必经之路,婚姻和生育是对女性的某种剥削。图为2月10日在北京王府井商业大街消费购物的女性。(彭博社) 中国人民大学新闻学院教授周小普接受《联合早报》采访时说,中国实施计划生育政策以来,独生子女现象改变了中国的家庭结构与关系,也影响了新生代的家庭与婚恋观念,他们的自我意识强,更关注自我价值。 中国年轻一代婚育观念正发生改变……}

\entryitemWithDescription{赞比亚反对中国要求让世行参与债务重组}{https://www.zaobao.com/news/china/story20230214-1362824}{中国要求让世界银行和其他多边贷款机构参与赞比亚的债务重组,遭到赞比亚的反对。赞比亚财长警告,这样的要求干扰债务减免进程,正阻碍该国的经济复苏。 据英国《金融时报》星期一(2月13日)报道,赞比亚财长穆索科图瓦内(Situmbeko Musokotwane)受访时说,今年是完成赞比亚约130亿美元(约173亿新元)外债重组的关键时刻,但中国政府提出的要求已对债务重组的谈判形成了一种干扰……}

\entryitemWithDescription{美媒:跨国企业高管陆续重返中国寻商机}{https://www.zaobao.com/news/china/story20230214-1362825}{(北京综合讯)随着中国重新开放,跨国公司的高层管理人员正陆续重返中国,以寻找重新开放所带来的商机。 据《华尔街日报》报道,大众汽车首席执行官奥博穆在1月底至2月初访问了中国。他也是1月初中国取消大部分防疫限制以来,首批访华的大型跨国公司高管之一。 报道引述知情人士透露,预计苹果公司首席执行官库克和辉瑞公司首席执行官艾伯乐,将在下个月访问中国。马赛地---奔驰集团说,集团董事长康林松也计划访问中国……}

\entryitemWithDescription{在日诞生中国大熊猫``香香''将回国}{https://www.zaobao.com/news/china/story20230214-1362826}{在日本东京上野动物园出生的5岁雌性大熊猫``香香'',将在下星期二(2月21日)归还中国。香香是上野动物园近29年来首次诞生的熊猫宝宝。 据日本共同社报道,上野动物园园长福田丰对香香的离开表示祝福,``虽然很不舍,但希望香香能努力尽快适应环境,找到好伴侣、留下后代''。 香香2017年6月出生,父母是中国旅日大熊貓``力力''和``真真''。它在2017年12月首次对公众亮相便大受欢迎……}

\entryitemWithDescription{湖北青年自愿入伍后拒服兵役被罚}{https://www.zaobao.com/news/china/story20230214-1362827}{中国湖北沙洋县一名青年因拒服兵役,除了要缴纳3万4584元(人民币,下同,约6746新元)罚款,还被罚两年内不得经商、贷款、出国等。 据澎湃新闻等报道,根据沙洋县政府通报,今年22岁的张佳豪在去年秋季自愿应征报名入伍,但入伍后不久,就以无法适应部队工作生活为由,多次提出终止服役申请……}

\entryitemWithDescription{武汉抗议医保改革人群不退 中国多地政府相继发声解释}{https://www.zaobao.com/news/china/story20230213-1362473}{湖北武汉上周接连出现退休人士聚集维权,抗议医疗保险改革导致个人账户资金减少。中国多地政府部门连日来相继发声解释医保新规,试图减少改革面对的民意阻力。 网传视频显示,上星期三(2月8日)有大批老年人冒雨在武汉市政府门前聚集,其中不少人打伞在警车旁合唱《国际歌》,还有人高喊:``还我养命钱''。此后两天仍陆续有年长者上街聚集……}

\entryitemWithDescription{台学者:求学时就被贴上地位标签 中国大陆精英教育固化社会阶级流动}{https://www.zaobao.com/news/china/story20230213-1362474}{长期关注中国大陆教育课题的台湾社会学者姜以琳,在北京追踪28名家庭优渥的精英学生,如何在八年内从中学生一步步出国求学,再进入全球高端市场就业,成为跨国精英阶级的历程。(缪宗翰摄) 大陆社会精英运用金钱资源,确保下一代得到最好的教育,进而成为跨国精英。而在中国二三线城市里,出身经济条件与社会地位较低家庭的子弟,只能仰赖读书、考试,希望能挤进高校窄门,翻转自身阶级……}

\entryitemWithDescription{国台办发唁函悼念星云法师 大陆团赴台遇阻引两岸争议}{https://www.zaobao.com/news/china/story20230213-1362475}{以国务院台办副主任龙明彪为团长的中国大陆官方吊唁团,2月12日下午3时出席在江苏宜兴大觉寺举行的星云法师吊唁仪式。图为佛光山``祖庭''江苏宜兴大觉寺吊唁星云法师仪式现场。(新华社) (北京/台北综合讯)中国大陆国台办主任宋涛发唁函,深切悼念星云法师,并感怀他坚定促统,反对台独。与此同时,两岸政界也因大陆官方代表团取消赴台湾吊唁行程一事,展开舆论攻防战……}

\entryitemWithDescription{指到处干涉别国内政 中国官媒批美国``假民主真霸权''}{https://www.zaobao.com/news/china/story20230213-1362476}{中美关系因气球事件滑坡之际,中国官媒《人民日报》连续两天发文,强烈批评美国的民主制度是``假民主真霸权'',将民主工具化、武器化到处干涉别国内政,已尽显病态和颓势……}

\entryitemWithDescription{单日在院死例维持100多起 大陆冠病疫情持续趋稳}{https://www.zaobao.com/news/china/story20230213-1362477}{中国冠病疫情持续趋稳,官方公布今年2月首个星期累计在院冠病感染相关死亡病例912起。这意味着中国过去一周平均单日在院死亡病例维持在100多起,与官方上周公布的单日数字相似。 中国疾病预防控制中心星期六(2月11日)发布2023年2月3日至9日冠病疫情数据显示,中国累计在院冠病感染相关死亡病例有912起,当中27起属冠病感染导致呼吸功能衰竭,其余885起属基础疾病感染……}

\entryitemWithDescription{美国得州考虑禁中国公民买房地产 引发当地民众抗议}{https://www.zaobao.com/news/china/story20230213-1362479}{中美关系日益紧张之际,美国得克萨斯州基于国家安全考量,正考虑禁止中国公民购买房地产。这一做法在当地引发了民众抗议,中国外交部也批评这是将经贸行为政治化,违背国际贸易规则。 据法新社报道,约300名抗议者星期六(2月11日)在得州第一大城休斯顿唐人街游行,高喊``停止仇恨中国人''和``得州是我们的家''。报道称,抗议者多为美籍华裔和新移民……}

\entryitemWithDescription{早说}{https://www.zaobao.com/news/china/story20230213-1362480}{随着这部戏的火爆,我最担心的事还是发生了。该来的总会来的,我只能面对。 ------在中国热播反黑刑侦剧《狂飙》中饰演毒贩的演员含笑,星期天(2月12日)在微博发文承认自己曾吸毒,并向剧组、观众和缉毒警察道歉。《狂飙》剧组回应说,将对含笑的参演片段进行修改删除,并强调始终秉持对涉毒零容忍的态度……}

\entryitemWithDescription{于泽远:中国拼经济的外部挑战}{https://www.zaobao.com/news/china/story20230213-1362481}{今年以来,摆脱疫情影响的中国经济呈现明显复苏景象,各地方政府也纷纷发出拼经济的强音。观察人士纷纷预计中国经济今年有望一扫三年疫情的阴霾,强势反弹。 但近期突发的``气球事件''引起的中美新一轮博弈,不仅给已经脆弱不堪的中美关系又蒙上一层阴影,也让中国外部环境进一步恶化。如处理不当,中国拼经济的努力有可能受到外部挑战的重大干扰……}

\entryitemWithDescription{涉隐私争议停映 港片《给十九岁的我》再掀蓝黄对骂}{https://www.zaobao.com/news/china/story20230213-1362482}{香港著名导演张婉婷历时10年拍摄的纪录片《给十九岁的我》虽然备受好评,但因涉及伦理争议在公映四天后就下架。(香港中通社) 《给十九岁的我》以导演张婉婷的母校英华女学校为拍摄背景,追踪六位性格及家境各异的千禧世代女生,在2011年至2021年的成长故事。影片屡次入选影展还获奖,但因有多名片中女生控诉遭不公对待而停映……}

\entryitemWithDescription{港渣打马拉松3万余人参赛}{https://www.zaobao.com/news/china/story20230213-1362483}{第25届渣打香港马拉松2月12日早晨开跑,此次赛事参赛名额由2万5000个增至3万7000个,设有全马马拉松、半马拉松和10公里三个距离组别。图为参赛者跑过港岛铜锣湾区。(路透社) (香港综合讯)延迟了三个月举行的香港渣打马拉松,星期天(2月12日)吸引超过3万4000人参加,是香港复常后举行的最大型运动赛事。 特区行政长官李家超说,香港已经全面复苏,将陆续举办多项体育和国际盛事……}