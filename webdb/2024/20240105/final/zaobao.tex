\entryitemWithDescription{韩咏红:大陆对台介选 只有大陆吗?}{https://www.zaobao.com/news/china/story20240105-1460097}{台湾大选投票日一天天逼近,政坛上的攻伐、爆料、乃至腥膻色故事也如期而至。新年的第二天(1月2日),疑似民进党籍人物罗致政的不雅私密影片外流,在台湾手机Line群组广传。 罗致政当晚公开否认自己是影片男主角,指控影片是深伪造假,为境外网军介选操作……}

\entryitemWithDescription{吴钊燮呼吁国际严防中国大陆操纵民主选举}{https://www.zaobao.com/news/china/story20240104-1460094}{台湾外交部长吴钊燮称,中国大陆通过在台湾的代理人设立虚假组织和虚假新闻网站,进行虚假民调,并使用数千个虚假社交媒体账户来操纵公众舆论。(法新社) 台湾将在1月13日举行总统和立委选举,台湾外交部长吴钊燮撰文称,中国大陆意图侵蚀台湾民主体制,呼吁国际社会严防大陆操纵民主国家的选举结果,共同捍卫以规则为基础的国际秩序。 吴钊燮接受英国《经济学人》杂志邀稿,以``即将来临的台湾大选:风险何在……}

\entryitemWithDescription{紧张局势开年持续 中美军舰同时在南中国海巡航}{https://www.zaobao.com/news/china/story20240104-1460091}{美菲两军星期三在南中国海展开为期两天的联合巡逻,美菲各派出四艘舰艇。图为菲律宾武装部队星期四在美国核动力航母卡尔文森号上对菲律宾海军直升机做检查。(法新社) 南中国海局势在2024开年持续紧张,中美军舰分别同时在这海域巡航两天,针锋相对意味浓……}

\entryitemWithDescription{学者:区议会单打独斗拼经济 不如港府划一统筹}{https://www.zaobao.com/news/china/story20240104-1460086}{学者建议振兴香港夜经济,如香港旅游发展局与商户合作推出的全新庙街夜市去年12月15日正式开锣后,半个月内已吸引到逾20万人次,平均人流比之前上升3至4倍,预计逾三成是游客。(法新社) 香港经济,尤其是餐饮零售业在过去一年表现疲弱,全港多区的区议会星期四(1月4日)纷纷提出推动地区经济发展的不同活动建议,但有学者质疑,这些活动难以提升地区经济,建议港府划一统筹,各区不宜``单打独斗''……}

\entryitemWithDescription{境外炮弹落入云南境内致五人受伤 中国表示强烈不满}{https://www.zaobao.com/news/china/story20240104-1460055}{网传视频显示,炮弹掉落后,一男子躺倒在人行道上,衣服上可见血迹。(路透社) (北京综合讯)来自缅甸的炮弹星期三(1月3日)落入中国云南边城,造成当地五人受伤,中国对此表示强烈不满,并称已向有关方面提出严正交涉。 炮弹是落在与缅甸果敢自治区接壤的云南省镇康县县城南伞镇上,南伞和缅甸果敢杨龙寨一桥相连,是中国西南最极边的一座岸城一体化``边地新城''……}

\entryitemWithDescription{美军指中国军机不安全拦截行为减少}{https://www.zaobao.com/news/china/story20240104-1460052}{(北京/华盛顿综合讯)美媒报道,随着中美关系出现改善迹象,中国解放军对美国军机的不安全拦截事件有所下降。 中美两军去年10月在海空领域的摩擦急剧增加,双方曾相互公开指责对方行为危险。五角大楼当时说,中国军方过去两年的180多次拦截行动超过之前十年总数,美军认为中国正试图迫使美国停止相关军事活动……}

\entryitemWithDescription{中国多省极端大雾持续 上海起飞航班延误}{https://www.zaobao.com/news/china/story20240104-1460047}{中国多省份持续出现强浓雾现象。图为被浓雾遮住的杭州钱塘江。(中央气象台微博) (上海/安徽/南京综合讯)中国多省份持续出现强浓雾现象,导致能见度低至危险水平,数十班上海起飞航班被延误。 据中国中央气象台网站星期四(1月4日)消息,山东、江苏、安徽、湖北、江西等地的部分地区有能见度低于500米的浓雾,局地有能见度低于200米的强浓雾……}

\entryitemWithDescription{比亚迪纯电车上季度销量超越特斯拉 成为全球最大纯电动车制造商}{https://www.zaobao.com/news/china/story20240103-1459873}{2023年4月18日在第20届上海国际汽车工业展览会上,人们走过比亚迪(BYD)的展位。(法新社) 中国汽车企业比亚迪2023年最后一季度的电车销量,超越了美国巨头特斯拉,成为全球最大的纯电动车制造商……}

\entryitemWithDescription{高端疫苗合约解密与否 成台湾选举攻防战}{https://www.zaobao.com/news/china/story20240103-1459871}{台湾自行研制的高端疫苗,最近成了台湾蓝绿政党的攻防焦点。(自由时报) (台北综合讯)台湾总统大选倒数计时,高端疫苗成攻防焦点。国民党质疑民进党政府的高端疫苗采购案涉不法行为,要求行政院长陈建仁星期五(1月5日)到立法院专案报告公布高端合约。陈建仁说,高端疫苗合约保密五年期限尚未到期,得经股东会同意才能决定是否公布。 国民党连日来猛打高端疫苗议题……}

\entryitemWithDescription{香港3.7\%受访中学生过去一年计划过自杀}{https://www.zaobao.com/news/china/story20240103-1459869}{香港学生自杀个案近年不断飚升,更有3.7\%的受访中学生在过去一年里计划过自杀。受访的教育界专家指出,香港学童自杀数字上升,可能与他们在疫情期间及社会复常后面临的较大挑战有关,建议商界营造工作与生活平衡的职场,给予员工更多亲子时间……}

\entryitemWithDescription{台剧从专攻大陆市场转向国际 内容更重视西方价值观}{https://www.zaobao.com/news/china/story20240103-1459865}{学者分析,台湾影视作品的目标市场近年从中国大陆转向全球,内容呈现上也更重视女性自主等符合西方价值观的设定。图为2023年表现亮眼的台剧《人选之人---造浪者》剧照。(互联网) 不少台湾影视作品近年在国际串流平台上表现亮眼,关注华语影视创作的台湾学者分析,过去八年来,台剧的目标市场从大陆转向全球,内容呈现上也更重视女性自主等符合西方价值观的设定……}

\entryitemWithDescription{赖清德被指有私生子 立委罗致政陷私密影片疑云 两人指控中国大陆介选}{https://www.zaobao.com/news/china/story20240103-1459860}{民进党新北市立委候选人罗致政陷入了私密影片疑云。(互联网) 台湾大选倒数10天,负面消息满天飞。国民党不分区立委候选人谢龙介和前立委邱毅,质疑民进党总统候选人赖清德有私生子。赖清德竞选总部表示,这完全是恶意中伤的错假讯息,已委由律师对谢、邱二人提出告诉。 民进党新北市立委候选人罗致政也陷入私密影片疑云。罗致政表示,该部影片是深伪影片,影中人不是他,此事意在影响选举结果,他已向新北市调查处告发……}

\entryitemWithDescription{央视新闻联播画面显示 中国第三艘航母``福建舰''近完工}{https://www.zaobao.com/news/china/story20240103-1459835}{星期二(1月2日)在央视新闻联播视频中出现的福建舰,与2022年下水时的画面相比,其电磁弹射装置上的施工棚已经拆除,三条弹射轨道清晰可见。(《新闻联播》视频截图) (北京/台北综合讯)中国第三艘航空母舰``福建号''的最新画面,星期二在中国中央电视台《新闻联播》中出现。军事评论员据此研判,福建舰已进入建造尾声,预计不久将进行海试航行测试……}

\entryitemWithDescription{赵少康:多次致电郭台铭都进语音信箱}{https://www.zaobao.com/news/china/story20240103-1459832}{台湾在野的国民党副总统候选人赵少康称,他曾多次致电鸿海创办人郭台铭(图),不仅都直接进入语音信箱,对于赵少康的留话和简讯,郭台铭也都不闻不问。(路透社) (台北综合讯)台湾在野的国民党副总统候选人赵少康说,他曾多次致电鸿海创办人郭台铭,不仅都直接进入语音信箱,对于赵少康的留话和简讯,郭台铭也都不闻不问……}

\entryitemWithDescription{杨丹旭:悬念中开启2024}{https://www.zaobao.com/news/china/story20240103-1459650}{台湾将在1月13日举行总统和立委选举。(彭博社) 从2023年岁末迈入2024年开端,大中华区渡过了一个相对平静的周末。略微掀起波澜的,是台湾大选前唯一的一轮正副总统候选人电视辩论会。 两场辩论会各方唇枪舌剑,火力交锋,两岸课题再度被炒热……}

\entryitemWithDescription{台湾大选进入民调封关期 三组候选人选情走势迷雾重重}{https://www.zaobao.com/news/china/story20240103-1459652}{民进党总统候选人赖清德(左二),日前在高雄出席竞选活动,与支持者相互击掌打气。(路透社) 台湾将在1月13日举行总统和立委大选,各政党和许多民调机构陆续在星期二(1月2日)民调封关日前公布民调。多数民调显示,执政的民进党``赖萧配''仍小幅领先,在野国民党``侯康配''紧追在后,民众党``柯盈配''落后前两者有段差距……}

\entryitemWithDescription{早知:封关民调知多少?}{https://www.zaobao.com/news/china/story20240102-1459638}{截至1月3日台湾民调封关,民进党的正副总统候选人仍保持微弱领先。图为去年12月22日,民进党支持者参加该党总统候选人赖清德在高雄举行的造势大会。(路透社) •封关民调有什么规定? 台湾法律规定,任何人在投票日前10天起到投票截止前,不得发布、报道、散布、评论或引述民调报告……}