\entryitemWithDescription{台湾蓝白合继续僵持 进入政党协商延长赛}{https://www.zaobao.com/news/china/story20231026-1446005}{台湾在野``蓝白合''继续僵持,双方星期四(10月26日)宣告无法就比民调或初选投票方式选出总统参选人达成共识,进入政党协商的延长赛。 台湾将在两个半月后的2024年1月13日举行总统与立委选举,执政的民进党(绿)参选人赖清德民调支持度持续领先,两大在野党、国民党(蓝)总统参选人侯友宜和民众党(白)总统参选人柯文哲支持度都在两成左右。过半民意希望两党整合,实现政党轮替,推动两党探讨蓝白合……}

\entryitemWithDescription{中国国安部护航《反间谍法》 指未强制收集外企数据}{https://www.zaobao.com/news/china/story20231026-1445988}{(北京综合讯)中国官方说,国安机构调取数据受到严格制约,并且不涉及外资企业和外籍人员的合法数据活动。 中国施行新修订的《反间谍法》以来,部分外企对正常商务行为可能被针对并进行惩罚感到担忧。 中国国家安全部星期四(10月26日)在微信公众号发文说,认为``《反间谍法》过度强调数据安全并使国家安全机关有权查阅数据,从而给企业和个人带来`数据安全风险'\,''是一种误解……}

\entryitemWithDescription{神舟十七号载人飞船升天 与空间站成功对接}{https://www.zaobao.com/news/china/story20231026-1445983}{中国神舟十七号载人飞船星期四(10月26日)顺利发射升空,完成与空间站对接。图为神舟十七号三名航天员汤洪波(前右)、江新林(后右一)和唐胜杰(后右二),与驻扎空间站的神舟十六号航天员景海鹏(前左)、桂海潮(后左一)和朱杨柱(后左二)在空间站天和核心舱合影。(中国载人航天工程办公室) (酒泉/北京综合讯)中国成功发射神舟十七号载人飞船,完成与中国空间站核心舱的对接,将三名航天员送入空间站……}

\entryitemWithDescription{中国国务委员谌贻琴当选全国妇联主席}{https://www.zaobao.com/news/china/story20231026-1445952}{64岁的谌贻琴是白族人,是首位担任全国妇联主席的少数民族女性。(互联网资料图) (北京综合讯)中国国务委员谌贻琴当选第十三届全国妇联主席,是首位兼任全国妇联主席的国务委员,打破35年来的人事惯例。 据中华全国妇女联合会(简称全国妇联)官网新闻稿,全国妇联第十三届执行委员会第一次会议星期三(10月25日)召开,以无记名投票方式选出全国妇联主席、副主席和常务委员……}

\entryitemWithDescription{俄总理:俄中关系达到前所未有水平并将继续加强}{https://www.zaobao.com/news/china/story20231026-1445912}{中国国务院总理李强当地时间星期三(10月25日)下午在比什凯克出席上海合作组织成员国政府首脑(总理)理事会第22次会议期间,与俄罗斯总理米舒斯京会面。(新华社) (比什凯克 /北京综合讯)中俄两国元首会面一周后,俄罗斯总理米舒斯京在与中国总理李强会面时说,俄中全面战略协作伙伴关系达到了前所未有的水平,并在继续加强……}

\entryitemWithDescription{王纬温:中美合作管控以哈冲突升级?}{https://www.zaobao.com/news/china/story20231026-1445754}{中美在管控以哈冲突上并没有根本分歧,近期还出现一些合作的迹象。图为以加边界上看到加沙地带的烟雾。(路透社) 中国外长王毅星期四(10月26日)将到美国访问三天,是今年2月气球事件引发中美紧张以来,访美的中国最高级别官员,明显是为中美元首可能11月在旧金山再度会晤铺路。在以哈冲突持续近三周之际,王毅还将与美国高层就国际地区问题深入交换意见,为两国可能加强合作管控中东乱局预留伏笔……}

\entryitemWithDescription{香港特首称将在2024年完成基本法第23条立法}{https://www.zaobao.com/news/china/story20231025-1445755}{香港特首李家超(前排)星期三在立法会发表《施政报告》,与他的领导团队一同穿戴由香港知专设计学院(HKDI)师生携手设计制作的绿色领带和领巾。不少议员、政治助理到立法会,也穿上绿色的衣服、配饰、领带等。(彭博社) 近年国际政治形势日益复杂,中国政府越来越重视国家安全,香港特区政府将于明年完成《基本法》第23条立法。有受访学者相信,这项立法不会影响香港的投资环境……}

\entryitemWithDescription{中美问题专家:美国不会在习拜会做出让步}{https://www.zaobao.com/news/china/story20231025-1445751}{美国的中美关系专家韩磊(Paul Haenle)认为,习拜会虽然不会解决中美关系中的所有问题,但提供了一个让中美关系变得更积极的机会。(互联网) 卡内基中国莫里斯·格林伯格荣誉主任韩磊(Paul Haenle)认为,随着中国外交部长王毅确定出访美国,下个月的习拜会相信能如期举行。他说,虽然美国不会在习拜会中做出让步,但两国领导人见面,能创造让中美互动变得更积极且具建设性的机会……}

\entryitemWithDescription{港府为提振楼市宣布``减辣'' 但投资者反应冷淡}{https://www.zaobao.com/news/china/story20231025-1445750}{香港特首李家超星期三宣布多项楼市``减辣''措施'',包括买家印花税减半等。(路透社) (香港综合讯)香港特首李家超星期三(10月25日)宣布多项楼市``减辣''措施'',包括买家印花税减半等。不过,投资者对此反应冷淡,香港主要地产开发商的股价星期三收跌……}

\entryitemWithDescription{中国``神舟十七号''将于10月26日发射}{https://www.zaobao.com/news/china/story20231025-1445736}{中国将于星期四上午发射``神舟十七''号载人飞船,执行神舟十七号载人飞行任务的航天员乘组汤洪波(中)、唐胜杰(右)、江新林(左)三名航天员星期三上午在酒泉卫星发射中心问天阁与中外媒体记者集体见面。 (中新社) (北京中新电)中国将于星期四(10月26日)上午发射``神舟十七''号载人飞船,并将首次在``天宫''空间站舱外进行试验性维修作业……}

\entryitemWithDescription{台湾朝野互批``亲中卖台'' 互相质疑对台湾的忠诚}{https://www.zaobao.com/news/china/story20231025-1445729}{国民党立委马文君(左一)和民进党立委赵天麟一同出席立法院外交及国防委员会。(自由时报) 台湾朝野互批``亲中卖台'',执政的民进党猛攻在野的国民党立委马文君泄露国防机密,要求她退选;国民党反质疑民进党立委赵天麟被中国大陆间谍色诱,呼吁暂停其立委职务展开调查。 台湾立法院外交及国防委员会星期三(10月25日)和星期四审查国防和机密预算两天,事关巨额预算和国安,也成了朝野选战攻防的战场……}

\entryitemWithDescription{分析:中国增发国债兼具稳增长、提信心和化风险作用}{https://www.zaobao.com/news/china/story20231025-1445726}{中国增发1万亿元人民币国债的利好消息带动陆港股市终结四连跌。(路透社) 中国增发1万亿元(人民币,下同,1894亿新币)国债的利好消息带动陆港股市终结四连跌。分析指出,在今年中国经济增速达标无忧的背景下,新一轮刺激措施主要为明年稳增长发力,同时起到提振信心和化解风险的作用。 沪深300指数星期三(10月25日)开盘后一度上涨近1.3\%,过后震荡回落,全天上涨0.5%……}

\entryitemWithDescription{长峰医院火灾:北京副市长及41公职人员被问责}{https://www.zaobao.com/news/china/story20231025-1445724}{北京长峰医院住院部东楼今年4月18日发生火灾致21人死亡。图为火灾后的医院外观。(法新社)(法新社) (北京讯)今年4月造成29人死亡的北京丰台长峰医院大火调查报告公布,北京副市长靳伟被问责,并给予政务警告处分,另有41名公职人员被问责,20名涉案人员被警方立案侦查。 长峰医院今年4月18日的重大火灾事故造成了29人死亡、42人受伤,直接经济损失3832万元人民币(717万新元)……}

\entryitemWithDescription{恒大创始人许家印身家跌至新低}{https://www.zaobao.com/news/china/story20231025-1445723}{中国债务规模最大的房地产企业恒大集团创始人许家印目前其净资产已跌至新低。图为许家印2017年3月在香港出席新闻发布会。(路透社) (纽约彭博电)中国债务规模最大的房地产企业恒大集团创始人许家印,目前其净资产已跌至新低。 据彭博社报道,彭博亿万富豪指数统计显示,许家印的净资产已跌至9.79亿美元(约13.38亿新元)。 自8月末复牌以来,恒大股价已下跌86\%……}

\entryitemWithDescription{国台办:台企应推动两岸关系和平发展}{https://www.zaobao.com/news/china/story20231025-1445708}{中国大陆政府对台企富士康集团在多个省的重点企业展开税务和用地调查。大陆国台办星期三对此回应说,台资企业应为推动两岸关系和平发展发挥积极作用。(路透社) (台北综合讯)针对富士康税务稽查风波,中国国务院台湾事务办公室回应称,台资企业在中国大陆分享增长红利、获得长足发展的同时,也应为推动两岸关系和平发展发挥积极作用……}

\entryitemWithDescription{中国国防部批美报告 臆测解放军核力量发展}{https://www.zaobao.com/news/china/story20231025-1445704}{中国国防部星期三抨击美国官方发布的中国军力报告罔顾事实、胡编乱造。图为中国海军战略弹道导弹核潜艇``长征十一号'',2019年4月23日在山东青岛外海参加阅兵仪式,庆祝中国人民解放军海军成立70周年。(路透社) (北京综合讯)针对美国官方发布的中国军力报告,中国国防部星期三抨击这份报告罔顾事实、胡编乱造,妄加臆测中方在核、太空、网络等领域军力发展……}

\entryitemWithDescription{杨丹旭:中国防长被免后香山论坛看什么}{https://www.zaobao.com/news/china/story20231025-1445483}{失踪近两个月的中国国防部长李尚福,24日正式被免职。不过,中国全国人大常委会会议当天并没有任命新的防长。(法新社) 失踪近两个月的中国国防部长李尚福,星期二(10月24日)正式被免职。 和两个月前秦刚被免除外交部长职务时不同,这一回中国高层干脆利落,连李尚福的国务委员职务也一并免去。虽然官媒的通报没有披露李尚福被免职的原因,但这样的处理方式,已间接证实了他涉腐的传闻……}

\entryitemWithDescription{李尚福被免防长和国务委员 秦刚被免国务委员 分析:如何定性值得关注}{https://www.zaobao.com/news/china/story20231025-1445484}{中国全国人大常委会会议星期二(10月24日)决定,免去李尚福(左)的国务委员、国防部长职务,免去前外长秦刚(右)的国务委员职务。(路透社/法新社) 中国官方星期二(10月24日)决定,免去李尚福的国务委员、国防部长职务,同时免去中国前外交部长秦刚的国务委员职务。这是李尚福在公众视野中消失近两个月后,官方首次披露他的消息,间接证实此前外界关于他涉腐被调查的传闻……}

\entryitemWithDescription{侯友宜可接受``柯侯配'' 柯文哲不满``逼婚''但愿续谈}{https://www.zaobao.com/news/china/story20231024-1445480}{国民党总统参选人侯友宜(右)与国民党主席朱立伦星期二开记者会表示,蓝白合的政治联姻要欢喜甘愿,为两岸也为下一代。(温伟中摄) 台湾最大在野党、国民党(蓝)总统参选人侯友宜首度表态可接受``柯侯配'',但民众党(白)总统参选人柯文哲不满国民党的合作选项像在``逼婚''。国民党开记者会回应绝无逼婚,在野合作两情相悦最重要。蓝白合持续推进中,有望在双方主帅拍板下成局……}