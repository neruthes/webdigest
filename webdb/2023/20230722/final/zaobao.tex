\entryitemWithDescription{新闻人间:消失的``他''}{https://www.zaobao.com/news/china/story20230722-1416242}{由朱一龙、倪妮等人主演的悬疑电影《消失的她》6月下旬上映后,热度一直不减,并打破了中国国产悬疑片的票房纪录。 现实中也有一个``消失的他'',社会关注度已远超上述热映影片,但截至7月21日,``他''的下落仍是一个谜。 ``他''就是中国国务委员兼外交部长秦刚……}

\entryitemWithDescription{庄慧良:中国大陆尖子生游宝岛}{https://www.zaobao.com/news/china/story20230722-1416284}{北京、清华、复旦、武汉和湖南大学五所中国大陆高校37名师生,应马英九基金会之邀来台展开九天八夜之旅。这些名校顶尖学生的自信风采、落落大方的谈吐,立即攫取台湾民众目光。宝岛民众对他们的友善与热情,也深深烙印在这些学生心中。 此团先前被台湾官方冠上``统战''之名,差点无法成行。马英九基金会想方设法减人数、调行程,终于在出发前四天获得许可……}

\entryitemWithDescription{侯友宜规划9月上旬访美 或与佩洛西会面}{https://www.zaobao.com/news/china/story20230721-1416271}{国民党总统参选人侯友宜规划9月上旬访问美国。(法新社) 国民党总统参选人侯友宜规划9月上旬访问美国,计划安排至少一场公开演讲,同时可能与美国众议院前议长佩洛西会面。 据台湾《中国时报》星期四(7月20日)报道,国民党立委江启臣已提前赴美为侯友宜踩点,届时也将担任侯友宜访美团团长。据悉,侯友宜访美时间主要配合参众议院开议时间,计划发表至少一场公开演讲,并可能与佩洛西会面……}

\entryitemWithDescription{中国第二口万米深井在四川盆地开钻}{https://www.zaobao.com/news/china/story20230721-1416266}{继中国新疆塔里木盆地的``深地塔科1井''后,四川省广元市剑阁县星期四(7月20日)首次开钻``深地川科1井'',是中国开钻的第二口万米深井。 综合新华社、《人民日报》报道,``深地川科1井''位于四川盆地西北部剑阁潜伏构造,地面海拔717米,设计井深10520米。这片区域超深层叠置多套优质储层,成藏条件优越,一旦成功将有望发现新的超深层天然气增储目标区……}

\entryitemWithDescription{台湾调查CPTPP机密文件泄露事件}{https://www.zaobao.com/news/china/story20230721-1416240}{两名知情官员透露,台湾政府正调查一起官方机密文件泄露事件,泄露的文件包括有关台湾申请加入《跨太平洋伙伴全面进展协定》(CPTPP)的外交电报和机密报告……}

\entryitemWithDescription{香港高院下周裁定 是否禁播《愿荣光归香港》}{https://www.zaobao.com/news/china/story20230721-1416237}{香港高等法院星期五(7月21日)再度审讯律政司申请禁止传播反修例歌曲《愿荣光归香港》一案,大批警员在法院外戒备,以防有人到场闹事。(法新社) 香港律政司申请禁止传播反修例歌曲《愿荣光归香港》案件,星期五(7月21日)再讯。处理此案的法官陈健强称需时间考虑,将案件押后至下星期五(28日)裁决……}

\entryitemWithDescription{香港高球团体强烈抗议港府收回球场建公屋}{https://www.zaobao.com/news/china/story20230721-1416235}{针对香港特区政府表示将收回逾百年历史的粉岭高球场兴建公共住房,香港高尔夫球团体强烈不满,并加大力度抗议及反对此计划的执行。 据路透社星期五(7月21日)报道,粉岭高尔夫球场占地172公顷,而港府计划收回其中的五分之一,并用九公顷的土地建造1万2000套公共住房。香港特首李家超也一再表示,将遵循法定程序执行收回高球场地计划……}

\entryitemWithDescription{特稿:为稳定供电和净零碳排 台湾目前无法放弃核能}{https://www.zaobao.com/news/china/story20230721-1416224}{为了稳定供电和降低空污,台中发电厂计划在2025年和2026年上线各一部燃气机组,以四部天然气机组取代现有的四部燃煤机组。(台湾电力公司提供照片) 台湾今夏天气暴热、停电频传,学者分析认为,为达致稳定供电和净零碳排的目标,现阶段无法放弃核能,必须滚动修正能源政策、积极开发多元绿能。 今年7月全球平均气温屡创新高,全台湾有如``烤番薯''……}

\entryitemWithDescription{韩咏红:百岁老人基辛格访华能化解什么}{https://www.zaobao.com/news/china/story20230721-1415927}{百岁老人、传奇外交家基辛格本周突然到访北京,在国际媒体与中国民间引起了不少关注。 短短一个月内已有四位美国政要与前政要踏足中国,他们的资格是一个比一个老,依序为国务卿布林肯、财长耶伦、拜登总统气候特使克里,以及基辛格。不少中国民众对于基辛格到了期颐之年仍风尘仆仆赴华推动中美关系改善,颇为感慨……}

\entryitemWithDescription{克里总结访华行:美中应对全球暖化使命仍需更多时间取得新突破}{https://www.zaobao.com/news/china/story20230721-1415936}{美国总统气候问题特使克里总结访华行程时说,华盛顿和北京在应对全球暖化的共同使命上,仍需更多时间取得``新突破''。 综合彭博社和路透社报道,克里星期三(7月19日)晚间结束访华行程前受访时说:``我们的会谈已取得长足的进展,但行政时间和能力不足,未能最终达成目标。不过商谈仍在进行中。''美中结束这轮会谈后并没有发表一份联合声明……}

\entryitemWithDescription{山东在建高铁被举报偷工减料}{https://www.zaobao.com/news/china/story20230720-1415903}{在官方和地方媒体报道称其工程存在偷工减料、重大安全隐患的问题后,中国当局正在对一条连接山东省北部莱西市和荣成市的在建高铁展开调查。 综合路透社、《经济参考报》报道,山东莱荣高铁三标段的劳务分包商、河南省三捷实业有限公司实名举报工程施工总承包单位中国建筑第八工程局有限公司,指控后者涉嫌偷工减料,部分路基段螺纹桩施工存在质量问题,该段大部分螺纹桩的施工桩长不满足设计要求,存在重大安全隐患……}

\entryitemWithDescription{中国驻美大使谢锋回应秦刚去向:让我们等着看看}{https://www.zaobao.com/news/china/story20230720-1415895}{中国驻美大使谢锋星期三(7月19日)在阿斯彭安全论坛上,回应中国外长秦刚去向时说:``让我们等着看看。'' 谢锋是今年阿斯彭安全论坛(Aspen Security Forum)``炉边谈话''的嘉宾。 身为主持人的网络媒体Semafor创始主编克莱蒙斯(Steve Clemons)在开场时,借美国前国务卿基辛格访华之事问谢锋,基辛格有没有机会与秦刚会面……}

\entryitemWithDescription{港警国安处调查被通缉前港议员家属}{https://www.zaobao.com/news/china/story20230720-1415875}{香港立法会前议员郭荣铿早前被香港警务处国家安全处悬红通缉,其父母和兄嫂星期四(7月20日)被国安处带走接受调查。 综合《明报》《星岛日报》和``香港01''等港媒报道,消息人士透露上述消息,并称其大哥郭荣臻被带到西区警署接受调查。 郭荣臻当天下午2时许走出警署,直接乘坐德士离开,未回应在外等候的媒体的提问……}

\entryitemWithDescription{中国房地产大亨张力承认行贿 与美检方达成协议免入狱}{https://www.zaobao.com/news/china/story20230720-1415871}{中国房地产大亨张力承认在美行贿,但在与美国检方达成协议后,将不会面对任何刑事指控,免受牢狱之灾。 据彭博社引述美国加州北区检察官办公室星期三(7月19日)发布的新闻稿报道,富力集团联合创始人张力为获得旧金山一项建筑工程的许可证,贿赂当地一名市政工程高官。 报道说,现年70岁的张力生于中国广州,2021年5月因一项刑事罪名被控。去年12月英国政府应美国政府要求在伦敦逮捕了张力……}

\entryitemWithDescription{宏观经济前景疲弱 台积电二度下修全年营收展望}{https://www.zaobao.com/news/china/story20230720-1415869}{超微(AMD)执行长苏姿丰本周访台,带动台湾AI概念股飙升。但虽然AI成长强劲,还不足以拉动台湾整体经济。图为苏姿丰(挥手者)7月20日在阳明交通大学获颁名誉博士学位。(路透社) 由于全球智能手机和个人电脑需求下滑,台湾积体电路制造公司(台积电)二度下修全年营收展望。台积电总裁魏哲家警告,尽管人工智能(AI)需求强劲,但宏观经济前景比预期疲弱……}

\entryitemWithDescription{谈中美关系 谢锋:当务之急是挡住赖清德访美这头灰犀牛}{https://www.zaobao.com/news/china/story20230720-1415868}{对于台湾副总统赖清德将在8月过境美国,中国大陆驻美大使谢锋星期三(7月19日)说,当务之急就是挡住赖清德访美``这头正向我们冲来的灰犀牛''。受访学者分析,在华盛顿尝试恢复与北京的关系之际,美方对赖清德过境的安排会尽可能低调。 台湾外交部星期一(7月17日)证实,赖清德8月将以台湾总统蔡英文指派特使的身份,率团出席巴拉圭新任总统就职典礼,也将循例过境美国……}

\entryitemWithDescription{李家超星期日率团访新印马 分析:有助港加入RCEP}{https://www.zaobao.com/news/china/story20230720-1415852}{香港特首李家超将于星期日(7月23日)率领代表团访问新加坡、印度尼西亚及马来西亚,加强香港与这三国在经贸和投资的合作。有分析认为,港府去年申请加入《区域全面经济伙伴关系协定》(RCEP),至今仍未有进展。李家超这次出访部分亚细安成员国,相信可对加入协定起到加速作用……}

\entryitemWithDescription{陈婧:中国GDP``保5''都悬?}{https://www.zaobao.com/news/china/story20230720-1415591}{中国第二季经济数据出炉后,多家投资银行纷纷下调对全年经济增长的预测,其中最低的降到了5%。 花旗、摩根大通和法国兴业银行均把对全年中国GDP增幅预测,从此前的5.5%下调至5%,摩根史丹利更将预测大幅下调0.7个百分点至5%。 这让人想到今年初中国刚走出冠病疫情影响时经济强势反弹,不少机构乐观预测全年经济增幅可突破6\%。中国政府3月份将官方增长目标定为5%左右时,还被认为过于保守……}

\entryitemWithDescription{早说}{https://www.zaobao.com/news/china/story20230719-1415589}{(互联网) 选举靠大家,民调不靠谱,胜利是自己打拼来的,只要我们有信心,一定会赢。创造卢秀燕的奇迹,侯友宜会当选总统! ------台湾总统大选最新民调星期三(7月19日)出炉,国民党总统参选人侯友宜再次垫底。台中市长卢秀燕当晚出席侯友宜在台中千人造势晚会,提到自己当初参选市长时,所有民调也说她会输。她强调民调怎么样没有关系,全力以赴认真拉票,就会创造奇迹……}

\entryitemWithDescription{AIT处长:北京没理由因赖清德过境美国而挑衅}{https://www.zaobao.com/news/china/story20230719-1415563}{美国在台协会(AIT)处长孙晓雅说,北京没有理由以台湾副总统赖清德过境美国为借口,进行任何挑衅行动。 综合路透社和《联合报》报道,孙晓雅星期三(7月19日)在台北举行的记者会上提到赖清德过境议题时说,这符合美国长期以来的作法,也符合美台之间非官方关系的本质,同时符合美国的一个中国政策。这次赖清德过境美国,是跟过去一样的惯例……}

\entryitemWithDescription{学者:中国料不会把气候议题独立于其他中美议题处理}{https://www.zaobao.com/news/china/story20230719-1415558}{中国国家副主席韩正(右)7月19日上午在北京会见美国总统气候问题特使克里(左)。(新华社) 美国总统气候问题特使克里和中国国家副主席韩正会面,呼吁中美把气候变化议题与更广泛的外交问题分开处理;韩正则敦促美国为中美各领域交流合作创造良好环境。 克里星期天(7月16日)起对中国展开四天访问,并在星期三(19日)访华最后一天与韩正会面……}