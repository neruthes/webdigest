\entryitemWithDescription{中美防长通话后数小时 美军机飞越台海}{https://www.zaobao.com/news/china/story20240418-3461732}{美国海军第七舰队发布的声明称,一架P-8A``海神''巡逻机(也用于执行反潜任务)当地时间星期三(4月17日)在国际空域飞越台湾海峡。(互联网) 一架美国海军巡逻机星期三飞越敏感的台湾海峡;此前数小时,中美防长刚刚举行一年多来的首次实质性接触。受访学者认为,这是当前中美关系既有分歧竞争,又需要对话合作的生动写照……}

\entryitemWithDescription{中美再次就产能过剩交锋 学者指美国为对华实施更多制裁铺路}{https://www.zaobao.com/news/china/story20240417-3461419}{美国财政部长耶伦(中)星期二(4月16日)在华盛顿与主持中美经济工作组会议的中方代表中国财政部副部长廖岷(右),以及中国人民银行副行长宣昌能(左)会面。(法新社) 中美经济工作组会议再次就中国产能过剩问题进行交锋,美国重申担忧中国工业产能过剩,中国则关切美国对华经贸限制措施……}

\entryitemWithDescription{香港国安法落地近四年291人被捕 学者:社会政治化氛围过去几年大减}{https://www.zaobao.com/news/china/story20240417-3461703}{香港保安局星期三(4月17日)披露,截至3月8日,共有291人涉嫌从事危害国家安全的行为和活动被捕。图为摄于1月18日香港维多利亚港旁的海滨长廊。(法新社) 《香港国安法》实施近四年,共有291人因涉嫌违反相关法律被拘捕。受访学者认为,香港社会的政治化氛围在过去几年确实大大减少,这有赖于国安法的实施,但当局未来仍有必要加强国安教育宣传……}

\entryitemWithDescription{台湾自造潜舰专案召集人黄曙光请辞 蔡英文尚未批准}{https://www.zaobao.com/news/china/story20240417-3460534}{台湾``总统府国造潜舰专案小组''召集人黄曙光上将,以身心俱疲为由请辞,声称与他人和政治因素都无关。(台湾国防部海军司令部提供) 台湾第一艘自造潜舰仍在测试中,``国造潜舰''专案召集人黄曙光却向台湾总统蔡英文请辞,声称身心俱疲,辞职无关政治。但他未来还得厘清针对他的贪污指控,与他发起的诽谤官司,台湾海军能否建构不对称战力也成关注焦点……}

\entryitemWithDescription{中国田协计划出台指导意见 规范路跑赛事商业竞争}{https://www.zaobao.com/news/china/story20240417-3460520}{中国马拉松选手何杰(右二)与三名非洲选手,在上星期天(4月14日)的北京半程马拉松比赛中,几乎是并排跑向终点。(路透社) (北京综合讯)北京半程马拉松比赛疑似造假风波持续发酵之际,中国田径协会星期二召开会议,提出将尽快研究出台规范路跑赛事商业竞争的指导性意见,加强市场规范化等方面的指导和培训力度。 中国田协在官网发布消息称,今年春季以来,中国各地路跑赛事集中举办,群众参赛热情高涨……}

\entryitemWithDescription{缅北冲突再起 解放军中缅边境空防实弹演习}{https://www.zaobao.com/news/china/story20240417-3460874}{(北京综合讯)中国军队从星期三起在中缅边境中方一侧,举行空防实兵实弹演习。随着缅甸北部局势近期再出现不确定性,北京借此次演习向各方发出休战的明确信号。 解放军南部战区星期三(4月17日)在微信公众号上宣布,根据年度训练计划,自当天起,南部战区组织陆军、空军部队,在中缅边境中国一侧举行空防实兵实弹演习。 南部战区称,此次演习旨在检验战区部队侦察预警、立体封控、警示驱离、防空打击能力……}

\entryitemWithDescription{中国国台办副主任罕见会见美国务院高官 就台湾问题阐明立场}{https://www.zaobao.com/news/china/story20240417-3460340}{国台办星期二(4月16日)在官网发布消息称,国台办副主任仇开明(图中)星期一(15日)与美国分管东亚与太平洋事务助理国务卿康达等人会面。(互联网) (北京/台北综合讯) 美国助理国务卿康达访华期间,罕有地与中国对台系统官员会面,双方就台湾问题进行讨论。这一不同寻常之举,显示台海议题未来一段时间对中美关系走势的重要性……}

\entryitemWithDescription{中国驻欧盟使团前团长傅聪 出任常驻联合国代表}{https://www.zaobao.com/news/china/story20240417-3460196}{中国新任常驻联合国代表、特命全权大使傅聪(右)星期二(4月16日)在纽约联合国总部,向联合国秘书长古特雷斯递交了全权证书。(中新社) (纽约综合讯)中国驻欧盟使团前团长傅聪,出任中国常驻联合国代表、特命全权大使。 根据中国常驻联合国代表团网站的消息,傅聪星期二(4月16日)在美国纽约的联合国总部,向联合国秘书长古特雷斯递交了全权证书。 傅聪表示,中国始终是联合国事业的坚定支持者和积极贡献者……}

\entryitemWithDescription{杨丹旭:中国对美舆论要调整?}{https://www.zaobao.com/news/china/story20240417-3455076}{作者说,中国媒体对耶伦热衷中华美食的报道,淡化了火药味,拉升了民间对她的好感。(法新社) 美国财政部长耶伦最近访华,有中国学者注意到,中国媒体对耶伦此行的报道,与近年来对美国的报道有些不同,``这次舆论的松弛感更强一些''。 回想起来,4月4日下午耶伦飞抵广州,斜跨一个帆布袋、手拎一个公文包走下飞机,一身朴素的装扮在中国社交媒体引发热议。一些网民感叹,她就像一位邻家老太太……}

\entryitemWithDescription{赖清德内政大权一把抓 外交国防延续蔡英文路线}{https://www.zaobao.com/news/china/story20240416-3452088}{台湾候任总统赖清德4月10日宣布,委任民进党前主席卓荣泰出任行政院长。(法新社) 台湾候任总统赖清德的新政府人事布局,被视为内政大权一把抓,外交国防则延续现任总统蔡英文的路线。 现任民进党主席的赖清德将在5月20日宣誓就任总统,新内阁团队也将于同日上任。他在4月10日宣布,委任民进党前主席卓荣泰出任行政院长……}

\entryitemWithDescription{中国3月新房售价降幅为近九年最大 分析:北京或决心摆脱地产拖累}{https://www.zaobao.com/news/china/story20240416-3454763}{路透社基于中国国家统计局的数据计算得出,中国3月新建商品住宅价格同比下降2.2\%,为2015年8月以来的最大降幅。图为民众3月15日在北京一售楼处了解楼盘信息。(中新社) 中国房地产市场持续低迷,3月新建商品住宅售价降幅为近九年最大。分析认为,中国房地产风险仍在出清中,但中国第一季经济增长优于预期,可能会让北京坚定通过转换增长模式,逐步摆脱房地产对经济的拖累……}

\entryitemWithDescription{五角大楼:美中防长举行视讯通话}{https://www.zaobao.com/news/china/story20240416-3455107}{(华盛顿综合电)美国五角大楼发布文告称,美国国防部长奥斯汀与中国国防部长董军星期二(4月16日)举行视讯通话,是美中防长近18个月以来首度实质对话。 综和法新社和路透社的报道,两人讨论美中国防关系以及区域和全球安全议题。奥斯汀强调尊重国际法所保障的公海航行自由重要性,特别是在南中国海。 美国国防部指出,两人也谈及俄乌战争、朝鲜问题……}

\entryitemWithDescription{中越防长会晤 签署设立海军热线谅解备忘录}{https://www.zaobao.com/news/china/story20240416-3454244}{(北京综合讯)中国国防部长董军上星期率团出席中越边境国防友好交流活动,并与越南国防部长潘文江举行会谈,并签署设立中越海军热线的谅解备忘录。 据中国国防部官网消息,中越第八次边境国防友好交流活动上星期四至五(4月11日至12日),先后在越南老街省老街市及中国云南省红河州河口县国际口岸举行,董军与潘文江分别率团出席并举行会谈。 这是董军去年12月出任中国国防部长后,首次踏出国境进行军事外交访问……}

\entryitemWithDescription{王毅:相信伊朗能把握好局势 赞赏强调不针对周边国家}{https://www.zaobao.com/news/china/story20240416-3453901}{(北京综合讯)中国外交部长王毅星期一称,中国注意到伊朗表示对以色列采取的行动是有限度的。他相信伊朗能够把握好局势,避免中东局势进一步动荡。 伊朗上星期六(4月13日)对以色列发动大规模导弹与无人机袭击,以报复4月1日在伊朗驻叙利亚首都大马士革的使馆馆舍遇袭事件。国际社会对此高度关注,呼吁伊以两国保持克制……}

\entryitemWithDescription{中国经济首季增长5.3\%超预期 分析:增长势头仍不稳固}{https://www.zaobao.com/news/china/story20240416-3454359}{中国国家统计局星期二(4月16日)公布中国第一季国内生产总值(GDP)同比增长5.3\%,超出预期,分析师认为制造业生产为主要的驱动力。图为江苏南京一家制造企业内2月29日的生产情况。(中新社) 中国第一季经济同比增长5.3\%,明显超出市场预期。受访分析师认为强劲的制造业和出口活动拉动了增长,但3月出现的数据波动凸显当前的增长势头并不稳固……}

\entryitemWithDescription{百度AI聊天机器人文心一言用户数突破2亿}{https://www.zaobao.com/news/china/story20240416-3454640}{百度创始人李彦宏星期二在百度AI开发者大会上说,``文心一言''用户数已突破2亿。(彭博社) (香港路透电)中国互联网巨头百度说,该公司开发的人工智能(AI)聊天机器人``文心一言''用户数已突破2亿。 百度创始人、董事长兼首席执行官李彦宏星期二(4月16日)在百度AI开发者大会上说,文心一言目前每天应用程序编程接口(API)的调用量也突破了2亿次,服务客户数达到8.5万……}

\entryitemWithDescription{欧盟拟对中国医疗器材采购启动调查 迈瑞:一向依法合规}{https://www.zaobao.com/news/china/story20240416-3453855}{(华盛顿/广州综合讯)据报欧盟将对中国医疗器材采购启动调查,以消除对中国政府偏袒国内供应商的担忧。消息传出后,中国A股医疗器械板块星期二(16日)集体走低。 彭博社星期一(4月15日)引述知情人士报道,欧盟可能最早在4月中旬宣布对中国医疗器材采购启动调查。这一调查将先从企业和成员国收集信息,然后开始与中国就市场公平和开放进行谈判……}

\entryitemWithDescription{北京半马疑似出现造假 中国选手被指获``保送''夺冠}{https://www.zaobao.com/news/china/story20240416-3453318}{北京星期天举行的半程马拉松比赛上,三名非洲跑手在冲刺阶段被指刻意让中国选手何杰(右二)夺冠。(路透社) (北京综合讯) 北京半程马拉松比赛疑似出现造假丑闻,三名非洲跑手冲刺时被指故意留力,刻意让中国选手何杰夺冠。 陷入争议的这场赛事,是星期天(4月14日)举行的北京国际长跑节---北京半程马拉松男子组比赛……}

\entryitemWithDescription{北京强化中华民族论述应对``去中国化''}{https://www.zaobao.com/news/china/story20240416-3451118}{分析认为,北京藉由习马二会强化中华民族论述,意在应对民进党长期执政造成的``去中国化''现象。 图为台独团体2月在台北街头放置的宣传旗帜。(法新社) 习马二会侧重中华民族论述,引发两岸舆论关注,甚至传出台湾候任总统赖清德若能把握``中华共识'',就能促使两岸关系趋缓。不过学者分析认为,北京此时强化中华民族论述,意在应对民进党长期执政造成的``去中国化''现象……}

\entryitemWithDescription{戴庆成:以玉女剑法破解吸星大法}{https://www.zaobao.com/news/china/story20240416-3438350}{作者说,香港去年在疫后重新通关,零售业的表现一直非常疲弱,步入2024年仍未有起色。(彭博社) 香港武侠小说名家金庸一生创作了15部武侠小说,作品具有深刻的人文、社会及艺术价值,陪伴了一代又一代香港人成长,可以说是许多港人的集体回忆……}

\entryitemWithDescription{朔尔茨访上海 要求中国不要倾销和生产过剩}{https://www.zaobao.com/news/china/story20240415-3438777}{德国总理朔尔茨(右)星期一(4月15日)上午在上海参观科思创亚太创新中心。(新华社) (上海综合讯)德国总理朔尔茨星期一强调,他支持开放和公平的市场,要求中国不要倾销和生产过剩,同时呼吁欧盟不要出于保护主义的私利而采取行动。 朔尔茨星期日抵达中国,是今年首位访华的西方大国领导人。他星期一(4月15日)上午飞抵上海,继续访华行程,外界密切关注他如何就经贸问题表态……}

\entryitemWithDescription{广交会开幕 加强``新三样''相关展区}{https://www.zaobao.com/news/china/story20240415-3438344}{广交会星期一(4月15日)在广州开幕,共有2万9000家企业参展。图为新能源汽车展区吸引采购商参观。 (中新社) (广州/北京综合讯)中国规模最大的国际贸易展会广交会星期一(4月15日)在广州开幕,在美国与欧盟关注中国外贸``新三样''产能过剩之际,本届展会将加强培育新能源汽车及智慧出行等展区……}