\entryitemWithDescription{杨丹旭:中国反向``卡脖子''}{https://www.zaobao.com/news/china/story20230705-1410748}{美国财政部长耶伦访华前夕,北京以国家安全为由出台最新的原材料出口管制令,让中美科技战加速升温。 中国商务部和海关总署星期一(7月3日)发布公告,从8月1日起,中国将对镓、锗相关物项实施出口管制,理由是``维护国家安全和利益''。 根据公告,这些物项要出口,必须办理出口许可手续,出口商得通过省级商务主管部门向商务部提出申请,还得上报物项的最终接收方和用途……}

\entryitemWithDescription{侯友宜改口称唯有台海和平才缩短兵役 台媒评论:忌惮美方反弹}{https://www.zaobao.com/news/china/story20230704-1410747}{侯友宜星期一刚宣示若当选总统,将把兵役由一年恢复四个月;不料隔天就改口,称重要前提是台海两岸必须先和平稳定。图为台军一辆架设美制拖式导弹射击系统的悍马军车,星期一在屏东准备进行实弹射击演习。(路透社) 台湾在野国民党总统参选人侯友宜,星期一(7月3日)首度表态接受``合乎中华民国宪法的九二共识'',并宣示若当选总统,``确定两岸稳定和平以后,我就(把兵役由一年)恢复四个月''……}

\entryitemWithDescription{英美强烈谴责香港通缉流亡海外民运人士}{https://www.zaobao.com/news/china/story20230704-1410741}{香港警方国安处首次悬红通缉八名涉嫌违反《香港国安法》且潜逃海外的民主派人士,引起美国强烈批评,认为是威胁全球人民人权和基本自由的危险先例。香港特首李家超回应时表示,支持警方行动,会用尽一切合法手段终身追究他们的法律责任。 李家超星期二(7月4日)出席行政会议前表示,这次行动是要发出明确信号,表明港府不会容忍有关八人的违法行为。他相信警方会拘捕到他们,无论相关人士走到天涯海角,特区政府都执法必严……}

\entryitemWithDescription{定格百年摄影展苏州启幕 新中历史照片最吸引中国读者}{https://www.zaobao.com/news/china/story20230704-1410730}{伴随着``定格百年''的背景灯亮起,《联合早报》中国摄影展星期二(7月4日)在苏州金鸡湖畔的苏州中心启幕,标志着早报下来两个月在中国的五城摄影巡展拉开序幕。 这是早报继今年4月在新加坡办摄影展后,在海外举办的首场百年报庆摄影展。 《联合早报》总编辑吴新迪在启幕礼上,讲述了早报与中国过去30多年的缘分……}

\entryitemWithDescription{蒋万安今起访新四天借镜城市治理}{https://www.zaobao.com/news/china/story20230704-1410727}{台北市长蒋万安表示,希望借镜新加坡的城市治理,提升市民生活环境。(台北市政府提供) 台北市长蒋万安星期三(7月5日)率团访问新加坡四天进行市政交流,拜访建屋发展局、陆路交通管理局、市区重建局和永续发展与环境部,以及参观组屋区、小贩中心、滨海湾花园和星耀樟宜。 44岁的蒋万安是国民党中生代领袖,也是台湾已故前总统蒋经国的孙子。这是他上任半年来第一次出访……}

\entryitemWithDescription{贺建奎再提争议性研究 用基因编辑预防阿尔茨海默病}{https://www.zaobao.com/news/china/story20230704-1410716}{曾因``基因编辑婴儿''获刑三年的中国科学家贺建奎发布一份研究提案,希望利用基因编辑技术来预防阿尔茨海默病,他的这一提议再次引发争议。 综合彭博社与CNN报道,贺建奎上周四(6月29日)在推特账号公布一项研究计划,他认为,阿尔茨海默病可能有基因特性,提议对某段特定的基因序列进行编辑,来测试这能对阿尔茨海默氏症产生的预防作用……}

\entryitemWithDescription{国台办回应侯友宜九二共识说 重申两岸同属一中}{https://www.zaobao.com/news/china/story20230704-1410695}{台湾最大在野党国民党总统参选人侯友宜星期一(7月3日)首度表态支持``合乎中华民国宪法''的``九二共识''后,中国大陆隔日回应,重申``九二共识''的核心意涵是两岸同属一中,北京愿在坚持``九二共识''、反对``台独''的共同政治基础上,与国民党保持良性互动……}

\entryitemWithDescription{【东谈西论】国民党``侯老三''放大招,总统选情能起死回生吗?}{https://www.zaobao.com/news/china/story20230704-1410659}{2024年台湾总统大选候选人(从左到右):侯友宜(中国国民党)、柯文哲(台湾民众党)、赖清德(民主进步党)。(互联网) 2024年台湾总统选举进入半年倒计时,各政党厉兵秣马加速冲刺。 民调落后的最大在野党国民党候选人侯友宜星期一(7月3日)晚上突然放大招,松口表示支持九二共识。侯友宜还承诺维护两岸和平,让台湾的兵役制度从一年恢复到四个月……}

\entryitemWithDescription{戴庆成:香港保得住``廉洁之都''地位吗?}{https://www.zaobao.com/news/china/story20230704-1410367}{上周六(7月1日)是香港回归中国26周年纪念日。连月来,建制派媒体持续不断地报道和宣传香港过去26年在经济、文化、教育等领域取得的一系列成就。有趣的是,竟然没有人指出香港九七年回归后在廉政方面的成绩。 ``香港,胜在有ICAC。''香港廉政公署(ICAC)自1974年成立以来一直独立运作,不惧不偏地奋力反贪……}

\entryitemWithDescription{台总统选战或四角混战 侯友宜首次承认九二共识}{https://www.zaobao.com/news/china/story20230703-1410388}{台湾总统大选进入最后半年倒计时,而最大在野党国民党参选人侯友宜的民调持续落后,反观鸿海创办人郭台铭正酝酿独立参选。在选情告急下,侯友宜星期一(7月3日)首次松口表示支持符合``中华民国宪法''的``九二共识''。 北京认为九二共识体现``一个中国''原则,将之定调为两岸关系发展的政治基础,但现执政的民进党政府拒绝接受,指九二共识就是``一国两制''……}

\entryitemWithDescription{港警国安处悬红800万港元 通缉八名违反国安法的海外港人}{https://www.zaobao.com/news/china/story20230703-1410323}{香港警务处国安处总警司李桂华星期一(7月3日)在记者会上介绍悬红通缉八名被指违反《香港国安法》而潜逃海外的香港人。(路透社) 香港警务处国家安全处星期一宣布,悬红800万港元(约138万新元)通缉八名被指违反《香港国安法》的海外港人,包括三名立法会前议员许智峰、罗冠聪和郭荣铿……}

\entryitemWithDescription{盗全校资料建颜值网 中国名校硕士毕业生被刑拘}{https://www.zaobao.com/news/china/story20230703-1410321}{中国人民大学一名硕士毕业生盗取全校学生个人信息并制作成颜值打分网站,为此被警方刑事拘留。 据北京海淀区公安局的官方微博星期一(7月3日)通报,经警方调查,嫌疑人马某某(男,25岁)涉嫌非法获取人民大学部分学生个人信息等违法犯罪行为。 通告称,目前马某某已被该局依法刑事拘留,案件正在进一步调查中。警方高度重视公民个人信息保护,对于相关违法犯罪,将依法予以严厉打击……}

\entryitemWithDescription{港保安局长:23条立法涵盖间谍及网络罪行}{https://www.zaobao.com/news/china/story20230703-1410288}{香港特首李家超早前表示,会在今年或最迟明年完成《基本法》第23条立法。港府保安局局长邓炳强近日透露,立法方向及部分条例内容会涵盖间谍活动及网上违反国家安全的罪行。 根据香港《大公报》星期一(7月3日)刊登的专访内容,邓炳强指出,一些危害中国国家安全的分子仍以不同方式进行鼓吹和渗透,以及串谋勾结外国势力……}

\entryitemWithDescription{于泽远:中国空军又添新``杀器''}{https://www.zaobao.com/news/china/story20230703-1410019}{中国空军明星装备歼-20隐身战机,这两天又成了广大军迷议论的热点,原因是它装备了两台国产涡扇-15发动机并成功首飞。一些军事观察人士认为,歼-20终于彻底克服了``心脏病'',成为真正可以与美军F-22和F-35隐身战机一较高下的新``杀器''……}

\entryitemWithDescription{中国股汇两市上半年低迷收官 分析:资本市场短期内仍低位徘徊}{https://www.zaobao.com/news/china/story20230702-1410047}{中国股市和汇市今年初均强势回弹,但随着经济复苏势头放缓,资本市场的疲态在第二季度逐渐显露。 中国股市和汇市上半年低迷收官,折射出投资者对中国经济的悲观预期,以及对官方经济刺激力度不足的担忧。分析预计,由于中国政府出台大规模刺激政策的概率不高,资本市场短期内仍将在低位徘徊……}

\entryitemWithDescription{韩正:有影响力的大国应劝和促谈}{https://www.zaobao.com/news/china/story20230702-1410036}{中国国家副主席韩正星期天表示,中国在全球地缘政治局势动荡的大环境下,愿同各国共同维护世界和平安全,并指出有影响力的大国,应根据当事国需要和愿望劝和促谈、斡旋调停,通过对话解决纷争。 据中国外交部网站、澎湃新闻消息,韩正星期天(7月2日)在清华大学出席第十一届世界和平论坛开幕式并致辞时,作出上述表述……}

\entryitemWithDescription{中国游客在法国遇袭 驻马赛总领馆向法方提出交涉}{https://www.zaobao.com/news/china/story20230702-1410034}{法国警察开枪打死非裔少年引发连日骚乱,一辆载着中国游客的旅游巴士在法国东南部城市马赛也被攻击,有游客受伤。中国驻马赛总领馆向法国提出交涉,要求确保中国公民安全。 据中国外交部领事司``领事直通车''微信公号7月2日消息,中国驻马赛总领馆在得知中国旅游团在法国遭袭后启动领事保护应急机制,协助旅行团报警并紧急处置。 法国一名17岁阿尔及利亚少年上星期二(6月27日)因拒绝警方拦检遭开枪击毙……}