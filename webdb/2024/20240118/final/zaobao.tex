\entryitemWithDescription{王纬温:``双海''加紧联动下印太博弈再升级?}{https://www.zaobao.com/news/china/story20240118-1462762}{美国和菲律宾去年4月大幅升级军事关系,并在南中国海和台海拉开``双海''联动帷幕后,菲律宾总统小马可斯的对华政策不仅日益强硬,与中国在南中国海摩擦也不断加剧。 小马可斯本周更罕见向赖清德发文,祝贺这位中国大陆眼里的``顽固台独工作者''当选台湾总统。此举在为马尼拉与台北关系加温的同时,也加码挑动北京敏感神经,还进一步坐实双海联动对华围堵的看法,中西方在印太的博弈大局今年预计将比去年更紧张……}

\entryitemWithDescription{解放军台湾大选后首次执行联合战备警巡 18架次军机在台海活动}{https://www.zaobao.com/news/china/story20240118-1462763}{(台北/北京综合讯)台湾国防部星期三晚间公告,中国大陆出动空军战机18架次在台湾空域配合海军执行联合战备警巡。这是自今年台湾大选结束后,解放军在台海执行的大规模军事活动。 台湾国防部星期三(1月17日)晚间9时50分发出新闻稿,指出台军自当天晚上7时50时起,陆续侦获解放军苏恺-30、运-8等各型主、辅战机计18架次出海活动……}

\entryitemWithDescription{澳洲驻美大使陆克文:中国需减少强调意识形态 以恢复企业信心}{https://www.zaobao.com/news/china/story20240117-1462760}{中国国际经济交流中心副理事长朱民星期三出席世界经济论坛时表示,中国经济有韧性,但挑战也是巨大的。(路透社) 澳大利亚驻美国大使陆克文说,中国当局须减少对意识形态的强调,并重新强调市场经济中正常的企业效益问题,才能从根本上恢复民营企业信心……}

\entryitemWithDescription{赖清德退出民进党最大派系新潮流 学者:为朝野政党创造和谐协商的弹性空间}{https://www.zaobao.com/news/china/story20240117-1462745}{台湾民进党主席赖清德在1月13日举行的大选中,以逾558万得票数胜出,当选新一任总统。图为他在胜选后的国际记者会上发表讲话。(路透社) 台湾民进党主席、候任总统赖清德星期三(1月17日)宣布退出党内最大派系新潮流,理由是为了客观推动国政、团结领导党内。受访学者分析,赖清德此举或可为太偏重自身派系解套,也为朝野政党创造议题合作、和谐协商的弹性空间……}

\entryitemWithDescription{中驻澳大使:中国与太平洋岛国建交非军事安全战略}{https://www.zaobao.com/news/china/story20240117-1462733}{中国驻澳大利亚大使肖千1月17日在中澳媒体新年酒会上致辞说,中澳双方在相互尊重基础上妥善解决各自合理贸易关切。自澳洲总理阿尔巴尼斯上任以来,中澳关系有回暖迹象。(中新社) (堪培拉综合讯)中国驻澳大利亚大使肖千在太平洋岛国瑙鲁与中国建交后说,中国与该地区国家建立政治关系非关军事安全战略,澳大利亚无须为此感到焦虑……}

\entryitemWithDescription{台湾前外长:邦交不至于归零 但两岸不缓和挑战必增}{https://www.zaobao.com/news/china/story20240117-1462730}{对于台湾与诺鲁断交,台湾前外长林永乐1月17日表示, 他不认为台湾的邦交国会归零,但只要两岸关系不缓和下来,台湾新任政府想拓展国际空间势必会面临更多挑战。(缪宗翰摄) 南太平洋岛国瑙鲁决定与台湾断交后,台湾前外长林永乐星期三(1月17日)出席座谈会时指出,台湾的邦交国不至于归零;不过两岸关系不缓和,台湾国际空间挑战势必增加,而目前也还看不出新任总统当选人赖清德能与北京对话的基础……}

\entryitemWithDescription{中欧贸易紧张升级 李强重申希望欧盟审慎出台限制性经贸政策}{https://www.zaobao.com/news/china/story20240117-1462726}{中国总理李强(右)1月16日在瑞士达沃斯出席世界经济论坛2024年年会期间,与欧盟委员会主席冯德莱恩(左)会面。(新华社) (达沃斯/北京综合讯)在中欧贸易紧张局势升级之际,中国总理李强重申希望欧方在经贸问题上公正、透明对待中国企业,审慎出台限制性经贸政策。 据新华社报道,李强于当地时间星期二(1月16日)在瑞士达沃斯出席世界经济论坛2024年年会期间,与欧盟委员会主席冯德莱恩会面……}

\entryitemWithDescription{中国人口连续两年萎缩 长期风险上升冲击市场信心}{https://www.zaobao.com/news/china/story20240117-1462722}{2023年是中国全面解除冠病防控后首年,死亡人数同比增至1960年以来最高水平,新生儿人口则降至1949年以来最低水平。图为1月12日在北京公园里散步的祖孙老少。(路透社) 中国人口总数连续两年下降,且降幅进一步扩大,凸显出这个传统人口大国面对的长期风险持续上升,进一步加剧外界对中国经济前景的担忧……}

\entryitemWithDescription{清华大学再否认80\%毕业生出国留学}{https://www.zaobao.com/news/china/story20240117-1462713}{(北京综合讯)曾因毕业生出国留学问题被推上舆论风口浪尖的清华大学再次发文,否认80\%毕业生出国。 清华大学星期三(1月17日)在微信公众号发文,介绍毕业生去向……}

\entryitemWithDescription{大陆国台办回应``武统''声音:愿尽最大努力争取和平统一}{https://www.zaobao.com/news/china/story20240117-1462696}{中国大陆福建省平潭岛的一名男子,1月15日眺望台湾海峡波涛汹涌的海面。平潭岛是大陆距离台湾本岛最近的地方。(法新社) (北京/台湾综合讯)赖清德当选为新一任台湾总统后,中国大陆网络再次出现``武统''声音。中国国务院台湾事务办公室星期三称,这反映了台湾各界担忧民进党可能加剧台海形势紧张动荡……}

\entryitemWithDescription{杨丹旭:孙任泽的非正常死亡}{https://www.zaobao.com/news/china/story20240117-1462534}{刚看到同事传来的这篇报道,我以为我穿越了。这样的事情在十几二十年前时有耳闻,没想到还在上演。 中国媒体《财新》上周六发表一篇题为《嫌疑人孙任泽之死》的特稿。这名当时不满31岁的年轻人在2018年3月因为涉嫌寻衅滋事,被新疆伊犁的公安机关刑事拘留。同年9月,孙任泽在看守所接受审讯时昏迷,辗转在多家医院接受救治后,最终在当年11月不治身亡……}

\entryitemWithDescription{AIT主席罗森伯格对瑙鲁与台断交表示``遗憾''}{https://www.zaobao.com/news/china/story20240116-1462531}{美国在台协会(AIT)主席罗森伯格(Laura Rosenberger,右)星期二(1月16日)由发言人邓艾德(Ed Dunn)陪同,在台北开记者会回应瑙鲁与台湾断交议题。(路透社) 美国在台协会(AIT)主席罗森伯格(Laura Rosenberger)星期二(1月16日)对南太平洋岛国瑙鲁决定与台湾断交表示``遗憾''(unfortunate)……}

\entryitemWithDescription{国民党评估拿下立院龙头可能性高 赵少康建议立院副院长让给民众党}{https://www.zaobao.com/news/china/story20240116-1462517}{台湾立法院龙头改选在即,以关键少数自居的在野民众党立委开出改革立法院四条件,但执政的民进党立院党团以立法院正、副院长是否适任为重,最大在野国民党也以尊重立院党团意见为由,均给民众党软钉子碰。 依立法院长选举办法,第一轮选举若未过半,可以有第二轮选举,故除非民众党八席立委集体支持民进党,否则作为第一大党的国民党不分区首席当选立委韩国瑜,笃定当选立法院长……}

\entryitemWithDescription{小马可斯罕见祝贺赖清德当选 学者:应审慎看待不过度解读}{https://www.zaobao.com/news/china/story20240116-1462512}{菲律宾总统小马可斯罕见发文祝贺赖清德当选台湾总统后,菲律宾外交部和总统府隔天相继重申其``一个中国''政策。不过,北京仍对小马可斯言论表达强烈不满,并召见菲律宾驻华大使提出严正交涉。 受访台湾学者认为,应审慎看待小马可斯此次发文,不应过度解读,菲律宾对台议题的政策大方针未变,也未见台菲关系有重大突破迹象。 台湾民进党主席赖清德星期六(1月13日)在总统选举中胜出……}

\entryitemWithDescription{北京就美日欧等国拟派团赴台致贺严正交涉}{https://www.zaobao.com/news/china/story20240116-1462504}{针对美国、日本和欧洲几个国家据报表示拟于台湾选举后派团赴台致贺,中国大陆外交部回应称,中国大陆已对有关国家表示强烈不满和坚决反对。图为台湾民众1月14日在台北著名景点中正纪念堂附近散步。(法新社) (北京/新加坡综合讯)针对台湾媒体报道,美国、日本和欧洲几个国家表示拟于台湾选举后派团赴台致贺,中国大陆外交部发言人在例行记者会上回应称,中国大陆已对有关国家表示强烈不满和坚决反对,并已提出严正交涉……}

\entryitemWithDescription{预计达90亿人次 中国今年春运流动量将创新高}{https://www.zaobao.com/news/china/story20240116-1462494}{(北京综合讯)中国2024年春运将在下星期五(1月26日)启动,预计今年春运中国跨区域人员流动量将创历史新高,达到90亿人次。 据中国青年网报道,中国交通运输部副部长李扬星期二(1月16日)在发布会上说,今年中国春运发生了结构性变化,传统营业性运输包括铁路、公路、民航、水运客运出行人次预计达18亿人次,其余80\%都将自驾车出行,自驾车出行将创历史新高。 今年中国春运将持续至3月5日……}

\entryitemWithDescription{新疆阿勒泰多处雪崩 逾千游客滞留景区一周}{https://www.zaobao.com/news/china/story20240116-1462479}{经过连日清雪作业,新疆阿勒泰地区布尔津县的雪阻路段于星期二(1月16日)疏通,受困近一周的喀纳斯景区游客纷纷开车离开。(新华社) (北京/武汉综合讯)受持续暴雪影响,新疆阿勒泰地区通往喀纳斯景区部分路段发生雪崩,积雪深度达数米,逾千名游客近一周来被困景区,雪阻路段星期二(1月16日)终于疏通……}

\entryitemWithDescription{【东谈西论】赖清德当选会如何冲击两岸关系?}{https://www.zaobao.com/news/china/story20240116-1462449}{民进党赖清德以558万票(40.05\%)当选台湾总统,被外界认为是``少数总统''领导一个弱势政府。 (法新社) 2024年台湾总统选举已经在1月13日落幕。民进党籍的候选人赖清德以四成选票在三角战中胜出。 迎接赖清德的绝非坦途。民进党失去了立法院多数优势,他将是一个弱势总统,领导一个弱势政府,面对来自内外的压力。美国总统拜登在第一时间就直言,美国不支持台独……}

\entryitemWithDescription{戴庆成:香港告别``三权分立''?}{https://www.zaobao.com/news/china/story20240116-1462300}{香港律政司宪制及政策事务科上个月在《基本法简讯》一篇探讨香港政制的文章中,提及特区权力全属中央政府转授,使用``三权分立''描述政制并不妥当。(路透社档案图) 台湾民进党在刚刚过去的周六(1月13日)如预料赢得总统大选,不过在立委选举的表现却不理想,由国民党取而代之成为立法院最大党……}

\entryitemWithDescription{被誉为``台湾曼德拉''的民进党前主席施明德过世}{https://www.zaobao.com/news/china/story20240116-1462314}{1993年11月,施明德接任民进党主席,并于1994年5月蝉联党主席。1996年3月,因民进党在台湾选举中大败,施明德请辞党主席。(自由时报) 被誉为``台湾曼德拉''的民进党前主席施明德星期一(1月15日)与世长辞,享寿83岁。 这一天也是施明德的生日,施明德的女儿施笳和施蜜娜在脸书发文说,施明德没有忌日,只有生日,一生所捍卫的精神和价值会在世界诞生、长长久久地茁壮……}

\entryitemWithDescription{美代表团选后访台 大陆表示坚决反对干涉台事务}{https://www.zaobao.com/news/china/story20240116-1462313}{美国跨党派代表团星期一(1月15日)在台北的民进党中央党部与民进党籍的正副总统当选人赖清德(右二)和萧美琴(右一)会面。美国代表团成员包括前国安顾问哈德利(右三)。(法新社) 由美国前高官组成的跨党派代表团,在台湾大选隔天高调抵台,并在星期一(1月15日)一天内会见台湾总统蔡英文、民进党籍的正副总统当选人赖清德和萧美琴,以及落败的两名在野党候选人。对此,中国大陆表示坚决反对……}

\entryitemWithDescription{瑙鲁与台湾断交 学者:北京给赖清德下马威 释出坚决反台独信号}{https://www.zaobao.com/news/china/story20240115-1462310}{瑙鲁星期一(1月15日)毫无预警宣布与台湾断交后,位于台北天母的使馆特区外悬挂的瑙鲁国旗同一天被撤下。(路透社) 台湾民进党主席赖清德赢得总统选举后不到48小时,台湾友邦瑙鲁星期一(1月15日)毫无预警宣布与台湾断交,与中国大陆复交。 台海两岸之间的``邦交国战'',一向是北京压缩台湾国际空间的手段之一,以及北京释放警告的方法……}

\entryitemWithDescription{王毅首度就红海紧张局势公开发声 分析:不点名吁胡塞及美英保持克制}{https://www.zaobao.com/news/china/story20240115-1462301}{中国外长王毅星期天(1月14日)在埃及开罗首度就红海紧张局势公开发声,呼吁停止袭扰民船。(路透社) 也门胡塞武装持续袭击红海船只近两个月之际,正在非洲出访的中国外长王毅在埃及首度就红海紧张局势公开发声,呼吁停止袭扰民船,也表示安理会从未授权任何国家对也门使用武力,应避免给局势火上浇油,不点名呼吁胡塞武装及美国和英国在红海保持克制……}