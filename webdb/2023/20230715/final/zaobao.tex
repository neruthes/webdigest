\entryitemWithDescription{黄小芳:急速升温的气候变化挑战}{https://www.zaobao.com/news/china/story20230715-1413981}{2021年刚到北京的第一个夏天,36、37摄氏度的高温令我感到惊讶。和朋友聊起北京夏天的天气时,他笑称``不用担心,北京的最高温度不会超过39摄氏度''------当时没有明白他的笑点,后来才发现根据官方规定,若气温达到40摄氏度以上,当地就必须停止室外露天作业。 不过,朋友的说法并不可靠……}

\entryitemWithDescription{上海报业集团副总程峰被查}{https://www.zaobao.com/news/china/story20230713-1413665}{上海报业集团副总经理程峰涉嫌严重违纪违法,正接受上海市纪委监委审查调查。(互联网) 上海市人大常委会主任董云虎落马次日,上海报业集团副总经理程峰也被调查。加上去年底被查的上海东方网原总裁徐世平,上海市宣传系统八个月来已有三名大员卷入反腐风暴。 上海市政府新闻办微信号``上海发布''星期四(7月13日)通报,程峰涉嫌严重违纪违法,目前正接受上海市纪委监委纪律审查和监察调查……}

\entryitemWithDescription{业者:香港进口禁令或将致两三成日料店倒闭}{https://www.zaobao.com/news/china/story20230713-1413620}{香港餐饮业者预计,日本10个都县的水产品若被禁进口,当地两至三成的日式餐厅将倒闭。 据香港01报道,香港餐饮联业协会会董、日式餐厅负责人陈强星期四(7月13日)在电台节目上推测,禁令如果实行,当地日式餐厅生意估计将下跌五成,若情况持续两至三个月,相信将有两至三成餐厅倒闭。 陈强透露,日式餐厅现在生意已下跌两至三成,而且有食客反映,若日本排放核处理水,将考虑改吃其他食物……}

\entryitemWithDescription{重庆万州新一轮强降雨 致1.6万余人受灾}{https://www.zaobao.com/news/china/story20230714-1414005}{重庆市万州区星期五(7月14日)再度遭遇新一轮暴雨袭击,停靠在江南新区路边的部分车辆陷在洪水淤泥里。(新华社) 10天前因暴雨发生洪涝导致重大伤亡事故的重庆市万州区,再度遭遇新一轮暴雨袭击。当地官方星期五(7月14日)清晨紧急升级防汛应急响应,要求做好断道、停工、停学、停业、停运、停游、停航等措施……}

\entryitemWithDescription{学者:中美交流恐成``相互指责''互动}{https://www.zaobao.com/news/china/story20230714-1413997}{美国丹佛大学国际关系教授赵穗生星期五(7月14日)在新加坡国立大学东亚研究所举办的线上研讨会上,分享对中美关系前景的看法。(线上研讨会截屏) 中美高层近期频密互动释放两国关系回暖信号,但长期研究中国外交政策的学者认为,这些互动不足以稳定两国关系;在双方不做出重大妥协的情况下,这些访问与沟通已变成了一种相互指责的互动……}

\entryitemWithDescription{华春莹图文并茂质问:是谁颠覆了国际秩序?}{https://www.zaobao.com/news/china/story20230714-1413952}{中国外交部部长助理华春莹星期四(7月13日)深夜在推特发图文质问:``是谁颠覆了国际秩序?''(取自华春莹推特账号) 针对北大西洋公约组织维尔纽斯峰会公报对中国提出的指控,中国外交部部长助理华春莹在个人推特账号上图文并茂回怼:``是谁颠覆了国际秩序……}

\entryitemWithDescription{陆委会民调显示逾八成台湾民众不接受``一国两制''}{https://www.zaobao.com/news/china/story20230714-1413949}{据台湾政府的大陆委员会星期四(7月13日)公布今年度第二次民意调查结果,台湾八成以上的民众不赞成中国大陆提出的``一国两制'',近九成不认同解放军的军机军舰持续在台湾周边活动。 陆委会星期四在官网公布的新闻稿中指出,针对大陆持续对台综合施压,台湾人民坚决反对……}

\entryitemWithDescription{于泽远:福建舰终于要出海了?}{https://www.zaobao.com/news/china/story20230714-1413630}{网络近日流传的图片显示,仍在码头舾装的中国第三艘航母福建舰已拆除了位于斜甲板的三号电磁弹射器的安装工棚。军事学者分析,由于福建舰三条弹射器工棚的安装时间接近,三号弹射器安装工棚被拆除后,一号、二号弹射器安装工棚也将在近期被拆除,这意味着已下水一年多的福建舰距离出海海试的日子不远了……}

\entryitemWithDescription{马英九基金会拟再增加申请三大陆人士来台被拒}{https://www.zaobao.com/news/china/story20230713-1413644}{台湾前总统马英九成立的基金会邀请北京大学等五所大陆高校37名师生7月15日来台交流,原拟再增加申请三人来台协助行政庶务,大陆委员会星期四(7月13日)以联合审查会已做决定,若有任何更动须依照程序提出申请间接拒绝。 这是三年多来,大陆高校首次组团赴台交流,备受两岸关注。 7月13日是马英九73岁生日,他在脸书发文说,促成大陆师生顺利来访,就是给他最好的生日礼物,也是两岸人民最想要的礼物……}

\entryitemWithDescription{美反潜巡逻机穿航台海 解放军东部战区:全程跟监警戒}{https://www.zaobao.com/news/china/story20230713-1413634}{美国一架P-8A巡逻侦察机星期四飞过台湾海峡的国际上空,引来战机全程跟监警戒。图为P-8A军机2016年在南中国海海域航行的莫姆森号驱逐舰上空飞过。(路透社) 中国大陆军机连日在台海周边展开大规模活动之际,美国海军一架巡逻侦察机星期四(7月13日)飞越了敏感的台湾海峡,并引来中国战机全程跟监警戒。 综合路透社与联合新闻网报道,涉事的是也可用于反潜任务的P-8A海神海上巡逻和侦察机……}

\entryitemWithDescription{台湾通过性平三法修法 权势性骚扰最重判三年}{https://www.zaobao.com/news/china/story20230713-1413622}{台湾政坛和娱乐圈等相继爆出多起性骚扰事件后,行政院星期四(7月13日)通过性平三法修正草案,权势性骚扰面临最重三年监禁,及最高100万元(新台币,下同,约4万2000新元)罚款。 综合联合新闻网和自由时报报道,台湾行政院会星期四通过``性别平等工作法''、``性别平等教育法''、``性骚扰防治法''修正草案。 本次修正主要是防堵权势性骚,台湾立法院临时会预计7月底完成性平三法修法……}

\entryitemWithDescription{中国将对生成式人工智能服务分类分级监管}{https://www.zaobao.com/news/china/story20230713-1413615}{中国将从8月15日起,对生成式人工智能服务实行包容审慎和分类分级监管。(路透社档案图) 中国将从8月15日起,对生成式人工智能服务实行包容审慎和分类分级监管。 据``网信中国''微信公众号星期四(7月13日)消息,中国国家网信办联合国家发展改革委、教育部、科技部、工业和信息化部、公安部、广电总局公布《生成式人工智能服务管理暂行办法》……}

\entryitemWithDescription{中国商务部:正与美方沟通雷蒙多访华事宜}{https://www.zaobao.com/news/china/story20230713-1413608}{中国商务部称,中国正就美国商务部长雷蒙多访华与美国沟通。(法新社) 中美高层近期互动频繁,中国商务部星期四(7月13日)称,正就美国商务部长访华与美国沟通,同时呼吁美国取消对中国企业的单边制裁。 根据中国商务部官网发布的信息,商务部新闻发言人束珏婷在例行记者会上回应媒体提问时说,中国对美国商务部长雷蒙多访华持开放和欢迎态度,正就此与美国沟通……}

\entryitemWithDescription{朱立伦强调国民党绝无可能``换侯'' 侯友宜和郭台铭互动过招}{https://www.zaobao.com/news/china/story20230712-1413315}{鸿海集团创办人郭台铭(左二)和新北市长侯友宜,7月12日在台东临别时才握手互相致意。(侯友宜竞选办公室) 台湾在野的国民党主席朱立伦星期三(7月12日)指出,国民党绝无可能撤换总统参选人侯友宜。他呼吁国民党同志团结一致,汇集所有在野力量共同下架执政的民进党。 鸿海集团创办人郭台铭独立参选消息甚嚣尘上,``拔朱换侯''逼宫之声四起,朱立伦星期三在国民党中常会致词时,斩钉截铁作上述表示……}

\entryitemWithDescription{学者:中国高层表态支持平台经济 不代表再度放松监管}{https://www.zaobao.com/news/china/story20230712-1413309}{中国总理李强7月12日主持召开平台企业座谈会,听取对更好促进平台经济规范健康持续发展的意见建议。中国副总理丁薛祥以及中国国务委员兼国务院秘书长吴政隆也出席座谈会。(新华社) 中国互联网平台企业整改潮告一段落后,高层发声力挺平台经济……}

\entryitemWithDescription{港府:日本若排放福岛核废水即禁水产进口}{https://www.zaobao.com/news/china/story20230712-1413299}{香港特区政府星期三(7月12日)宣布,一旦日本启动排放福岛核废水,香港将立即禁止从日本10个都县进口水产品。 据香港政府新闻的新闻稿,港府政务司司长陈国基、环境及生态局局长谢展寰星期三与日本驻港总领事冈田健一会面,表明特区政府这一立场。 这10个都县包括东京、福岛、千叶、栃木、茨城、群马、宫城、新潟、长野与琦玉,拟禁止进口所有鲜活、冷冻、冷藏、干制或以其他方式保存的水产品,还包括海盐与海藻……}

\entryitemWithDescription{港七惩教人员涉轮奸案被捕停职}{https://www.zaobao.com/news/china/story20230712-1413281}{香港七名监狱警官因涉及一起上周末发生的轮奸案而被捕,并被勒令停职。 综合路透社与网媒``香港01''报道,事发在上星期六(7月8日),31岁的女性受害者和朋友到尖沙咀一间公寓参加派对,七名惩教人员也到场,众人饮酒玩乐至翌日凌晨,受害者的友人离开单位,只剩下她与七名惩教人员继续饮酒,她后来感到头晕并失去意识。 她逐渐恢复意识时,怀疑遭三人在未经其同意下强行性交,其余四人则在旁自慰……}

\entryitemWithDescription{32架次大陆军机越台海中线 三个月来规模最大}{https://www.zaobao.com/news/china/story20230712-1413279}{台湾国防部星期三(7月12日)公布,中国大陆解放军实施海空联训,截至当天上午6时的24小时内,发现32架次大陆军机越过台湾海峡中线(简称台海中线),是三个月来规模最大的一次。 三个月前,北京在台湾总统蔡英文4月5日过境美国与美国众议院议长麦卡锡会面后,发动三天环台军演,并在4月10日派出91架次军机绕台、其中54架次越过台海中线,双双写下历来最高纪录……}

\entryitemWithDescription{中国驻伦敦使馆新址之争成为外交僵局}{https://www.zaobao.com/news/china/story20230712-1413256}{数名中英两国官员称,围绕中国驻伦敦大使馆新址的争议,已经演变成外交僵局。图为中国驻伦敦大使馆新址的外观。(路透社) 中国驻伦敦大使馆新址计划去年底遭到当地议会否决后,中英两国官员称目前使馆选址争议已升级为外交僵局(diplomatic standoff),破坏了两国修复关系的努力……}