\entryitemWithDescription{新闻人间:政坛``蜜獾''徐巧芯}{https://www.zaobao.com/news/china/story20240420-3471815}{新闻人间:政坛``蜜獾''徐巧芯(联合早报制图) 台湾在野国民党立委徐巧芯,因先生刘彦澧的姐姐刘向婕和姐夫杜秉澄涉入诈骗洗钱被羁押,饱受质疑;加上亲执政的民进党名嘴与侧翼对她发起总攻,尤其原生家庭的责难与不谅解,让上星期天(4月14日)在花莲为地震赈灾的她,说出``我若利用职权协助诈骗,我便辞职;若我的先生涉入,我会与先生离婚''的重话,旋即哭倒在国民党党团总召傅昆萁怀里……}

\entryitemWithDescription{温伟中:交流避险与朝贡舔共}{https://www.zaobao.com/news/china/story20240420-3470656}{台海和平稳定备受瞩目,不过两岸官方零互动已长达八年。两岸执政党之间缺乏沟通管道,误判局势的风险越来越大。 自称``务实台独工作者''的民进党主席赖清德,今年初在总统大选胜出;眼看在他未来四年的执政下,两岸关系不容易改善或融冰,最大在野党国民党领袖正前赴后继到中国大陆与对方高级官员交流。 据蓝营公开说明访问大陆的理由,不外都是要推动对话与和平,避免战争风险,或简言之:交流避险……}

\entryitemWithDescription{研究:中国近半城市正在下沉 洪水风险将加剧}{https://www.zaobao.com/news/china/story20240419-3470481}{(华盛顿综合讯)一项研究指出,中国近一半的主要城市正面临``中度至重度''程度的下沉;在海平面上升的情况下,这将使数百万人面临洪水风险。 据路透社报道,由华南师范大学北斗研究院副研究员敖祖锐领导的研究小组基于全国卫星雷达数据进行的这项研究,星期五(4月19日)发表于《科学》杂志上。研究测量了占中国城市人口四分之三的82个主要城市的地表在2015年至2022年间的升降幅度……}

\entryitemWithDescription{北京半马调查结果出炉 取消冠军与三外籍选手比赛成绩}{https://www.zaobao.com/news/china/story20240419-3471373}{中国马拉松选手何杰(右二)与埃塞尔比亚选手比基拉(右一),以及两名肯尼亚选手凯特尔(左二)和姆南加特参加上星期天(4月14日)的北京半程马拉松赛。(路透社) 北京半程马拉松比赛上星期天引发巨大争议,中国选手何杰被质疑因三位外籍选手``故意放水''才赢得男子组冠军。组委会星期五经调查后宣布,取消何杰及三位外籍选手的比赛成绩……}

\entryitemWithDescription{``熊猫外交''重启 旧金山明年将迎来一对中国大熊猫}{https://www.zaobao.com/news/china/story20240419-3471153}{(北京综合讯)随着中美关系出现企稳迹象,中国开始重启与美国的``熊猫外交''。继美国圣地亚哥动物园今年将迎来旅美大熊猫后,中美机构再签意向书,计划在2025年实现一对大熊猫前往旧金山。 据《北京日报》报道,中国野生动物保护协会与美国旧金山动物园星期五(4月19日)在北京签署《大熊猫国际保护合作意向书》……}

\entryitemWithDescription{赖清德就职倒数一个月 分析:北京加大施压不会停}{https://www.zaobao.com/news/china/story20240419-3470784}{距离台湾候任总统赖清德就职剩下一个月,中国大陆星期五(4月19日)宣布启用紧邻台湾金门、马祖两座离岛空域的衔接航线,由西向东运行。这是2023年12月20日,从金门远眺兴建中的中国大陆福建厦门翔安国际机场……}

\entryitemWithDescription{中国航母福建舰出现舰载机模型 引发近期海试猜测}{https://www.zaobao.com/news/china/story20240419-3470607}{最新曝光的网络照片显示,中国第三艘航空母舰福建舰上出现四型五架舰载机模型。(互联网) 中国第三艘航空母舰福建舰上近日出现了四型五架舰载机模型,航母的救生筏也已安装完毕。有军迷推测,福建舰或将在4月23日中国海军节当天进行海试……}

\entryitemWithDescription{韩咏红:美国``堤丰''中导系统部署菲律宾}{https://www.zaobao.com/news/china/story20240419-3467212}{备受瞩目的美日菲三边峰会结束后不久,美国太平洋陆军司令部当地时间本星期一(4月15日)就宣布,美国陆军第一多域特遣队(1st MDTF)已经在菲律宾吕宋岛北部部署新型陆基中程导弹发射系统``堤丰''(Typhon),作为美菲联合军事演习``盾牌24'' 的一部分。 美军文告形容这项部署为``历史性首次'',提升了美菲陆军协同能力、备战水平和防务能量……}

\entryitemWithDescription{AI热潮大赢家台积电业绩超预期 带动台湾科技业}{https://www.zaobao.com/news/china/story20240418-3465386}{全球最大晶片制造商台积电预估,第二季营收将比去年同期大增27.6\%。图为去年7月5日,公众在新竹科学园区的台积电创新馆观看晶圆介绍视频。(法新社) 因人工智能(AI)半导体需求激增,全球最大晶片制造商台湾积体电路制造公司(简称台积电)今年第一季净利同比增长8.9\%,超出市场预期。受访专家指出,AI浪潮下,台湾以台积电为中心的供应链上中下游厂商都将获益,前景备受看好……}

\entryitemWithDescription{民进党中生代``大阿哥''郑文灿 据报任海基会董事长}{https://www.zaobao.com/news/china/story20240418-3466657}{受台湾政府委托处理两岸事务的海峡交流基金会(简称海基会)董事长一职,据报将由台湾行政院副院长郑文灿出任。(互联网) (台北综合讯)受台湾政府委托处理两岸事务的海峡交流基金会(简称海基会)董事长一职,据报将由台湾行政院副院长郑文灿出任。 《镜周刊》星期三(4月17日)报道,台湾候任总统赖清德阵营的内部已锁定,海基会董事长将推派郑文灿出任……}

\entryitemWithDescription{中国整体就业形势稳定 25至29岁非在校生青年失业率上升}{https://www.zaobao.com/news/china/story20240418-3466669}{中国官方数据显示,3月份中国城镇不含在校生的16岁至24岁劳动力失业率为15.3\%,与前一个月持平,但25岁至29岁劳动力失业率上升,显示青年人就业压力仍然存在。图为人们2月19日在中国河南省郑州市参加一场招聘会……}

\entryitemWithDescription{区域竞争加剧 香港港口吞吐量首次跌出全球十大之列}{https://www.zaobao.com/news/china/story20240418-3466051}{曾连续11年蝉联第一的香港首次跌出全球港口吞吐量排行前十。图为堆放在香港葵青货柜码头的集装箱。(彭博社) 香港货柜码头曾连续11年蝉联全球港口货柜吞吐量排行第一,但近年不断萎缩,去年更首次跌至全球第11位。受访学者指出,香港航运成本高,除非港府降低土地税,否则有关问题短期内难以改善……}

\entryitemWithDescription{布林肯据报将在下周访华四天}{https://www.zaobao.com/news/china/story20240418-3466044}{美国国务卿布林肯据报将在下个星期访华四天。图为布林肯4月18日在意大利卡普里岛出席七国集团外长会议第二天的会议。(路透社) (华盛顿/北京综合讯)继美国财政部长耶伦等美国高级官员后,美国国务卿布林肯据报将在下个星期访华四天。这是他时隔不到一年再度访华……}

\entryitemWithDescription{赖清德入选《时代》周刊百大最具影响力人物}{https://www.zaobao.com/news/china/story20240418-3465067}{美国《时代》周刊公布2024年100位最具影响力人物名单,台湾候任总统赖清德入选。图为赖清德今年4月10日在台北举行的新闻发布会上讲话。(法新社) (台北/纽约综合讯)美国《时代》周刊公布2024年100位最具影响力人物名单,台湾候任总统赖清德入选……}

\entryitemWithDescription{陈婧:中国经济乍暖还寒}{https://www.zaobao.com/news/china/story20240418-3461490}{中国经济今年第一季迎来``开门红'',打破外界的悲观预期。但随之而来的疑问,是这样的强劲表现能持续多久? 中国国家统计局星期二(4月16日)发布的数据显示,第一季中国国内生产总值(GDP)同比增长5.3\%。这不仅高于去年第四季的5.2%,更远超高盛、摩根士丹利等投行,以及路透社和彭博社调查分析师的预测,为实现5%左右的全年增长目标开了个好头……}

\entryitemWithDescription{中美防长通话后数小时 美军机飞越台海}{https://www.zaobao.com/news/china/story20240418-3461732}{美国海军第七舰队发布的声明称,一架P-8A``海神''巡逻机(也用于执行反潜任务)当地时间星期三(4月17日)在国际空域飞越台湾海峡。(互联网) 一架美国海军巡逻机星期三飞越敏感的台湾海峡;此前数小时,中美防长刚刚举行一年多来的首次实质性接触。受访学者认为,这是当前中美关系既有分歧竞争,又需要对话合作的生动写照……}

\entryitemWithDescription{中美再次就产能过剩交锋 学者指美国为对华实施更多制裁铺路}{https://www.zaobao.com/news/china/story20240417-3461419}{美国财政部长耶伦(中)星期二(4月16日)在华盛顿与主持中美经济工作组会议的中方代表中国财政部副部长廖岷(右),以及中国人民银行副行长宣昌能(左)会面。(法新社) 中美经济工作组会议再次就中国产能过剩问题进行交锋,美国重申担忧中国工业产能过剩,中国则关切美国对华经贸限制措施……}

\entryitemWithDescription{香港国安法落地近四年291人被捕 学者:社会政治化氛围过去几年大减}{https://www.zaobao.com/news/china/story20240417-3461703}{香港保安局星期三(4月17日)披露,截至3月8日,共有291人涉嫌从事危害国家安全的行为和活动被捕。图为摄于1月18日香港维多利亚港旁的海滨长廊。(法新社) 《香港国安法》实施近四年,共有291人因涉嫌违反相关法律被拘捕。受访学者认为,香港社会的政治化氛围在过去几年确实大大减少,这有赖于国安法的实施,但当局未来仍有必要加强国安教育宣传……}

\entryitemWithDescription{台湾自造潜舰专案召集人黄曙光请辞 蔡英文尚未批准}{https://www.zaobao.com/news/china/story20240417-3460534}{台湾``总统府国造潜舰专案小组''召集人黄曙光上将,以身心俱疲为由请辞,声称与他人和政治因素都无关。(台湾国防部海军司令部提供) 台湾第一艘自造潜舰仍在测试中,``国造潜舰''专案召集人黄曙光却向台湾总统蔡英文请辞,声称身心俱疲,辞职无关政治。但他未来还得厘清针对他的贪污指控,与他发起的诽谤官司,台湾海军能否建构不对称战力也成关注焦点……}

\entryitemWithDescription{中国田协计划出台指导意见 规范路跑赛事商业竞争}{https://www.zaobao.com/news/china/story20240417-3460520}{中国马拉松选手何杰(右二)与三名非洲选手,在上星期天(4月14日)的北京半程马拉松比赛中,几乎是并排跑向终点。(路透社) (北京综合讯)北京半程马拉松比赛疑似造假风波持续发酵之际,中国田径协会星期二召开会议,提出将尽快研究出台规范路跑赛事商业竞争的指导性意见,加强市场规范化等方面的指导和培训力度。 中国田协在官网发布消息称,今年春季以来,中国各地路跑赛事集中举办,群众参赛热情高涨……}