\entryitemWithDescription{钟南山建议中国民众 4月以后可不戴口罩}{https://www.zaobao.com/news/china/story20230403-1378917}{钟南山建议中国民众4月后在条件允许的情况下,可以不戴口罩。图为3月27日北京一处火车站等候登车的旅客,多数戴着口罩。(法新社) 按照中国目前的冠病疫情形势,中国工程院院士钟南山认为,佩戴口罩可成为非强制性的措施,建议今年4月以后在条件允许情况下让民众不戴口罩。 据《南方都市报》报道,钟南山星期五(3月31日)在广州一场有关冠病药物的研讨会上指出,过去三年,戴口罩是预防病毒的有效措施……}

\entryitemWithDescription{在湖南大学解说一中 马英九:两岸都是中华民国同属中国}{https://www.zaobao.com/news/china/story20230403-1378918}{马英九(中左)星期天带领台湾青年学生,与湖南大学校长段献忠(中右)等师生交流座谈,现场气氛友好热烈。(新华社) 马英九重申两岸同文同种,双方都坚持一个中国的立场,尽管目前分属两边,有各自的制度和政策,但希望双方真诚交流、减少不必要隔阂,并呼吁两岸青年生力军继续交流。 台湾前总统马英九星期天(4月2日)在与湖南大学师生座谈时说,``台湾和中国大陆都是中华民国'',但同属一个中国……}

\entryitemWithDescription{蔡英文访危地马拉 盼两地邦谊``3000年也不为过''}{https://www.zaobao.com/news/china/story20230403-1378920}{台湾总统蔡英文(左)与友邦危地马拉总统贾马特一同参访玛雅文明所在的蒂卡尔国家公园,承诺在中国大陆的外交攻势面前保持牢固关系。(路透社) 台湾总统蔡英文星期天(4月2日)与中美洲邦交国危地马拉总统贾马特一同面对媒体时说,希望两地邦谊永固,``3000年也不为过''……}

\entryitemWithDescription{湖南一原局长受审 昔日同事现场``围观''}{https://www.zaobao.com/news/china/story20230403-1378921}{湖南省益阳市国土资源局原局长贺国伟涉嫌受贿犯罪一案开审,座无虚席的庭审现场竟有33名原单位的前同事 ``围观''。 据``清风益阳''微信公众号消息,贺国伟受贿案近日在沅江市人民法院开审。贺国伟曾任益阳市国土资源局党组书记、局长,以及市自然资源和规划局党组书记 出席旁听庭审的除了来自益阳市纪委监委、市委办、市委政法委等多个党政机关,共144名党员领导干部,还包括贺国伟原单位的33名前同事……}

\entryitemWithDescription{台媒:侯友宜若获征召参选总统将向郭台铭请益}{https://www.zaobao.com/news/china/story20230403-1378922}{台湾媒体报道,一旦国民党新北市长侯友宜确定获征召,代表国民党参加2024年总统大选,他将向鸿海创办人郭台铭在内的党内外要角请益,争取支持。 国民党上月决定以征召方式提名总统大选人选。综合《联合报》、《中国时报》、TVBS等台媒报道,多名国民党立委连日来已向党中央喊话,应尽速征召侯友宜参选总统,立法院国民党团预估超过九成立委力挺侯友宜……}

\entryitemWithDescription{中国愿与亚细安加快推进 《南中国海行为准则》磋商}{https://www.zaobao.com/news/china/story20230403-1378923}{中国总理李强与马来西亚首相安华举行会谈时说,中国愿同马来西亚等亚细安国家加快推进《南中国海行为准则》磋商,共同维护南中国海和平稳定。 据新华社报道,李强星期六(4月1日)在北京人民大会堂与访华的安华举行会谈时,发表上述谈话。李强也说,``亚洲是我们的共同家园,合作共赢是唯一正确选择'',并称中国愿同马来西亚等亚细安国家积极推进中国亚细安自贸区3.0版谈判,共同实施好《区域全面经济伙伴关系协定》……}

\entryitemWithDescription{美军参联会主席米利:美中并未处在战争边缘}{https://www.zaobao.com/news/china/story20230403-1378924}{美国参谋长联席会议主席米利(Mark Milley)说,美国与中国并未处于``战争边缘'',而且北京要攻下台湾并非易事。 为开战言论降温据美国新闻网站``防务一号''(Defense One)星期五(3月31日)刊登的报道,米利在专访中指出,美国必须避免让美中开战的话题和言论变得``过热 ``。 继今年2月初的气球风波,中美关系持续紧张……}

\entryitemWithDescription{中国足球反腐还会牵出谁?}{https://www.zaobao.com/news/china/story20230403-1378925}{4月1日,中国体育总局副局长、足协党委书记杜兆才被官宣落马。这是120多天以来,中国足球领域落网的第九条``大鱼''。 去年11月26日,中国男足国家队原主教练李铁被查。随后,反腐风暴蔓延至整个足球圈……}

\entryitemWithDescription{湖北锣圈岩天坑 动植物生命之源}{https://www.zaobao.com/news/china/story20230403-1378927}{在中国湖北省西部山区的宣恩县矅天眼景区,春季研学的师生在深达290余米的锣圈岩天坑及周边溶洞中进行科普学习。这个喀斯特天坑四周植物茂密,坑内峭壁森然。在天坑底部,阳光从坑口射入,清水从天空及绝壁而降,汇成小溪,成为天坑``动植物王国''的生命之源。图为4月1日拍摄的锣圈岩天坑景色……}

\entryitemWithDescription{陈茂波︰新马学生对港抢人才抢企业政策感兴趣}{https://www.zaobao.com/news/china/story20230403-1378929}{香港财政司长陈茂波说,新加坡和马来西亚学生对香港推出的``高端人才通行证计划''(简称高材通计划)很感兴趣,并称这项政策让学生``更积极考虑到香港发展''。 陈茂波上周访问新加坡和马来西亚。他星期天(4月2日)在网络博客写道,为了与新马年轻人有更深入的互动,他分别到访新加坡国立大学和马来亚大学……}

\entryitemWithDescription{美媒:马云海外推动阿里巴巴分拆计划}{https://www.zaobao.com/news/china/story20230403-1378930}{美国媒体报道,阿里巴巴集团创始人马云在海外时,推动了阿里巴巴的分拆计划。 《华尔街日报》3月31日引述知情人士称,马云最近几个月与包括现任阿里巴巴集团董事局主席兼首席执行官张勇在内的阿里高管多次通话,敦促他们分拆公司。马云称,此举将使阿里巴巴在中国市场中更加灵活、更具竞争力。 知情人士称,尽管马云2019年就不再担任阿里巴巴董事局主席,但对公司仍有影响力,并积极参与公司战略决策……}

\entryitemWithDescription{中国特稿:送年幼孩子留学泰国 中国中产家庭盼什么?}{https://www.zaobao.com/news/china/story20230402-1377928}{泰国曼谷哈罗国际学校操场一角。(严宣融) 近年越来越多中国中产级家庭将年幼孩子送到泰国的国际学校。去年冠病疫情缓和后,留泰的中国学生人数大反弹,清迈的国际学校的中国学生人数更达40\%。是什么因素让中国家长选择让孩子到这个微笑之国留学?留泰之后,孩子会有怎样的未来? 几年前辞去中国国内媒体工作的成都妈妈刘媛(化名,46岁)去年6月和先生带着11岁的儿子杰森到泰国国际学校读书……}

\entryitemWithDescription{签署七谅解备忘录 新中加强绿色及数码经济等创新合作}{https://www.zaobao.com/news/china/story20230402-1378632}{贸工部长颜金勇(前排左)和中国商务部长王文涛(前排右)星期六(4月1日)在李显龙总理(后排左)和中国总理李强(后排右)的见证下,签署新中自贸协定升级后续谈判实质性完成的谅解备忘录。(邝启聪摄) 除了贸易投资,两国星期六也在进出口食品安全、国际商事争议处理、水和环境科研、文化艺术交流、湿地和红树林保护这五个领域签署谅解备忘录……}

\entryitemWithDescription{去年4月1日陷入史无前例停摆 封城两个月改变上海居民人生轨迹}{https://www.zaobao.com/news/china/story20230402-1378633}{上海陆家嘴金融区路上行人如织。此图摄于2023年2月28日。                         (路透社) 新加坡漫画家刘之华去年5月10日在上海封控期间画的漫画。         (作者提供) 封城初期,物流受阻、物资匮乏、防疫乱象百出,引发上海民怨沸反盈天……}

\entryitemWithDescription{偕四姐妹湘潭祭祖 马英九哽咽落泪}{https://www.zaobao.com/news/china/story20230402-1378634}{马英九(左三)4月1日与四名姐妹马以南(左起)、马莉君、马乃西、马冰如到湖南湘潭祭祖后,在祖父马立安墓碑前合影。                                                      (马英九办公室) (台北综合讯)正在中国大陆访问的台湾前总统马英九星期六(4月1日)携同四名姐妹,回到湖南湘潭祭拜祖父马立安……}

\entryitemWithDescription{半导体之争辟新战线 中国对美芯片商美光科技产品启动网安审查}{https://www.zaobao.com/news/china/story20230402-1378635}{中国宣布将对美国最大存储芯片制造商美光科技在华销售产品开展网络安全审查,显示中美半导体之争开启了新战线。 综合彭博社和路透社报道,中国网络安全审查办公室星期五(3月31日)公告,为保障关键信息基础设施供应链安全,防范产品问题隐患造成网安风险,维护国家安全,将对美光(Micron)在华销售的产品实施网安审查,但未具体说明涉及哪些产品……}

\entryitemWithDescription{与危地马拉总统会谈 蔡英文盼创双赢发展}{https://www.zaobao.com/news/china/story20230402-1378636}{台湾总统蔡英文与危地马拉总统贾马特会谈后说,台湾与危地马拉能在困难中互相伸出援手,是真正的朋友,期盼台、危在坚实合作基础上持续携手努力,创造双赢发展。 综合《联合报》《自由时报》等报道,结束过境美国纽约行程的蔡英文,当地时间星期五(3月31日)下午飞抵危地马拉首都危地马拉市,开展三天两夜的访问行程……}

\entryitemWithDescription{【视频】中国家庭扎堆泰国国际学校}{https://www.zaobao.com/news/china/story20230402-1378582}{泰国作为旅游胜地长期以来深受中国游客青睐。作为东南亚第二大经济体,泰国留学也成为中国人的热门选择之一。 不过,和几年前入读泰国大学的群体不同,近年越来越多的中国中产家庭将目光投向泰国的国际学校,并以幼儿园到小学的低龄儿童为主。 这一群体在经过疫情的短暂回流后,去年开始人数出现明显反弹,清迈不少国际学校的中国学生人数比例已达40\%。 是什么原因吸引中国家长送低龄孩子去泰国留学……}

\entryitemWithDescription{蔡英文过境访美领奖: 台湾是两岸关系中负责任方}{https://www.zaobao.com/news/china/story20230401-1378341}{台湾总统蔡英文(中)出席华府智库哈德逊研究所晚会,在该所所长瓦特斯(左)见证下,从董事会主席史登手中接受全球领导力奖。(路透社) 受访学者指出,美方处理蔡英文过境异常低调,包括一向友台的美国前国务卿蓬佩奥都没有如预期出席对话,显示美国内外都有迫切议题要协商,必须避免节外生枝。 台湾总统蔡英文过境美国纽约,获华府智库哈德逊研究所颁发全球领导力奖……}