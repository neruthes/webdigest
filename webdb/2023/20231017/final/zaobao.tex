\entryitemWithDescription{``蓝白合''陷僵局 民众党基层出现与郭台铭合作的呼声}{https://www.zaobao.com/news/china/story20231016-1443340}{台湾在野国民党总统参选人侯友宜的竞选办公室,星期一(10月16日)晚上致函民众党总统参选人柯文哲竞选办公室,呼吁三天内举行第二次会面,尽速整合在野力量。民众党尚无回应,唯该党基层已出现与独立参选的鸿海集团创办人郭台铭合作的呼声。 国民党和民众党上星期六(10月14日)针对2024年总统大选和立委选举首次举行``蓝白合''磋商会议,隔天双方就撕破脸……}

\entryitemWithDescription{中国央行净投放2890亿人民币MLF 规模为近三年最大}{https://www.zaobao.com/news/china/story20231016-1443334}{中国央行通过中期借贷便利(MLF)向市场注入2890亿元人民币(548亿新元)资金。图为中国央行位于北京的总部大楼。(路透社) 中国央行通过中期借贷便利(MLF)向市场注入2890亿元(人民币,下同,548亿新元)资金,净投放规模为近三年来最大。分析预计,年底前央行还会持续大幅加量操作,为稳经济政策``组合拳''保驾护航……}

\entryitemWithDescription{美国在台协会主席上任半年多三度访台 将见蓝绿白总统参选人}{https://www.zaobao.com/news/china/story20231016-1443330}{美国在台协会(AIT)主席罗森伯格今年第三度访台,她星期一(10月16日)与台湾总统蔡英文见面时强调,美国当务之急是要提高全球对台海和平的重视。 44岁的罗森伯格(Laura Rosenberger)星期天(10月15日)抵台进行五天访问,她在美国国务院亚太局资深顾问罗峻平(Michael Pignatello)陪同下,准备与台湾政府高层、部会官员和各界人士讨论台美关系、区域安全、经贸投资等议题……}

\entryitemWithDescription{极兔快递拟在港上市 有望成香港今年第二大IPO}{https://www.zaobao.com/news/china/story20231016-1443292}{(香港综合讯)中国科技巨企腾讯支持的物流公司极兔快递已在香港启动首次公开募股(IPO)程序,计划募集39.2亿港元(6.86亿新元),有望成为香港今年迄今第二大IPO。 据彭博社报道,极兔快递环球有限公司(J\&T Global Express)星期一(10月16日)开启招股,计划10月27日在港交所主板挂牌上市。 在此之前,顺丰控股和阿里巴巴旗下的菜鸟也于过去两个月,向港交所提交了上市申请书……}

\entryitemWithDescription{中行原董事长刘连舸被捕 分析:金融反腐频或为平息民众对经济下行不满}{https://www.zaobao.com/news/china/story20231016-1443284}{中国官方星期一(10月16日)证实,中国银行原党委书记、董事长刘连舸(图)因涉嫌受贿、违法发放贷款,已被逮捕。(中新社档案照) 中国官方星期一(10月16日)证实,中国银行原党委书记、董事长刘连舸因涉嫌受贿、违法发放贷款,已被逮捕。中国今年至少已有108名金融官员和高阶主管,受到调查或处罚。学者分析,官方或许尝试透过对金融领域的集权反腐,平息民众对经济乏力的不满……}

\entryitemWithDescription{香港据报将设立抗战纪念馆 强化爱国主义教育}{https://www.zaobao.com/news/china/story20231016-1443273}{(香港讯)香港特区政府据报将通过新施政报告,设立宣扬中华文化办公室和抗战纪念馆,坚定文化自信,强化爱国主义教育。 据《星岛日报》星期一(10月16日)报道,香港特首李家超将在下星期三(10月25日)发表施政报告。消息称,新施政报告将提出进一步完善治理体系,在不同领域成立小组及办公室;部分部门则因应性质、权责而改组合并……}

\entryitemWithDescription{台交通部长:陆客今年都不会来}{https://www.zaobao.com/news/china/story20231016-1443271}{台湾台北市区的夜景。图为知名的台北101大厦。(路透社) (台北综合讯)在中国大陆和台湾无法就恢复两岸观光交流达成一致的背景下,台湾交通部长王国材坦言,大陆旅客今年都不会到台湾来。 综合《中国时报》、风传媒等报道,王国材星期一(10月16日)在立法院交通委员会回答国民党立委傅崐萁提问大陆旅游团客何时会来台时指出,台湾已对大陆释出善意,起初预计下半年双方就会同步解除旅游禁令……}

\entryitemWithDescription{新中法律专家齐聚狮城 探讨基础设施项目、知识产权和海事争议解决方法}{https://www.zaobao.com/news/china/story20231016-1441581}{我国麦士威国际争议解决中心(Maxwell Chambers)是全球第一个提供全面解决方案的争议解决中心。(韩宝镇摄) 基础设施项目争议,除了通过诉讼、仲裁和调解来解决,还能够通过什么方法在争议未发生前,就进行预防?知识产权纠纷日益增加,新中两国司法界对此各采取了哪些举措?联合国公约对海事业者带来什么益处……}

\entryitemWithDescription{于泽远:中国将现新一波人事变动}{https://www.zaobao.com/news/china/story20231016-1442981}{中国全国人大常委会会议将于10月20日至24日召开。(法新社) 中国全国人大常委会会议将于10月20日至24日召开,外界预计会议将任命刘振立为国防部长,蓝佛安为财政部长,阴和俊为科技部长。 现任国务委员兼国防部长李尚福已有一个半月未在公开场合露面,尤其没有出席9月28日中国高层举行的国庆招待会,从侧面证实他可能涉及腐败案件,已不能正常履职……}

\entryitemWithDescription{学者:王毅最新表态显示 北京对以哈冲突的外交姿态已出现根本变化}{https://www.zaobao.com/news/china/story20231015-1443045}{中国外交部长王毅14日首次公开点名批评以色列的行为已超越自卫范围,应停止对加沙民众的集体惩罚。(法新社) 以哈冲突持续一周之际,中国外交部长王毅星期六(10月14日)首次公开点名批评以色列的行为已超越自卫范围,应停止对加沙民众的集体惩罚。受访学者研判,王毅的最新表态显示北京对以哈冲突的外交姿态已出现根本变化,对以色列计划摧毁哈马斯可能引发新一轮中东战乱表现出前所未有的担忧……}

\entryitemWithDescription{``蓝白合''未达共识 两党言语展露火药味}{https://www.zaobao.com/news/china/story20231015-1443035}{(台北综合讯)台湾在野的国民党和民众党首次磋商会隔天,双方仍未达成共识。双方对于最强候选人的产生方式分歧依然突出,两党言语间更展露火药味。 综合《联合报》《自由时报》等报道,在蓝白两党星期六(10月14日)的磋商会议中,民众党主张``比民调'',国民党则主张采用``开放式民主初选''的方式……}

\entryitemWithDescription{中俄互免签证重启 俄人到黑河吃早餐}{https://www.zaobao.com/news/china/story20231015-1443023}{图为在星期日(10月15日)黑龙江省黑河市的早市,跨境前来消费的俄罗斯游客与中国摊贩进行买卖。(中新社) (黑河综合讯)中俄团体免签旅游上月在中国黑龙江省黑河市率先恢复,吸引大批俄罗斯居民跨境到黑河街头消费。 综合新华社、《星岛日报》报道,9月中下旬,黑河市率先恢复因疫情中断的中俄互免签证团体旅游业务,随后黑河口岸出入境人员数量迅速激增……}

\entryitemWithDescription{第三届中新国际商事争议解决论坛10月20日登场}{https://www.zaobao.com/news/china/story20231015-1443018}{因为疫情停办三年的中新国际商事争议解决论坛去年4月7日恢复以线上线下方式在新加坡、北京和厦门三地举行。图为在新加坡举行的其中一场专题讨论,左起:安睿雅士律师事务所顾问许廷芳律师、艾伦格禧律师事务所合伙人王文辉、新加坡王律师事务所合伙人曾福庆。(主办当局提供) 探讨加强新中商事纠纷解决合作的第三届中国---新加坡国际商事争议解决论坛,将于星期五(10月20日)在新加坡登场……}

\entryitemWithDescription{普京:俄罗斯从一带一路看到合作的愿望 而非征服别人的企图}{https://www.zaobao.com/news/china/story20231015-1443016}{(莫斯科综合讯)俄罗斯总统普京星期二(17日)将访华参加``一带一路''论坛,他在接受中国官媒采访时说,俄罗斯从一带一路看到的是合作的愿望,而不是中国征服别人的企图。 据中国央视新闻报道,普京在央视星期天(15日)播出的《高端访谈》专访节目中,作出上述表态。 普京说,一带一路倡议与俄罗斯发展欧亚经济联盟的设想完全契合,参与国将实现互惠互利,俄罗斯愿和中国共同推进该倡议的实施……}

\entryitemWithDescription{微店要求平台所有商家西藏英译``Tibet''改``Xizang''}{https://www.zaobao.com/news/china/story20231015-1442986}{(香港综合讯)继中国外交部将西藏译为``Xizang''后,中国电商平台微店也要求平台商家将产品英文描述从英文翻译``Tibet''改为汉语拼音``Xizang''。 香港《南华早报》星期天(10月15日)报道,微店星期三(10月11日)发布通知,要求所有商家更改西藏翻译名行,若未更改,将下架对应商品。 微店是中国中小卖家常用的手机电商平台……}

\entryitemWithDescription{美国本周或宣布AI芯片对华出口新管制措施 弥补现有出口管制漏洞}{https://www.zaobao.com/news/china/story20231015-1442983}{(华盛顿综合讯)据美国媒体报道,拜登政府即将宣布对销售到中国的人工智能(AI)芯片和设备的新措施,以弥补现有的出口管制漏洞。 美国新闻网站Axios星期六(10月14日)引述美国官员报道,美国政府可能最快在这个星期初就宣布新措施。与那些已被禁止向中国出口的芯片相比,性能相对较低的芯片及其生产设备也将受到限制。 报道认为,此举将有助于白宫进一步阻止中国在AI领域取得军事优势……}

\entryitemWithDescription{幕后老板中国人 很多人都是自愿 缅甸诈骗园区 为何难以整治?}{https://www.zaobao.com/news/china/story20231015-1442434}{中国和缅甸执法部门在9月初针对缅北诈骗园区展开了联合执法行动,图为9月16日由缅方移交给云南普洱公安机关的109名缅北电信网络诈骗嫌犯。(中新社) 有关中国民众被困在缅北诈骗园区的消息近年不时见诸报端,使该灰色产业备受关注。缅甸这个长期处于内战的神秘国度是如何成为电信诈骗温床?中缅两国的联合执法能否整治诈骗活动?邻国旅游业是如何受牵连……}

\entryitemWithDescription{博雷利盼中国同等对待欧盟 学者:欧盟重申对华政策独立自主}{https://www.zaobao.com/news/china/story20231014-1442786}{中国国家副主席韩正(右)星期五(10月13日)在北京会见欧盟外交与安全政策高级代表博雷利。(新华社) 欧盟外交与安全政策高级代表博雷利表明,欧盟高度重视中国,期望中国同等对待欧盟,而不是通过其与第三国关系的视角来看待它们。 受访学者认为,欧盟希望强调在安全问题上虽与美国关系紧密,但在对华政策上独立自主……}

\entryitemWithDescription{塔利班据报将派员参加一带一路高峰论坛}{https://www.zaobao.com/news/china/story20231014-1442776}{(喀布尔路透电)阿富汗塔利班政府代理商业和工商部长,将前往北京参加一带一路国际合作高峰论坛,显示中国和塔利班政府的官方联系日渐密切。 阿富汗临时政府塔利班的代理商务和工业部长阿齐兹将参与论坛,并将邀请大型投资者前往阿富汗投资。 他也会和北京继续商讨修建穿越瓦罕走廊的公路计划。阿富汗北部的瓦罕走廊是阿富汗与中国接壤的狭长山区地带……}

\entryitemWithDescription{``蓝白合''首次磋商会:候选人产生方式未达共识}{https://www.zaobao.com/news/china/story20231014-1442771}{国民党和民众党星期六(10月14日)举行在野力量整合会议,秘书长黄健庭(左起)和侯友宜竞选办执行长金溥聪代表国民党出席,民众党代表是柯文哲竞选办总干事黄珊珊和竞选办主任周榆修。双方会后展示共同签署的协议。(自由时报) (台北综合讯)台湾在野的国民党和民众党为磋商2024年总统及立委选举合作事宜而举办的首场会议结束,两党同意公开辩论,但对如何产生最强候选人的方式各有主张,未能达成共识……}

\entryitemWithDescription{中国足球名嘴段暄失联 据报涉向奥运备战办原主任行贿}{https://www.zaobao.com/news/china/story20231014-1442722}{中国足球名嘴段暄据报涉嫌向奥运备战办原主任行贿。(互联网) (北京综合讯)中国知名体育媒体人段暄从公众视野消失已有数月,知情人士透露,他涉嫌向今年4月落马的中国赛艇协会和中国皮划艇协会原主席、国家体育总局奥运备战办原主任刘爱杰行贿。 财新网星期六(10月14日)引述多名知情人士报道,段暄被指涉嫌向刘爱杰行贿,以谋取承揽备战北京冬奥会跨项跨界选材等工作的相关利益……}