\entryitemWithDescription{新闻人间:``揭弊天王''邱毅再现江湖}{https://www.zaobao.com/news/china/story20230812-1422850}{台湾政坛``揭弊天王''邱毅(联合早报) 台湾政坛``揭弊天王''邱毅教授出手了,从天天点评时政、批判蓝绿白政治人物,到重炮揭露有关执政党领导人的弊案,网络声量忽然飙升。 67岁的邱毅近年来深居简出,几乎在台湾所有媒体平台绝迹,重心转到经营网络直播,谈两岸统一、文化和经贸交流等课题。 从去年``九合一''选举以来,他开始热衷点评台湾时政,今年3月中以来更几乎天天在脸书发一篇``社论''……}

\entryitemWithDescription{黄小芳:防疫后遗症}{https://www.zaobao.com/news/china/story20230812-1422905}{黑龙江8月以来成为中国舆论焦点,除了暴雨洪灾,当地建设方舱医院的消息也成为持续延烧的热点话题。 8月2日,黑龙江佳木斯要建方舱医院的消息登上微博热搜。网上流传的公示板信息显示,佳木斯前进区计划建造一座方舱医院,用地面积6388平方米。根据公开资料,单是在一期,该建设项目涉及资金规模就超过4000万元人民币(748万新元)……}

\entryitemWithDescription{中国医疗反腐掀起``举报潮'' 今年已近170医院高层被查}{https://www.zaobao.com/news/china/story20230812-1422912}{中国近期医疗反腐风暴持续,重庆、四川等多地星期五(8月11日)跟进公布集中整治举报方式,推升举报医疗行业人员的浪潮。据统计,中国各地今年已有近170名医院院长等高层被查,人数超过去年全年的两倍。 受访学者研判,宏观经济不振之际,官方下半年势必加大力度整治出现在医疗、教育等攸关基层民生领域的腐败,以更好保障民众利益,避免激起更多民愤……}

\entryitemWithDescription{赖清德出访巴拉圭过境美国 学者:中国大陆初步``有节制'' 军演}{https://www.zaobao.com/news/china/story20230811-1422910}{台湾副总统、执政的民进党总统参选人赖清德星期六(8月12日)出访友邦巴拉圭,并过境美国。赖清德出访前夕,中国大陆宣布自星期六起在东海举行三天军事演习。 受访学者专家认为,大陆目前公布的军演相当``有节制'',但要观察赖清德过境美国情况,不排除大陆再升高军演动作。 赖清德星期六启程前往巴拉圭,参加巴拉圭当选总统佩纳就职典礼,其间将过境美国纽约和旧金山,8月18日返台,共计七天六夜……}

\entryitemWithDescription{中国一军工人员公派国外被吸收 向美国提供情报}{https://www.zaobao.com/news/china/story20230811-1422876}{中国国家安全部星期五(8月11日)通报破获一起间谍案,一名中国军工集团工作人员公派意大利期间被吸收,为美国中央情报局(CIA)提供大量核心情报,收取间谍经费。~ 据中国国家安全部微信公众号发布,52岁的曾姓嫌疑人属重要涉密人员,被单位公派至意大利留学进修期间,美国驻意大利使馆官员塞斯主动与其结识,两人通过聚餐、郊游、观赏歌剧等活动逐步建立密切关系……}

\entryitemWithDescription{传中国暂停在伦敦建新使馆计划 中英外交紧张恐加深}{https://www.zaobao.com/news/china/story20230811-1422871}{中国规划在伦敦桥附近一带兴建新大使馆的外部景观。(路透社) 中英两国都希望修复受损的双边关系之际,据报中国将暂停在伦敦建新使馆的计划,这可能会加剧两国外交紧张。 中国2018年以2.5亿英镑(约4.4亿新元)的价格,在伦敦塔附近的皇家造币厂旧址购买土地,计划将现在位于波特兰大街的大使馆迁至该地。但这一计划去年底被当地市议会以存在安全和隐私风险为由,拒绝授予许可证……}

\entryitemWithDescription{在华被拘三年 澳洲记者发公开信描述关押经历}{https://www.zaobao.com/news/china/story20230811-1422841}{因涉嫌国家安全罪而被中国拘留的澳大利亚籍记者成蕾,首次发表公开信,描述自己在中国被关押的经历。(路透社) 澳大利亚籍记者成蕾被中国以涉嫌国家安全罪拘留三年后,首次发表公开信,描述自己在中国被关押的经历。 综合路透社、法新社、彭博社星期五(8月11日)报道,成蕾发表了一封被称为``给2500万人的情书'',是她在北京被拘留三年以来首次发表公开信……}

\entryitemWithDescription{中国大陆出境游名单不包括台湾 民进党政府指责任在大陆官方}{https://www.zaobao.com/news/china/story20230811-1422839}{中国大陆进一步开放出境团队旅游,但不包括台湾。有台湾旅行社业者质疑民进党政府未推动两岸主管部门接触洽商。台湾陆委会则强调,台方一再提议两岸主管部门直接沟通安排,是大陆官方一再拒绝回应。 据中国文化和旅游部官网消息,文旅部办公厅星期四(8月10日)公告,进一步恢复大陆旅行社经营中国公民出境团队旅游业务。在所公布开放的第三批国家和地区名单中,不包括台湾……}

\entryitemWithDescription{中国科企抢跑人形机器人赛道}{https://www.zaobao.com/news/china/story20230810-1422546}{以扫地机器人闻名的追觅科技,今年3月推出第二代仿生机器狗和通用人形机器人。(陈婧摄) 步入苏州追觅科技的展厅,四周陈列着吸尘器和扫地机等招牌产品,摆在正中的却是一个和普通男子体型相仿的机器人,以及一只摇头摆尾地和访客互动的机器狗。 这家以扫地机器人闻名的独角兽企业,今年3月推出第二代仿生机器狗和通用人形机器人……}

\entryitemWithDescription{港警国安处拘捕10名涉违国安法``612基金''人士}{https://www.zaobao.com/news/china/story20230810-1422534}{香港社运人士叶宝琳(右)8月10日被香港警方带到她工作的天主教书局塔冷通心灵书舍蒐证后,再被押走。(路透社) 香港警务处国家安全处星期四(8月10日)拘捕10人,指他们涉嫌捐助流亡海外的港人,违反《香港国安法》的``串谋勾结外国或者境外势力危害国家安全''罪及煽动暴动罪。 被捕的四男六女,年龄介于26至43岁……}

\entryitemWithDescription{乌克兰驻华大使称中俄关系是``权宜联姻''}{https://www.zaobao.com/news/china/story20230810-1422529}{在中国罕见公开批评俄罗斯、中国参与乌克兰和平峰会,中俄乌三国关系出现微妙变化之际,乌克兰驻华大使形容,中俄关系是一场为了资源的``权宜婚姻''。 据彭博社报道,乌克兰驻华大使利亚比肯(Pavlo Riabikin)星期二(8月8日)接受乌克兰新闻通讯社(RBC)采访时说,北京根据自身利益制定外交政策,对中国来说,俄罗斯既是政治伙伴,也是资源来源……}

\entryitemWithDescription{美国将限制对华敏感技术投资 中国称将保留采取措施权利}{https://www.zaobao.com/news/china/story20230810-1422527}{美国总统拜登8月9日签署行政令,限制美国公司和个人投资中国的敏感技术。图摄于8月9日。(路透社) 美国总统拜登星期三(8月9日)签署行政命令,限制美国公司和个人投资中国的敏感技术,包括半导体、量子运算与人工智能;中国外交部和商务部隔天提出严正交涉和强烈不满,誓言保留采取措施的权利。 受访学者分析,行政令将带来外溢效应,美国盟友有可能会跟进采取类似措施,其他国家的企业对华投资也将更谨慎……}

\entryitemWithDescription{中国警方侦破79起利用AI换脸欺诈案件}{https://www.zaobao.com/news/china/story20230810-1422520}{中国公安部通报,在针对``AI(人工智能)换脸''欺诈问题的专项行动中,全国共侦破相关案件79起,抓获犯罪嫌疑人515名。 据中新经纬报道,中国公安部网络安全保卫局副局长李彤星期四(8月10日)在发布会上说,犯罪分子用于实施``AI换脸''的物料主要为照片,特别是身份证照片,同时结合人员姓名、身份证号来突破人脸识别验证系统……}

\entryitemWithDescription{中国就驻伦敦使馆新址之争向英国抗议}{https://www.zaobao.com/news/china/story20230810-1422519}{中国驻伦敦大使馆新址计划去年底被当地议会否决后,至今仍没有进展。图为使馆新址伦敦塔附近的景观。(路透社) 中国驻伦敦大使馆新址计划去年底被当地议会否决后,至今没有进展。中国因此向英国政府提出抗议,指其未能履行外交义务。 中国在2018年斥资2.55亿英镑(约4.4亿新元)购置了伦敦塔附近的皇家铸币厂旧址,以便将现在位于波特兰大街的使馆迁至该地……}

\entryitemWithDescription{《愿荣光》禁令上诉案 香港律政司:法庭应遵行政机关判断}{https://www.zaobao.com/news/china/story20230810-1422491}{香港律政司不服高等法院拒绝就传播反修例歌曲《愿荣光归香港》批出禁制令而提出上诉,列出的上诉理由包括,法庭应``遵从于行政机关的判断''。 此前,香港律政司在6月5日入禀高院申请禁制令,禁止公众在网上或任何平台传播2019年反修例运动期间出现的歌曲《愿荣光归香港》,但香港高院上月拒批临时禁制令。 香港律政司星期一(8月7日)提出上诉,并在星期三公开草拟上诉文件……}

\entryitemWithDescription{中国恢复对日韩美澳等国出境团队游}{https://www.zaobao.com/news/china/story20230810-1422481}{中国宣布恢复对日本、韩国、澳洲等国的出境团队游业务。图为人们6月14日在日本涩谷人行道上行走。(法新社) 中国文化和旅游部宣布恢复对日本、韩国、澳洲、英国、德国和美国等多国的出境团队游业务。消息发布后,中国多家旅行社与在线旅行服务平台陆续上线相关产品,游客咨询量迅速攀升……}

\entryitemWithDescription{中国大陆批麻生涉台言论是想把台民众推向火坑}{https://www.zaobao.com/news/china/story20230810-1422470}{针对日本前首相麻生太郎访问台湾时提出,为避免台海发生战争,要具备吓阻实力、执行吓阻决心,中国大陆外交部批评称,日本政客访台必言战,摆出唯恐台海不乱的架势,是想把台湾民众推向火坑。 中国大陆外交部发言人星期三(8月9日)在官网就日本自民党副总裁麻生太郎访台时的言论,以回答记者问形式作出上述回应……}

\entryitemWithDescription{中国首条直通中越边境高速铁路开始铺轨}{https://www.zaobao.com/news/china/story20230809-1422213}{中国铁路南宁局集团称,从广西防城港至东兴的防东铁路星期二已开始铺轨。图为7月20日,防东铁路西湾跨海双线特大桥在进行桥梁架设。(中新社) 据中国铁路南宁局集团消息透露,从广西防城港至东兴的铁路星期二开始铺轨,意味着中国首条直通中越边境的高速铁路------防东铁路开始铺轨……}

\entryitemWithDescription{侯友宜主张将核电列入能源选项 民进党讥``侯友宜参选人打脸侯市长''}{https://www.zaobao.com/news/china/story20230809-1422209}{台湾在野国民党总统参选人、新北市长侯友宜星期三(8月9日)批判执政的民进党``2025非核家园''目标不可能达成,主张将核电正式列入能源选项。 侯友宜承诺当选后,第一任期内将完成核一(第一核能发电厂)、核二、核三检查检修工作,安全延役,并邀请顶尖核能安全学者专家成立核四总体安全审查委员会,在安全无虞下,推动核四安全重启……}

\entryitemWithDescription{中国大陆再派大批军机舰台海警巡}{https://www.zaobao.com/news/china/story20230809-1422207}{台湾国防部星期三(8月9日)通报,中国大陆派出10架次军机越过台海中线,并配合五艘大陆军舰执行联合战备警巡,是本周内第二次大规模军机舰在台海周边活动。 通报说,星期三自早上9时起即陆续侦获大陆各型军机共计25架次出海活动,其中10架次逾越台海中线及延伸线进入台湾西南空域。 台国防部强调,台军运用联合情监侦手段绵密掌握,并检派任务军机军舰及岸置导弹系统,严密监控应处……}