\entryitemWithDescription{中国早点:陆港器官移植合作引发争议}{https://www.zaobao.com/news/china/story20230516-1395079}{近些年来,陆港融合渐渐成了香港的政治主旋律,特区政府每隔一段日子就会公布融入国家发展,尤其是大湾区的新政策。最近一个例子,便是港府正研究与中国大陆设立恒常的器官移植互助机制。 香港医务卫生局局长卢宠茂上周出席一个公开活动时透露,医院管理局这个星期将率领医护团队到大陆访问,协助拟定两地器官互助机制安排……}

\entryitemWithDescription{在华咨询公司被查 分析:公司测试中国法律极限和政府底线}{https://www.zaobao.com/news/china/story20230515-1395052}{中国对在华经营的外国尽职调查和咨询公司公开执法,促使一些公司对业务进行审查。有分析指出,中国在解除严格疫情管控后,一些公司测试了中国的法律极限和政府底线,以满足咨询激增的需求。 据路透社报道,这些咨询公司的主要业务是帮助投资客户,如全球对冲基金、私募股权公司,与行业专家和调查员搭线,从中获取有价值的信息,以便在中国市场做出投资决定……}

\entryitemWithDescription{郭侯争选总统本周揭晓 国民党能否避免分裂?}{https://www.zaobao.com/news/china/story20230515-1395046}{郭台铭5月14日在脸书公开与高雄前市长韩国瑜``四手交迭''的合照,并透露自己当面向韩国瑜道歉。(取自郭台铭脸书) 国民党最快在星期三(5月17日)征召台湾总统候选人,呈现一方全力冲刺、另一方鸭子划水的两强竞逐局面。最大悬念是如果落选方不服输,作为最大在野党的国民党能否避免闹分裂,导致未战先败……}

\entryitemWithDescription{台湾交通部有信心今年能达600万人次旅客目标}{https://www.zaobao.com/news/china/story20230515-1395045}{冠病疫情后台湾积极推动后观光旅游,因中国大陆旅客仍无法赴台,面对朝野立委质疑,交通部次长陈彦伯对达成旅客600万人次的目标有信心,但对中国大陆旅客仍``维持现在的政策''。 台湾立法院交通委员会星期一(5月15日)邀请交通部,就``疫后国际航线规划及全台整体观光行销规划及愿景''进行专题报告,陈彦伯作上述表示……}

\entryitemWithDescription{高温天气威胁中国电力供应和经济复苏力度}{https://www.zaobao.com/news/china/story20230515-1395037}{中国多个地区自3月以来遭受热浪袭击。其中,北京在星期一(5月15日)发出高温蓝色预警。一名女子在北京街道上使用购物袋遮挡烈日照射。 (中新社) 中国多个城市近期发布高温预警,破纪录的高温天气将威胁电力供应、农作物产量,以及脆弱的中国经济。 据路透社报道,中国多个地区自3月以来遭受热浪袭击,气候一向温和的云南省最近的气温超越40摄氏度。酷热的天气促使数百万家庭开空调降温,对当地电网造成巨大负担……}

\entryitemWithDescription{时隔163年 俄罗斯对华开放符拉迪沃斯托克港}{https://www.zaobao.com/news/china/story20230515-1395026}{russia(联合早报) 中俄关系在俄乌战争后越加紧密。中国海关总署宣布,中国将增加俄罗斯符拉迪沃斯托克港(中国称海参崴)为内贸货物跨境运输中转口岸;这意味着俄罗斯在163年后重新对华开放重要的枢纽港口,中国东北内陆地区将打开出海通道。 受访学者分析,这更进一步证实,随着俄罗斯在俄乌战争后被孤立,中俄关系的天平已经向中国倾斜……}

\entryitemWithDescription{中国特别代表李辉赴乌克兰商讨政治解决方案}{https://www.zaobao.com/news/china/story20230515-1395015}{中国政府欧亚事务特别代表李辉星期一(5月15日)启程访问乌克兰,就乌克兰危机商讨''政治解决方案''。他是去年2月俄乌战争爆发以来,到访乌克兰的最高级别中国官员。 中国外交部发言人汪文斌上星期五(5月12日)宣布,李辉此行将到访乌克兰、波兰、法国、德国和俄罗斯,就政治解决乌克兰危机同各方进行沟通。中国外交部并没有透露详细行程……}

\entryitemWithDescription{中国将在20个地市试点营造生育友好社会环境}{https://www.zaobao.com/news/china/story20230515-1394992}{中国今年将在河北邯郸市、广东广州市等20个地市试点,开展新时代婚育文化建设,营造生育友好社会环境。 据新华社报道,中国计划生育协会上星期四(5月11日)宣布这一消息。中国计生协常务副会长王培安当日在广州举行的新时代婚育文化建设主题宣传活动上说,人口问题是``国之大者'',在降低生育、养育、教育成本的同时,应大力推进新时代婚育文化建设……}

\entryitemWithDescription{香港八所公立大学的大陆学者占比首超本地学者}{https://www.zaobao.com/news/china/story20230515-1394991}{香港大学教育资助委员会数据显示,在本学年度,香港八所公立大学聘用的学者中,来自中国大陆学者的占比首次超越香港本地学者。 据《南华早报》星期一(5月15日)报道,香港八所公立大学本学年度聘用5120名学者,出生于中国大陆的有1815人,较五年前的1224人增加,占比达35\%;香港本地学者从五年前的1924人下降到1670人,比例从五分之二缩减至三分之一;海外学者比例则从34\%下降到32\%……}

\entryitemWithDescription{香港国安法生效后 图书馆逾两年涉政治资料少四成}{https://www.zaobao.com/news/china/story20230515-1394987}{香港媒体统计,当地公共图书馆涉政治题材和人物的录影资料及书籍,在过去两年多内已有四成下架。 《明报》星期一(5月15日)报道,该报在2020年底起就香港公共图书馆馆藏资料,整合468项涉及政治题材及人物的录影资料和书籍,发现目前至少有195项资料已下架,即有四成在两年多内下架。其中,有96项是在过去一年内被移除。 《香港国安法》在2020年6月30日生效……}

\entryitemWithDescription{美籍港人梁成运间谍罪成 在中国被判无期}{https://www.zaobao.com/news/china/story20230515-1394983}{中国江苏省苏州市中级人民法院星期一(5月15日)宣判,持有美国护照的78岁香港永久居民梁成运间谍罪名成立,被判终身监禁。 据苏州市中级人民法院微信公众号发布,梁成运因涉嫌从事间谍活动,苏州市国家安全局于2021年4月15日对其采取强制措施。法院经审理后一审认定梁成运犯间谍罪,判处无期徒刑,剥夺政治权利终身,并没收个人财产人民币50万元(9.6万新元)。声明并未说明梁成运从事哪些间谍活动……}

\entryitemWithDescription{于泽远:《漫长的季节》重现当年下岗潮}{https://www.zaobao.com/news/china/story20230515-1394778}{下岗人员的名词虽已较少使用,但中国就业的压力并未减轻多少,那个时代其实并未远去……}

\entryitemWithDescription{我国副总理黄循财与上海市长龚正会面}{https://www.zaobao.com/news/china/story20230514-1394648}{到访中国的我国副总理兼财政部长黄循财(左),星期天傍晚与上海市长龚正会面。(通讯及新闻部提供) 正在中国访问的我国副总理兼财政部长黄循财,星期天(5月14日)在上海与当地官员、企业家和旅沪新加坡人会面交流,探讨新沪两地深化合作的机遇。 黄循财此次访华的首个公开活动,是星期天上午参访中国金融科技巨头蚂蚁集团的上海办公楼,并与蚂蚁董事长兼首席执行官井贤栋等集团高层会谈……}

\entryitemWithDescription{山西男子因情感纠纷杀人后开车撞人 致七死11伤}{https://www.zaobao.com/news/china/story20230514-1394645}{中国山西省一名男子因情感纠纷,先杀人、后开车在路上撞人,导致七死11伤。 综合每经网、潮新闻、新京报报道,山西吕梁兴县人民政府新闻办公室星期天(5月14日)通报,一名27岁郭姓男子因感情纠纷,星期六(13日)下午约2时在兴县奥家湾乡沟门前村将一名21岁郭姓女子致伤,并将她的婆婆、丈夫和儿子杀害。 随后,郭姓男子驾驶一辆轿车逃窜,过程中将一名出警人员和13名行人撞倒。该案件共致七人死亡,11人受伤……}

\entryitemWithDescription{中国4月贷款和社融增量环比断崖式下降}{https://www.zaobao.com/news/china/story20230514-1394643}{中国央行公布的金融统计数据显示,中国4月份贷款和社会融资规模增量环比出现断崖式下降,下降幅度远超预期。 据《华尔街日报》报道,中国央行星期四(5月11日)公布的数据显示,4月份人民币贷款增加7188亿元(下同,约1383亿新元),社会融资规模增量为1万2200亿元,月底社会融资规模存量同比增长10\%,同时广义货币供应量(M2)同比增长12.4\%……}

\entryitemWithDescription{金庸创作遭大陆作家侵权终审获胜 获赔188万元人民币}{https://www.zaobao.com/news/china/story20230514-1394642}{金庸2015年发现在中国大陆发行的小说《此间的少年》,所描写人物的名称均来自他的四部作品,且人物间的相互关系、性格特征及故事情节与其作品实质性相似,认为作者江南侵权,将他告上法庭。(互联网) 香港已故武侠小说作家金庸,生前起诉中国大陆作家江南抄袭其小说人物的``同人作品案''二审判决结果出炉。江南权侵和不正当竞争罪名成立,被要求赔偿188万元(人民币,下同,约36万新元)……}

\entryitemWithDescription{广州深圳2022年常住人口出现负增长}{https://www.zaobao.com/news/china/story20230514-1394636}{受冠病疫情、企业用工需求下降、外来人口回流等因素影响,广东省两大城市广州和深圳在2022年均出现人口负增长。图为4月7日广州一家咖啡厅。(法新社) 受冠病疫情、企业用工需求下降、外来人口回流等因素影响,广东省两大城市广州和深圳在2022年均出现人口负增长……}

\entryitemWithDescription{诚品疑似个资外泄 台民众购书后称接统战电话}{https://www.zaobao.com/news/china/story20230514-1394633}{台湾诚品书店疑似出现顾客个人资料外泄情况,有台湾民众称在购书后接到``统战''电话。(互联网) 台湾诚品书店疑似出现顾客个人资料外泄情况,有台湾民众称在购书后接到``统战''电话。 综合《经济日报》《自由时报》等台媒报道,``台湾伫遮计划''副执行秘书杨欣慈2月购入《阿共打来怎么办》一书后,星期六(5月13日)接获自称来自诚品书局的回访市调电话,关切为何要买内容``不恰当''的书……}

\entryitemWithDescription{香港图书馆疑似下架政治学者著作}{https://www.zaobao.com/news/china/story20230514-1394624}{香港媒体报道,香港公共图书馆网站已搜索不到多名香港政治学者的著作,包括香港中文大学政治与行政学系副教授马岳、国际关系学者沈旭晖的作品疑似被下架。 ``香港01''星期六(5月13日)报道,在香港公共图书馆网站搜寻马岳、沈旭晖、香港立法会法律界前议员吴霭仪、前岭南大学文化研究系客席副教授许宝强以及香港政治学者方志恒的名字,均显示``没有符合的检索结果'',即有关作者的书籍可能已被下架……}

\entryitemWithDescription{郭台铭宴请国民党中常委 寻求挺郭反对征召侯友宜}{https://www.zaobao.com/news/china/story20230514-1394618}{鸿海创办人郭台铭积极争取国民党提名参加2024年台湾总统大选。图为郭台铭星期五(5月12日)在新北市参加宗教活动。(路透社) 积极争取国民党提名参选台湾总统的鸿海创办人郭台铭,星期天(5月14日)宴请部分国民党中常委和立委。据台媒报道,郭阵营的目的是,一旦来临的中常会提出征召新北市长侯友宜参选,挺郭的中常委能反对这项决议。 国民党未来一周预计将宣布总统参选人,星期三(17日)的中常会可能会有结果……}