\entryitemWithDescription{庄慧良:``蓝白合''还有胜算吗?}{https://www.zaobao.com/news/china/story20231121-1451288}{台湾执政的民进党总统参选人赖清德和驻美代表萧美琴组成的``赖萧配'',星期一开心亮相。但在野国民党(蓝)总统参选人侯友宜和民众党(白)总统参选人柯文哲,为了民调误差范围认知差异几近分手,外界高度关注两人能否在星期五(11月24日)中央选举委员会登记截止前搭档参选。 国民党和民众党是两个完全不同的政党,套句柯文哲的话,两党的DNA完全不同,合作本非易事……}

\entryitemWithDescription{民进党赖萧配登场 在野蓝白副手未决}{https://www.zaobao.com/news/china/story20231120-1451283}{台湾现任副总统、民进党主席兼总统参选人赖清德(左),星期一(11月20日)下午在台北的竞选总部记者会,正式宣布副手为台湾驻美代表萧美琴。(法新社) 民进党总统参选人赖清德星期一(11月20日)下午正式宣布,他的副总统搭档为台湾驻美代表萧美琴。赖萧配星期二(21日)率先登记,在野蓝白两党的副手仍未敲定。若蓝白合本周内破局,赖萧配备受看好能延续民进党的执政权……}

\entryitemWithDescription{两岸关系低迷 香港赴台学生人数骤降}{https://www.zaobao.com/news/china/story20231120-1451241}{(香港 / 台北综合讯)据台媒报道,当前两岸关系低迷,香港特区政府配合北京,限制台湾招生资讯进入香港校园,赴台就读的香港新生人数近年急剧下降。 《旺报》星期一(11月20日)引用台湾教育部十年数据报道,在马英九执政的2015年,赴台的香港学士与硕博士新生有5038人。2016年蔡英文上任后,这一数字逐年下降。不过在2020年,赴台香港学生人数突然大幅增长,2021年更达到6037人,创历史新高……}

\entryitemWithDescription{美共和党议员促军方履行对台军援承诺}{https://www.zaobao.com/news/china/story20231120-1451239}{(华盛顿彭博电)一群美国共和党议员警告说,台湾向美国采购的F-16战斗机项目面临进一步延误交付的``高风险''。他们敦促美国国防部``集中精力'',履行对台湾的军援承诺。 据彭博社报道,逾20名共和党众议员上星期五(11月17日)致函美国空军部长弗兰克·肯德尔(Frank Kendall),在信中赞赏拜登政府设法加快交付台湾军机项目的速度,但担心过去存在的生产与转让困难可能会持续下去……}

\entryitemWithDescription{中国据报拟在阿曼建军事基地 分析:或增加美处理台海问题的复杂性}{https://www.zaobao.com/news/china/story20231120-1451228}{(华盛顿综合讯)中国据报有意在阿曼兴建军事设施,分析认为,这可能迫使美国不得不重新评估在中东地区乃至全球的军事战略布局与资源配置,或增加其处理台海问题的复杂性。 美国之音星期天(11月19日)的报道引述台湾国策研究院资深顾问陈文甲说,阿曼基地可被视为中国在非洲吉布提后,补充和扩展其在印太地区的军事存在,进一步增强其全球军事布局,可能会对美国在中东地区的军事优势构成挑战与威胁……}

\entryitemWithDescription{分析:中国将持续以非军事手段 维持南中国海主动权}{https://www.zaobao.com/news/china/story20231120-1451220}{中菲两国海警船与民用船只在10月22日发生碰撞事件以来,双方在南中国海的摩擦持续。图为一艘中国海警船11月10日在南中国海有争议的阿云津礁(中国称仁爱礁),靠近执行补给任务的菲律宾海警船……}

\entryitemWithDescription{中国初婚人数九年间大减 去年首次不足1100万}{https://www.zaobao.com/news/china/story20231120-1451196}{(北京综合讯)中国的结婚人数近年持续下降,结婚年龄也普遍推迟,初婚人数从2013年到2022年已大减逾半。 第一财经星期天(11月19日)引述《中国统计年鉴2023》报道,中国2022年全年结婚登记数降至683.5万对,较上年减10.6\%。初婚人数为1051.76万人,比2021年减少9.16\%至106.04万人,这也是多年来初婚人数首次低于1100万人……}

\entryitemWithDescription{中国首只全流程国产克隆猫诞生}{https://www.zaobao.com/news/china/story20231120-1451192}{中国首只全流程采用国产设备、试剂和耗材培育的体细胞克隆猫星期天(11月19日)在青岛农业大学哺乳动物体细胞克隆基地诞生。(中新社) (青岛综合讯)中国首只全流程采用国产设备、试剂和耗材培育的体细胞克隆猫诞生,标志着中国在动物克隆领域拥有完整的产业链。 综合《科技日报》和青岛新闻报道,青岛农业大学星期天(11月19日)发布消息说,克隆猫当天在该校的哺乳动物体细胞克隆基地诞生……}

\entryitemWithDescription{美研究:中国政策性银行正将绿色能源拒之门外}{https://www.zaobao.com/news/china/story20231120-1451044}{(波士顿讯)美国研究发现,尽管中国曾承诺向发展中国家提供更多绿色和低碳能源项目的资金支持,但其政策性银行尚未实现这一承诺。 《南华早报》星期天(11月19日)报道,波士顿大学全球发展政策中心研究发现,包括中国进出口银行和中国国家开发银行在内的中国政策性银行,在2021至2022年间并未向能源部门注入资金……}

\entryitemWithDescription{国民党推韩国瑜领军不分区立委 盼柯文哲回心转意任侯友宜副手}{https://www.zaobao.com/news/china/story20231119-1451043}{台湾在野国民党(蓝)星期天举行临时中常会,通过以高雄市前市长韩国瑜领军的不分区立委名单,希望民众党(白)总统参选人柯文哲回心转意,担任国民党总统参选人侯友宜的副手,也要抢回支持柯文哲和鸿海集团创办人郭台铭的韩粉票源。 民众党同日(11月19日)也举行誓师大会,柯文哲回应支持群众的热烈要求,强调会继续用台湾民众党总统候选人的身份拼战到底……}

\entryitemWithDescription{合资公司获发清算业务核可证 万事达卡正式进军中国市场}{https://www.zaobao.com/news/china/story20231119-1451031}{中国央行宣布,已向万事达在中国的合资企业核发银行卡清算业务许可证,标志着``万事达''品牌人民币银行卡将可在中国市场发行。(路透社档案照) (北京综合讯)中国央行宣布,已向万事达在中国的合资企业核发银行卡清算业务许可证,标志着``万事达''品牌人民币银行卡将可在中国市场发行……}

\entryitemWithDescription{官媒:中国不再依赖投资刺激经济增长的老路}{https://www.zaobao.com/news/china/story20231119-1451016}{中国官媒《经济日报》发表评论称,中国经济不能、也不会再走依赖投资刺激增长的老路,而是要将投资重点聚焦于关键领域和薄弱环节,坚持精准有效的投资导向。图为福建省厦门市建设中的楼盘。(中新社档案照) 中国投资增速持续放缓之际,官媒《经济日报》发表评论称,中国经济不能、也不会再走依赖投资刺激增长的老路,而是要将投资重点聚焦于关键领域和薄弱环节,坚持精准有效的投资导向……}

\entryitemWithDescription{保障融资 中国财政部提前下达明年部分地方债限额}{https://www.zaobao.com/news/china/story20231119-1450997}{民众在中国金融中心上海外滩上行走和休息。(彭博社) (北京综合讯)中国财政部表示,将提前下达2024年度部分新增地方政府债务额度,合理保障地方融资需求……}

\entryitemWithDescription{加媒:在华被捕加拿大人正向加国政府求偿}{https://www.zaobao.com/news/china/story20231119-1450993}{(渥太华综合讯)加拿大媒体报道,曾在中国被拘捕近三年的加拿大人斯帕弗,正在向加国政府寻求赔偿,声称他被捕是因为无意中向加国政府提供了有关朝鲜的情报。 加拿大《环球邮报》星期六(11月18日)引述两名消息人士报道,斯帕弗(Michael Spavor)称,他向康明凯(Michael Kovrig)分享情报,但他并不知道这些信息会被转交给加拿大政府及加国的五眼情报伙伴……}

\entryitemWithDescription{中国人口第一大县安徽临泉关停50所幼儿园}{https://www.zaobao.com/news/china/story20231119-1450978}{(阜阳综合讯)中国人口第一大县安徽阜阳市临泉县,今年已有50所幼儿园停止办学,占全县民办幼儿园总数近四分之一。 21世纪经济报道,临泉县教育局今年8月发布当地50所民办幼儿园终止办学公告,包括12所自愿申请终止和38所因无实际招生活动、办学许可证到期的幼儿园。停办数字占2022年全县幼儿园总数11.8\%,占全县民办幼儿园的比例为24.5\%……}

\entryitemWithDescription{台湾中研院院士吁晶片和平取代晶片战争}{https://www.zaobao.com/news/china/story20231119-1450976}{美国2022年推出晶片法案引发中美晶片战争,同时因地缘政治关系而大力推动台湾半导体业巨头台积电到美国亚利桑那州投资设厂。(Pixabay) 台湾中央研究院院士王平呼吁美国以晶片和平取代晶片战争,建议美国取消一些对中国大陆的出口禁令和制裁,换来大陆承诺不攻打台湾,这对陆美台三方是最好的结果……}

\entryitemWithDescription{中国特稿:中国政策摇摆 外企等待定心丸}{https://www.zaobao.com/news/china/story20231119-1450482}{美国百事公司今年首次参加在上海举办的进博会,展台前排起等待进场的长队。(新华社) 多家在中国发展的外国企业接连撤出,外国直接投资总额转负,令中国政府对外开放的承诺遭受质疑。是哪些因素冲击了外资信心?哪类企业最受影响?中国市场对外资还有多大吸引力? 时隔三年,中国国际进口博览会本月全面恢复线下举办。不过,这场宣示中国对外开放决心的展会,却被几条同期发生的新闻抢了热度……}

\entryitemWithDescription{香港紫荆党吁港府特赦反修例运动中被捕学生}{https://www.zaobao.com/news/china/story20231118-1450873}{(香港综合讯)由中国大陆``海归派''人士组成的香港紫荆党在港媒发文,呼吁港府特赦在2019年反修例风波中被捕的学生。 紫荆党星期六(11月18日)在网媒``香港01''论坛栏目发表题为《特赦被捕学生》的文章,作出上述呼吁。 文章引述媒体报道说,香港目前仍有约6000人在修例风波中被捕后未被起诉。港府公开资料显示,截至2020年3月初,在修例风波中被捕的7700多人中,学生占了40\%……}

\entryitemWithDescription{民调误差率谈不拢 蓝白合再卡关 台在野总统候选人悬而未决}{https://www.zaobao.com/news/china/story20231118-1450870}{民众党主席柯文哲星期六(11月18日)在新庄竞选总部的临时记者会上说,柯侯配的民调支持度都不输侯柯配。(路透社) 因民调误差范围谈不拢,台湾``蓝白合''再卡关,在野党总统候选人悬而未决。双方都称蓝白合未破局,会继续政党协商,但最终能否合力打选战,再添新变数。 台湾明年1月13日举行总统大选,候选人必须在下星期五(11月24日)下午5时前登记……}