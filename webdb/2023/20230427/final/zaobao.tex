\entryitemWithDescription{马英九亲自反应称谓再被误植``台北前领导人'' 希腊总统致歉}{https://www.zaobao.com/news/china/story20230427-1387827}{台湾前总统马英九星期三(4月26日)在希腊出席第八届德尔菲经济论坛(Delphi Economic Forum)开幕式时,当面向应邀致词的希腊总统和论坛创办人反应称谓被误植为``前领导人'',希腊总统致歉,并请主办单位更正。 稍早前,总统府和行政院都指责主办单位作法极不尊重。行政院长陈建仁更说,名称改来改去,``我想绝大部分的人都会取消去演讲''……}

\entryitemWithDescription{陈婧:除了冰淇淋,上海车展还有什么?}{https://www.zaobao.com/news/china/story20230427-1387828}{中国欧盟商会本周的一场记者会上,商会主席伍德克(Joerg Wuttke)问在场媒体,对今年上海车展有什么印象。记者们异口同声:``冰淇淋!'' 德国车企宝马旗下的MINI品牌,因为被拍到在车展上派送冰淇淋时区别对待,只给外国访客免费冰淇淋而陷入舆论旋涡。多家参展车商纷纷蹭热度,承诺送冰淇淋给所有人,但主办方很快禁止所有展台派送冰淇淋……}

\entryitemWithDescription{美专家:中国正进行大规模核力量扩张}{https://www.zaobao.com/news/china/story20230426-1387661}{2019年10月1日的国庆阅兵上,中国对外展示东风41洲际弹道导弹。(法新社) 多名美国专家认为,中国正在进行历史上规模最大的核力量扩张。 根据斯德哥尔摩国际和平研究所(SIPRI)数据,2019年中国的核弹头数量为290枚,2020年增加到320枚,目前为350枚。 法新社星期三(4月26日)报道称,中国的核弹头数量不如美国和俄罗斯,但增长速度很快……}

\entryitemWithDescription{中俄签署加强海上执法合作谅解备忘录}{https://www.zaobao.com/news/china/story20230426-1387656}{中国海警局与俄罗斯联邦安全总局举行高级别工作会晤,并签署加强海上执法合作的谅解备忘录。 据中国官媒``央视军事''微信公号星期三(4月26日)消息,此前两天(4月24日至25日),双方在俄罗斯摩尔曼斯克举行高级别工作会晤,共同签署《关于加强海上执法合作的谅解备忘录》,并观摩了北极海警论坛下的``北极巡航2023''海上实战演习……}

\entryitemWithDescription{台学者:国民党若赢下总统大选 两岸制度协商会加速}{https://www.zaobao.com/news/china/story20230426-1387654}{台湾陆委会前副主委吴安家在纪念汪辜会谈30周年的座谈会上研判,两岸制度化协商估计要到明年初总统大选后才能恢复,若国民党胜选,协商会加速,否则会放缓。他也呼吁台湾建立超党派的两岸政策,以智慧处理、不要硬拼。 1993年4月27日,中国大陆海峡两岸关系协会(海协会)前会长汪道涵和台湾海峡交流基金会(海基会)前董事长辜振甫,在新加坡举行三天的``汪辜会谈'',签署四项事务性协议……}

\entryitemWithDescription{台汉光演习将侧重反制封锁和战力保存}{https://www.zaobao.com/news/china/story20230426-1387609}{台湾国防部宣布,将基于先前中国大陆的环台军演来制定今年度的汉光演习,并将侧重在反制封锁和战力保存上。 台湾国防部星期三(4月26日)召开记者会,宣布今年度``汉光39号''演习将分成5月``电脑辅助指挥所演习''及7月``实兵演练''两个阶段实施。其中,电脑辅助指挥所演习将于5月15至19日实施,实兵演习则将于7月24至28日实施……}

\entryitemWithDescription{摘下口罩 传播欢乐}{https://www.zaobao.com/news/china/story20230426-1387408}{(路透社) 疫情三年,口罩遮住了人们的笑容。法国艺术家JR邀请香港市民摘下口罩,拍摄隐藏已久的笑脸照片,组成社区共创艺术展,一起向社会传播欢乐。这项展览4月24日(星期二)起在香港海港城一带展出,在五光十色的城市中,巨型的黑白相片与周遭环境形成强烈对比,吸引公众的目光与注意力……}

\entryitemWithDescription{杨丹旭:卢沙野的失言让谁尴尬?}{https://www.zaobao.com/news/china/story20230426-1387381}{从敢于斗争到善于斗争之间,还有很深的学问,尤其是遇到国际法和主权相关的敏感话题……}

\entryitemWithDescription{中国火星车祝融号遇到沙尘堆积 无法自主苏醒}{https://www.zaobao.com/news/china/story20230425-1387253}{中国火星探测器祝融号因沙尘累积影响发电能力,无法自主唤醒。图为2021年6月1日拍摄的祝融号火星车与着陆平台合影。(中国国家航天局) 本应于去年12月结束休眠的中国火星探测器祝融号,至今仍然没有任何活动迹象。中国官方在沉默数月后回应称,这可能是因为沙尘累积影响了火星车的发电能力。 据路透社报道,按照计划,祝融号在2022年5月为了度过火星的冬季进入计划休眠模式后,本应该在12月自主唤醒……}

\entryitemWithDescription{美民调:46\%受访者支持禁TikTok 不同党派和年龄存在明显分歧}{https://www.zaobao.com/news/china/story20230425-1387250}{美国一项民调显示,近一半的美国选民支持禁止社交平台TikTok应用,但在频繁使用TikTok的受访人群中,只有12\%的人赞成禁令;不同党派、年龄段的人对TikTok禁令态度存在明显分歧。 根据《华尔街日报》4月开展的民意调查,46\%的受访者支持在全美实施TikTok禁令,35\%的受访者反对……}

\entryitemWithDescription{早说}{https://www.zaobao.com/news/china/story20230425-1387246}{与其说是面试不如说是彼此认识。 ------在美国访问的台湾民众党主席、已表态有意参加2024台湾总统大选的柯文哲,星期五(21日)接受美国之音访问,回答此次访美是否有``被面试''的感觉时如是说。有意角逐总统大位的台湾政治人物到华盛顿与美国官员会面,经常被台湾媒体解读为``面试''……}

\entryitemWithDescription{香港特首李家超:不容许区议会成``港独''平台}{https://www.zaobao.com/news/china/story20230425-1387242}{香港特区行政长官李家超星期二(4月25日)出席行政会议前见记者时指出, 区议员今后将以多种方式产生。(中通社) 备受关注的香港区议会改革方案即将出台。特首李家超星期二(4月25日)指出,不容许区议会沦为``港独''平台,日后区议会将保留一定选举成分,同时议员也会以多种方式产生。受访学者估计,以后的区议会将只有三成或以下议席由直选产生,防止民主派垄断议会……}

\entryitemWithDescription{朔尔茨邀李强访德 缓和中欧紧张关系}{https://www.zaobao.com/news/china/story20230425-1387240}{知情人士透露,为缓和中欧紧张关系,德国总理朔尔茨邀请中国总理李强6月20日访问柏林。 彭博社星期二(4月25日)引述知情人士称,在中欧关系恶化之际,朔尔茨的邀请意在为政治对话敞开大门。 该知情人透露,朔尔茨的目标是在为任何改变台海现状举动划红线的同时,争取北京成为主要合作伙伴,共同推动全球和平、应对气候变化。朔尔茨也希望应对一些棘手问题,例如欧洲企业的市场准入、人权和台湾问题等……}

\entryitemWithDescription{马克龙批卢沙野言辞不当 学者:言论风波不会根本冲击中国拼外交努力}{https://www.zaobao.com/news/china/story20230425-1387238}{中国驻法大使卢沙野上星期五(4月21日)在法国LCI电视台《大访谈》栏目直播专访中,声称``前苏联国家没有有效的国际法地位'',引发欧美哗然。((视频截图)) 中国驻法大使卢沙野近日质疑前苏联加盟共和国的主权国家地位,持续在欧洲引发舆论声讨。近期才访华的法国总统马克龙星期一(4月24日)批评卢沙野言辞不当,强调这些国家的边界是``不可侵犯的'',展现欧洲一致对外立场……}

\entryitemWithDescription{为总统初选备战 新北市长侯友宜提前4月27日起赴议会备询}{https://www.zaobao.com/news/china/story20230425-1387219}{台湾新北市长侯友宜须参加的新北市总质询日期,从6月5日提早至4月27日开始,预定5月6日结束,让他能更早宣布参加国民党的总统初选。这是侯友宜上周访问新加坡时接受媒体采访。(谢智扬摄) 为了让新北市长侯友宜早日争取2024年国民党总统初选提名,该党新北市议会党团星期二(4月25日)借多数优势变更议程,顺利将侯友宜的总质询从原定6月5日提早至4月27日开始,预定进行到5月6日结束……}

\entryitemWithDescription{入境中国抗原检测替代核酸 登机前不再查验}{https://www.zaobao.com/news/china/story20230425-1387218}{中国进一步放宽边境防疫管控,从星期六(4月29日)起,赴华人员可在登机前48小时内以抗原检测代替核酸检测。图为北京大兴机场的旅客。(路透社) 中国外交部宣布,自星期六(4月29日)起,赴华人员可以登机前48小时内的抗原检测代替核酸检测,航空公司也不再查验登机前检测证明。 中国外交部发言人毛宁星期二(4月25日)在例行记者会上宣布上述调整时说,为便利中外人员往来,中国将进一步优化远端检测安排……}

\entryitemWithDescription{中国防长李尚福周四赴印度出席上合防长会议}{https://www.zaobao.com/news/china/story20230425-1387214}{中国国务委员兼国防部长李尚福星期四(4月27日)赴印度出席上海合作组织成员国国防部长会议。 根据中国国防部星期二(25日)在微信公众号发布的信息,李尚福从4月27日至28日将应邀出席在印度新德里举行的上海合作组织成员国国防部长会议。 李尚福将在会议期间作大会发言,并与有关国家代表团团长会面,就国际和地区形势以及防务安全合作等议题进行沟通交流。李尚福预料将与印度防长辛格会面……}

\entryitemWithDescription{中国资深媒体人董郁玉被拘一年后面临间谍罪指控}{https://www.zaobao.com/news/china/story20230425-1387207}{《光明日报》评论部副主任董郁玉被指控间谍罪。(互联网) 中国官方媒体《光明日报》评论部副主任董郁玉被拘留一年后,家人首次发声,称董郁玉已被检方以间谍罪起诉。 综合《华尔街日报》和《纽约时报》星期二(4月25日)报道,去年2月,董郁玉与一名日本外交官在北京市中心一家餐厅会面时被拘留。这名日本外交官数小时后获释,而董郁玉此后一直被拘……}

\entryitemWithDescription{台民族党副主席在大陆被控涉``分裂国家罪''}{https://www.zaobao.com/news/china/story20230425-1387206}{中国大陆以``涉嫌分裂国家罪'',指控并逮捕台湾民族党副主席兼创办人之一的杨智渊。 据中国最高人民检察院官方微博星期二(4月25日)消息,浙江省温州市国家安全局侦查终结后,已将杨智渊一案移送温州市检察院审查起诉。温州市检察院以涉嫌分裂国家罪对其批准逮捕……}

\entryitemWithDescription{台湾彰化食品厂失火 7死15伤}{https://www.zaobao.com/news/china/story20230425-1387197}{台湾联华食品公司彰化厂星期二(4月25日)清晨发生大火,造成七人死亡,15人受伤。(香港中通社) 台湾彰化一家食品工厂星期二(4月25日)发生火患,造成七人死亡,15人受伤。 综合《联合报》和中时新闻网报道,联华食品公司彰化厂在星期二清晨起火。22名员工受困,10人到院无呼吸心跳,都是被浓烟呛伤,七人不治,三人在抢救中。 联华食品发声明说,起火原因初步判断为生产机台意外起火所致……}

\entryitemWithDescription{``五一''长假中国民众担心冠病疫情再起 专家:暴发新一轮疫情概率小}{https://www.zaobao.com/news/china/story20230424-1386861}{星期一(4月24日)下午的重庆网红景点洪崖洞,游客依旧熙熙攘攘。(王纬温摄) 中国多地民众声称感染冠病不满半年复阳,不少网民刊登他们抗原检测第二次阳性的图片,担心冠病感染潮在``五一''长假期间卷土重来。中国疾病控制预防专家评估,五一期间冠病感染人数有可能增加,但不太可能造成大规模流行,原因是民众已形成群体免疫,五一前后气温上升也并不适合病毒生存……}