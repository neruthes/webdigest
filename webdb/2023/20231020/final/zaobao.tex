\entryitemWithDescription{台积电总裁魏哲家:半导体市场很接近触底复苏}{https://www.zaobao.com/news/china/story20231019-1444344}{台积电预计的第四季收入和全年资本支出,都超越分析师预期。(彭博社) 台湾积体电路制造公司(台积电)预计的第四季收入和全年资本支出,都超越分析师预期,总裁魏哲家评估,晶片产业已``很接近''谷底。此一表态支持主流观点,即全球科技需求明年有望从疫情后的低迷中复苏……}

\entryitemWithDescription{塔利班政府冀正式加入``一带一路''倡议}{https://www.zaobao.com/news/china/story20231019-1444337}{阿富汗临时政府代理商工部长阿齐齐(Haji Nooruddin Azizi)说,阿富汗希望加入中巴经济走廊和一带一路倡议,并称``今天正在讨论技术问题''。(路透社) (北京路透电)阿富汗塔利班政府希望正式加入中国的``一带一路''倡议,并将派遣技术团队到中国商谈……}

\entryitemWithDescription{美国加州州长宣布访华 分析:显示中美地方接触也在恢复中}{https://www.zaobao.com/news/china/story20231019-1444336}{美国加利福尼亚州州长纽森宣布下周访华一星期。(路透社) 中美高层官员恢复接触后,美国加利福尼亚州州长纽森宣布下周访华一星期,行程包括香港和上海特斯拉工厂。 受访学者分析,这显示中美上至行政部门、下至地方级别的接触都在逐步恢复。 纽森办公室在美国时间星期三(10月18日)发布纽森下周访华的消息。纽森将访问香港、深圳、广州、北京、上海和江苏……}

\entryitemWithDescription{被指涉嫌从事间谍活动 日籍药企员工被中国逮捕}{https://www.zaobao.com/news/china/story20231019-1444334}{日本媒体报道,中国正式逮捕了因涉嫌间谍活动而被拘留的日本安斯泰来制药公司的在华日籍员工。(互联网) (北京综合讯)日本政府说,中国正式逮捕了因涉嫌间谍活动而被拘留的在华日本制药公司的日籍员工。 综合路透社、日本共同社和日本经济新闻等报道,日本内阁秘书长松野博一星期四(19日)在记者会上说,一名在中国的50多岁日本男子,``确认已在10月中旬被逮捕'',并称日本政府将继续要求中国尽早释放该名男子……}

\entryitemWithDescription{中国新房价格连续三个月下跌}{https://www.zaobao.com/news/china/story20231019-1444332}{楼市宽松政策频出之下,中国新房价格仍在9月录得近一年来最大跌幅,显示市场需求持续低迷。图为中国房企碧桂园在江苏宿迁的一处楼盘。(法新社) 楼市宽松政策频出之下,中国新房价格仍在9月录得近一年来最大跌幅,显示市场需求持续低迷。分析认为,当前政策仅能刺激刚需和改善性购房需求,要使整体楼市回暖,还有待民众对宏观经济信心进一步提升……}

\entryitemWithDescription{``五眼联盟''警告:中国间谍活动日益针对科技公司}{https://www.zaobao.com/news/china/story20231019-1444305}{美联调局官员估计,中国超过一半以窃取美国技术为重点的间谍活动发生在旧金山湾区。图为2020年7月的旧金山湾区。(彭博社) (伦敦/纽约综合讯)``五眼联盟''成员国在本周的一次会议上警告,中国的间谍活动已触及美国及其伙伴国的国家安全、外交和先进商业技术的各个方面,并且日益针对科技公司……}

\entryitemWithDescription{台民调:五成二民众支持蓝白合及柯文哲的``比民调''方案}{https://www.zaobao.com/news/china/story20231019-1444282}{(台北讯)亲绿的台湾民意基金会星期四(10月19日)公布最新民调,显示有五成二的台民众支持蓝白合,也有过半民众接受民众党总统候选人柯文哲主张的``比民调''方案。 这份民调在10月15日至17日进行……}

\entryitemWithDescription{碧桂园辟谣 否认创始人父女离开中国}{https://www.zaobao.com/news/china/story20231019-1444269}{中国房地产巨头碧桂园创始人杨国强2018年3月在香港出席记者会。(路透社) (北京/重庆综合讯)外传陷入债务危机的中国房地产巨头碧桂园创办人父女已离开中国,碧桂园星期四(10月19)辟谣否认,并称该谣言被 ``别有用心''地发布在网络。 碧桂园微信公号发文称,公司关注到有谣言称``创始人父女或已离境'',造成恶劣影响……}

\entryitemWithDescription{徐芳达:``一带一路''价值观与新加坡强化国际供应链的目标一致}{https://www.zaobao.com/news/china/story20231019-1444111}{我国交通部代部长兼财政部高级政务部长徐芳达10月18日在北京``一带一路''国际合作高峰论坛互联互通高级别论坛上强调,供应链的韧性、可持续性和长远规划,是改善全球互联互通的关键领域。(取自徐芳达脸书) 供应链的韧性、可持续性和长远规划,是改善全球互联互通的关键领域。新加坡大力支持``一带一路''倡议,也是因为其价值与我国要发展以开放、多边和以规则为基础的全球供应链体系的目标一致……}

\entryitemWithDescription{陈婧:中国经济还得下猛药?}{https://www.zaobao.com/news/china/story20231019-1444020}{``我们初步测算,如果要完成全年预期目标,四季度只要增长4.4\%以上\ldots\ldots 从这个角度来讲,我们对完成全年预期目标是非常有信心的。'' 中国第三季经济增长数据星期三(10月18日)出炉后,中国国家统计局副局长盛来运在新闻发布会上回答媒体关于全年经济增长能否达标的问题时,语调显得轻松许多……}

\entryitemWithDescription{台湾在野整合陷僵局 蓝营向白营抛橄榄枝:全民调与初选可并行}{https://www.zaobao.com/news/china/story20231018-1444017}{国民党总统参选人侯友宜(左)和民众党总统参选人柯文哲,星期三先后出席远见高峰会``向未来领袖提问''论坛。(2023远见高峰会提供) 台湾在野``蓝白合''濒临破局之际,国民党向民众党抛橄榄枝,称民调和初选并行也可考虑,呼吁尽快进行第二次幕僚会,以达成折衷方案……}

\entryitemWithDescription{中国第三季GDP增长4.9\%超出预期 学者:经济走出年中低谷}{https://www.zaobao.com/news/china/story20231018-1444011}{中国第三季度国内生产总值(GDP)同比增长4.9\%,超出预期。专家认为,中国市场需求逐步回升,经济已经走出年中低谷。图为北京一家餐馆星期二(10月17日)的热闹情况。(路透社) 中国第三季度经济增长超出预期,有助完成全年经济增长目标。经济学家分析,中国经济已走出年中低谷,内需逐步回升,政府出台更多宏观刺激政策的压力降低……}

\entryitemWithDescription{苏起:美中无意发生战争 但台湾或成为美中的``意外''}{https://www.zaobao.com/news/china/story20231018-1444006}{台湾国安会前秘书长苏起认为,美中都把对方看成最大的威胁,无意发生战争,也不想退让,``但台湾可能会变成美中的意外''。 现任台北论坛基金会董事长的苏起,星期三(10月18日)在2023年第21届远见高峰会以``当前美中台关系的特色''为题发表演说,发表上述看法。 他指出,台湾现任总统蔡英文宣称``维持现状'',实际已``改变现状''……}

\entryitemWithDescription{蓝鲸资本创始人:中国正成为``投资信息黑洞''}{https://www.zaobao.com/news/china/story20231018-1443992}{(香港综合讯)做空机构蓝鲸资本(Blue Orca Capital)创办人说,中国限制海外企业取得本土数据将会赶走部分投资者,削弱投资者投资中国的意愿。 据彭博社星期三(10月18日)报道,蓝鲸资本创办人安达尔(Soren Aandahl)说,咨询和尽职公司遭到搜查,海外公司在访问企业资料库方面越来越困难,正将中国变成一个完全的``投资信息黑洞''……}

\entryitemWithDescription{鸿海和英伟达联手建AI工厂}{https://www.zaobao.com/news/china/story20231018-1443981}{英伟达创办人黄仁勋(右二)星期三在富士康年度科技日上 通过一张手绘草图,解释AI工厂如何接收和处理自动驾驶电动车的数据。右一为鸿海集团董事长刘扬伟。(彭博社) (台北综合讯)台湾科技巨头鸿海集团(富士康)和美国芯片制造商英伟达,将联手建造一座人工智能(AI)工厂,用于推动电动车等下一代产品的制造……}

\entryitemWithDescription{美在台协会主席``面试''蓝绿白总统参选人}{https://www.zaobao.com/news/china/story20231017-1443681}{台湾总统选举将在2024年1月13日举行,第11届立法委员选举也在同日举行。(路透社) 美国在台协会(AIT)主席罗森伯格访台五天,安排与蓝绿白三大党总统参选人会面,被视为``面试''台湾未来总统人选。受访学者解读,美国希望当面了解,他们准备如何在复杂国际情势下维持台海现状,并对选情做出第一手评估……}