\entryitemWithDescription{北京应对赖清德就任 分析:``以在野包围执政''模式成型}{https://www.zaobao.com/news/china/story20240501-3547539}{台湾候任总统赖清德将于5月20日宣誓就职。受访学者分析,北京或将运用台湾``朝小野大''局面,藉由在野党在立法院的优势,试图改变民进党政府抗中政策。图为赖清德4月25日在台北召开记者会,宣布即将上任的内阁成员。(路透社) 中国大陆近期加强与台湾在野的国民党交流,并释出陆客赴台可望解禁消息;国民党也传出研拟修改民进党政府制订的《反渗透法》……}

\entryitemWithDescription{中国第三艘航母福建舰展开首次海试}{https://www.zaobao.com/news/china/story20240501-3546477}{中国第三艘航空母舰福建舰星期三(5月1日)上午8时许,从上海江南造船厂码头解缆启航,开展首次航行试验。(新华社) (北京/上海综合讯)中国第三艘航空母舰福建舰在``五一''黄金周假期首日展开首次海试。有分析称,福建舰将显著提升中国的海军能力。 据中国央视报道,福建舰星期三(5月1日)上午8时许从上海江南造船厂码头启航,赴相关海域开展首次航行试验……}

\entryitemWithDescription{广东梅州连续强降雨 高速公路塌陷24死}{https://www.zaobao.com/news/china/story20240501-3548423}{中国广东省梅州市的一条高速公路路面星期三(5月1日)凌晨塌陷,导致20辆车陷落,造成24人死亡、30人受伤。图为无人机拍摄的事故救援现场。(新华社) (广州综合讯)受连续强降雨影响,中国广东省梅州市的一条高速公路路面,在``五一''黄金周假期首日塌陷,导致20辆车陷落,造成24人死亡,30人受伤……}

\entryitemWithDescription{台湾NCC四委员任期将届满 总统府否认蔡英文主导后续人事提名}{https://www.zaobao.com/news/china/story20240501-3548347}{台湾的国家通讯传播委员会(NCC)四名委员任期,将于今年7月31日届满。 在野国民党认为NCC近年的作为高度政治化,主张修正NCC组织法,将NCC委员提名权由行政院提名,改为按立法院政党比例提名。执政的民进党指责这是剥夺行政院人事权的违宪之举 ,坚决反对。 立法院司法及法制、交通委员会联席会议,星期三(5月1日)审查NCC组织法修正案……}

\entryitemWithDescription{华南暴雨影响五一出行 有旅客一天内四趟航班被取消}{https://www.zaobao.com/news/china/story20240501-3547698}{受雷雨天气影响,深圳宝安机场星期三(5月1日)航班表大屏幕显示,有约一半的航班延误或取消。(林煇智摄) 华南地区连日暴雨影响中国``五一''假期出行,深圳、广州机场出现航班大规模延误,高达90\%航班受影响,有旅客在24小时内经历四次航班取消。 中国从星期三(5月1日)开始,为期五天放假调休,许多人计划利用假期外出旅游或回家与家人相聚……}

\entryitemWithDescription{台国安局长:重点关注520后大陆是否借军演对台施压}{https://www.zaobao.com/news/china/story20240501-3547796}{(台北综合讯)距离台湾候任总统赖清德就职还剩不到三星期,台湾国安局长蔡明彦称,由于赖清德就职后的时间点,是中国大陆例行军演的季节,因此会重点关注大陆会否借此向台湾施压。 综合《联合报》、《自由时报》和Yahoo奇摩新闻网报道,台湾立法院外交及国防委员会星期三(5月1日)邀请蔡明彦以及行政院国土安全办公室报告并备询……}

\entryitemWithDescription{中美恢复气候合作谈判 刘振民5月访美}{https://www.zaobao.com/news/china/story20240501-3547431}{美国和中国是全球最大的两个碳排放国。图为美国德克萨斯州橡树林发电厂附近的草原。(法新社) (华盛顿综合讯)中美关系因贸易和安全议题日趋紧张之际,两国负责气候事务的最高官员将恢复双边气候合作谈判。 据路透社报道,美国气候特使波德斯塔(John Podesta)和中国气候变化事务特使刘振民将于本月在华盛顿会面。这也是刘振民1月接任这个职务后首次访问华盛顿。他的前任解振华因身体原因卸任……}

\entryitemWithDescription{中国司法部原副部长刘志强落马 学者:被查中管干部人数大概率将创新高}{https://www.zaobao.com/news/china/story20240430-3540394}{已退休近一年的中国原副司法部长刘志强,星期二(4月30日)被通报落马。 (互联网) 中国司法部过去一个月持续震荡,继原司法部长唐一军4月初落马后,已退休近一年的原副部长刘志强星期二(4月30日)也被通报落马。今年被查的中管干部由此升至第23人,首四个月已达去年被查中管干部总人数近半……}

\entryitemWithDescription{中菲围绕南中国海争议海域再起争端}{https://www.zaobao.com/news/china/story20240430-3539807}{菲律宾海岸警卫队星期二(4月30日)公布的视频显示,两艘中国海警船在斯卡伯勒浅滩(中国称黄岩岛)附近的南中国海争议海域,用高压水炮夹击一艘菲律宾海岸卫队船艇,导致船艇受损。(法新社) (马尼拉/北京综合讯)中国与菲律宾海警船只星期二(4月30日)围绕南中国海有争议海域再起争端。菲律宾海岸警卫队指责中国海警船当天骚扰并损坏了一艘菲律宾海警船,中国则敦促菲律宾停止挑衅,不要挑战中国维护主权的决心……}

\entryitemWithDescription{中国公安机关专项整治``机闹'' 至今拘留311人次}{https://www.zaobao.com/news/china/story20240430-3539780}{(北京综合讯)飞机上霸座、吸烟、闹事等``机闹''问题在中国航班上频繁发生,中国民航公安机关自去年7月以来开展专项整治``机闹''行动,至今已行政拘留311人次。 据中国公安部网站星期二(4月30日)消息,整治``机闹''行动开展以来,民航公安机关共打击处理669起事件,采取刑事强制措施六人次,行政拘留311人次……}

\entryitemWithDescription{中国大陆再公布海警船在金门水域画面 强调``常态化执法巡查''}{https://www.zaobao.com/news/china/story20240430-3539414}{继上个月后,中国大陆星期一(4月29日)再度公布海警船在金门水域执法巡查的画面。(中国海警微博) (北京综合讯)今年2月在金厦水域发生的翻船事故至今尚未结案,中国大陆再度公布海警船在金门水域执法巡查的画面,并强调是``常态化执法巡查'',显示两岸过去对金门、马祖禁限制水域的默契已被打破。 中国海警局官方微博星期一(4月29日)通报,福建海警当天在金门附近海域依法开展常态化执法巡查……}

\entryitemWithDescription{传福建舰五一期间首次海试}{https://www.zaobao.com/news/china/story20240430-3539450}{网传中国第三艘航空母舰福建舰在星期一(4月29日)驶离上海江南造船厂的泊位。(社交平台X) (华盛顿/香港/北京综合讯)美国军事媒体报道,中国第三艘航空母舰福建舰已驶离泊位,并可能即将进行首次海试。有军事专家预计,福建舰将在星期三(5月1日)开始海试……}

\entryitemWithDescription{中国外交部:哈马斯和法塔赫磋商取得积极进展}{https://www.zaobao.com/news/china/story20240430-3538411}{(北京综合讯)中国外交部透露,巴勒斯坦政治势力哈马斯和法塔赫的代表在北京就推进巴勒斯坦内部和解进行磋商,并取得积极进展。 根据中国外交部官网,中国外交部发言人林剑星期二(4月30日)在例行记者会上说,巴勒斯坦民族解放运动(法塔赫)和伊斯兰抵抗组织(哈马斯)代表,应中方邀请赴北京磋商,就推进巴勒斯坦内部和解进行``深入坦诚对话''。 不过,他没有具体说明磋商时间与双方代表身份……}

\entryitemWithDescription{马斯克清除特斯拉在华营运两大障碍}{https://www.zaobao.com/news/china/story20240429-3531301}{特斯拉据报将与中国互联网巨头百度合作,以百度车道级地图信息为基础,在中国提供自动驾驶服务。图为特斯拉位于上海的展厅。(彭博社) 美国电动汽车巨头特斯拉首席执行官马斯克星期一(4月29日)中午搭乘私人飞机离开北京,结束不到24小时的闪电访华行程。有舆论形容马斯克此行是分水岭时刻,据报为特斯拉在华发展清除自动驾驶及数据安全两项重大障碍,北京也已初步批准特斯拉的全自动驾驶科技在华运营……}

\entryitemWithDescription{傅崐萁开记者会说明访问大陆成果 呼吁台湾候任总统赖清德改善两岸关系}{https://www.zaobao.com/news/china/story20240429-3531430}{台湾在野的国民党立法院党团总召集人傅崐萁,星期一(4月29日)说明上周末率团访问中国大陆成果,强调此行降低两岸紧张气氛,更让两岸同胞全面交流,也打开农产品销往大陆之门。他呼吁候任总统赖清德改善两岸关系,让2300万民众安居乐业,这是他们此行最重要的意义……}

\entryitemWithDescription{拜登签署对台军援等法案 中国暗示将采取报复行动}{https://www.zaobao.com/news/china/story20240429-3531265}{(北京综合讯)美国总统拜登签署台湾军事援助法案,以及要求TikTok从中国母公司字节跳动剥离的法案后,中国官方星期一(4月29日)暗示,可能会采取报复行动。 据路透社报道,拜登星期三(24日)签署了一揽子对外援助法案,包括对乌克兰、以色列和台湾的军事援助,同时签署了另一项与援助立法相关的法案,规定如果字节跳动在未来九个月到一年内未能剥离TikTok,将禁止TikTok这一应用程序在美国上架……}

\entryitemWithDescription{台美代表在台北展开实体贸易谈判}{https://www.zaobao.com/news/china/story20240429-3530318}{台湾与美国代表在台北展开为期五天的实体贸易谈判,台方称要争取出口农产品到美国。不过学者评估难度很大,认为这更像是在政治上向社会与农民交代,重点还是希望维持整体对美国的出口增长。 台美去年6月在华盛顿签署``台美21世纪贸易倡议''首批协定,内容包括促进双方贸易便捷化。目前进入第二阶段谈判,课题聚焦农业、劳工与环境……}

\entryitemWithDescription{中国商务部批日本拟实施半导体出口管制}{https://www.zaobao.com/news/china/story20240429-3530700}{(北京综合讯)针对日本宣布拟扩大半导体、量子技术出口限制范围,中国商务部批评日本此举损人不利己,也损害全球供应链的稳定,将严重影响中日企业间的正常贸易往来。 据中国商务部官网消息,发言人星期一(4月29日)就日本拟对半导体等领域相关物项实施出口管制答记者问时,敦促日本从双边经贸关系大局出发,及时纠正错误做法,共同维护全球产业链供应链稳定。 发言人也强调中国将采取必要措施,坚决维护企业正当权益……}

\entryitemWithDescription{台湾宝林茶室中毒案再夺一命 累计四人死亡}{https://www.zaobao.com/news/china/story20240429-3530085}{(台北综合讯)台湾台北市素食餐厅宝林茶室食物中毒案再增一例死亡个案,一名40岁女性经抢救一个多月后离世,以致死例增至四起。 综合《联合报》和《自由时报》报道,台湾卫生福利部次长王必胜星期一(4月29日)证实,自上个星期六(27日)出现死例后,星期一又新增一起死例,也就是说三天内有两人死亡。 据报道,最新死例在3月中旬到宝林茶室吃完粿条后出现恶心、呕吐及腹泻等症状,经医疗团队抢救,病情一度回稳……}