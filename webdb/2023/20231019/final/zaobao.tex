\entryitemWithDescription{陈婧:中国经济还得下猛药?}{https://www.zaobao.com/news/china/story20231019-1444020}{``我们初步测算,如果要完成全年预期目标,四季度只要增长4.4\%以上\ldots\ldots 从这个角度来讲,我们对完成全年预期目标是非常有信心的。'' 中国第三季经济增长数据星期三(10月18日)出炉后,中国国家统计局副局长盛来运在新闻发布会上回答媒体关于全年经济增长能否达标的问题时,语调显得轻松许多……}

\entryitemWithDescription{台湾在野整合陷僵局 蓝营向白营抛橄榄枝:全民调与初选可并行}{https://www.zaobao.com/news/china/story20231018-1444017}{国民党总统参选人侯友宜(左)和民众党总统参选人柯文哲,星期三先后出席远见高峰会``向未来领袖提问''论坛。(2023远见高峰会提供) 台湾在野``蓝白合''濒临破局之际,国民党向民众党抛橄榄枝,称民调和初选并行也可考虑,呼吁尽快进行第二次幕僚会,以达成折衷方案……}

\entryitemWithDescription{中国第三季GDP增长4.9\% 超出预期 经济走出年中低谷}{https://www.zaobao.com/news/china/story20231018-1444011}{中国第三季度国内生产总值(GDP)同比增长4.9\%,超出预期。专家认为,中国市场需求逐步回升,经济已经走出年中低谷。图为北京一家餐馆星期二(10月17日)的热闹情况。(路透社) 中国第三季度经济增长超出预期,有助完成全年经济增长目标。经济学家分析,中国经济已走出年中低谷,内需逐步回升,政府出台更多宏观刺激政策的压力降低……}

\entryitemWithDescription{苏起:美中无意发生战争 但台湾或成为美中的``意外''}{https://www.zaobao.com/news/china/story20231018-1444006}{台湾国安会前秘书长苏起认为,美中都把对方看成最大的威胁,无意发生战争,也不想退让,``但台湾可能会变成美中的意外''。 现任台北论坛基金会董事长的苏起,星期三(10月18日)在2023年第21届远见高峰会以``当前美中台关系的特色''为题发表演说,发表上述看法。 他指出,台湾现任总统蔡英文宣称``维持现状'',实际已``改变现状''……}

\entryitemWithDescription{蓝鲸资本创始人:中国正成为``投资信息黑洞''}{https://www.zaobao.com/news/china/story20231018-1443992}{(香港综合讯)做空机构蓝鲸资本(Blue Orca Capital)创办人说,中国限制海外企业取得本土数据将会赶走部分投资者,削弱投资者投资中国的意愿。 据彭博社星期三(10月18日)报道,蓝鲸资本创办人安达尔(Soren Aandahl)说,咨询和尽职公司遭到搜查,海外公司在访问企业资料库方面越来越困难,正将中国变成一个完全的``投资信息黑洞''……}

\entryitemWithDescription{鸿海和英伟达联手建AI工厂}{https://www.zaobao.com/news/china/story20231018-1443981}{英伟达创办人黄仁勋(右二)星期三在富士康年度科技日上 通过一张手绘草图,解释AI工厂如何接收和处理自动驾驶电动车的数据。右一为鸿海集团董事长刘扬伟。(彭博社) (台北综合讯)台湾科技巨头鸿海集团(富士康)和美国芯片制造商英伟达,将联手建造一座人工智能(AI)工厂,用于推动电动车等下一代产品的制造……}

\entryitemWithDescription{美在台协会主席``面试''蓝绿白总统参选人}{https://www.zaobao.com/news/china/story20231017-1443681}{台湾总统选举将在2024年1月13日举行,第11届立法委员选举也在同日举行。(路透社) 美国在台协会(AIT)主席罗森伯格访台五天,安排与蓝绿白三大党总统参选人会面,被视为``面试''台湾未来总统人选。受访学者解读,美国希望当面了解,他们准备如何在复杂国际情势下维持台海现状,并对选情做出第一手评估……}

\entryitemWithDescription{俄罗斯提出以哈停火决议草案 未获联合国安理会通过}{https://www.zaobao.com/news/china/story20231017-1443680}{联合国安理会星期一(10月16日)晚就以哈冲突举行紧急会议,并就俄罗斯主导的决议草案进行表决。图为俄罗斯常驻联合国代表涅边贾(前排左一)举手支持决议草案。(路透社) 俄罗斯拟的以哈停火决议草案,星期一(10月16日)以美英法日四国反对、中俄等五国支持、六国弃权,未获联合国安全理事会通过……}

\entryitemWithDescription{台防长避谈自制潜舰泄密后续影响 外长:台湾信用和防务安全均受损}{https://www.zaobao.com/news/china/story20231017-1443675}{韩国去年初起诉协助台湾制造潜舰的承包商,台湾国防部长邱国正星期二(10月17日)在立法院受访时说,政府已掌握相关资讯,因检调单位已展开调查,他不再多所置喙,避免被外界穿凿附会,造成更大的风波。 台湾外交部长吴钊燮则在立法院会备询时指出,国际友人愿意帮忙台湾,结果台湾有人泄密,既损伤台湾的信用,未来若没有他国愿意帮忙,台湾防务也会受损……}

\entryitemWithDescription{许连碹:新中通过``一带一路''开展绿色项目潜力巨大}{https://www.zaobao.com/news/china/story20231017-1443631}{新加坡和中国过去15年来制定绿色发展框架、打造绿色建筑示范区,让中新天津生态城从一片盐碱地变成有超过10万人居住的绿色家园,两国未来通过``一带一路''开展绿色项目的合作潜力巨大。 新加坡永续发展与环境部兼交通部高级政务部长许连碹博士星期二(10月17日)在北京第三届``一带一路''国际合作高峰论坛一场以``绿色丝绸之路新展望''为主题的高层研讨会上,作出上述发言……}

\entryitemWithDescription{香港区议会选举接受报名 紫荆党计划首次派人参选}{https://www.zaobao.com/news/china/story20231017-1443627}{香港2023年区议会一般选举将于12月10日举行,提名期由星期二(17日)开始,至10月30日结束。图为民建联主席陈克勤(后排右二)17日率领该党港岛区及离岛区九名参选人到海港政府大楼报名。(香港中通社) 香港实施新选举制度后的首次区议会选举,周二(17日)开始接受报名。除了传统的建制和泛民政团相继公布出选名单,代表中国大陆``海归派''的紫荆党也计划首次派人参选……}

\entryitemWithDescription{中日船只又在钓鱼岛周围海域发生对峙}{https://www.zaobao.com/news/china/story20231017-1443611}{(北京/东京综合讯)中国和日本的船只又在有领土争议的钓鱼岛周围海域发生对峙。 根据中国海警局网站的消息,中国海警局新闻发言人甘羽说,日本``鹤丸''号船只和数艘巡视船星期一(10月16日)非法进入钓鱼岛(日本称尖阁诸岛)领海,中国海警舰艇对其采取必要管控措施并警告驱离。 甘羽称,中国海警舰艇在相关海域开展海上维权执法活动,敦促日本立即停止在该海域的一切违法活动……}

\entryitemWithDescription{路透:中国收紧公务员和国企员工海外旅行的限制}{https://www.zaobao.com/news/china/story20231017-1443600}{(上海/香港路透电)英国媒体报道,中国公务员和国有企业员工正面临更严格的私人出国旅行限制,他们的海外联系也受到更严格的审查。 路透社星期二(10月17日)引述10名现任和前任公务员和国有企业员工说,相关限制措施自2021年以来扩大,包括禁止海外旅行、收紧旅行频率和境外逗留时间、繁琐的审批流程,以及离境前的保密培训。他们称,这些措施都与冠病疫情无关……}

\entryitemWithDescription{``蓝白合''陷僵局 民众党基层出现与郭台铭合作的呼声}{https://www.zaobao.com/news/china/story20231016-1443340}{国民党和民众党14日首次举行``蓝白合''磋商会议,隔天双方就撕破脸。民众党基层现出现与独立参选的鸿海集团创办人郭台铭合作的呼声。(路透社) 台湾在野国民党总统参选人侯友宜的竞选办公室,星期一(10月16日)晚上致函民众党总统参选人柯文哲竞选办公室,呼吁三天内举行第二次会面,尽速整合在野力量。柯办冷回称,看不出侯办有具体内容,不用为见面而见面……}

\entryitemWithDescription{中国央行净投放2890亿人民币MLF 规模为近三年最大}{https://www.zaobao.com/news/china/story20231016-1443334}{中国央行通过中期借贷便利(MLF)向市场注入2890亿元人民币(548亿新元)资金。图为中国央行位于北京的总部大楼。(路透社) 中国央行通过中期借贷便利(MLF)向市场注入2890亿元(人民币,下同,548亿新元)资金,净投放规模为近三年来最大。分析预计,年底前央行还会持续大幅加量操作,为稳经济政策``组合拳''保驾护航……}

\entryitemWithDescription{美国在台协会主席上任半年多三度访台 将见蓝绿白总统参选人}{https://www.zaobao.com/news/china/story20231016-1443330}{美国在台协会(AIT)主席罗森伯格今年第三度访台,她星期一(10月16日)与台湾总统蔡英文见面时强调,美国当务之急是要提高全球对台海和平的重视。 44岁的罗森伯格(Laura Rosenberger)星期天(10月15日)抵台进行五天访问,她在美国国务院亚太局资深顾问罗峻平(Michael Pignatello)陪同下,准备与台湾政府高层、部会官员和各界人士讨论台美关系、区域安全、经贸投资等议题……}