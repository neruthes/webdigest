\entryitemWithDescription{韩咏红:特朗普阴影笼罩2024年}{https://www.zaobao.com/news/china/story20231208-1454841}{英国《经济学人》杂志上个月推出2024年趋势的预测系列,封面设计留给了地球和阴影------美国前总统特朗普的侧面剪影,如日蚀般覆盖半个地球。《经济学人》称,他们的年度预测从未像2024年般被一个人的阴影笼罩,``特朗普构成2024年全世界最大的危险''。 《经济学人》一贯擅长设定议题和打造概念……}

\entryitemWithDescription{王金平任侯友宜的全台后援会总会长}{https://www.zaobao.com/news/china/story20231207-1454828}{国民党总统候选人侯友宜(左二)和副手赵少康(左一),以及国民党主席朱立伦(右一)星期四(12月7日)上午拜会前立法院长王金平。(侯友宜竞选办提供) (台北综合讯)在国民党正副总统候选人侯友宜和赵少康,以及党主席朱立伦亲自力邀下,台湾前立法院长王金平点头答应担任侯友宜的全台后援会总会长……}

\entryitemWithDescription{台网曝情治机关监听政治人物 官方:境外认知作战}{https://www.zaobao.com/news/china/story20231207-1454826}{台湾社群网站出现一份疑似情治机关监听多名政治人物及外国驻台机构的匿名资料。内政部和法务部等相关单位星期四(12月7日)纷纷强调,该份错假资料来自境外,意图在大选期间对台进行认知作战,呼吁民众切勿轻信。 不过,在野国民党正副总统候选人侯友宜、赵少康等人呼吁检调应先厘清事实,若此事为真,监听是否依法请法官开监听票?为何资料会外泄?若此事为假,便不应以讹传讹……}

\entryitemWithDescription{中国户籍改革停滞}{https://www.zaobao.com/news/china/story20231207-1454806}{(北京综合讯)中国全面放宽民众在大城市落户的步伐可能戛然而止。 路透社星期四(12月7日)报道,近几个月来,中国一些经济学家原本预期实施70多年的户籍制度会终止,但参与中央政府内部讨论的两名消息人士透露,有关政策的讨论已经停滞,而且不大可能取得进一步突破,尤其是在中国的大城市……}

\entryitemWithDescription{港府无所不用其极为星期日区选``谷票''}{https://www.zaobao.com/news/china/story20231207-1454789}{港府将于区选投票日前一晚(12月9日)在西九文化区举行大型户外音乐会``共建美好社区大汇演'',宣扬重塑区议会对市民的好处,以及积极参与区议会选举的重要性。免费入场券12月6日在街头派发。(中通社) 香港将在来临的星期日(12月10日)举行新一届区议会选举。港府连月来用尽不同方法呼吁市民投票。有受访学者认为,这次区选若达不到前年立法会选举约三成的投票率,将有损港府形象……}

\entryitemWithDescription{中新天津生态城知多少?}{https://www.zaobao.com/news/china/story20231207-1454782}{2008年的生态城彩虹桥一带(中新天津生态城投资开发有限公司提供) 在距离北京约两个小时车程的天津滨海新区,有一个可能让不少新加坡人感到眼熟的区域;这里街道上绿意盎然,绿化面积高于中国城市平均值,还设有一座小型的鱼尾狮模型。 中新天津生态城里头的中新友好花园设有一座鱼尾狮。(中新天津生态城投资开发有限公司提供) 这是中国和新加坡第二个政府间合作项目中新天津生态城……}

\entryitemWithDescription{四川开设世界最深最大地下实验室}{https://www.zaobao.com/news/china/story20231207-1454775}{世界最深、最大的极深地下实验室锦屏大设施星期四(12月7日)投入科学运行。图为摄于11月8日的中国锦屏地下实验室所在的锦屏山隧道口。(新华社) (成都新华电)世界最深、最大的极深地下实验室------位于中国四川省凉山彝族自治州的锦屏大设施,正式投入科学运行,将探索宇宙暗物质之谜。 锦屏地下实验室二期极深地下极低辐射本底前沿物理实验设施(简称锦屏大设施)土建公用工程星期四(12月7日)完工……}

\entryitemWithDescription{侧记:黄循财首次以总理接班人之姿率团访华}{https://www.zaobao.com/news/china/story20231207-1454627}{副总理兼财政部长黄循财(前排右二)12月6日在人民大会堂与中国总理李强(前排右一)会晤,参加会见的新加坡政要包括(前排右三起)贸工部长颜金勇、教育部长陈振声、国家发展部长李智陞、交通部代部长兼财政部高级政务部长徐芳达、交通部兼永续发展与环境部高级政务部长许连碹博士、国家发展部兼外交部高级政务部长沈颖;以及(后排右起)贸工部兼文化、社区及青年部政务部长刘燕玲,卫生部兼律政部高级政务次长拉哈尤玛赞……}

\entryitemWithDescription{萧美琴被指``战猫外交''背后是卑躬屈膝 台外交部:在野党扭曲台美关系}{https://www.zaobao.com/news/china/story20231206-1454620}{台湾在野国民党立委候选人星期三(12月6日)揭露一份台美谈判密件,指时任驻美代表的民进党副总统候选人萧美琴,去年4月便知台湾同意开放美国莱克多巴胺猪肉和日本福岛五县市食品的政治代价,仍无法被美国纳入印太经济框架(IPEF)的事实,政府吹嘘萧美琴的``战猫外交'',背地里却是卑躬屈膝……}

\entryitemWithDescription{中国部分地方健康码重新上线 受访民众不认为官方会收紧防疫}{https://www.zaobao.com/news/china/story20231206-1454608}{《联合早报》记者星期三(12月6日)登陆``天府健康通'',发现四川的健康码已重新上线。(手机截图) 中国今年入冬以来呼吸道感染患者持续增多,医院儿科就诊量居高不下,广东、四川等地近期重现健康码,引发部分民众担忧严格防疫措施时隔一年卷土重来。 但有迹象显示,重新上线的健康码并未投入使用,有分析相信官方不会在经济不佳之际收紧防疫,以免引起民众反弹……}

\entryitemWithDescription{全球首座第四代核电站在山东投入商业运行}{https://www.zaobao.com/news/china/story20231206-1454549}{华能石岛湾高温气冷堆核电站正式投入商业运行。图为摄于2016年的华能石岛湾高温气冷堆示范工程首台反应堆压力容器吊装。(新华社) (北京新华电)全球首座第四代核电站:位于山东省荣成市的华能石岛湾高温气冷堆核电站正式投入商业运行……}

\entryitemWithDescription{港反修例中枪抗争者:参加更生计划后 学会避免被人煽动}{https://www.zaobao.com/news/china/story20231206-1454539}{曾参与香港反修例抗争而中枪的曾志健,在香港无线电视推出的国安法资讯节目《有法安国》中现身说法但未露脸。(香港警察脸书截图) (香港综合讯)曾在反修例抗争期间中枪被捕的香港青年在国安法节目上说,他参加了更生计划,并从中学会如何管理情绪,``避免被人煽动和唆摆''。 香港警方星期二(12月5日)在脸书发布一段时长两分钟的视频,同时介绍该视频源自香港无线电视11月底开播的国安法资讯节目《有法安国》……}

\entryitemWithDescription{美海军巡逻机飞越台海 中国大陆战机跟监警戒}{https://www.zaobao.com/news/china/story20231206-1454537}{(北京/台北综合讯)美国海军一架巡逻机星期三(12月6日)飞越台湾海峡,中国大陆解放军东部战区称已组织战机跟监警戒。 据路透社报道,美国海军第七舰队星期三发布声明称,P-8A海神海上巡逻侦察机在国际空域飞越了台海。P-8A是美军长程反潜机,能执行海上巡逻、侦察与反潜任务。 声明称,此举表明了美国对自由开放的印太地区的承诺。``美国军队在国际法允许的任何地方飞行、航行和行动……}