\entryitemWithDescription{外商对华投资连月下滑 中国政府一周二度吁支持外资参与大规模设备更新}{https://www.zaobao.com/news/china/story20240702-4025761}{中国副总理何立峰(台上右三)星期一(7月1日)在北京主持召开外资工作座谈会。(新华社) 外商对中国直接投资连续12个月下滑后,中国副总理何立峰本周要求进一步放宽市场准入,破除不合理限制,一视同仁支持内外资企业参与大规模设备更新、政府采购和招投标等……}

\entryitemWithDescription{台湾在野党将邀赖清德赴立法院进行国情报告}{https://www.zaobao.com/news/china/story20240702-4025719}{台湾立法院改革法案6月27日生效,尽管执政的民进党政府四路齐发声请释宪,但在野的国民党和民众党仍决定成立《镜电视》调查小组等查弊,后者更拟于7月12日将邀请总统赖清德赴立法院作国情报告排上议程。 总统府发言人郭雅慧星期二(7月2日)回应称,各宪政机关都应尊重宪法严肃性。虽然宪法法庭裁判的结果未知,但既然已有违宪疑虑,应等候宪法法庭判决结果,于大法官尚未做出释宪结果前不宜采取行动……}

\entryitemWithDescription{中国公民菲律宾遭绑架遇害 使馆敦促菲尽快缉拿严惩凶手}{https://www.zaobao.com/news/china/story20240702-4025564}{(马尼拉 / 北京综合讯)中菲关系紧张之际,一名中国公民在菲律宾遭绑架遇害。中国驻菲律宾大使馆敦促菲律宾官方加大办案力度,尽快缉拿并严惩凶手。 中国驻菲律宾大使馆星期二(7月2日)在官网公告,称日前接到一名中国公民在菲律宾遭绑架的报案求助,立即启动应急机制,全力开展营救工作……}

\entryitemWithDescription{中国拟更广泛推动脑机接口技术 与西方对手竞争}{https://www.zaobao.com/news/china/story20240702-4025357}{(北京综合讯)中国计划筹建脑机接口标准化技术委员会,被视为中国拟更广泛推动脑机接口技术,加快开发这一新兴技术,与西方对手竞争。 中国工业和信息化部星期一(7月1日)在官网发布委员会筹建方案,以听取社会各界意见,截止日期为7月30日。 根据这项方案,官方计划邀请脑机接口领域的企业、科研院所、高校等产业和技术专家,参与开展脑信息编解码、数据通信、数据可视化等脑机接口数据关键标准的制修订工作……}

\entryitemWithDescription{台亲绿民调:赖清德6月执政不满意度 大涨9.4百分点}{https://www.zaobao.com/news/china/story20240702-4025171}{(台北讯)台湾亲绿媒体的最新民意调查显示,34.7\%的受访民众不满意民进党籍总统赖清德上任一个月后的执政表现,比5月大涨9.4个百分点。 《美丽岛电子报》星期一(7月1日)在官网公布6月国政民调的结果,有47.6\%民众满意赖清德的执政表现,比5月增加2.4个百分点;不满意的则有34.7\%,比上月陡增9.4个百分点。 至于信任度,民调显示53.7\%民众信任赖清德,比上月微增0.7个百分点……}

\entryitemWithDescription{中国为南太平洋岛国瓦努阿图建新总统府}{https://www.zaobao.com/news/china/story20240702-4024006}{中国全国政协副主席胡春华(左)星期一(7月1日)在援瓦努阿图总统府、财政部、外交部办公楼项目移交仪式上,和瓦努阿图总理萨尔维共同手执一个印着``中国援助''字样的巨大金色钥匙模型合影。(法新社) (悉尼 / 维拉港综合讯)财政资金短缺的瓦努阿图政府很快将迁入由中国出资兴建的几栋新官方大楼,此举很可能会再次引发外界对北京在这个南太平洋岛国扩大影响力的担忧……}

\entryitemWithDescription{戴庆成:李家超会争取连任香港特首吗?}{https://www.zaobao.com/news/china/story20240702-4017561}{星期一(7月1日),既是香港回归中国27周年的纪念日,也是本届特区政府上任两周年的大日子。连月来,港府除了推出一系列庆祝回归的活动和优惠,也积极对外宣传过去两年的政绩。 行政长官李家超近日就连续两天在社交网络平台发布影片,粉墨登场总结两年以来的施政成果,表明港府未来会继续做好搭台和导演角色,为香港创造机遇。 香港在2019爆发史无前例的反修例风波,对社会带来严重影响……}

\entryitemWithDescription{军工企宣传片显示 中国或已装备新隐身战机}{https://www.zaobao.com/news/china/story20240702-4018498}{沈飞集团宣传片中披露的J-31B``鹘鹰''隐身战机图像。(互联网) 中国沈阳飞机工业集团上周发布一则宣传片,在片尾展示了一款编号为J31-B隐身战机的电脑生成图,并配上``从陆基到海基''的字样。 有军事专家分析,J-31B编号的出现意味着距离解放军装备该机型为期不远;但也有分析认为,不排除这仅仅是沈飞集团展示对新一代舰载机志在必得的决心……}

\entryitemWithDescription{中国航母菲律宾近海现踪 分析指对美国军事反威慑}{https://www.zaobao.com/news/china/story20240701-4018223}{菲律宾总统小马可斯上星期四(6月27日)强硬表示,有必要采取更多行动应对南中国海冲突,中国航空母舰``山东号''同日出现在菲律宾吕宋岛以西海域约200海里(约360公里)处海域……}

\entryitemWithDescription{日媒:苏州遇袭案嫌犯或随机行凶 非针对日本人}{https://www.zaobao.com/news/china/story20240701-4018292}{(苏州讯)日本媒体报道,日本母子苏州遇袭案嫌犯可能是随机行凶,并非针对日本人。 日本共同社星期天(6月30日)引述多名日中关系消息人士报道,关于遇袭事件,当地有关部门已告知日本政府,中国籍男子的作案动机疑为对社会不满,该案正在侦办中。 消息人士透露,该男子没有工作,也没有家人,感到被孤立。这起事件可能并非针对日本人,而是随机无差别袭击。警方认为,嫌犯针对特定人物行凶的可能性较低……}

\entryitemWithDescription{波音据报已全面恢复对华交付飞机}{https://www.zaobao.com/news/china/story20240701-4018176}{在2022年7月20日的英国范堡罗航空航天展上,美国波音公司的标志出现在波音737 MAX飞机的侧面。(路透社) (北京路透电)知情人士说,美国波音公司已全面恢复对华的飞机交付。 此前,由于波音737 MAX客机发生两起空难,加上中美因科技、国家安全等问题而关系紧张,波音自2019年对华的新飞机交付一直断断续续……}

\entryitemWithDescription{北京出台新规:非中国籍港澳永久居民可申请回乡证}{https://www.zaobao.com/news/china/story20240701-4017562}{适逢香港回归中国27周年纪念日,香港特首李家超在庆祝酒会上宣布,中国中央政府同意再赠送香港一对大熊猫。图为香港小学生近距离观看大熊猫的档案照片。(中新社) (香港/澳门/北京综合讯)适逢香港回归中国27周年纪念日,北京出台新规允许香港和澳门的非中国籍永久居民,从下星期三(7月10日)起,可申请办理往返中国大陆的通行证件……}

\entryitemWithDescription{中国民企研制火箭试车出意外 升空后坠地起火}{https://www.zaobao.com/news/china/story20240701-4015860}{中国许多网民星期天(6月30日)发视频称,河南省巩义市发生巨响,有圆柱形物体从空中坠落并在落地后起火。(互联网) (郑州综合讯)中国民营商业航天企业天兵科技自主研制的液体运载火箭,星期天(6月30日)在河南巩义市试车时发生事故,火箭意外升空后坠地起火,不过事件未造成人员伤亡。 当地民众星期天上传到互联网的视频显示,一枚火箭在升空时冒起黑烟,之后从高空坠落到山林里,现场出现火球,浓烟滚滚……}

\entryitemWithDescription{台湾民众党反对蓝营提高罢免门槛修法 周四审议再添变数}{https://www.zaobao.com/news/china/story20240701-4016165}{台湾朝野自5月针对《立法院职权行使法》等改革法案修法以来,冲突不断;而立法院内政委员会7月4日预计排审提高罢免门槛的法案,也势必再掀波澜。图为5月28日台湾执政党与在野党党团,在立法院内展示各式攻防内容的海报与横幅……}

\entryitemWithDescription{苏州校车袭击案后 中国社交媒体平台整治极端言论}{https://www.zaobao.com/news/china/story20240630-4009480}{中国江苏省苏州发生日本籍人士遇袭事件后,中国互联网上出现极端言论。图为摄于6月5日在人来人往的上海黄浦区街道。(法新社) (北京综合讯)中国苏州发生日本籍人士遇袭事件后,中国互联网上出现极端言论,微博、网易、抖音等多个社交媒体平台相继整治相关极端言论;中国官媒也发评论称,不会接受个别人炒作``仇外情绪''。 苏州日本人学校校车6月24日载学生回家时,遭到一名中国男子持刀袭击,一对日本母子受伤……}

\entryitemWithDescription{中国对澳洲实施免签 澳洲仍维持中国旅行警告}{https://www.zaobao.com/news/china/story20240630-4009183}{(墨尔本综合讯)中国早前对澳大利亚护照给予免签证待遇后,澳洲政府并未因此降低对国民出访中国的旅行警告,依然把中国列为须高度谨慎的地区。 中国总理李强6月17日访问澳洲时宣布,将对澳洲实施单方面免签政策。根据澳洲九号新闻,这项消息发布后仅30分钟内,旅游网站Trip.com的搜索数据显示,澳洲用户中与中国有关的关键词搜索量激增80\%……}

\entryitemWithDescription{特稿:中国北方首个地级市GDP破万亿 烟台靠创新与工业升级突围``破圈''}{https://www.zaobao.com/news/china/story20240630-3877755}{烟台市生物医药产业蓬勃发展,已形成相对成熟的创新平台和产业集群,图为中国药企绿叶制药在烟台的厂区。(绿叶制药提供) 新加坡人吃的苹果,不少是产自中国山东烟台。 这个苹果之乡去年晋级中国``GDP万亿城市俱乐部'',是中国北方除了北京、天津两个直辖市以外的第五个万亿GDP城市。不过,烟台的底气并不主要来自农业,制造业才是烟台的主要经济引擎……}

\entryitemWithDescription{中国出台新规明定稀土资源属国有}{https://www.zaobao.com/news/china/story20240630-4009138}{(北京综合讯)中国《稀土管理条例》近日出台,立法明定``稀土资源属于国家所有,任何组织和个人不得侵占或者破坏稀土资源'',并强调国家对稀土资源实行``保护性开采''。 有观点指出,《条例》强调全产业链监管体系,中国在稀土开采、冶炼和流通制定法规,旨在以国家安全名义来保护供应,谋求牢牢掌控这个战略资源……}

\entryitemWithDescription{中国特稿:最重可判死刑 大陆惩独意见完备合法攻台?}{https://www.zaobao.com/news/china/story20240630-3987763}{(插图 / 何汉聪) 受访学者分析,中国大陆发布22条惩治台独的司法意见,借此主张对台管辖权外,更进一步精细化《反分裂国家法》中对台动武的前提。图为北京一处巨型户外屏幕5月24日播放着从地面、空中和海上向台湾发射导弹的动画视频。(路透社) 中国大陆在台湾新任政府上台满一个月之际,发布22条惩治台独的司法意见,引发台湾舆论哗然……}