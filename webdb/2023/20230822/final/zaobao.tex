\entryitemWithDescription{中国早点:香港中文大学再陷政治风波}{https://www.zaobao.com/news/china/story20230822-1425862}{趁着8月暑假还没过去,我上星期回了一趟母校,和当年的老师和同窗聚旧。香港中文大学依山而建,我们相约在山上的新亚书院餐厅吃午饭。 当天我在山下准时上了校巴,岂料校巴行驶了一两分钟就停止前进。原来,大学的马路是单线路,前面有一辆不熟悉交通情况的私人轿车突然迎面驶来,结果两个方向的汽车对望着都动弹不了,后面的车子也被迫停了下来。六七名学校保安在场手忙脚乱地维持秩序,但一时间也不知道怎样处理……}

\entryitemWithDescription{中国荣耀手机重返印度市场}{https://www.zaobao.com/news/china/story20230821-1425873}{中国智能手机品牌荣耀 (Honor)重返印度,并计划于2024年初在印度制造生产。 根据路透社,荣耀能重新进入印度市场,是得益于与一家新成立的荣耀科技公司达成的许可协议。这项协议未公开跟技术和硬件转让相关的``商定成本''。 综合路透社、《上海证券报》、《中国证券报》报道,荣耀科技公司由印度当地股东全资拥有,将在印度生产、销售荣耀手机并提供服务……}

\entryitemWithDescription{台最新民调:赖清德支持度破四成 总统选情已非三足鼎立}{https://www.zaobao.com/news/china/story20230821-1425870}{台湾民意基金会最新民调显示,有43.4\%受访者支持赖清德,超过柯文哲和侯友宜支持率之和。图为赖清德出访巴拉圭、过境美国后,8月18日抵达桃园机场。(路透社) 距离2024年台湾总统选举投票还有四个多月,最新民调显示,目前选民最看好现任副总统、民进党总统参选人赖清德,他的支持率突破四成。 台湾民意基金会星期一(8月21日)在网站公布最新民调结果……}

\entryitemWithDescription{台海局势日益紧张 台湾明年总体国防预算创新高}{https://www.zaobao.com/news/china/story20230821-1425863}{图为台湾陆军今年7月27日在新北市的淡水河口,出动八轮装甲步兵战斗车,实施``汉光39号演习''场面最盛大的``八里海滩联合反登陆操演''。(法新社) 台海两岸关系日益紧张之际,2024年度台湾总体国防预算将达到6068亿元(新台币,下同,约257亿新元),创历史新高,占经济生产总值(GDP)的比率将达到2.5\%……}

\entryitemWithDescription{中国加强打击间谍活动 两周内公开两起个人向美提供情报案}{https://www.zaobao.com/news/china/story20230821-1425833}{中国正在加大力度打击涉嫌为美国提供情报的间谍活动,两周内公开两起由个人向美国中央情报局提供情报的间谍案。 中国国家安全部星期一(8月21日)披露,近日破获一起与中情局相关的间谍案,一名39岁的郝姓国家部委干部向中情局提供情报并收取间谍经费。 根据``国家安全部''微信公众号消息,郝某是在日本留学期间被中情局的东京站人员策反。他因需要办理到美国的签证,与美国驻日本大使馆一个名叫泰德的官员认识……}

\entryitemWithDescription{分析:美日韩对华态度有落差 考验领导人峰会合作成效}{https://www.zaobao.com/news/china/story20230821-1425805}{美国总统拜登(中)、日本首相岸田文雄(右)和韩国总统尹锡悦(左)8月18日在美国戴维营举行峰会。图为三方会后共同接受媒体提问。(法新社) 台湾学者分析认为,这次美日韩领导人峰会关切台海安全,加大对北京施压,合作趋向制度化,或许能确保北京犯台时,能提供更迅速的集体回应。但美日韩对印太合作,以及如何回应中国的扩张仍有不同考量,也考验三方合作的实际成效……}

\entryitemWithDescription{于泽远:中国大陆围台军演已成常态}{https://www.zaobao.com/news/china/story20230821-1425537}{台湾副总统、民进党总统参选人赖清德8月18日刚结束出访过境美国,中国大陆解放军东部战区19日(上周六)在台岛周边展开海空联合演训。台湾国防部称,当天9时起,侦获解放军空警500、运9、歼10、歼11、歼16、苏-30以及直9等各型军机,共计42架次出海活动,其中26架次逾越所谓台海中线及延伸线;另有解放军军舰8艘配合执行联合战备警巡……}

\entryitemWithDescription{美对解放军台海演训反应低调 学者指美中或有默契}{https://www.zaobao.com/news/china/story20230820-1425573}{图为台湾海军成功级导弹护卫舰``田单''号的一名水兵,8月19日通过望远镜监视中国大陆海军054A型导弹护卫舰``徐州''号的动向。(台湾海军司令部) 台湾副总统赖清德返台后,解放军在台海展开演训,美国国务院发言人敦促北京停止在军事、外交和经济上施压台湾。受访学者评估,美国政府反应相当低调,说明中美事前可能已有共识,解放军新一轮军演对中美关系影响有限……}

\entryitemWithDescription{中国青年失业率居高不下 多省设国企招聘目标}{https://www.zaobao.com/news/china/story20230820-1425557}{中国青年失业率居高不下之际,包括广东、安徽、贵州等多个省市相继明确国有企业责任,设定面向高校毕业生招聘目标,以提高年轻人就业。 据澎湃新闻报道,广东省政府办公厅近日出台《关于优化调整稳就业政策措施全力促发展惠民生的通知》(简称``稳就业16条''),提出拓宽渠道促进青年就业……}

\entryitemWithDescription{就业形势严峻 中国青年夜不能寐}{https://www.zaobao.com/news/china/story20230820-1425542}{今年6月中国年龄介于16岁到24岁的青年失业率高达21.3\%,不过保险行业仍然在扩大招聘保险代理人员。图为8月19日在北京的一家保险公司的招聘展位,不少青年正在询问业务要求。(法新社) 心理学专业毕业生张同学(音译)已向雇主投出了数千份简历,但仍未能在她心仪的市场研究领域找到一份工作,这让她倍感焦虑。 张同学上周末在北京的一场招聘会上说,她每发出十份简历,只有一份会收到回复……}

\entryitemWithDescription{台积电走向全球分散风险 深耕本土主导研发}{https://www.zaobao.com/news/china/story20230820-1423389}{台积电7月28日在新竹总部启用可容纳8000名研发人员的全球研发中心,宣示根留台湾和深耕本土的决心。(路透社) 据《联合早报》了解,台积电2028年有望先在台湾量产出1.4纳米制程晶片(也称芯片),并通过开发新技术、新材料和主导产学合作,在2031年之后推出一纳米和更尖端的制程。 身处中美科技冷战的暴风眼,台湾积体电路制造公司(台积电)走向全球分散风险,也深耕本土主导研发……}

\entryitemWithDescription{中国2022年生育率跌至1.09为历史新低}{https://www.zaobao.com/news/china/story20230820-1425500}{北京一名女子7月10日推着婴儿车,等待过马路。(法新社) 中国官方报告显示,中国总和生育率已从2020年的1.30跌至1.09,为历史新低。这一数字不仅低于人口快速老龄化的日本,更在人口破亿的国家中垫底。 据《华尔街日报》报道,《每日经济新闻》8月15日引述中国人口与发展研究中心的一份报告,公布了上述数据。该报道目前已被撤下,相关词条在微博上被限流……}

\entryitemWithDescription{大湾区磁悬浮列车线路设计方案亮相 学者:须与其他城市群连接才可发挥更大作用}{https://www.zaobao.com/news/china/story20230820-1425498}{连接广州、深圳、香港的磁悬浮线路设计方案,在8月初亮相。这项规划旨在进一步推进粤港澳大湾区城市的互联互通,为广深港实现半小时同城化。 受访学者认为,磁悬浮轨道交通是中国接下来寻求技术突破的领域,广深港线路必须与中国其他城市群连接,才可发挥更大作用……}

\entryitemWithDescription{中国特稿:台商投资降温 对陆转趋保守观望}{https://www.zaobao.com/news/china/story20230820-1424679}{中国大陆在冠病疫情后面临严峻的经济下行压力,内需市场萎缩也令台商却步。图为北京一家购物中心停业的食品摊位。(法新社) 冠病疫情结束后,中国大陆经济面临严峻挑战,官方大力推动招商引资,台商自然是必须招揽的对象。不过,大陆开始对台商精准``挑商选资'',着重延揽资讯通信、机械制造等具战略价值的企业。但对台商而言,随着大陆内外不稳定因素增多,投资意愿也渐趋保守观望……}

\entryitemWithDescription{中国恢复15天免签 徐芳达:交通部将推动增加新中直飞航班}{https://www.zaobao.com/news/china/story20230819-1425307}{代交通部长兼财政部高级政务部长徐芳达(台上右三)星期六(8月19日)在北京参加中国新加坡商会国庆晚宴,与宾客``饮胜''祝酒。带领祝酒的还有新加坡驻华大使陈海泉(左三)、中国新加坡商会会长王子元(右二)、经济发展局属下新加坡全球联系司副司长许文强(左二)。左一和右一是晚宴主持人……}

\entryitemWithDescription{价格优势不灵 品牌信心不足 中国电动车欧洲遇挑战}{https://www.zaobao.com/news/china/story20230819-1425297}{一名女子8月17日坐在中国电动汽车制造商蔚来位于德国柏林的展厅内。(路透社) 在中国电动汽车制造商蔚来位于德国柏林的展厅里,一些人正围在蔚来ET5车型旁看车。图摄于8月17日。(路透社) (柏林路透电)中国电动车品牌在本土市场击败外国汽车品牌,但在进军欧洲市场的过程中,却碰到了多项挑战……}

\entryitemWithDescription{解放军在赖清德返台后展开台海演训}{https://www.zaobao.com/news/china/story20230819-1425285}{央视新闻8月19日发布解放军东部战区海空联合演训的视频,称各作战平台在台湾周边海空域进行多方向、长时段抵近慑压。(央视新闻视频截图) 台湾副总统、民进党总统参选人赖清德出访过境美国引发的风波持续延烧,中国大陆解放军东部战区星期六(8月19日)在台湾周边展开海空联合演训,称这是对台独分裂势力与外部势力勾连挑衅的严重警告;台湾外交部长吴钊燮则批评北京试图影响即将举行的台湾选举……}

\entryitemWithDescription{中泰缅老四国警方合作打击赌诈犯罪}{https://www.zaobao.com/news/china/story20230819-1425281}{缅甸地区电信诈骗愈发猖獗,中泰缅老四国警方联合启动针对电信诈骗和网络赌博的专项打击行动。 据央视新闻报道,中国公安部、泰国警察总署、缅甸警察总部、老挝公安部8月15日至16日在泰国清迈联合举行针对赌诈及衍生犯罪的专项合作打击行动启动会……}

\entryitemWithDescription{台湾狠批解放军台海演训 称中国大陆才是麻烦制造者}{https://www.zaobao.com/news/china/story20230819-1425276}{中国大陆8月19日在台湾周边海空域举行联合军事演训之际,台湾派出一艘军舰停靠在北端的基隆港。(路透社) 针对中国大陆星期六(8月19日)在台湾周边展开海空联合演训,台湾各部门作出了强烈谴责和抗议,除了批评北京试图影响即将举行的台湾选举,也指解放军的行径证明大陆才是麻烦制造者……}

\entryitemWithDescription{新闻人间:香港旅游业不振,杨润雄成箭靶}{https://www.zaobao.com/news/china/story20230819-1425022}{``爱国者治港''下的新香港,没有了以往的恶性政治斗争,官员不会再被反对派动辄指骂,但这不等于他们可以``躺平''不用努力工作。文化体育及旅游局局长杨润雄近来就因为表现欠佳,而卷入了舆论的风口浪尖。 在香港,文化体育及旅游局又称为文体旅游局,是特区政府去年重组决策局时成立的一个新部门,主要负责文化艺术、体育、创意产业及旅游事务事宜。 但生不逢时,一成立就遇上冠病疫情,无法大展拳脚……}

\entryitemWithDescription{赖清德过境美国后续效应 北京拟中止部分对台关税优惠}{https://www.zaobao.com/news/china/story20230818-1425069}{台湾副总统赖清德星期五(8月18日)结束出访巴拉圭和过境美国行程返台,中国商务部星期四表示,将结合台湾针对中国大陆的贸易壁垒调查,研究采取中止《海峡两岸经济合作框架协议》(ECFA)部分对台关税优惠。这番表态被学者解读为以商逼政。 赖清德以总统特使身份出访南美洲邦交国巴拉圭,来回过境美国纽约和旧金山,星期五凌晨4时30分搭华航客机抵达桃园国际机场,结束七天六夜行程……}

\entryitemWithDescription{温伟中:台在野势力分击执政新共主}{https://www.zaobao.com/news/china/story20230818-1425051}{面对大野狼的威胁,三只小猪能否团结合力,还是会被逐个击破? 酝酿参选总统的鸿海集团创办人郭台铭,星期四(8月17日)参加号称``挺郭大会''的``主流民意大联盟''云林旅外乡亲后援会活动,向坐满83桌宴席的与会者重新诠释这个童话故事……}