\entryitemWithDescription{陈婧:医药反腐风暴卷了谁}{https://www.zaobao.com/news/china/story20230814-1423435}{过去一个月来,中国医药反腐风暴持续升级,几乎每周都有大动作发生。 7月15日,中国国家医保局、财政部等四部门联合发布通知,指出要彻查医药领域各类违法违规行为。中国卫健委7月21日会同多个部门召开视频会议,部署开展为期一年的全国医药领域腐败问题集中整治工作。一周后,中纪委国家监委召开动员会,要求``全链条、全领域、全覆盖''整肃医药腐败……}

\entryitemWithDescription{接连传出被带走调查 中国前奥运冠军王濛、邓亚萍辟谣}{https://www.zaobao.com/news/china/story20230813-1423412}{中国前短道速滑奥运冠军王濛、前乒乓球奥运冠军邓亚萍,过去一周接连传出被带走的消息,两人均发文辟谣。 综合中新社、《证券时报》报道,王濛持股的湾道体育445万(人民币,83万新元)股权被冻结一事上周引起大众关注,有传闻称王濛``被带走配合调查''。 王濛工作室8月9日在微博转发湾道体育声明称,``王濛一直生活在阳光下'',``没有`被带走',没有`配合调查',没有`联系不上'\,''……}

\entryitemWithDescription{陕西西安突发山洪泥石流 21人遇难六人失联}{https://www.zaobao.com/news/china/story20230813-1423407}{中国陕西省西安市突发山洪泥石流,造成21人遇难、六人失联。 据``西安应急管理''微信公众号星期天(8月13日)发布消息,西安市长安区滦镇街道喂子坪村鸡窝子组星期五(11日)突发山洪泥石流,造成27人失联,21人遇难、仍有六人失联。另有三处电力基础设施毁坏及一条35千伏线路故障,900户居民停电,55个通信基站停机。 专家研判,初步分析,致灾原因为短时暴雨引发突发山洪泥石流……}

\entryitemWithDescription{赖清德抵达纽约 中国大陆批麻烦制造者}{https://www.zaobao.com/news/china/story20230813-1423386}{台湾副总统赖清德8月12日过境美国纽约,当晚下榻曼哈顿乐天纽约皇宫酒店。酒店街区两侧人行道的集会区站满数百名支持者,高举欢迎标语,高喊台湾加油等口号。(路透社) 台湾副总统赖清德率团访问巴拉圭,于当地时间星期六(8月12日)过境纽约,中国大陆星期天(13日)批评赖清德是``麻烦制造者'',并重申反对美台官方互动……}

\entryitemWithDescription{中驻德大使驳斥留学生间谍风险论:是严重的``恐华症''}{https://www.zaobao.com/news/china/story20230813-1423370}{针对德国教育部长指中国留学生增加了学术间谍风险的言论,中国驻德国大使吴恳批评这是严重的``恐华症'',``需要好好治一治''。 中国驻德国大使馆官网星期六(8月12日)发布吴恳接受德国《柏林报》专访的问答,表明上述立场。 据德国之声报道,德国教育部长瓦钦格7月29日曾警告,中国政府提供给青年学者的政府奖学金是战略工具,增加了学术间谍的风险……}

\entryitemWithDescription{内蒙古一周内出现三起鼠疫病例 三人为亲属}{https://www.zaobao.com/news/china/story20230813-1423368}{中国内蒙古过去一周累计报告三起鼠疫病例,三名患者为一起居住的亲人。 新华社星期六(8月12日)报道,内蒙古自治区锡林郭勒盟苏尼特右旗卫生健康委员会星期一(8月7日)通报第一起鼠疫病例。 报道称,患者已在定点医院进行救治,密切接触者已按要求及时进行隔离管控,均未出现异常情况……}

\entryitemWithDescription{持续高温下电力负荷屡破新高 深圳部分区域发生停电}{https://www.zaobao.com/news/china/story20230813-1423367}{深圳下梅林区围面村7月26日凌晨疑因电力负荷过高发生停电。(林煇智摄) 在罕见高温笼罩下,中国南方今年夏天的用电量持续攀升,电力负荷屡破新高,对深圳一些地区的电网带来压力,部分地区因设备不胜负荷而跳闸断电。 受全球暖化和厄尔尼诺现象叠加影响,北半球今年入夏以来平均气温接连打破最高纪录,亚洲、欧洲和美国都面临着极端酷暑天气……}

\entryitemWithDescription{中国特稿:老牌商场等不到春天 疫后``百货''萧条 伤心``太平洋''}{https://www.zaobao.com/news/china/story20230813-1422537}{经营了30年的太平洋百货徐汇店,曾是上海徐家汇商圈的地标之一。(陈婧摄) 熬过三年冠病疫情,经营30年的上海商场``太平洋百货''没能等来迟到的春天,成为又一家退出市场的百货业者。业态老化、电商崛起和疫情冲击,本就令传统百货业风雨飘摇;疫后复苏缓慢的消费,成为压垮许多老牌商场的最后一根稻草。官方政策是否足以提振疲弱的消费?传统百货又要如何摆脱经营困境……}

\entryitemWithDescription{赖清德低调过境美国 大陆首日军演相对克制}{https://www.zaobao.com/news/china/story20230812-1423158}{台湾副总统赖清德(中)星期六(8月12日)登机出访巴拉圭和过境美国前,在桃园国际机场向公众挥手示意。总统府秘书长林佳龙(右)也随行。(路透社) 台湾副总统、执政的民进党总统参选人赖清德星期天(8月13日)低调过境美国纽约,中国大陆首日军演强度相对克制,受访学者认为中美或正通过管控危机,来稳定紧绷的关系……}

\entryitemWithDescription{中国恢复团队游头一天 53艘中国邮轮申请停靠济州港}{https://www.zaobao.com/news/china/story20230812-1423121}{中国时隔六年五个月恢复赴韩团队游后仅过一天,就有53艘从中国出发的邮轮申请中途停靠济州道的港口。 据韩联社星期五(8月11日)引述济州道政府消息报道,中国文化和旅游部星期四(8月10日)宣布即日起恢复对韩国等78国出境团队游业务后,至隔天早上已有53艘从上海启程的邮轮申请途中停靠济州港口,济州港和江汀港直至明年3月的停靠申请都已排满……}

\entryitemWithDescription{中国官媒称部分发达国家领导人不在一带一路论坛邀请之列}{https://www.zaobao.com/news/china/story20230812-1423114}{中国将举办第三届``一带一路''国际合作高峰论坛。图为论坛咨询委员会7月12日在北京召开第五次会议。(中国外交部网站) 中国官媒驳斥有关欧洲国家领导人回避参与中国一带一路国际合作高峰论坛的说法,称一些发达国家领导人本就不在论坛的邀请之列……}

\entryitemWithDescription{澳总理:已通过最高层级要求北京释放澳籍记者成蕾}{https://www.zaobao.com/news/china/story20230812-1423111}{澳大利亚总理阿尔巴尼斯说,堪培拉已通过最高层级向中国政府施压,要求释放被拘留三年的澳大利亚籍华裔记者成蕾。 综合法新社和澳大利亚广播公司报道,阿尔巴尼斯星期六(8月12日)到访昆士兰时,对成蕾的健康情况表示担忧。 阿尔巴尼斯说,成蕾在中国被关押三年太长了,应该获释。``成蕾是澳大利亚公民,她不应该受到这样的对待……}

\entryitemWithDescription{新闻人间:``揭弊天王''邱毅再现江湖}{https://www.zaobao.com/news/china/story20230812-1422850}{台湾政坛``揭弊天王''邱毅(联合早报) 台湾政坛``揭弊天王''邱毅教授出手了,从天天点评时政、批判蓝绿白政治人物,到重炮揭露有关执政党领导人的弊案,网络声量忽然飙升。 67岁的邱毅近年来深居简出,几乎在台湾所有媒体平台绝迹,重心转到经营网络直播,谈两岸统一、文化和经贸交流等课题。 从去年``九合一''选举以来,他开始热衷点评台湾时政,今年3月中以来更几乎天天在脸书发一篇``社论''……}

\entryitemWithDescription{黄小芳:防疫后遗症}{https://www.zaobao.com/news/china/story20230812-1422905}{黑龙江8月以来成为中国舆论焦点,除了暴雨洪灾,当地建设方舱医院的消息也成为持续延烧的热点话题。 8月2日,黑龙江佳木斯要建方舱医院的消息登上微博热搜。网上流传的公示板信息显示,佳木斯前进区计划建造一座方舱医院,用地面积6388平方米。根据公开资料,单是在一期,该建设项目涉及资金规模就超过4000万元人民币(748万新元)……}

\entryitemWithDescription{中国医疗反腐掀起``举报潮'' 今年已近170医院高层被查}{https://www.zaobao.com/news/china/story20230812-1422912}{中国近期医疗反腐风暴持续,重庆、四川等多地星期五(8月11日)跟进公布集中整治举报方式,推升举报医疗行业人员的浪潮。据统计,中国各地今年已有近170名医院院长等高层被查,人数超过去年全年的两倍。 受访学者研判,宏观经济不振之际,官方下半年势必加大力度整治出现在医疗、教育等攸关基层民生领域的腐败,以更好保障民众利益,避免激起更多民愤……}

\entryitemWithDescription{赖清德出访巴拉圭过境美国 学者:中国大陆初步``有节制'' 军演}{https://www.zaobao.com/news/china/story20230811-1422910}{台湾副总统、执政的民进党总统参选人赖清德星期六(8月12日)出访友邦巴拉圭,并过境美国。赖清德出访前夕,中国大陆宣布自星期六起在东海举行三天军事演习。 受访学者专家认为,大陆目前公布的军演相当``有节制'',但要观察赖清德过境美国情况,不排除大陆再升高军演动作。 赖清德星期六启程前往巴拉圭,参加巴拉圭当选总统佩纳就职典礼,其间将过境美国纽约和旧金山,8月18日返台,共计七天六夜……}

\entryitemWithDescription{中国一军工人员公派国外被吸收 向美国提供情报}{https://www.zaobao.com/news/china/story20230811-1422876}{中国国家安全部星期五(8月11日)通报破获一起间谍案,一名中国军工集团工作人员公派意大利期间被吸收,为美国中央情报局(CIA)提供大量核心情报,收取间谍经费。~ 据中国国家安全部微信公众号发布,52岁的曾姓嫌疑人属重要涉密人员,被单位公派至意大利留学进修期间,美国驻意大利使馆官员塞斯主动与其结识,两人通过聚餐、郊游、观赏歌剧等活动逐步建立密切关系……}

\entryitemWithDescription{传中国暂停在伦敦建新使馆计划 中英外交紧张恐加深}{https://www.zaobao.com/news/china/story20230811-1422871}{中国规划在伦敦桥附近一带兴建新大使馆的外部景观。(路透社) 中英两国都希望修复受损的双边关系之际,据报中国将暂停在伦敦建新使馆的计划,这可能会加剧两国外交紧张。 中国2018年以2.5亿英镑(约4.4亿新元)的价格,在伦敦塔附近的皇家造币厂旧址购买土地,计划将现在位于波特兰大街的大使馆迁至该地。但这一计划去年底被当地市议会以存在安全和隐私风险为由,拒绝授予许可证……}

\entryitemWithDescription{在华被拘三年 澳洲记者发公开信描述关押经历}{https://www.zaobao.com/news/china/story20230811-1422841}{因涉嫌国家安全罪而被中国拘留的澳大利亚籍记者成蕾,首次发表公开信,描述自己在中国被关押的经历。(路透社) 澳大利亚籍记者成蕾被中国以涉嫌国家安全罪拘留三年后,首次发表公开信,描述自己在中国被关押的经历。 综合路透社、法新社、彭博社星期五(8月11日)报道,成蕾发表了一封被称为``给2500万人的情书'',是她在北京被拘留三年以来首次发表公开信……}

\entryitemWithDescription{中国大陆出境游名单不包括台湾 民进党政府指责任在大陆官方}{https://www.zaobao.com/news/china/story20230811-1422839}{中国大陆进一步开放出境团队旅游,但不包括台湾。有台湾旅行社业者质疑民进党政府未推动两岸主管部门接触洽商。台湾陆委会则强调,台方一再提议两岸主管部门直接沟通安排,是大陆官方一再拒绝回应。 据中国文化和旅游部官网消息,文旅部办公厅星期四(8月10日)公告,进一步恢复大陆旅行社经营中国公民出境团队旅游业务。在所公布开放的第三批国家和地区名单中,不包括台湾……}