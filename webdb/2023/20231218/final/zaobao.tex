\entryitemWithDescription{京沪楼市新政出台 首个周末买气各不同}{https://www.zaobao.com/news/china/story20231217-1456707}{北京和上海房地产新政出台后首个周末,两地楼市温度不同。图为被积雪覆盖的北京亮马河沿岸人行道。(彭博社) 北京和上海房地产新政出台后首个周末,两地楼市温度不同。分析师预测,京沪楼市有望迎来一波回暖行情,而整体政策仍有进一步放宽空间。 京沪两城上周四(12月14日)同步出台新政策,包括调整普通住宅认定标准、降低首付比例,以及下调房贷利率下限等……}

\entryitemWithDescription{中国明年5月起实施《非银行支付机构监督管理条例》}{https://www.zaobao.com/news/china/story20231217-1456701}{(北京综合讯)中国支付领域发展迅速,但存在违法挪用资金、提供电信诈骗资金转移通道等,官方为防范风险,明年5月起将实施相关监管条例。 综合中国政府网和新华社星期天(12月17日)的消息,中国总理李强日前签署国务院令,公布《非银行支付机构监督管理条例》。该条例将于明年5月1日施行,是中央金融工作会议后出台的首部金融领域行政法规……}

\entryitemWithDescription{台湾选举周末拜票 蓝绿白都打出气势}{https://www.zaobao.com/news/china/story20231217-1456697}{台湾总统与立法委员选举倒数第四个周末,蓝绿白在寒天中从南到北拜票。蓝营主打两岸和平、经济发展;绿营亮出蔡英文路线,呼吁支持``英德美''(谐音赢得美);白营连办两天万人大造势,柯文哲民调回稳。 台湾2024年1月13日举行总统与立委选举,星期一(12月18日)倒数26天,各阵营把握周末冲刺,四处造势和拜票。 不巧周末寒流来袭,台湾进入入冬以来最低温,多处低至11摄氏度……}

\entryitemWithDescription{台湾陷入与大陆``灰区地带冲突''执法困境}{https://www.zaobao.com/news/china/story20231217-1456693}{(台北综合讯)台湾官方指中国大陆渔船和海警在海上执行任务时,严重威胁台湾安全。当处理这类``灰区地带冲突'',台方何时决定使用武力,成为棘手问题。 《联合报》星期天(12月17日)报道,中国大陆渔船、海警形同``海上民兵'',主要负责海上维权并执行部分军事任务,并可能配备武器。 大陆船只执行任务过程中,与台湾易陷入一种既非战争也非和平模糊状态,即所谓``灰区地带冲突''……}

\entryitemWithDescription{网友质疑释义不当《新华字典》被起诉}{https://www.zaobao.com/news/china/story20231217-1456690}{(杭州综合讯)一名中国网民对《新华字典》某些词条释义提出质疑,认为表述不当且涉嫌违法,已向法院提起诉讼。 综合极目新闻、看度新闻等报道,一名为``钱妞,春天''网民上周五(12月15日)发布视频,称第12版《新华字典》某些内容``涉嫌违反相关法律法规,并可能对未成年人造成不良影响'',已将诉讼材料提交至杭州市余杭区法院,预计七个工作日内会有初步结果……}

\entryitemWithDescription{中疾控:目前中国冠病感染处于较低流行水平}{https://www.zaobao.com/news/china/story20231217-1456684}{中国疾病预防控制中心指出,中国冠病病毒感染目前处于较低流行水平。图为北京一名戴着口罩的女子,12月14日走在白雪皑皑的街道上。(路透社) (北京综合电)中国疾病预防控制中心指出,中国冠病病毒感染目前处于较低流行水平……}

\entryitemWithDescription{购房者担忧不确定经济环境 深圳放宽购房限制未能持续刺激需求}{https://www.zaobao.com/news/china/story20231217-1456683}{为提振房地产市场,深圳宣布降低居民购买二套住房的资金门槛,把住房贷款最低首付款比例从70\%-80\%统一下调至40\%。(林煇智摄) 深圳放宽购房限制提高了消费者升级住房的兴趣,但在不明朗的经济环境下,新政策对刺激实际购房需求作用不大,交易量在政策出台第二周已开始减缓。 受访专家认为,不确定的经济环境是购房者的主要担忧,大多数购房者对市场信心仍不足,担心未来房价继续下跌而选择观望……}

\entryitemWithDescription{特稿:拼多多撼动阿里京东电商地位}{https://www.zaobao.com/news/china/story20231217-1456653}{疫后消费降级的大潮,助推了主打性价比的中国电商平台拼多多,令它在中国和海外市场同时飞速发展。(路透社) 刚刚过去的``双十二''网购节,上海白领刘静怡(29岁)最大一笔花销,是在购物网站``拼多多''上买了一台iPad平板电脑。 刘静怡告诉《联合早报》,这台iPad在拼多多售价比官网便宜近800元(人民币,下同,150新元),也比淘宝和京东便宜三四百元不等……}

\entryitemWithDescription{澳洲有信心中国明年会取消对澳酒类关税}{https://www.zaobao.com/news/china/story20231217-1456648}{(悉尼综合讯)中澳贸易关系取得新进展,澳大利亚政府有信心中国会在2024年初撤销对澳洲酒类征收关税。 综合英国``天空新闻''、路透社等报道,澳洲贸易部长法雷尔星期天(12月17日)说,``我非常有信心,在新年伊始我们将从中国当局那里得到一个理想结果:中国将解除对澳酒类的禁令。'' 法雷尔指出,澳洲葡萄酒业正进入加工季,这项举措对葡萄酒制造商来说意义非凡。 中国曾是澳洲最大葡萄酒出口市场……}

\entryitemWithDescription{台湾特稿:打贪反诈抑房价 蓝绿白内政比拼 选民更信谁?}{https://www.zaobao.com/news/china/story20231217-1456375}{台北市信义区豪宅``陶朱隐园'',一户成交价高达18亿新台币(约7600万新元),许多买不起房子、只能租房的``无壳蜗牛''只能望屋兴叹。(温伟中摄) 台湾总统选战倒数四周,蓝绿对决形势日渐清晰,中间选民将左右战局,民生和内政议题让民众最有感。民进党八年执政包袱酝酿政党轮替呼声,但在野党战力不足、整合受挫,导致思变的民心虽继续焖烧,却尚未沸腾……}

\entryitemWithDescription{美国公布新一轮对台军售 学者:向中国重申美国印太战略不变}{https://www.zaobao.com/news/china/story20231216-1456562}{美国国务院批准向台湾出售总额约3亿美元(约4亿新元)的军备,来维护战术信息系统。图为台中军事基地的一名义务兵,11月23日专心聆听台湾总统蔡英文在台上发表讲话。(路透社) 台湾总统大选倒数不到一个月,美国国务院批准向台湾出售总额约3亿美元(约4亿新元)的军备,来维护战术信息系统。 受访学者认为,美国希望通过军售向台湾保证美方的支持,同时也向中国大陆重申,其在印度洋-太平洋地区的战略不会改变……}

\entryitemWithDescription{台湾农业部否认800项农产品受贸易壁垒影响}{https://www.zaobao.com/news/china/story20231216-1456557}{(台北综合讯)对媒体报道称,中国大陆认定台湾对其贸易限制措施构成贸易壁垒,台湾将会有800项农产品受影响,台湾农业部称媒体夸大与错误引用报道,并强调台湾已降低对中国大陆市场依赖,故影响有限。 台湾经济部贸易局今年1月宣布禁止进口一批大陆产品。中国大陆商务部星期五(12月15日)发布公告,认定台湾对大陆贸易限制措施构成贸易壁垒……}

\entryitemWithDescription{人工智能巨头商汤科技创始人汤晓鸥病逝}{https://www.zaobao.com/news/china/story20231216-1456550}{汤晓鸥1996年获得麻省理工学院博士学位,1997年回国,加入香港中文大学任教。(香港中文大学网站) (上海综合讯)中国人工智能巨头商汤科技创始人汤晓鸥去世,享年55岁。 商汤科技官方微信公众号星期六(12月16日)发布讣告称,公司创始人、人工智能科学家、浦江实验室主任、上海人工智能实验室主任、香港中文大学教授汤晓鸥因病救治无效,于星期五(12月15日)深夜11时45分去世……}

\entryitemWithDescription{建成多年的海南环岛高铁儋州海头站开始售票}{https://www.zaobao.com/news/china/story20231216-1456546}{(上海综合讯)此前因投资4000余万元人民币、建成多年未启用而受到舆论关注的海南环岛高铁儋州海头站,已于近日开始售票。 据澎湃新闻报道,中国铁路12306网站显示,目前已经可以购买海头站的火车票。该网站客服人员受询时称,一般在12306官网上能出售的车票,就表示这个站已经开始营业。不过,该工作人员向澎湃新闻提供的海头站联系电话为空号……}

\entryitemWithDescription{美国驻华大使:中美科技合作协定续签未定}{https://www.zaobao.com/news/china/story20231216-1456539}{美国驻华大使伯恩斯(图)2022年7月4日出席在北京清华大学举行的世界和平论坛。(路透社档案照) (华盛顿综合讯)美国驻华大使伯恩斯说,他已开始与中国就续签中美科技合作协定展开磋商,虽然该协定的内容须要与时俱进,但新协定能否签成仍充满变数。 据路透社报道,中美科技合作协定是两国在1979年正式建交后,签署的第一份协定……}

\entryitemWithDescription{东方甄选内讧风波收尾:CEO被免职 俞敏洪和董宇辉共同直播}{https://www.zaobao.com/news/china/story20231216-1456527}{新东方创始人、董事长俞敏洪(右)和``小作文''风波主角、网红主播董宇辉(左),星期六(12月16日)晚上共同出现在俞敏洪的抖音直播间。(俞敏洪抖音直播间截图) (北京综合讯)陷入``小作文''风波的中国直播销售平台东方甄选宣布,免去该公司执行董事和首席执行官(CEO)孙东旭的职务,由母公司新东方创始人、董事长俞敏洪兼任CEO,即日生效……}

\entryitemWithDescription{台湾中研院院士吴玉山:成功的大陆研究才是``护国神山''}{https://www.zaobao.com/news/china/story20231216-1456524}{台湾中央研究院院士吴玉山(图)12月15日提出,对台湾而言,真正的``护国神山''应是成功的中国大陆研究,因为这是攸关战争危机与生死的研究。(缪宗翰摄) 在中美战略竞争之下,台湾的晶片制造业近年被形容为``矽盾'',但台湾中央研究院院士吴玉山星期五(12月15日)指出,台湾真正的护国神山应该是成功的中国大陆研究,因为这攸关战争危机与生死,更值得国家投入资源……}

\entryitemWithDescription{新闻人间:曾经``感动中国''的高耀洁}{https://www.zaobao.com/news/china/story20231216-1456397}{被誉为``中国民间防艾(艾滋病,即爱之病)第一人''的高耀洁医生当地时间12月10日在美国纽约去世,享年95岁。《纽约时报》等美国主流媒体报道了高耀洁逝世的消息,但对很多中国读者来说,高耀洁这个名字已十分陌生。 不过,高耀洁也曾大名鼎鼎……}

\entryitemWithDescription{温伟中:台北诚品与24小时阅读}{https://www.zaobao.com/news/china/story20231216-1456429}{在台北101大楼地标附近,号称全球最大繁体中文书店的诚品信义店,将在12月24日平安夜熄灯。(彭博社) 在台北101大楼地标附近,号称全球最大繁体中文书店,也是全台湾唯一24小时书店、诚品信义店,将在12月24日平安夜熄灯。 幸好,这则坚持了24年的故事还没结束,还有接棒店、代班店续写都市传奇,让任何人任何时候都能免费阅读,24小时不打烊……}

\entryitemWithDescription{香港38\%市民想移居海外 20\%打算移居大陆}{https://www.zaobao.com/news/china/story20231215-1456387}{(香港讯)香港中文大学一份民意调查结果显示,近三成八的香港市民打算移居海外,两成市民有移居中国大陆的打算。 香港中文大学亚太研究所星期四(12月14日)在网站发出新闻稿,公布上述民调。 根据民调,约37.7\%受访市民表示有打算移居外地,比去年9月进行的上一轮调查(28.4\%)高出9.3\%。这轮调查有58\%受访者没打算移居海外,另有4.3\%回答``不知道/很难说/没有想过''……}

\entryitemWithDescription{中国国安部称唱衰中国经济危害国家安全}{https://www.zaobao.com/news/china/story20231215-1456385}{(北京综合讯)中国社媒平台微博提醒用户避免发表有关经济的悲观言论之际,中国国安部发文指出,唱衰中国经济是对中国特色社会主义制度及道路的攻击与否定和对中国的战略围堵打压。 中国国安部是在星期五(12月15日)在官方微信账号上发表题为《国家安全机关坚决筑牢经济安全屏障》的文章,论及唱衰中国经济的危害……}

\entryitemWithDescription{北京地铁昌平线雪夜发生追尾事故 致102人骨折}{https://www.zaobao.com/news/china/story20231215-1456384}{北京地铁昌平线星期四(12月14日)晚发生列车脱节事故,是因两辆列车追尾导致。事故造成515人送医院检查,102人骨折。(互联网) (北京综合讯)北京地铁昌平线星期四(12月14日)晚发生列车脱节事故,是因两辆列车追尾导致。事故造成515人送医院检查,102人骨折。 北京市交通委员会星期五(15日)在微信公号通报称,当天18时57分,地铁昌平线西二旗至生命科学园上行区间,两辆列车发生追尾事故……}

\entryitemWithDescription{英美批香港悬赏通缉五名海外港人}{https://www.zaobao.com/news/china/story20231215-1456382}{香港警务处国家安全处星期四(12月14日)举行记者会公布新一批悬红通缉名单,涉及五名现正身处海外的人士,即郑文杰、许颖婷、邵岚、霍嘉志及蔡明达,每人悬红100万港元。(香港中通社) 香港特区政府继早前通缉罗冠聪和许智峰等八名前民主派人士后,周四(12月14日)再通缉五名海外港人。美国和英国纷纷谴责有关做法,但中国政府强调任何触犯法律红线的人都难逃法网……}