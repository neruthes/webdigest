\entryitemWithDescription{陈婧:有中国特色的万圣节}{https://www.zaobao.com/news/china/story20231102-1447286}{刚过去的这个周末,上海朋友圈里流传这么一句问候语,言下之意是:你参加万圣节巡游吗? 被称为``网红路''的巨鹿路是今年万圣节巡游的中心,这几天都是水泄不通。我也尝试过去凑热闹,但很快就被汹涌的人潮挤出圈外。 不过,即便不到巨鹿路,在上海街头也能感受到浓郁的万圣节氛围:路旁的商家挂起南瓜灯和蜘蛛网,路上的年轻人也有不少盛装打扮,坦然接受行人的注目礼……}

\entryitemWithDescription{中国民众献花 李克强合肥故居外鲜花成山}{https://www.zaobao.com/news/china/story20231101-1447285}{中国原总理李克强10月27日凌晨在上海逝世,中国民众星期二(11月1日)继续到李克强位于安徽合肥的故居外献花,现场鲜花堆积如山。(彭博社) (合肥/北京综合讯)已故中国前总理李克强的遗体星期四(11月2日)将在北京火化。大批民众连日来到李克强位于安徽合肥的故居外献花。 李克强少年时曾居住在安徽省文史研究馆大院,也就是现在的合肥市红星路80号……}

\entryitemWithDescription{郭台铭低调递交连署书 蓝白对选总统仍无共识}{https://www.zaobao.com/news/china/story20231101-1447272}{鸿海集团创办人郭台铭(中)与副手赖佩霞(右三)11月1日下午前往台北市选委会递交台湾总统、副总统选举连署书。不过,郭台铭并未公布连署数量,仅表示会继续向前。(香港中通社) 台湾在野``蓝白合''前景依然不明朗,准备独立参选台湾总统的鸿海集团创办人郭台铭星期三(11月1日)下午递交连署书,但低调不受访。蓝白三主帅密会后,仍无法对如何决定正副总统人选达成共识,蓝营主张协商决定,白营仍坚持公开比赛……}

\entryitemWithDescription{黄永宏呼吁南中国海声索国达成渔业协定}{https://www.zaobao.com/news/china/story20231101-1447264}{南中国海局势近期升温,我国国防部长黄永宏医生呼吁声索国达成渔业协定,并加速《南中国海行为准则》磋商。 中国和菲律宾过去几个月在南中国海摩擦不断,两国船只上月还在阿云津礁(中国称仁爱礁)附近发生碰撞。 黄永宏星期三(11月1日)在北京接受媒体访问时,呼吁涉事各方为南中国海局势降温,并远离危险线和采取防范措施。 他说,各国应尽一切所能避免亚洲发生冲突……}

\entryitemWithDescription{中国互联网平台实施前台实名制}{https://www.zaobao.com/news/china/story20231101-1447253}{中国多家互联网平台宣布,将对有50万粉丝以上的社交媒体账号实施前台实名制,在个人账户的首页展示真实姓名。 中国主要的互联网平台抖音、微信、微博、快手、小红书、B站、百度与知乎,星期二(10月31日)在各自平台发布声明,将分批次引导粉丝量在50万以上和100万以上的自媒体账号,在个人主页展示真实姓名……}

\entryitemWithDescription{中科技部副部长出席英人工智能安全峰会}{https://www.zaobao.com/news/china/story20231101-1447246}{中国科技部副部长吴朝晖将率团出席在英国举行的人工智能安全峰会,并向参会各方介绍中国提出的《全球人工智能治理倡议》。(互联网) (北京/伦敦综合讯)中国科技部副部长率团出席在英国举行的人工智能安全峰会,被视为两国紧张关系进一步缓和。不过英国也称,中国不适合参加部分讨论……}

\entryitemWithDescription{中国北方雾霾将持续到11月中旬}{https://www.zaobao.com/news/china/story20231101-1447238}{北京商务中心区星期三(11月1日)被雾霾笼罩。(路透社) (北京综合讯)中国空气质量近日转差,京津冀等北方城市的雾霾情况预计将持续至11月中旬。 法新社引述瑞士空气净化信息科技公司IQAir称,北京星期三(11月1日)的PM2.5浓度,比世界卫生组织的标准高出二十多倍。IQAir也说,北京目前是世界上污染第三严重的城市……}

\entryitemWithDescription{中国通报多艘日本船只驶入钓鱼岛周围海域}{https://www.zaobao.com/news/china/story20231101-1447221}{(北京/东京综合讯)中国海警局通报,多艘日本船只驶入有领土争议的钓鱼岛(日本称``尖阁诸岛'')周围海域,并称已采取必要管控措施。 根据``中国海警''微信公众号消息,中国海警局新闻发言人甘羽说,日本``惠丸''\,``鹤丸''\,``第八泰生丸''三艘船只和数艘巡视船星期三(11月1日)非法进入钓鱼岛周围海域,中国海警舰艇依法采取必要管控措施……}

\entryitemWithDescription{台国防部:配合山东舰海空联训 37架次解放军军机越过海峡中线}{https://www.zaobao.com/news/china/story20231101-1447219}{(台北综合讯)台湾国防部说,侦察到中国大陆解放军``山东号''航母编队在台湾东南应变区外实施海空联训,并侦测到多架次解放军军机越过台海中线。 根据台湾国防部官网消息,自星期二(10月31日)上午6时至星期三(11月1日)上午6时,侦获中国大陆派出43架次军机、七艘次军舰持续在台海周边活动,其中37架次跨越海峡中线及其延伸线进入西南及东南空域……}

\entryitemWithDescription{在华被拘四年 澳籍作家杨恒均之子吁澳总理助父获释}{https://www.zaobao.com/news/china/story20231101-1447206}{在中国被拘留四年多的杨恒均又名杨军,是澳大利亚华裔时事评论家,网络作家。(互联网) (堪培拉综合讯)澳大利亚华裔作家杨恒均被中国以涉嫌间谍罪拘留四年多后,他的儿子写信请求澳洲总理阿尔巴尼斯,在访华期间敦促北京释放他们的父亲……}

\entryitemWithDescription{赖清德副手考虑六人 萧美琴是优先之上}{https://www.zaobao.com/news/china/story20231101-1447176}{台湾驻美代表萧美琴(右)今年8月在赖清德过境美国期间,在脸书上贴出两人看职棒时,分别穿上1号跟11号球衣的背影照,曾一度被解读为赖萧配已成形。(萧美琴脸书) (台北综合讯)台湾民进党总统参选人赖清德透露,考虑搭档的副总统人选目前有六人,外界热传的台湾驻美国代表萧美琴、文化部前部长郑丽君均在列,且萧美琴在名单中是``优先之上''……}

\entryitemWithDescription{台湾前副防长:希望对岸深刻体会到武力解决是``最笨方法''}{https://www.zaobao.com/news/china/story20231031-1447052}{台湾前海军司令陈永康上将(左)和战略学者、国民党驻美代表黄介正教授,在台湾学术界推动台海兵棋推演多年。(温伟中摄) 台湾安全研究中心星期一(10月30日)发表区域安全兵推报告,设想中国大陆2027年攻台情境,推演大陆如何施展各种围困手法,并全方位加强备战与避战之道……}

\entryitemWithDescription{黄永宏:避免亚洲出现实体冲突 是各国安全首长首要任务}{https://www.zaobao.com/news/china/story20231031-1447032}{我国国防部长黄永宏医生10月31日在北京香山论坛上致辞。(法新社) 我国国防部长黄永宏医生呼吁各国国防机构和军队必须接触,并强调未来的十年里,所有安全首长最重要的任务,是避免亚洲发生实体冲突。 黄永宏星期二(10月31日)在北京香山论坛上致辞时,以过去三年的冠病疫情、俄乌战争和以哈冲突为例,提醒和平并不牢固,也不能被任何国家视为理所当然……}

\entryitemWithDescription{北京指菲护卫舰闯黄岩岛邻近海域 马尼拉指中国炒作}{https://www.zaobao.com/news/china/story20231031-1447013}{(北京/马尼拉综合讯)中国和菲律宾在南中国海又发生摩擦,中国解放军星期一(10月30日)称,拦阻管制了一艘``非法闯入''黄岩岛(菲律宾称马辛洛克浅滩)邻近海域的菲律宾护卫舰。菲律宾则回应,船只``没有非法进入中国的主权空间'',并称将不惜一切代价保护领土……}

\entryitemWithDescription{中国国安部发现数百气象探测点 非法向境外传送数据}{https://www.zaobao.com/news/china/story20231031-1447008}{中国国安部发现藏身在居民区的非法气象探测设备。(中国国家安全部) (北京综合讯)中国国家安全部披露,20多个省份存在数百个非法气象探测站,实时向中国大陆境外传输数据,对国家安全构成风险。 中国国家安全部微信公众号星期二(10月31日)发布消息称,国安机关今年会同气象、保密等部门,在大陆各地开展对涉外气象探测的专项治理,查核了10余家境外气象设备代理商和3000多个涉外气象站点……}

\entryitemWithDescription{澳洲智库报告:中国对太平洋地区援助缩量增质}{https://www.zaobao.com/news/china/story20231031-1447006}{(悉尼综合讯)澳大利亚智库报告显示,中国对太平洋地区的援助正在减少,但针对性更强,重点转向了支持合作伙伴和战略国家。 综合《南华早报》、澳大利亚广播公司等报道,澳大利亚智库洛伊国际政策研究所星期二(10月31日)发表报告,指中国在太平洋地区的支出在2016年达到3.84亿美元(约5.24亿新元)峰值后,到2021年已缩减至2.41亿美元……}

\entryitemWithDescription{神舟十六飞船返回舱着陆}{https://www.zaobao.com/news/china/story20231031-1446999}{神舟十六号载人飞船返回舱在东风着陆场成功着陆,现场医监医保人员确认航天员景海鹏、朱杨柱、桂海潮身体健康状况良好,顺利出舱。 图为航天员景海鹏出舱。 (中新社) (北京综合讯)中国神舟十六号载人飞船结束了为期五个月的太空空间站任务后,星期二(10月31日)早上在内蒙古的东风着陆场着陆……}

\entryitemWithDescription{台学者:东南亚是``一带一路''下阶段发展重点}{https://www.zaobao.com/news/china/story20231031-1446962}{台湾政治大学东亚研究所10月31日举办``一带一路的十年回顾:进展与影响''专题演讲。由政大东亚所教授薛健吾(右)主讲,淡江大学中国一带一路研究中心执行长洪耀南(左)与谈。 (缪宗翰摄) 台湾学者认为,北京下一步会把``一带一路''发展重心放在东南亚,然而由于不少东南亚国家与中国存在海洋冲突,可预期这些国家会采取一定的避险措施,并维持``经济靠中国、安全靠美国''的策略……}

\entryitemWithDescription{【东谈西论】郭台铭选总统内外交困}{https://www.zaobao.com/news/china/story20231031-1446891}{8月28日宣布以独立参选人身份投入台湾2024年总统大选的鸿海集团创办人郭台铭,近期被传旗下的富士康集团正被中国大陆调查。(路透社) 2024年台湾总统选举进入倒计时。在四组参选人中,排名第四的台湾首富郭台铭最近内外交困。 郭台铭创办的鸿海富士康集团,在中国大陆被查税和土地违规使用。在台湾,郭台铭也陷入``花钱买连署''的丑闻……}