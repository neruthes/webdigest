\entryitemWithDescription{杨丹旭:瓦格纳兵变给中国提了什么醒?}{https://www.zaobao.com/news/china/story20230628-1408527}{戏剧性的瓦格纳集团兵变,在刚过去的周末是中国舆论场最热门的话题。 消息传出后,挺乌克兰的中国网民难掩兴奋,等着看俄罗斯``内讧'',还把瓦格纳集团首脑普里戈任比作俄版``安禄山'';挺俄派则忧心忡忡,担心俄罗斯陷入内战,引发俄乌战局大转折,让美国得以腾出手来对付中国。 不过,与舆论场上热火朝天的讨论相比,中国官方采取的态度要冷很多,并没有在兵变发生后的第一时间做出任何评论……}

\entryitemWithDescription{陈诗龙:中国疫后在政策上增加了韧力}{https://www.zaobao.com/news/china/story20230627-1408515}{我国人力部长兼贸工部第二部长陈诗龙医生(左二)星期二(6月27日)到访位于成都高新区的新川兰花文化博览园,赠予园方一盆珍贵罕见的胡姬品种``Aranda Noorah Alsagoff''。左一是四川副省长杨兴平……}

\entryitemWithDescription{台在野党批教育部为延兵役牺牲教育}{https://www.zaobao.com/news/china/story20230627-1408511}{为配合2024年义务役恢复为一年,台湾国防部与教育部近日推出``3+1''修课新方案,让男大学生能三年完成学业、一年当兵。首批适用对象为2005年以后出生的役男,今年9月进入大学即适用。 在野的国民党和民众党立委星期二(6月27日)各自举行记者会,齐轰民进党政府执政七年造成两岸兵凶战危,如今再把年轻人推到最前线,让学制大乱,要求教育部长潘文忠下台。 ``3+1''方案在民间引起不小反弹……}

\entryitemWithDescription{全台首个AI科技主播 引发主播饭碗被抢热议}{https://www.zaobao.com/news/china/story20230627-1408496}{台湾民视6月26日推出全台首个AI电视主播,虚拟气质美女今后每逢星期一至五在黄金时段播报气象新闻。(民视提供) 台湾推出全台首个人工智能(AI)电视主播,从星期一(6月26日)起在黄金时段由虚拟美女播报气象新闻,揭示台湾正迈向人机相融的新闻新时代,也引发主播饭碗是否被抢的讨论……}

\entryitemWithDescription{李强:政府不要越俎代庖替企业评估风险}{https://www.zaobao.com/news/china/story20230627-1408489}{中国总理李强(左)星期二(6月27日)在夏季达沃斯开幕大会上致辞后,与世界经济论坛主席施瓦布交谈。(法新社) 在去风险成为欧美对华政策主流之际,中国总理李强称``降依赖、去风险''是一道伪命题,他呼吁政府和相关组织不要越俎代庖替企业评估风险,更不要把去风险政治化和意识形态化……}

\entryitemWithDescription{李家超:香港会对接国家的要求 支持爱国主义教育}{https://www.zaobao.com/news/china/story20230627-1408469}{中国全国人大常委会近日在审议制订《爱国主义教育法(草案)》,当中列明港澳台人士等要自觉维护国家统一和民族团结。香港特首李家超说,港府会对接国家的要求,支持爱国主义教育。 李家超星期二(6月27日)出席行政会议前会见媒体时强调,香港已进行了大量爱国教育的工作,包括中史课、国安法教育等,也有推动学生访问中国大陆不同省市。这些措施都是推动爱国主义教育、加强认识国家、推动两地人民交流、认识中华文化的工作……}

\entryitemWithDescription{香港骨灰龛场高楼化 解决安置地短缺问题}{https://www.zaobao.com/news/china/story20230627-1408432}{图为6月7日善心生命文化纪念馆的一名职员,指着放置骨灰盒的壁龛。(法新社) 一座拥有白色大理石门厅和奢华吊灯的12层高楼,或许会被误认作香港最新的酒店;事实上它将为数千人提供更长久的寄宿,成为他们骨灰最终的安息之地。 法新社星期二(6月27日)报道,上个月开幕的善心生命文化纪念馆计划提供2万3000个骨灰龛位。这是香港政府十年来引入私营公司以缓解死亡关怀行业压力的部分应对措施……}

\entryitemWithDescription{李尚福:中国愿与越南加强军事合作}{https://www.zaobao.com/news/china/story20230627-1408414}{中国国务委员兼国防部长李尚福星期二(6月27日)在北京向到访的越南国防部长潘文江表示,中国愿同越南一道加强两军高层沟通,深化务实合作。 据中国国防部官网发布的新闻稿,李尚福说,当前国际局势变乱交织,亚太地区安全仍面临一些挑战,中越双方应继续在社会主义新征程中并肩携手、紧密团结,维护两国共同战略利益,为地区和平稳定作出积极贡献……}

\entryitemWithDescription{戴庆成:李家超上任一周年}{https://www.zaobao.com/news/china/story20230627-1408122}{``在我们这儿,要保持原地不动,你得飞快地跑动才行。'' 这句话出自英国童话小说《爱丽丝镜中奇遇记》中红皇后与爱丽丝的对话。美国进化生物学家范华伦(Leigh van Valen)后来引用这句话提出了``红皇后假说''。根据该假说,在适者生存的大自然,比其他生物体进化速度缓慢的生物体,最终会面临被淘汱和灭亡。 去年7月1日,警察出身的李家超在外界普遍不看好的情况下出任香港第六任特首……}

\entryitemWithDescription{陈诗龙:新中合资医院将助成都成为宜居城市}{https://www.zaobao.com/news/china/story20230627-1408161}{新加坡人力部长兼贸工部第二部长陈诗龙医生(右二),星期一(6月26日)在成都走访企业,并品尝当地茶饮。右一是新加坡驻中国大使陈海泉。(王纬温摄) 到访四川省会成都的新加坡人力部长兼贸工部第二部长陈诗龙医生,星期一(6月26日)为新落户成都的一家新中合资医院主持启动仪式……}

\entryitemWithDescription{柯文哲称若当选总统 将重启两岸货贸服贸谈判}{https://www.zaobao.com/news/china/story20230626-1408150}{民众党总统参选人柯文哲(右)星期一(6月26日)上广播节目,谈两岸服务贸易谈判。(民众党提供) 台湾民众党总统参选人、党主席柯文哲表示,2024年若当选总统,会先通过《两岸协议监督条例》建立规范,再依续和中国大陆重启《海峡两岸货品贸易协议》和《海峡两岸服务贸易协议》(简称货贸和服贸)谈判。 他星期一(6月26日)在接受广播节目专访时作此表示……}

\entryitemWithDescription{新粤首季度双边贸易增五成 疫后首次线下合作理事会召开}{https://www.zaobao.com/news/china/story20230626-1408139}{新加坡卫生部长、新加坡---广东合作理事会新方联合主席王乙康(右)星期一(6月26日)与广东省省长、理事会粤方联合主席王伟中,签署广东---新加坡深化合作备忘录。(梁麒麟摄) 新加坡和中国广东省的双边贸易额再创新高,新加坡卫生部长王乙康说,新粤经济联系强大且有韧性,未来双方将在具有前瞻性的绿色经济、数码经济和健康医药领域展开合作……}

\entryitemWithDescription{台北市长蒋万安下周赴新 开展上任后首次出访}{https://www.zaobao.com/news/china/story20230626-1408118}{台北市长蒋万安(中)将于下周出访新加坡。(自由时报) 台北市长蒋万安上任满半年后,将于7月5日至8日到新加坡参观访问。 综合《联合报》与《中国时报》消息,蒋万安星期一(6月26日)出席台北信义区``市长与里长有约''座谈时受访称,感谢新加坡驻台北商务办事处的邀请,他将就交通建设、都市发展、环境永续、智慧城市等议题参访新加坡……}

\entryitemWithDescription{中国启动2023年高校毕业生就业服务攻坚行动}{https://www.zaobao.com/news/china/story20230626-1408087}{中国青年失业率持续高企,高校毕业生正面临严峻的就业形势。中国人力资源社会保障部星期天(6月25日)开始启动2023年高校毕业生等青年就业服务攻坚行动,为期共六个月,多举措推动青年就业创业。 新华社报道,中国人力资源社会保障部星期天发布了上述通知。当中要求,在今年底前,力争有就业意愿的未就业毕业生和登记失业青年都能实现就业,或是参加到就业准备活动中……}

\entryitemWithDescription{台外长吁台澳互派武官 防止``最坏情况发生''}{https://www.zaobao.com/news/china/story20230626-1408085}{台湾外交部长吴钊燮接受澳洲媒体访问时呼吁澳政府,台澳应互派武官进驻,以防止``最坏情况发生''。图为吴钊燮6月14日在捷克首都布拉格举行的欧洲价值峰会上发表演说。(路透社) 在两岸关系和地缘政治局势持续紧张之际,台湾外交部长吴钊燮呼吁澳大利亚政府,台澳双方应互派武官进驻,以防止``最坏情况发生''。 吴钊燮今年6月初接受《澳洲人报》驻台记者专访,报道在星期天(6月25日)刊登……}

\entryitemWithDescription{美媒:北京尝试就台湾大选争取美国合作}{https://www.zaobao.com/news/china/story20230626-1408074}{美国媒体报道称,台湾总统选举是美国国务卿布林肯访华期间与中国大陆高级官员会谈的重要议题。北京表达对民进党籍候选人赖清德的担忧,美国则重申不干预台湾选举的立场。 《华尔街日报》星期一(6月26日)报道,知情人士透露,中国大陆官员向布林肯暗示,也是台湾副总统赖清德的政治立场可能会加剧台海紧张局势,进一步损害北京和华盛顿的关系。北京认为,赖清德是台独阵营的一员……}

\entryitemWithDescription{中国拟定爱国主义教育法 维护国家统一和民族团结}{https://www.zaobao.com/news/china/story20230626-1408072}{中国全国人大常委会会议正在审议爱国主义教育法草案,以维护国家统一和民族团结。草案将针对港澳台和海外华侨等不同群体,分别作出针对性规定的爱国主义教育,以构筑中华民族共有精神家园。 综合中国新闻网和《法治日报》微信公众号消息 十四届全国人大常委会第三次会议于星期一(26日)至星期三(28日),一连三天在北京审议《中华人民共和国爱国主义教育法(草案)》……}

\entryitemWithDescription{于泽远:俄罗斯兵变搅动中国舆论场}{https://www.zaobao.com/news/china/story20230626-1407780}{俄罗斯瓦格纳雇佣军老板普里戈任突然发动兵变,不到24小时又戏剧性收场,极大搅动了中国舆论场,也让中国的挺乌派和挺俄派在一天之内经历了冰火两重天。 6月24日(星期六)上午,有关瓦格纳发动兵变的消息开始充斥中国互联网。大量自媒体发布了普里戈任的表态和瓦格纳雇佣军的动向……}

\entryitemWithDescription{俄副外长赴北京 中国宣示支持俄罗斯维护稳定未提普京}{https://www.zaobao.com/news/china/story20230626-1407841}{俄国副外交部长亚历山大·鲁登科(左)星期天(6月25日)抵达北京与中国外长秦刚会面。(中国外交部官网) 俄罗斯兵变出现戏剧性转变之际,俄国副外交部长亚历山大·鲁登科抵达北京与中国外长秦刚会面,中国外交部并在同日晚间发表声明``支持俄罗斯维护国家稳定''……}

\entryitemWithDescription{新川贸易去年创新高 陈诗龙:双边经济关系实力韧性强}{https://www.zaobao.com/news/china/story20230625-1407835}{到访成都的我国人力部长兼贸工部第二部长陈诗龙医生(左)星期天(6月25日)在会见后,赠予四川省省长黄强(右)从新加坡带来的石斛兰。(王纬温摄) 我国人力部长兼贸工部第二部长陈诗龙医生星期天(6月25日)在四川省会成都说,新加坡和四川双边贸易去年尽管处疫情期间,仍创下27亿美元(36.5亿新元)的历史新高,同比增幅超过38%,足以说明新川经济关系实力和韧性之强。 陈诗龙星期六晚飞抵成都访问四天……}