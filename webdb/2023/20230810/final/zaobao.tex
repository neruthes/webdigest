\entryitemWithDescription{中国首条直通中越边境高速铁路开始铺轨}{https://www.zaobao.com/news/china/story20230809-1422213}{中国铁路南宁局集团称,从广西防城港至东兴的防东铁路星期二已开始铺轨。图为7月20日,防东铁路西湾跨海双线特大桥在进行桥梁架设。(中新社) 据中国铁路南宁局集团消息透露,从广西防城港至东兴的铁路星期二开始铺轨,意味着中国首条直通中越边境的高速铁路------防东铁路开始铺轨……}

\entryitemWithDescription{侯友宜主张将核电列入能源选项 民进党讥``侯友宜参选人打脸侯市长''}{https://www.zaobao.com/news/china/story20230809-1422209}{台湾在野国民党总统参选人、新北市长侯友宜星期三(8月9日)批判执政的民进党``2025非核家园''目标不可能达成,主张将核电正式列入能源选项。 侯友宜承诺当选后,第一任期内将完成核一(第一核能发电厂)、核二、核三检查检修工作,安全延役,并邀请顶尖核能安全学者专家成立核四总体安全审查委员会,在安全无虞下,推动核四安全重启……}

\entryitemWithDescription{中国大陆再派大批军机舰台海警巡}{https://www.zaobao.com/news/china/story20230809-1422207}{台湾国防部星期三(8月9日)通报,中国大陆派出10架次军机越过台海中线,并配合五艘大陆军舰执行联合战备警巡,是本周内第二次大规模军机舰在台海周边活动。 通报说,星期三自早上9时起即陆续侦获大陆各型军机共计25架次出海活动,其中10架次逾越台海中线及延伸线进入台湾西南空域。 台国防部强调,台军运用联合情监侦手段绵密掌握,并检派任务军机军舰及岸置导弹系统,严密监控应处……}

\entryitemWithDescription{游轮交通接驳服务严重不足 分析指香港旅游业竞争力下降}{https://www.zaobao.com/news/china/story20230809-1422205}{启德邮轮码头位于九龙九龙城启德承丰道33号,即前启德机场跑道的末端。特区政府期望发展启德邮轮码头,可以帮助香港把握亚太区游轮旅游业市场增长所带来的机遇,将香港发展成为亚洲的游轮中心。(互联网) 近年香港致力打造成为亚洲区内游轮旅游中心,但有游轮上周重临香港时,却出现交通接驳服务严重不足问题,引起旅客强烈不满。有受访学者认为,事件反映香港旅游业竞争力下降,当局有必要检讨和改进……}

\entryitemWithDescription{【视频】33人在北京洪灾中死亡 部分受灾户未完成清理}{https://www.zaobao.com/news/china/story20230809-1422186}{北京市政府星期三(8月9日)发布过去一个多星期的防汛救灾情况,至少33人在洪灾中遇难。 据《联合早报》记者现场观察,北京灾情最严重的门头沟区市区已大致恢复原貌,但部分村庄里较严重的受灾户仍未完成后续清理工作……}

\entryitemWithDescription{中国多省跟进医药反腐 已有上市药企高层被查}{https://www.zaobao.com/news/china/story20230809-1422180}{中国医药领域反腐行动不断扩大,截止星期二(8月8日),已有八个省份跟进集中整治,一些上市药企的高层人员也被报立案调查。 据澎湃新闻报道,江苏、上海、北京、海南四省的卫生监管部门已针对医药领域采取反腐行动,西藏自治区、陕西、山西、山东四省纪委部门也就整治医疗领域发声……}

\entryitemWithDescription{杨丹旭:不能说的中国经济}{https://www.zaobao.com/news/china/story20230809-1421909}{前阵子到浙江休假,朋友聚会逃不过一个话题:经济。 疫情虽然过去了,中国经济没有大幅反弹,下行的压力却只见加大,连经济发达的浙江都在发愁如何能让GDP(地区生产总值)数字更好看一点。 一名在二线城市开发区工作的朋友在饭桌上吐苦水,眼下最大的烦恼是招不到商,地方上有GDP压力,层层下放指标给各个部门,但招商部门却没有拿得出手的优惠政策……}

\entryitemWithDescription{麻生太郎在台湾演讲:要有战争觉悟 也要发挥吓阻反制能力避战}{https://www.zaobao.com/news/china/story20230808-1421901}{日本自民党副总裁、前首相麻生太郎8月8日在台湾外交部与远景基金会合办的``凯达格兰论坛-2023印太安全对话'' 演讲时说,台海和平稳定对日本和世界各国都很重要,为避免战争,必须具备吓阻实力、执行吓阻决心。(路透社) 日本自民党副总裁、前首相麻生太郎说,台海和平稳定对日本和世界各国都很重要,为避免战争,必须具备吓阻实力、执行吓阻决心,并让对手了解自身不惜动武维护台海稳定安全的意志……}

\entryitemWithDescription{美军太平洋前副司令:美中关系正危险地渐行渐远}{https://www.zaobao.com/news/china/story20230808-1421891}{美军太平洋司令部前副司令立夫(Daniel Leaf)星期二(8月8日)认为,美中关系目前正危险地渐行渐远,两国都不能承担双边关系走偏所带来的后果。 立夫星期二在重庆举办的美国史迪威将军诞辰140周年纪念活动上致辞时表示,无论喜欢与否,美中两国都离不开彼此,不论是两国政府还是人与人交流层面的关系,都将对21世纪有着深远影响,两国承受不了双边关系的失败……}

\entryitemWithDescription{蓝营县议长促郭柯合 不挺郭台铭独立参选}{https://www.zaobao.com/news/china/story20230808-1421890}{鸿海集团创办人郭台铭星期二(8月8日)在台北诚品信义店发布父亲节新书,次女郭晓如(左一)和长女郭晓玲(右一)到场献花和拥抱感谢,令他和妻子曾馨莹(左二)感动拭泪。(温伟中摄) 为挺鸿海集团创办人郭台铭而退党的彰化县议长谢典林星期一(8月8日)公开反对郭台铭独立参选,期待他与民众党参选人柯文哲合作以提高胜算。但民众党明确排除``郭柯配'',郭台铭则没正面回应,只强调团结主流民意,让执政党下台……}

\entryitemWithDescription{伦敦涂鸦墙出现中国社会主义核心价值观标语引争议}{https://www.zaobao.com/news/china/story20230808-1421878}{伦敦东区红砖巷成了中国政府支持者与反对者表达观点的画布。一组中国留学生上星期在红砖巷涂鸦,喷出中国``社会主义核心价值观''的标语后,一些反对者前去写上``愿荣光归香港'' 、``今日香港 明日台湾''等字样。图摄于8月7日……}

\entryitemWithDescription{北京试探小马可斯的中美平衡政策 连续两天就仁爱礁问题表态}{https://www.zaobao.com/news/china/story20230808-1421867}{1999年在仁爱礁(菲律宾称阿云津礁)搁浅坐滩的军舰,再次引发中菲交锋。(法新社摄2023年4月23日) 中国外交部连续两天就仁爱礁(菲律宾称阿云津礁)问题表态,敦促菲律宾立即拖走坐滩军舰;菲律宾则重申,绝不放弃在阿云津礁的驻岗。 受访学者分析,中国正在试探小马可斯的中美平衡政策。 中国海警船星期六(8月5日)在具争议的南中国海一带对菲律宾船只发射水炮……}

\entryitemWithDescription{台商父子被控为中国大陆收集军事演习情报}{https://www.zaobao.com/news/china/story20230808-1421856}{一对台商父子涉嫌利诱两名台湾现役军人,协助收集机密军事资料给中国大陆,在台南被提起公诉。 据《联合报》报道,台湾高等检察署台南分署星期一(8月7日)宣布侦结起诉台商父子涉国安案件。起诉书称,这对黄姓父子2015年移居福建经商后,结识了大陆情报机构官员。受大陆官员利诱,黄姓父子吸收时任台湾空军防空与导弹部门的叶姓与苏姓两名军官……}

\entryitemWithDescription{中国拟规范人脸识别技术应用 使用前应取得个人同意}{https://www.zaobao.com/news/china/story20230808-1421815}{中国拟规范人脸识别技术应用,除法律与行政法规规定不需取得个人同意的场景,使用人脸识别技术处理人脸信息时,应当取得个人的单独同意,或者依法取得书面同意。 中国网信办星期二(8月8日)通过官网发布公告,向社会就人脸识别技术应用管理规定征求意见,规范人脸识别技术应用……}

\entryitemWithDescription{【东谈西论】一次中国村超的亲体验}{https://www.zaobao.com/news/china/story20230808-1421760}{在中国榕江举行的``村超''迅速在网络爆红,比赛声势浩大,线上直播的观众超过5000万人次。(互联网) 你应该知道什么是英格兰超级足球联赛。但是你可曾听过``村超''吗? 村超是中国农村的足球联赛,全名是中国贵州省榕江和美乡村足球联赛。村超一共有20支乡村足球队参加,球员全是业余的。比赛中场会表演贵州少数民族歌舞,奖品也非常接地气,有活鸡、鸭、鹅、猪、牛、羊等……}

\entryitemWithDescription{台湾碳权交易所揭牌加速净零转型目标 但八成中小企业有``碳焦虑''}{https://www.zaobao.com/news/china/story20230807-1421545}{台湾碳权交易所星期一(8月7日)正式揭牌,总统蔡英文在典礼上表示,碳权交易所能让台湾跟上国际净零转型潮流。不过有调查显示,台湾高达八成中小企业因减碳压力而焦虑。 为因应2050年达成净零排放目标,台湾行政院国家发展基金和证券交易所共同出资设立碳交所。考量到高雄市列管高碳排的石化和钢铁厂等高达90家,碳排放占台湾总排放的20\%,故台湾政府将碳交所总部设在高雄,资讯交易中心设在台北,分工营运……}

\entryitemWithDescription{中国洪灾引发舆情 张国清赴黑龙江指导抗洪工作}{https://www.zaobao.com/news/china/story20230807-1421544}{洪灾导致黑龙江部分地区断电并与外界失联,图为8月6日居民在海林市长汀镇一广场上的应急通信车前与外界通话。 (新华社) 中国洪灾暴露地方治理问题,在河北居民疏散迟缓、外地捐赠物资遭拒等问题引发舆情后,中国国务院副总理张国清周末赴天津和黑龙江指导抗洪工作,强调要最大程度减少人员伤亡和洪涝灾害损失。 受访学者分析,京津冀和东北的严重灾情凸显地方政府预备不足,灾后巨额的赔偿问题也将为地方财政带来巨大压力……}

\entryitemWithDescription{港警退休副处长刘赐蕙 传将出任香港国安委新职}{https://www.zaobao.com/news/china/story20230807-1421531}{刘赐蕙是香港警务处首位负责国家安全处的警队副处长,警务生涯长达38年。(互联网) 已退休的香港警务处前副处长刘赐蕙,据报将加盟香港特区维护国家安全委员会(国安委)出任新职位。 据《星岛日报》星期一(8月7日)引述消息称,继香港警务处前高级助理处长林晓彤加入国安委担任副秘书长后,今年4月退休的刘赐蕙也将加盟国安委……}