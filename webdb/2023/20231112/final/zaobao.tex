\entryitemWithDescription{回归以来首次没有民主派参与(引) 香港新选制区选投票率不受看好}{https://www.zaobao.com/news/china/story20231112-1449085}{港府近来积极在地区宣传,呼吁市民在下月举行的区议会选举投票。图为志愿者向市民派发区选宣传单张。(中通社) 香港新一届区议会选举将于12月10日举行,这也是香港改革选举制度后的首届区选。伴随着街上随风飘扬的宣传标语,再加上义工和官员深入社区呼吁投票的声音,选举气氛近日愈来愈热烈。但和以往相比,新一届区议会在组成、职能等方面有了很大的不同,未来走向备受外界关注……}

\entryitemWithDescription{ONE冠军赛预计明年回归中国}{https://www.zaobao.com/news/china/story20231111-1449478}{因冠病疫情在中国停办三年的ONE冠军赛,预计明年上半年恢复举办。 ONE冠军赛中国区总裁李颖接受《联合早报》访问时披露,这家体育赛事公司今年来已在美国和泰国等地恢复办赛,希望明年能把综合格斗赛事带回中国。 除了上海、广州和长沙等原有市场,ONE冠军赛也计划开拓山东市场。集团星期六(11月11日)在新加坡---山东经贸理事会会议上,分别和山东省体育局、山东省体育产业集团签订战略合作协议……}

\entryitemWithDescription{新加坡与山东加强在绿色经济、现代服务等领域合作}{https://www.zaobao.com/news/china/story20231111-1449470}{新加坡代交通部长兼财政部高级政务部长徐芳达(右)和山东省长周乃翔星期六(11月11日)在新加坡---山东经贸理事会会议上,共同为山东(新加坡)企业服务中心揭牌。(新加坡企发局提供) 新加坡和中国山东省将加强在绿色经济、现代服务、贸易和互联互通等领域合作,进一步推动双向投资与贸易往来,为双边更广更深的合作铺路……}

\entryitemWithDescription{消费信心低迷 中国双11购物盛会热不起来}{https://www.zaobao.com/news/china/story20231111-1449467}{中国星期六(11月11日)迎来年度最大购物盛会``双11'',但今年的买气不如往年,民众在消费时更为谨慎。图为中国快递公司顺丰的工作人员为双11的包裹进行分拣和打包。(新华社) 在经济复苏乏力、消费情绪低迷下,中国今年``双11''购物节似乎热不起来,民众在消费时更为谨慎,更注重商品的实际价值。有公众也认为,各大平台的销售套路变得越来越复杂,让人产生``消费疲劳''……}

\entryitemWithDescription{台湾将与捷克合作助力乌克兰重建}{https://www.zaobao.com/news/china/story20231111-1449450}{(台北综合讯)台湾和捷克将合作协助乌克兰重建被战争摧毁的关键基础设施。 根据台湾外交部网站消息,台湾驻捷克代表柯良叡与捷克驻台代表史坦格星期五(11月10日)以视讯方式签署一项谅解备忘录,双方将合作协助乌克兰重建净水处理设施及气电共生能源系统……}

\entryitemWithDescription{中国官媒:华盛顿与英伟达的猫鼠游戏不应持续}{https://www.zaobao.com/news/china/story20231111-1449427}{中国官媒《环球时报》形容,美国政府的对华科技出口管制演变成一场和英伟达等公司的``猫鼠游戏''。(路透社档案照) (北京综合讯)中国官媒《环球时报》形容,美国政府的对华科技出口管制演变成一场和英伟达等公司的``猫鼠游戏'',这不应持续下去,因为不符合中美利益,也只会加速中国的自主科技进程……}

\entryitemWithDescription{新闻人间:社会泛国安化 叶刘淑仪躺着也中箭}{https://www.zaobao.com/news/china/story20231111-1449295}{新闻人间:叶刘淑仪 香港上星期举行同乐运动会(Gay Games),成为首个举办该项性小众界别盛事的亚洲城市。本来这是香港对外展现自己是多元共融国际大都会的机会,岂料社会上竟然掀起一股反对的声音,甚至连行政会议召集人叶刘淑仪躺着也``中箭''。 事缘早前有一批亲建制派的香港市民在立法会议员何君尧等人的支持下,到立法会递交请愿信,反对香港举办同乐运动会……}

\entryitemWithDescription{庄慧良:台湾大选后可能出现的宪政危机}{https://www.zaobao.com/news/china/story20231111-1449314}{11月2日,候友宜(左2)、郭台铭(右2)和柯文哲(右1)齐现身台北寺庙活动。(AFP) 台湾前总统马英九和国民党最有人气的高雄前市长韩国瑜,星期五(11月10日)相继呼吁国民党(蓝)和民众党(白)以``全民调''方式决定2024年总统候选人组合,让濒临破局的``蓝白合''再燃生机。 在野 ``蓝白合''商议多时,一直卡在正副总统产生方式未获共识……}

\entryitemWithDescription{马英九建议以全民调整合 化解在野蓝白合僵局}{https://www.zaobao.com/news/china/story20231110-1449318}{台湾前总统马英九建议,以全民调化解在野``蓝白合''僵局,高雄前市长韩国瑜也呼应赞成,为在野力量合力实现政党轮替重燃一线希望。但蓝白阵营都已做好各自作战的最坏准备。 台湾2024年1月13日举行总统与立委选举,候选人必须在11月24日前完成登记。在倒数两周内,国民党(蓝)总统参选人侯友宜与民众党(白)参选人柯文哲,必须决定是否合组正副搭档……}

\entryitemWithDescription{受贿近10亿人民币 中信银行前行长孙德顺被判死缓}{https://www.zaobao.com/news/china/story20231110-1449312}{中信银行原行长孙德顺2022年2月22日在山东济南中级法院受审,承认受贿9.08亿元人民币。(互联网) (济南综合讯)中信银行原行长孙德顺因受贿罪被判处死刑,缓期二年执行,剥夺政治权利终身,并处没收个人全部财产。 央视新闻报道,山东省济南市中级法院星期五(11月10日)对孙德顺案进行一审宣判……}

\entryitemWithDescription{天主教领袖联署促释黎智英 港府强烈不满批颠倒是非}{https://www.zaobao.com/news/china/story20231110-1449288}{(香港综合讯)全球10位天主教领袖联名,要求香港特区政府释放壹传媒创始人黎智英,港府对此表示坚决反对和强烈不满。 英国道迪街律师事务所于11月1日发布联署信,10名来自英美等国的天主教领袖联署称,75岁的黎智英坚持信仰,挑战专制与压迫而失去事业,并与家人失去联系。他已在囚超过1000天,可能面临更长的刑期,促请港府立即无条件释放黎智英……}

\entryitemWithDescription{蔡英文:望向APEC峰会传达台湾致力于促进地区和平与繁荣}{https://www.zaobao.com/news/china/story20231110-1449261}{台湾总统蔡英文(左)星期五(11月10日)在台北举行的记者会上与受委代表台湾出席APEC峰会的台积电创办人张忠谋合影。(路透社) (台北综合讯)正当亚太区域面临地缘政治不确定性以及全球供应链重组的变局之际,台湾将在下周的亚太经合组织(APEC)峰会上强调地区和平的重要……}

\entryitemWithDescription{香港养猪场验出非洲猪瘟 5600多头猪紧急销毁}{https://www.zaobao.com/news/china/story20231110-1449210}{(香港综合讯)香港一家猪场接连验出非洲猪瘟病毒,特区政府决定销毁全场5600多头猪。 香港渔农自然护理署(渔护署)官网星期二(11月7日)通报,11月6日在元朗流浮山一家持牌养猪场对32头猪进行样本检测,发现16头猪对非洲猪瘟病毒呈阳性。署方通报翌日再到该猪场抽检37头猪,又发现六头猪感染非洲猪瘟病毒……}

\entryitemWithDescription{陈婧:进博会重回疫情前盛况?}{https://www.zaobao.com/news/china/story20231110-1449025}{在今年进博会开幕当天询问首次来华的新加坡商家,参展过程中遇到的最大挑战是什么。对方想了想:``都挺顺利的\ldots\ldots 就是今天交通管制,走了好久才进场。'' 面对同一个问题,去年参展的企业给出惊险百倍的回答:由于在上海入住的隔离酒店工作人员感染冠病,公司全体代表被迫接受二次隔离,差点赶不上参展。 别说是从国外进来的参展商,就连常驻上海的我本人,去年都险些因为严苛的防疫规定而没能进场……}

\entryitemWithDescription{民进党政府拟将陆生纳入健保 在野党:为了选票}{https://www.zaobao.com/news/china/story20231109-1449074}{台湾民进党总统参选人赖清德8日主张将大陆学生纳入健保。国民党和民众党认为,民进党在反对陆生纳入健保12年后突然急转弯,是为了大选考量。(法新社) 台湾执政民进党总统参选人赖清德星期三(11月8日)在中常会主张将中国大陆学生比照外国学生纳入健康保险。在野国民党和民众党都认为,民进党在反对陆生纳入健保12年后突然政策急转弯,是为了2024年总统大选的政治考量……}

\entryitemWithDescription{巨资购地变烂泥 陆家嘴索赔百亿人民币}{https://www.zaobao.com/news/china/story20231109-1449073}{(苏州/上海综合讯)上海浦东国资委旗下上市企业陆家嘴因买到严重污染土地,向政府机构和国有企业索赔逾百亿元。 综合金融界、红星新闻等报道,上海陆家嘴金融贸易区开发股份有限公司11月4日公告,将江苏苏钢集团有限公司、苏州市环境科学研究所等五方列为被告,请求判令赔偿约100亿4400万元(人民币,下同,18亿7000万新元),并称如后续损失超出此数,将追加诉求……}

\entryitemWithDescription{中国官方密集发布中美积极互动信息 或为元首旧金山会晤造势}{https://www.zaobao.com/news/china/story20231109-1449072}{学者认为,中国密集发布积极信息旨在为会见创造氛围并塑造舆论,改善国内部分敌视美国的舆情,并提振民间对中美有能力进一步改善双边关系的信心。(路透社档案图) 中美元首据报11月15日将在旧金山举行时隔一年的面对面会谈。中国官方尚未证实消息,但星期四(11月9日)密集发布中美积极互动信息,包括两国星期三(11月8日)在加州的气候变化会谈达成``积极成果'',以及中美每周直航客运航班将增加近五成……}

\entryitemWithDescription{美网络软件巨头思杰母公司撤离中国}{https://www.zaobao.com/news/china/story20231109-1449052}{(北京综合讯)美国网络软件巨头思杰(Citrix)母公司云计算软件集团(Cloud Software Group)宣布将退出中国市场,成为最新一家撤出中国的美国公司。 《华尔街日报》星期四(11月9日)报道,该报看到一封思杰母企云计算软件集团星期一(11月6日)发给客户和合作伙伴的电邮,显示该公司已决定于12月3日在中国大陆和香港``停止所有新商业交易'',其理由是``商业运营成本上升''……}

\entryitemWithDescription{新中合作标杆项目``知识塔''封顶}{https://www.zaobao.com/news/china/story20231109-1449049}{302米的中新广州知识城地标建筑''知识塔''主体结构封顶。(中新广州知识城合资公司提供) (广州讯)中新广州知识城地标建筑``知识塔''主体结构星期三(11月8日)举行封顶仪式,标志着新中两国这一合作标杆项目取得阶段性进展。 新加坡驻华大使陈海泉、新加坡驻广州总领事罗德杰、新加坡企业发展局中国司司长胡丽燕、广州市黄埔区委副书记陈智勇等人出席了当日的活动……}

\entryitemWithDescription{招商银行原行长田惠宇一审被控受贿逾2.1亿人民币}{https://www.zaobao.com/news/china/story20231109-1449044}{中国招商银行原行长田惠宇一审被控受贿逾2.1亿元人民币,他当庭认罪,案件择日下判。(央视新闻) (长沙综合讯)中国招商银行原行长田惠宇一审被控受贿逾2.1亿元(人民币,下同,3920万新元),他当庭认罪,案件择日下判。 据新华社报道,湖南省常德市中级法院星期四(11月9日)一审公开审理招商银行原党委书记、行长田惠宇受贿、国有公司人员滥用职权、利用未公开信息交易、内幕交易、泄露内幕信息一案……}

\entryitemWithDescription{学者:选举结果不会改变大陆对台威慑}{https://www.zaobao.com/news/china/story20231109-1449000}{分析认为,在中美战略竞争下,不论明年台湾是否政党轮替,北京对台军事威慑,不太可能再退回到台海中线以西。图为过去曾是台湾军事要塞的马祖南竿26据点。(法新社) 台湾总统大选进入倒数两个月,``战争与和平''无可避免成为选战攻防话题。受访学者分析,在中美战略竞争下,不论明年台湾是否政党轮替,北京对台军事威慑,不太可能再退回到台海中线以西;民进党若长期执政,大陆恐将``丢掉幻想,准备斗争''……}

\entryitemWithDescription{台国防部:大陆解放军航母山东舰沿台海中线航行}{https://www.zaobao.com/news/china/story20231109-1448983}{台湾国防部9月13日发布侦测到中国大陆山东舰的图片。山东号航舰编队自11月8日下午起,沿台湾海峡中线以西由南向北航行。(法新社) (台北综合讯)台湾国防部说,中国大陆解放军山东号航舰编队自11月8日起,沿台湾海峡中线航行。 台湾国防部星期四(11月9日)发布新闻稿说,山东号航舰编队自11月8日下午起,沿台湾海峡中线以西由南向北航行。该编队在11月9日早晨8时左右通过北部海域并继续北行……}