\entryitemWithDescription{中国``文管''登场了?}{https://www.zaobao.com/news/china/story20230609-1402562}{继城管和农管之后,中国''文管''大队登场的话题,本星期引起了境外社交媒体关注。 ``文管''一词横空出世,导源于今年5月30日黑龙江省文化旅游厅举行的一场``文化市场综合执法队伍统一着装仪式暨规范化建设现场观摩会''。 根据主办方官网发布的消息与配图,黑龙江省各级文化市场执法队员代表共500余人参加了活动。代表们穿着整齐的制服,犹如纪律部队般出席了统一着装仪式……}

\entryitemWithDescription{侯友宜选总统民调低迷 保台论述和掌握民怨或是反弹关键}{https://www.zaobao.com/news/china/story20230608-1402564}{美国在台协会(AIT)主席罗森伯格(左一)两个月内二度访台,6月7日与侯友宜(右一)和新北市政府团队密谈一小时后,搭火车到平溪放天灯祈福,并聊到深夜。(新北市政府提供) 台湾在野的国民党总统候选人侯友宜民调低迷,半年内从领先到两度垫底,引发弃保担忧。但受访民调专家研判,美方本周专程派高官来台面谈,说明侯友宜仍具备实现政党轮替的政治能量,若能掌握民怨所在,说清楚保台论述和未来愿景,民调有望触底反弹……}

\entryitemWithDescription{中国将在古巴设立针对美国的秘密监听基地}{https://www.zaobao.com/news/china/story20230608-1402559}{民众在古巴首都哈瓦那街头的摊贩排队买食物。(法新社) 美国官员称,中国和古巴达成秘密协议,中国将在古巴建立电子窃听设施,对美国构成新的地缘政治挑战。 《华尔街日报》引述熟悉美国高度机密情报的官员说,设在古巴的电子窃听设施,距离佛罗里达州约160公里,这让中国情报部门能够窃听美国东南部的电子通信,以及监视美国的船只。美国东南部是许多军事基地的所在地……}

\entryitemWithDescription{中国国企领导与女子牵手逛街后被免职}{https://www.zaobao.com/news/china/story20230608-1402537}{中国一名国企高管在四川成都出差期间,被拍到与一名年轻女子亲密牵手逛街。视频本周在网络流传后,涉事两人均被停职调查。 一则街拍视频星期三(6月7日)短视频平台抖音上被疯转。视频显示,一名身穿玫粉色上衣的中年男子与一名衣着靓丽的年轻女子牵手说笑、漫步在成都闹市区街道……}

\entryitemWithDescription{谢锋:用``去风险''为``脱钩''掩护 给中美关系埋下更多钉子}{https://www.zaobao.com/news/china/story20230608-1402527}{中国驻美大使谢锋6月7日出席美中贸易全国委员会为他履新举行的欢迎活动时说,如果用``去风险''为``脱钩''打掩护,就会给中美关系埋下更多钉子。(中国驻美大使馆官网) 中国呼吁美国在高层交往中做好全过程管理,不能言行不一致,强调如果用``去风险''为``脱钩''打掩护,就会给中美关系埋下更多钉子……}

\entryitemWithDescription{大陆军机37架次现台周边空域 台军启动防御系统}{https://www.zaobao.com/news/china/story20230608-1402516}{台湾国防部星期四(6月8日)侦查到37架次中国大陆的军用飞机,飞入台湾防空识别区西南空域。台军随即启动防御系统,两军并没有发生冲突。 根据台湾国防部星期四发布的新闻稿,自当天早上五时起,侦查到中国大陆军机的军机,包括歼11、歼16、轰6、运油20及预警等各型军机,进入台湾的防空识别区。部分大陆编队随后飞入西太平洋,进行空中监视和远程航行训练……}

\entryitemWithDescription{美议员反对邀李家超出席APEC 港学者:特首获邀也不一定赴会}{https://www.zaobao.com/news/china/story20230608-1402510}{香港近年与美国关系紧张。美国有跨党派国会议员星期三(7日)致函国务卿布林肯,反对美国有意邀请被美方制裁的香港行政长官李家超,参加今年年底在美国举行的亚太经合组织(APEC)会议。 有受访学者认为,美国总统拜登正寻求连任,未必愿意在该议题上惹恼参众两院议员。但纵使美国邀请李家超出席,李家超也不一定赴会……}

\entryitemWithDescription{英政府部门将拆除中国制造监控设备 北京指责伦敦歧视打压中企}{https://www.zaobao.com/news/china/story20230608-1402505}{英国内阁公署要求各政府部门从敏感政府机关地点拆除中国制造的监控设备。中国驻英国大使馆指这是``歧视和打压中国企业''。 综合路透社和《金融时报》报道,英国内阁公署当地时间星期二(6月6日)在一份收紧采购条规的公告中,承诺将公布一份时间表,拆除受中国《国家情报法》约束的公司所生产的监控设备……}

\entryitemWithDescription{台湾性骚扰风波持续延烧 文坛与教育界两位重量级人物被点名}{https://www.zaobao.com/news/china/story20230608-1402464}{台湾``MeToo''风暴从政坛烧向文坛,著名现代诗人郑愁予、知名作家陈芳明,被曝曾经分别性骚扰女学生和女助理。 台湾媒体《放言》星期三(7日)报道,以一首《错误》享誉诗坛,诗歌入选海峡两岸教材的老牌诗人郑愁予,近日被指在东华大学任教时,曾对女学生有类似毛手毛脚的行为。 当事人在脸书发长文《踩在受害者的位置上,踩好踩满》,指控郑愁予在2005年担任东华大学客座教授期间性骚扰学生……}

\entryitemWithDescription{陈婧:落空的黄仁勋``登陆''行}{https://www.zaobao.com/news/china/story20230608-1402205}{人工智能芯片巨头英伟达(Nvidia)创办人黄仁勋已从台湾返回美国的消息在星期一(6月5日)得到证实,让坊间对他本周访问中国大陆的高涨期待落了空。 此前一周,关于黄仁勋``登陆''的消息已传得沸沸扬扬,连具体行程都有鼻子有眼。据说,黄仁勋会到访腾讯、字节跳动等科技公司,理想和比亚迪等电动车企,以及正进军电动车领域的小米公司……}

\entryitemWithDescription{高尔夫球场部分用地``未决定用途'' 港府新界兴建公屋计划或有变}{https://www.zaobao.com/news/china/story20230607-1402213}{香港土地长期供不应求,港府早前表示将收回有逾百年历史的粉岭高尔夫球场来兴建公营房屋。但当局近日向城市规划委员会提交的文件,却将部分用地暂时修订为``未决定用途''地带,让人察觉到建屋计划出现变数。 受访学者认为,目前香港房屋问题严峻,当局应学习新加坡政府果断收回克兰芝赛马场兴建房屋的魄力,尽快收回粉岭高尔夫球场高场并立即建公屋……}

\entryitemWithDescription{中美未否认布林肯将访华 分析:若成行意味气球事件翻篇}{https://www.zaobao.com/news/china/story20230607-1402190}{美国国务卿布林肯传出将在未来几周重启访华,白宫和中国外交部之后相继表态,虽未证实也没有否认。(路透社档案照) 美国国务卿布林肯星期二(6月6日)传出将在未来几周重启访华,白宫和中国外交部之后相继表态,虽未证实也没有否认。受访中国学者评估,布林肯今年2月因气球事件推迟访华,重启访华将意味着气球事件完全翻篇,也表明中美有意继续加强沟通对话为两国紧张降温……}

\entryitemWithDescription{英国指中国关闭秘密警察站 北京敦促英国停止诋毁中国}{https://www.zaobao.com/news/china/story20230607-1402186}{英国安全部部长图根哈特(Tom Tugendhat)称,中国已经关闭在英国各地的``警察服务站''。中国外交部加以否认,并敦促英国停止诋毁中国。 综合法新社、路透社等报道,非政府人权组织``保护卫士''(Safeguard Defenders)去年9月发表报告,指中国在英国境内三个地点设有警察站,旨在提供行政服务,但也被用来``监视和骚扰侨民社区,在某些情况下,强迫人们在合法渠道之外返回中国……}

\entryitemWithDescription{中俄联合空中战略巡航 日韩空军出动战机应对}{https://www.zaobao.com/news/china/story20230607-1402182}{中国和俄罗斯星期三(6月7日)完成第六次联合空中战略巡航第二阶段任务,引起了日本和韩国的担忧。 中国国防部旗下的微信公众号``国防部发布'',在星期三下午宣布上述消息。 这场联合空中巡航从星期二(6日)开始,中俄两国军机在日本海和东海进行联合空中战略巡航,引起了日本和韩国的担忧……}

\entryitemWithDescription{地缘政治压力下 红杉资本分拆中国业务 中美资本合作标杆成历史}{https://www.zaobao.com/news/china/story20230607-1402178}{红杉资本全球总部位于美国加州硅谷地区门洛帕克(Menlo Park)市。该公司星期二(6月6日)宣布将其在美洲和欧洲、印度和东南亚,以及中国的分部,分拆为完全独立的实体。(彭博社) 声名显赫的美国风险投资公司红杉资本,宣布将中国和美国的业务完全分拆。专家认为,红杉``一分为三''的决定,凸显出夹在中美地缘政治紧张下,资本如果不自我厘清身份,在两国的业务都将受限……}

\entryitemWithDescription{被指助伊朗获导弹技术 陆港企业被美国制裁}{https://www.zaobao.com/news/china/story20230607-1402174}{美国财政部对中国大陆、香港和伊朗的六家实体和七名个人实施制裁,理由是他们涉嫌帮助伊朗获得研制弹道导弹的零件与技术。 综合路透社、法新社、《华尔街日报》报道,伊朗在星期二(6日)早些时候展示了官方宣称的第一枚国产高超音速弹道导弹,可能会加剧西方国家对德黑兰导弹能力的担忧。 被美国制裁的人当中,包括伊朗驻北京的国防武官达沃德·达姆加尼 (Davoud Damghani)……}

\entryitemWithDescription{中国AI诈骗频发 相关法律保护仍处于真空地带}{https://www.zaobao.com/news/china/story20230607-1402170}{诈骗分子用AI进行人工``变脸'',让人防不胜防。有中国网民在网上演示变脸,警惕大家``网聊有风险聊天需谨慎''。(互联网) 有中国人上个月底因人工智能(AI)``换脸变声''技术被骗走430万元(人民币,约81万新元),使AI技术的安全隐患引起关注,也反映了相关法律保护仍处于``真空地带''……}

\entryitemWithDescription{台湾政坛频传性骚事件 恐冲击民进党明年选情}{https://www.zaobao.com/news/china/story20230607-1402147}{小英之友总会长颜志发因被指涉及性骚扰,向台湾总统府请辞资政。(自由时报) 台湾政坛近来频传性骚扰事件,恐怕会冲击执政的民进党在明年总统大选和立委选举的选情。在总统府资政颜志发被民进党前党工指控性骚扰而请辞后,台湾总统、民进党前主席蔡英文第二度向社会道歉……}

\entryitemWithDescription{王纬温:中美新共同威胁——人工智能}{https://www.zaobao.com/news/china/story20230607-1401864}{中美今年2月因气球事件紧张升温,之后数月两国关系走向波折反复,中西方舆论也很少再提双方可能的合作。但面对人工智能(AI)发展方兴未艾,可能升高犯罪和军事风险,对太平洋东西两岸都造成巨大威胁,是否共同应对因此逐渐成为北京和华盛顿绕不过的问题。 未来数年里,人类生存预计将主要受到三大威胁笼罩,除了人们已熟悉的气候变化及核武扩散,后果最难预测的当属后疫情时期开始走入家家户户的AI……}

\entryitemWithDescription{中俄第六次联合空中战略巡逻 八架军机一度进入韩国防识区}{https://www.zaobao.com/news/china/story20230606-1401867}{中国和俄罗斯星期二(6月6日)在日本海和东海进行联合空中战略巡航,韩国称已部署战斗机来应对领空附近的中俄战机。 韩国联合参谋本部(韩联参)称,四架中国军机和四架俄罗斯军机,在6日午餐时间左右,进入韩国南部和东部海域的防空识别区,韩国紧急调动战斗机应对。 根据韩联社报道,中俄军机进入防空识别区后,韩国军方发出警告,中国军机通过军用直通热线回答正在进行正常演习,俄军没有特别的回应……}