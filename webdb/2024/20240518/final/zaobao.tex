\entryitemWithDescription{新闻人间:北京管得住台湾名嘴的嘴?}{https://www.zaobao.com/news/china/story20240518-3677809}{新闻人间:北京管得住台湾名嘴的嘴?(联合早报制图) 表演能力不输专业演员的台湾名嘴摊上大事了。 中国大陆国台办发言人陈斌华5月15日在例行新闻发布会上回应媒体提问时,点名了北京要惩戒的五个台湾名嘴和其家属。罪名是``罔顾大陆发展进步的事实,蓄意编造有关大陆的虚假、负面信息'',大肆传播错误言论,蒙蔽部分台湾民众,挑动两岸敌意对立。 黑名单五人是:黄世聪、李正皓、王义川、于北辰、刘宝杰……}

\entryitemWithDescription{庄慧良:赖清德的瓶中信}{https://www.zaobao.com/news/china/story20240518-3682238}{台湾候任总统赖清德即将在下星期一(5月20日)正式就任;在``登基''大典前,他透过官方LINE推出``写一封瓶中信''活动,邀请民众写下对台湾未来的期许,希望 大家``写瓶中信,和赖清德、萧美琴一起共织台湾,民主前行''……}

\entryitemWithDescription{学者:中美防长若月底都在新加坡 极可能在香会正式对话}{https://www.zaobao.com/news/china/story20240517-3681824}{中国防长董军与美国防长奥斯汀据传本月底将在新加坡举行的香格里拉对话会见面;如若属实,这将是中美防长首次面对面接触。受访学者分析,中美目前已形成政治共识,不因有争议冲突而中断官方对话沟通。中美防长如果月底都在新加坡,很有可能举行正式对话。 奥斯汀与去年底出任防长的董军,上月中已举行视频会议并首次通话。美国众议院前议长佩洛西前年8月访台,中美防长正式对话因此中断长达近一年半……}

\entryitemWithDescription{中国买家据报对北极最后一块战略土地表兴趣}{https://www.zaobao.com/news/china/story20240517-3681699}{斯瓦尔巴最后一幅私有土地待售,拥有该地主权的挪威不希望地皮落入中国人之手。(联合早报制图) (奥斯陆综合电)中国买家据报有意购取北极群岛最后一幅私有战略土地,但拥有该地主权的挪威并不希望地皮落入外国之手。 综合彭博社和法新社报道,这块土地面积达60平方公里,大小跟曼哈顿差不多,离人烟稀少的朗伊尔城约64公里,有5公里长的海岸线,遍地高山冰川,是北极野生动物的家园,但没有基础设施……}

\entryitemWithDescription{指涉新疆强迫劳动 美国禁止26家中企进口产品}{https://www.zaobao.com/news/china/story20240517-3681943}{此照片摄于2021年4月1日,显示新疆维吾尔自治区一家棉质布料厂的工人在生产线上工作。(路透社档案照) (伦敦/巴黎综合讯)美国星期四(5月16日)宣布将26家中国企业纳入禁止进口名单,理由是这些企业生产产品的工厂涉嫌强迫新疆维吾尔人劳动。 综合路透社与法新社报道,26家中国公司被列入《防止维吾尔人强迫劳动法》实体清单,自星期五(17日)开始生效……}

\entryitemWithDescription{台湾立法院爆发肢体冲突 女立委遭扑倒抱腰袭臀 男立委被掌掴}{https://www.zaobao.com/news/china/story20240517-3678667}{民进党立委锺佳滨(右二)爬上立法院主席台,把阻挡他的国民党女立委陈菁徽扑倒在地,令旁观的立法院长韩国瑜(左二)大吃一惊。(法新社) 民进党立委郭国文(左)被指掌掴民众党团总召黄国昌。(取自Instagram) 台湾立法院星期五(5月17日)爆发一系列肢体冲突,国民党女立委陈菁徽遭民进党男立委锺佳滨扑倒抱腰袭臀,民众党团总召黄国昌遭民进党立委郭国文掌掴……}

\entryitemWithDescription{中国国安部:近年侦破多起航天领域间谍案}{https://www.zaobao.com/news/china/story20240517-3681782}{(北京/武汉综合讯)中国国家安全部称,太空领域的安全挑战日益严峻,中国近年侦破了多起航天领域间谍案,涉及以利诱、胁迫等手段窃取中国航天领域的最新研究进展。 中国国安部微信公众号星期五(5月17日)发文说,随着中国空间技术的发展,某些国家将中国视为太空领域主要竞争对手,``进行不遗余力的遏制打压''……}

\entryitemWithDescription{美媒:中国第四艘航母可能是全球首艘专用无人机航母}{https://www.zaobao.com/news/china/story20240517-3680994}{(华盛顿综合讯)中国第四艘航空母舰在建情况备受外界关注,美媒根据卫星图片分析称,这很可能是一艘专为操作无人机而打造的航母,将创下全球首例。 美国``海军新闻''网站星期三(5月15日)报道称,江苏大洋造船有限公司正在打造中国第四艘航母,这艘舰艇曾在2022年12月下水,但直到现在仍披着神秘面纱。 中国目前已有三艘航母,包括正在服役的辽宁舰和山东舰,以及本月初刚刚完成八天海试的福建舰……}

\entryitemWithDescription{韩咏红:中俄复杂关系下普京访华找活路}{https://www.zaobao.com/news/china/story20240517-3676473}{宣誓就职不到10天,俄罗斯总统普京就踏上了访问中国的旅程。这是普京第19次以总统身份访华,也是他开启新一个六年总统任期的外交首访,目的是在中俄建交75周年之际,深化两国``新时代全面战略协作伙伴关系''。 除了外长拉夫罗夫,普京此行还带了共六位副总理,以及财长、央行行长、银行与石油公司CEO、铝业大亨等……}

\entryitemWithDescription{中国预计周五宣布重磅楼市政策 乐观预期带动地产股大涨}{https://www.zaobao.com/news/china/story20240516-3676193}{中国多地近期加码推出宽松楼市政策,寄望加速消化存量住房,提振持续低迷的房地产市场。图为广西南宁的住宅楼。(彭博社) 市场对中国政府将收购存量住房的预期持续升温,官方预计最早在星期五(5月17日)宣布相关政策。受乐观情绪提振,陆港房地产股星期四全面大涨……}

\entryitemWithDescription{蓝绿专家:赖政府若展现善意 或可打开两岸新局}{https://www.zaobao.com/news/china/story20240516-3675297}{陆委会前主委夏立言(左)与海基会前董事长洪奇昌,星期四(5月16日)在台北出席研讨会,点评新政府与两岸关系新局面。(温伟中摄) 台湾陆委会前主委夏立言、海基会前董事长洪奇昌,星期四在研讨会乐观指出,只要候任总统赖清德对中国大陆展现善意,新政府或可打开两岸关系新局面。 夏立言和洪奇昌星期四(5月16日)在参加余纪忠文教基金会的``新政府与两岸关系新局面?''跨党派研讨会时,提出以上看法……}

\entryitemWithDescription{是否特赦陈水扁?台湾总统府:无此决定}{https://www.zaobao.com/news/china/story20240516-3676224}{台湾前总统陈水扁因贪污洗钱等案,2008年11月被羁押,2015年获准保外就医。(互联网) (台北综合讯)台湾政坛近期传出总统蔡英文将在5月20日卸任前特赦前总统陈水扁,台湾总统府星期四表示,实际上相关程序都必须依法进行,也无此决定。 综和《联合报》和《自由时报》报道,台湾总统府对于特赦陈水扁的议题,近期不断重申现阶段的立场,仍是希望确保陈水扁获得妥善的健康照顾,并依照相关的法律规范办理……}

\entryitemWithDescription{湖北官员为政策达标 私自登记村民为个体工商户}{https://www.zaobao.com/news/china/story20240516-3676002}{(北京综合讯)中国湖北省大悟县刘院村多位村民发现,在不知情情况下被注册工商营业执照,成为个体工商户,导致无法申请社会救助。据中国媒体报道,当地官员为达成政绩要求,私自替村民登记营业执照。 《中国青年报》旗下《冰点周刊》星期三(5月15日)报道,2011至2023年,在``市场主体增量行动''中,陆续有134名刘院村村民的个人信息被用于违规办理工商营业执照……}

\entryitemWithDescription{旅日中国教授据报涉间谍罪 被判六年有期徒刑}{https://www.zaobao.com/news/china/story20240516-3675263}{(北京综合讯)据日本媒体报道,旅日中国籍教授袁克勤暂时回国为母奔丧期间,因涉嫌间谍罪被拘留,已被判处六年有期徒刑。 袁克勤此前供职于日本北海道教育大学,2019年5月暂返中国后在吉林省长春市失联。中国外交部之后称,袁克勤已承认从事间谍活动并被拘留。 据日本共同社星期三(5月15日)报道,消息人士透露,袁克勤被判罪的罪名是违反《反间谍法》。吉林省长春市的中级人民法院(地方法院)今年1月底作出判决……}

\entryitemWithDescription{美国前国务卿蓬佩奥将出席赖清德就职典礼}{https://www.zaobao.com/news/china/story20240516-3674823}{美国前国务卿蓬佩奥宣布,他将以个人身份参加下星期一(5月20日)举行的台湾总统就职典礼。(路透社档案照) (华盛顿/台北综合讯)继美国跨党派代表团后,美国前国务卿蓬佩奥也宣布,他将以个人身份参加下星期一(5月20日)举行的台湾总统就职典礼。 曾在特朗普政府时期担任国务卿的蓬佩奥,星期三(5月15日)接受美国之音采访时说,他会以个人身份前往台北,参加台湾候任总统赖清德的就职典礼……}

\entryitemWithDescription{中国各大社交平台发公告 强调从严打击炫富拜金内容}{https://www.zaobao.com/news/china/story20240516-3674138}{(北京综合讯)中国各大社交平台在同一天发公告,强调要从严打击``炫富拜金''等不良价值导向内容,这意味着未来那些晒豪宅豪车、炫耀富二代身份的内容都可能被清理,相关账号也会被封禁。 综合封面新闻和财经网报道,腾讯、抖音、快手、微博、B站、小红书等平台星期三(5月15日)发布不良价值导向内容专项治理公告,针对近期网络上出现的奢靡浪费、炫富拜金等问题,从严打击,倡导理性、文明的消费观和价值观……}

\entryitemWithDescription{陈婧:迟来的刺激政策}{https://www.zaobao.com/news/china/story20240516-3670597}{``刺激政策终于找对了方向,力度也上来了。'' 最近访问的几名学者和经济师,不约而同发出这样的感慨。过去两周,中国各领域经济刺激政策如雨后春笋般层出不穷,力度也超出预期。 先是最受关注的房地产,两个一线城市北京和深圳在``五一''长假前后相继放宽限购政策,热点二线城市杭州和西安更全面解除限购,带动新一波楼市松绑潮。本周又有消息称,中央政府将要求地方出资收购存量住宅,帮助房企消化库存……}

\entryitemWithDescription{国台办宣布惩戒台湾五名嘴 学者:感觉有如拿明朝的剑斩清朝的官}{https://www.zaobao.com/news/china/story20240515-3669221}{民进党政策会执行长王义川曾说``大陆高铁没靠背'',引起反弹。(取自脸书) 台湾政论节目《关键时刻》主持人刘宝杰星期三(5月15日)主持节目时回应:``评论经济等同台独?大陆怕什么?''(视频截图) 台湾财经专家黄世聪在脸书回应时表示,他论及大陆的内容都来自媒体,并非凭空捏造,在台湾都是言论自由的范围……}

\entryitemWithDescription{中国如何反制美国加征关税 学者:可能以关税对关税}{https://www.zaobao.com/news/china/story20240515-3670468}{美国白宫星期二宣布,将对价值180亿美元(近244亿新元)的中国进口货品加征关税。图为摄于2021年1月21日在北京一家美国公司大楼外,飘扬着中美两国国旗。(路透社档案照) 针对美国宣布对中国电动车、锂电池和半导体等产品加征高额关税一事,中国外长王毅星期三(5月15日)直指这是``当今世界上最典型的霸道霸凌''。对于美国此举是否意味着中美关系倒退,中国外交部发言人在当天例行记者会上拒绝正面回应……}

\entryitemWithDescription{赖清德就职典礼在即 中国大陆军事活动越来越接近台湾}{https://www.zaobao.com/news/china/story20240515-3669582}{台湾候任总统赖清德星期三(5月15日)在台北出席台湾资安大会。(法新社) (台北/北京综合讯)台湾政府报告称,随着候任总统赖清德5月20日就职典礼的临近,中国大陆军方最近几周在台海周边活动的航行和飞行距离,比以往任何时候都要靠近台湾,一些军机还对进入台海的外国船只进行了模拟攻击……}