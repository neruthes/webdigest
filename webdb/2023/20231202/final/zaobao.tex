\entryitemWithDescription{新闻人间:坐困愁城的柯文哲}{https://www.zaobao.com/news/china/story20231202-1453654}{nao\_2023\_RGB-01(Lianhe Zaobao) ``蓝白合''功败垂成,这一个多星期以来,台湾在野民众党(白)总统候选人柯文哲立即面临金主抽腿、民调陡降、辅选行程难以迈出台北市的窘境,内心五味杂陈。 11月中旬在前总统马英九见证下,柯文哲与国民党(蓝) 签下合作的六点协议, 未料因彼此认知差异而闹僵,在君悦酒店与国民党的分手闹剧更在众目睽睽下两败俱伤,重创柯文哲的政治诚信……}

\entryitemWithDescription{黄小芳:从拼多多到军大衣}{https://www.zaobao.com/news/china/story20231202-1453659}{两年多前初到北京工作时,朋友向我大力推荐了拼多多。不到10元(人民币,下同,1.89新元)能买一件女装背心、十几元能买到充电宝,手机等3C产品的价格也都普遍低于一般市价。低得让人难以置信的价格,让我对商品质量存疑;尽管已下载拼多多APP好几年,我从未在该平台上购物。~ 不过,我显然已经过时了,拼多多的火热程度如今已超乎我的想象……}

\entryitemWithDescription{陕西贵州被点名统计造假 学者:地方数据灌水已严重影响中央资源调拨}{https://www.zaobao.com/news/china/story20231201-1453660}{中国官方本周罕见点名贵州省和陕西省存在统计造假问题。受访学者分析,这反映出地方数据灌水已经对中央政策和资源调拨造成严重影响,甚至出现资源错配的情况。 中国国家统计局专项督察组星期四(11月30日)分别向陕西省和贵州省反馈专项统计督察意见,指出两省在统计工作方面存在的问题。 据督察组通报,两省的问题包括:对统计造假问题严重性认识不到位,存在干预统计报送、统计造假行为,专项治理行动成效不够明显等……}

\entryitemWithDescription{中印召开边境事务磋商和协调会议 同意尽早举行下一轮军长级会谈}{https://www.zaobao.com/news/china/story20231201-1453644}{(北京 / 新德里综合讯)中国与印度举行由外交系统主导的边境事务磋商会议,同意尽早举行下一轮军长级会谈,早日实现边境局势翻篇。 据中国外交部网站星期五(12月1日)发布的消息,中印在星期四举行边境事务磋商和协调工作机制第28次会议。会议由中国外交部边界与海洋事务司司长洪亮和印度外交部东亚司联秘戴国澜共同主主持,两国外交、国防、移民等部门代表参加……}

\entryitemWithDescription{警惕呼吸道疾病 台湾抽验陆港澳入境旅客}{https://www.zaobao.com/news/china/story20231201-1453621}{(台北综合讯)中国大陆呼吸道疾病进入高发期,台湾从上周起在机场对来自大陆及港澳的旅客进行抽样检测,尚未发现肺炎支原体感染病例。 据台湾疾病管制署星期五(12月1日)发布的消息,当局自11月26日起在台北、桃园、台中和高雄四个国际机场,对来自中国大陆、香港和澳门的有症状旅客实施``定点监测,鼓励采检''措施。 检测范围包括流感、冠病、肺炎支原体等17种病毒和四种细菌……}

\entryitemWithDescription{美国审计监察机构对中国公司处790万美元罚款}{https://www.zaobao.com/news/china/story20231201-1453619}{(巴黎/伦敦综合讯)美国上市公司会计监督委员会(PCAOB)星期四(11月30日)宣布三项针对中国公司的执行令,总计罚款超790万美元(约1060万新元),对象是普华永道中国大陆和香港的分公司,以及一家中国审计公司。 综合法新社与路透社报道,PCAOB说,普华永道中国大陆与香港的分公司的1000多名员工,在内部训练考试中作弊,违反了审计品质管理标准,两家分公司同意合计支付700万美元了结指控……}

\entryitemWithDescription{中国复审澳洲葡萄酒反倾销关税}{https://www.zaobao.com/news/china/story20231201-1453614}{(北京/上海综合讯)中国启动对澳大利亚进口葡萄酒实施反倾销关税措施的复审,有望在未来数月内,取消这一实施逾三年的关税。 中国商务部官网星期四(11月30日)发布公告,决定即日起对原产于澳洲的进口相关葡萄酒所适用反倾销措施和反补贴措施进行复审,调查应于2024年11月30日前结束。 澳洲前总理莫里森2020年要求国际社会追溯冠病疫情起源后,澳中关系跌入低谷……}

\entryitemWithDescription{《南华早报》记者据报赴京采访后失联}{https://www.zaobao.com/news/china/story20231201-1453609}{(香港综合讯)香港《南华早报》一名资深记者据报在上月底到北京出差后失联。 日本共同社星期四(11月30日)引述知情人士说,《南华早报》的中国国防和外交事务记者陈敏莉(Minnie Chan),在前往北京报道10月29日至31日举行的香山论坛后与周围人失去联络。 陈敏莉的朋友说,她最后一次与外界联系是在11月1日,此后下落不明;朋友对她可能正接受当局调查表示担忧……}

\entryitemWithDescription{台湾总统大选前 北京据报赞助数百名台政界人士赴陆游}{https://www.zaobao.com/news/china/story20231201-1453605}{(台北综合讯)台湾明年1月大选临近,消息人士和文件透露,北京赞助了数百名台湾政界人士的中国大陆低价游,引起台湾官员对彼岸广泛介入选举的不安。 一名追查大陆活动的台湾安全官员告诉路透社,过去一个月,台湾各地的安全机构调查了400多起到大陆的访问,其中大多数是由地方的意见领袖------比如村长、里长等------率团前往……}

\entryitemWithDescription{赵天麟婚外情引统战疑虑 台学者:防范应从地方议员层级做起}{https://www.zaobao.com/news/china/story20231201-1453570}{两度担任台湾立法院外交及国防委员会召集委员的民进党籍立委赵天麟,被爆出与大陆女子发生婚外情,震动台湾社会,也引发舆论质疑有统战甚至间谍渗透疑虑。(互联网) 关于中国大陆对台湾立法委员的统战渗透,在来临选战中被民进党列为不分区立委第二名的台湾学者沈伯洋,星期五(12月1日)受访时指出,许多立委都是从地方议员出身,因此应要从地方上的统战围堵做起……}

\entryitemWithDescription{蔡英文称中国大陆因内部难题 目前不太可能攻台}{https://www.zaobao.com/news/china/story20231130-1453389}{《纽约时报》于美东时间星期三(11月29日)在纽约林肯表演中心举行年度活动``交易录峰会'',播放蔡英文(右)与专栏作家索金的预录访谈视频。(法新社) 台湾总统蔡英文在《纽约时报》的高峰访谈中说,中国大陆正面对经济和政治等内部难题,目前不太可能进攻台湾……}

\entryitemWithDescription{中国将严打境外诈骗集团领导者}{https://www.zaobao.com/news/china/story20231130-1453378}{(北京综合讯)中国最高人民检察院称,将从严惩治电信网络诈骗,严打境外诈骗集团领导者。 中国最高检网站星期四(11月30日)发布《检察机关打击治理电信网络诈骗及其关联犯罪工作情况(2023)》(简称《情况》),表示将进一步加强国(区)际执法司法合作,围绕重点地区、重大集团、重要案件,深挖彻查案件线索……}

\entryitemWithDescription{谷歌指中国大陆正加大对台湾的网络攻击}{https://www.zaobao.com/news/china/story20231130-1453352}{(北京/台北综合讯)美国科技巨头谷歌的网络安全专家说,北京正在加大对台湾的网络攻击,其中试图侵入台科技公司的黑客组织数量庞大。 据彭博社星期四(11月30日)报道,谷歌负责监控政府赞助黑客活动的威胁分析部门高级工程经理摩根(Kate Morgan)说,谷歌观察到北京在过去六个月左右的时间内,大幅增加对台湾的网络攻击……}

\entryitemWithDescription{湄洲妈祖赴台卡关?台陆委会:须补齐``必要性''资料}{https://www.zaobao.com/news/china/story20231130-1453349}{台湾政府的大陆委员会主委邱太三(图)星期四(11月30日)在台北出席研讨会时,再度呼吁北京逐步消弭两岸间的敌意螺旋,重建双方善意互信。 (缪宗翰摄) 中国大陆福建湄洲妈祖金身和漳州东山关帝金身应邀赴台巡安传出卡关,台湾政府的大陆委员会主委邱太三星期四(11月30日)表示,须补齐说明同行人员及来台``必要性''等资料……}

\entryitemWithDescription{王纬温:看好中国的美国投资大师芒格去世}{https://www.zaobao.com/news/china/story20231130-1453172}{美国知名投资人芒格当地时间星期二(11月28日)在加州一家医院逝世,享年99岁,距百岁生日仅月余。 这名叱咤风云的投资大师身价约26亿美元(34.6亿新元),是``股神''巴菲特执掌的美国投资巨舰伯克希尔·哈撒韦的副董事长。他也是巴菲特的老乡和老友,多年来更多以巴菲特``黄金搭档''\,``幕后智囊'',以及``中国巴菲特''被外界熟知……}

\entryitemWithDescription{柯文哲再爆大选掮客 赵少康要求说清楚}{https://www.zaobao.com/news/china/story20231129-1453168}{台湾在野民众党总统候选人柯文哲星期三(11月29日)出席九大工商团体举办的``2023台湾经济发展论坛'',说明他的经济愿景与政策。 (民众党提供) 台湾在野民众党总统候选人柯文哲及其竞选办公室近日相继爆料,称在总统选举登记前,国民党方面有人开价两亿美元(2.66亿新元)要求柯当副手。柯文哲星期三(11月29日)受访时说,希望这一页就此翻过,却同时释出数位掮客姓氏,在政坛引发诸多揣测……}

\entryitemWithDescription{特稿:中美人文交流寒冬下 在华美国留学生数量骤减}{https://www.zaobao.com/news/china/story20231129-1453166}{疫情三年阻隔,加上中美在各领域的矛盾加深,两国人文教育交流近几年进入寒冬。(路透社档案照) 美国田纳西大学的大一学生莫尼(Chapin Mohney,18岁)一直希望到中国学习,但他最近参加学校的海外游学展时得知,今年没有到中国的项目。 莫尼就读的大学在疫情前设有到上海游学的项目,今年改为去伦敦。就读化学工程专业、对中国感兴趣的莫尼接受《联合早报》采访时说:``只能明年再看看……}

\entryitemWithDescription{港最大国安案件进入结案阶段 控方指用社媒造谣也危害国家安全}{https://www.zaobao.com/news/china/story20231129-1453162}{香港涉案人数最多的国安案件民主派初选案,星期三(11月29日)开始总结陈词。图为被告之一、湾仔区议会前主席杨雪盈到庭应讯。(法新社) (香港综合讯)香港涉案人数最多的国家安全案件,本周进入结案陈词阶段。控方指出,武力非必要元素,使用社媒造谣也可危害国家安全……}