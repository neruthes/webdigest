\entryitemWithDescription{陈婧:中美芯片战谁是赢家?}{https://www.zaobao.com/news/china/story20230525-1397902}{历时一个多月,中国官方对美国芯片制造商美光(Micron)的网络安全审查,终于靴子落地。中国国家互联网信息办公室上星期天(5月21日)以``影响国家安全''为由,要求国内关键信息基础设施运营商停止采购美光公司的产品。 早在网信办宣布对美光进行网络安全审查之时,外界就预料到这个结果……}

\entryitemWithDescription{中国黑客被指对肯尼亚进行攻击 以获得债务信息}{https://www.zaobao.com/news/china/story20230525-1397912}{在肯尼亚首都内罗毕,一条由中国路桥工程公司与肯尼亚政府合作开发的快速路,连接了市中心和当地的国际机场。图片拍摄于今年5月7日。(路透社) 中国黑客被指对东非国家肯尼亚的政府部门和机构进行了持续多年的、广泛的网络入侵,以获取肯尼亚欠下中国债务的信息。 路透社星期三(5月24日)引用匿名人士报道称,中国黑客从2019年末开始对肯尼亚展开为期三年的网络入侵行动……}

\entryitemWithDescription{涉嫌歧视非英语旅客 国泰航空解雇三名空服员}{https://www.zaobao.com/news/china/story20230524-1397876}{香港国泰航空行政总裁林绍波(中),星期三在广州就旗下空服员涉歧视乘客事件,以普通话再次致歉并承诺全面检讨。图为林绍波接受媒体采访。 ( 中新社) 香港国泰航空空中服务员涉嫌歧视非英语旅客,在中国大陆引起激烈回响。国泰航空周二(5月23日)急速解雇涉事的三名空中服务员,意图为事件降温。但有受访学者认为,事件反映香港社会的仇中情绪炽热,要解决这个根本问题并不容易……}

\entryitemWithDescription{中国特别代表李辉:中法在乌克兰问题上存在不少共识}{https://www.zaobao.com/news/china/story20230524-1397854}{中国政府欧亚事务特别代表李辉(右)5月23日在巴黎与法国外交部政治和安全事务总司长蒙多洛尼举行会谈,双方就政治解决乌克兰危机和中法关系等问题交换意见。(中国外交部官网) 在法国访问的中国政府欧亚事务特别代表李辉说,中法在乌克兰问题上有不少共识,中方愿与各国加强对话,扩大政治解决危机的公约数……}

\entryitemWithDescription{中国新任驻美大使坦言 中美关系面临严峻挑战}{https://www.zaobao.com/news/china/story20230524-1397850}{新任中国驻美国大使谢锋(右)与夫人在美东时间星期二(5月23日)抵达纽约肯尼迪机场,并在机场与媒体和各界人士交流。(路透社) 中国新任驻美国大使谢锋抵美履新,填补了这一空缺多时的职位。谢峰坦言,当前中美存在深刻分歧,两国关系遭遇严重困难,面临严峻挑战。 据中国驻美国大使馆网站消息,谢锋于美东时间星期二(5月23日)抵达纽约,晚间抵达华盛顿……}

\entryitemWithDescription{中荷外长会谈 荷方:保护经济与网络空间安全是优先要务}{https://www.zaobao.com/news/china/story20230524-1397848}{中国国务委员兼外长秦刚(右)5月23日在北京同荷兰副首相兼外长胡克斯特拉会谈后共同会见记者。(中新社) 中国国务委员兼外长秦刚星期二(5月23日)在北京与荷兰副首相兼外长胡克斯特拉会谈后称,中荷应共同努力,维护两国之间正常的贸易秩序,胡克斯特拉则强调,荷兰有责任保护自己的国家安全和核心利益……}

\entryitemWithDescription{美国空军:F-16战机生产遇开发挑战 延迟对台交付}{https://www.zaobao.com/news/china/story20230524-1397822}{美国空军称,延迟交付售卖给台湾的F-16战机,是因为生产遇到了``复杂的开发挑战''。图为3月25日,一架美国产的F-16V战机在嘉义一处军事基地的跑道滑行。(法新社) 美国延迟交付售卖给台湾的F-16型战斗机,台湾早前指事因冠病疫情影响了供应链。美国空军在最新声明中表示,是因为F-16战机的生产遇到了``复杂的开发挑战''……}

\entryitemWithDescription{杨丹旭:拜登的``解冻''预言能否成真?}{https://www.zaobao.com/news/china/story20230524-1397596}{美国``政治''(Politico)新闻网站星期一(5月22日)爆料,新一任中国驻美大使谢锋将在当地时间星期二抵达华盛顿履新。这意味着悬空已有数月的驻美大使职位,总算要填上了。 要知道中美关系自1979年正常化以来,除了1995年台湾总统李登辉访美、北京召回时任驻美大使李道豫两个月,驻美大使的职位从来没有像今年这样,悬空如此之久……}

\entryitemWithDescription{中美互动频密 ``解冻''迹象明显}{https://www.zaobao.com/news/china/story20230523-1397553}{在美国总统拜登宣称中美关系``很快解冻''之后,本周内中国商务部长王文涛将访美、新任中国驻美大使谢锋上任,而美国驻华大使伯恩斯也刚到成都访问,种种迹象显示两国都在解冻做准备。 美国是在2020年7月21日限时中国72小时关闭驻休斯敦总领馆,中国随后宣布撤销对美驻成都总领事馆的设立和运行许可。那次事件后,美国驻华大使就没有再公开踏足成都……}

\entryitemWithDescription{香港取消捐献器官申请急升 警方介入调查}{https://www.zaobao.com/news/china/story20230523-1397542}{香港与中国大陆政府计划建立恒常的器官移植互助机制,引起部分港人担忧和抵制,近期申请取消捐献器官登记的数字急升,但其中逾半是从未登记却申请退出或重复取消的无效申请。特首李家超认为情况可疑,已要求警方调查。 李家超星期二(5月23日)出席行政会议前会见媒体时指出,器官捐献非常重要,可救人生命,是崇高之举,世界各地包括香港都鼓励及推动。港府也在推动与大陆合作,有个别香港患者从中受益……}

\entryitemWithDescription{俄罗斯总理访华 深化两国经济联系}{https://www.zaobao.com/news/china/story20230523-1397540}{俄罗斯总理米舒斯京(中)星期二在上海参观访问中国石油化工股份有限公司研究所。(法新社) 俄罗斯总理米舒斯京在中国展开为期两天的访问,以中俄两国在能源等领域的商贸合作为主,表明在受到西方制裁的背景下,俄罗斯在经济上向中国寻求更多支持。 米舒斯京于星期一(5月22日)晚间抵达上海,星期二早上参加了中俄商务论坛……}

\entryitemWithDescription{美媒:中国加紧建设卫星互联网网络 欲与马斯克的Starlink竞争}{https://www.zaobao.com/news/china/story20230523-1397532}{美媒称,中国正抓紧建设自己的卫星互联网网络,以和美国太空探索技术公司(SpaceX)的星链(Starlink)竞争,后者在世界各地迅速扩展,并已在俄乌战争的军事应用中展露身手。 《华尔街日报》报道称,中国加紧建设卫星互联网网络的背后原因,缘于地缘政治的紧张局势升温,以及担忧关键轨道资源饱和的风险,这也是中国政府在关键科技方面推动自给自足的举措之一……}

\entryitemWithDescription{末代皇帝溥仪手表以4000万港元拍出}{https://www.zaobao.com/news/china/story20230523-1397524}{中国清朝末代皇帝爱新觉罗溥仪曾经拥有的百达翡丽腕表,星期二(5月23日)在香港展出和拍卖,最终以4000万港元的价格拍出。(法新社) 中国清朝末代皇帝溥仪佩戴过的百达翡丽手表,在香港的一次现场拍卖中以4000万港元(687万8千新元)的价格拍出。 彭博社报道,这次拍卖于星期二(5月23日)在钟表拍卖行富艺斯位于香港的亚洲总部进行,总共进行了七分钟……}

\entryitemWithDescription{贸易制裁未解除 澳不公开支持中国加入CPTPP}{https://www.zaobao.com/news/china/story20230523-1397517}{消息人士和学者表示,澳大利亚不太可能公开支持中国加入跨太平洋伙伴全面进展协定(CPTPP),原因包括中国对澳大利亚的贸易制裁尚未解除,以及中国申请CPTPP存在的``地缘政治包袱''。 香港《南华早报》星期二(5月23日)报道,多位消息人士称,在澳大利亚贸易部长法雷尔5月初访华期间,中国希望澳大利亚明确支持中国加入CPTPP,并拒绝台湾加入,而且最好是``公开承诺''……}

\entryitemWithDescription{被举报空服员涉歧视非英语乘客 国泰航空两度发声明道歉}{https://www.zaobao.com/news/china/story20230523-1397504}{一名中国大陆网民星期一(5月22日)在社交网站,实名举报香港国泰航空的空服人员歧视非英语乘客。国泰航空对此连发两份声明道歉,称涉事空服人员已被暂停飞行任务,将在三天内公布处理结果。 综合《北京日报》与《明报》报道,这名大陆乘客发文称,在他搭乘5月21日国泰航空CX987航班由成都飞往香港时,座位靠近空中服务员备餐及休息处。在入座之后,不断听见空服人员用英文及粤语抱怨……}

\entryitemWithDescription{港议员称约150反修例抗争被捕者暂不被起诉}{https://www.zaobao.com/news/china/story20230523-1397471}{2019年香港反修例运动中逾6000名被捕、但一直未被检控人士,据说有150人已接获警方通知暂不作起诉或结案。 据网媒``香港01''星期一(5月22日)报道,一直关注反修例被捕人士情况的新界北立法会议员张欣宇向《香港01》表示,自2月初起至今,先后有约150名被捕人士向他透露,已接获警方通知暂不作起诉或结案。他们主要涉及非法集结、公众地方行为不检等罪行。他相信当局整体的检视程序已到尾声……}

\entryitemWithDescription{戴庆成:香港乐坛为何一蹶不振?}{https://www.zaobao.com/news/china/story20230523-1397228}{踏入五六月,香港开始迈入闷热的夏季。许多人晚上在家没事,一般都会开空调看电视或上网打发时间。我上星期天(5月21日)晚上在家打开电视,赫然看到无线电视台正播映特备节目《劲歌43年情· 让音乐高飞》,才察觉《劲歌金曲》这个音乐节目原来正式画上句号了。 《劲歌金曲》是大部分香港人的集体回忆,自从1981年首播以来,一直提供平台给歌手亲自上电视台宣传新歌,成为香港至今最长寿的电视音乐节目……}

\entryitemWithDescription{中国外交部回应拜登中美关系``很快改善''言论 质疑美方对沟通的诚意和意义}{https://www.zaobao.com/news/china/story20230522-1397227}{针对美国总统拜登称中美关系将很快得到改善,希望与中国打开更多沟通渠道,中国外交部回应指美国有意加强沟通的同时,却继续打压中国,质疑美方对沟通的诚意和意义。 中国外交部发言人毛宁星期一(5月22日)在例行记者会上指出,美国一边说要沟通,一边不择手段地打压遏制中国,对中方的官员、机构和企业实施制裁,她质问这样的沟通诚意和意义何在……}

\entryitemWithDescription{中国商务部长王文涛:继续欢迎美企在华发展}{https://www.zaobao.com/news/china/story20230522-1397219}{中国商务部部长王文涛向在华美资企业派定心丸,强调中国将把吸引外资放在更加重要位置,继续欢迎美企在华发展。 据中国商务部网站发布的消息,王文涛星期一(5月22日)在上海主持召开座谈会,听取美资企业在华经营情况以及对中国持续优化营商环境的意见建议。 上海美国商会以及强生、3M、陶氏、默克、霍尼韦尔等美资企业代表参会……}

\entryitemWithDescription{中美芯片战升级 学者认为有可能扩大到其他行业}{https://www.zaobao.com/news/china/story20230522-1397207}{北京指美光科技公司(Micron)的产品未能通过中国的网络安全审查。(路透社) 中国与美国的半导体科技战升级,北京当局宣布美光科技公司(Micron)的产品,未能通过中国的网络安全审查,要求中国的运营者禁止采购。美国随即回应将与盟友展开合作,应对中国的举措所造成的存储芯片市场扭曲问题。 受访专家学者认为,中国的这一举动可以理解为中国对美国管制半导体出口的报复,不过是一个顾全大局、留有情面的反击……}

\entryitemWithDescription{钟南山预测6月底中国再现冠病感染高峰 每周或有6500万起病例}{https://www.zaobao.com/news/china/story20230522-1397205}{中国工程院院士钟南山星期一(5月22日)预测,中国冠病第二波疫情高峰将发生在6月底,每周或将出现约6500万起冠病感染病例。 综合《北京商报》与彭博社报道,钟南山在2023大湾区科学论坛生物医药与健康分论坛上说,基于SEIRS模型的预测,中国第二轮疫情在4月中旬起峰,5月底出现一个小高峰,波峰感染数量约每周4000万例,而到6月底预计将出现本轮疫情的高峰……}