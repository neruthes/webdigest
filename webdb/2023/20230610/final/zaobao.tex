\entryitemWithDescription{新闻人间:王进洋——泛民政客沦为电骗集团骗徒}{https://www.zaobao.com/news/china/story20230610-1402857}{《香港国安法》实施三年以来,民主派遭受空前打击,一众头面人物要么身陷囹圄,要么移居海外,要么就退场转型另找工作。 在刚刚过去的星期三(6月7日),香港警方公布拘捕一名28岁王姓男子,指他涉嫌与九起``猜猜我是谁''的电话骗案有关,骗去九名老妇近100万港元(17万新元)。令人惊讶的是,这名被捕人士竟然是前任离岛区议员王进洋……}

\entryitemWithDescription{温伟中:从性骚扰到喂安眠药}{https://www.zaobao.com/news/china/story20230610-1402894}{从5月31日至今的10天以来,民进党爆发掩盖性骚扰丑闻、新北市政府被指拖延处理幼儿园喂安眠药事件,被视为台湾蓝绿总统候选人后院失火的破口事件。 绿军总统候选人、民进党主席赖清德,以及蓝军总统候选人、国民党籍新北市长侯友宜,已分别多次向社会道歉。但能否成功止血停损、清理战场,甚至化危机为转机;还是继续失血至无力回天,仍有待观察……}

\entryitemWithDescription{中国大使提醒首尔别在中美``下错赌注'' 韩国外长谴责他干涉内政}{https://www.zaobao.com/news/china/story20230609-1402878}{中国驻韩国大使邢海明星期四(6月8日)不点名批评尹锡悦政府``赌美国赢、中国输'',提醒首尔不要在中美竞争中``下错赌注''。韩国外长官朴振星期五谴责邢海明这番言论过分。 受访学者分析,邢海明的表态意在向韩国政商界发出警示,同时也分化韩国国内的友中反中势力,以及日益热络的日韩关系,表明中国对美日韩在东亚打造军事联盟越来越警惕……}

\entryitemWithDescription{美媒指中国在古巴设对美监听基地 中国白宫古巴皆驳斥报道不准确}{https://www.zaobao.com/news/china/story20230609-1402874}{《华尔街日报》报道指中国在古巴建造秘密监听美国基地,中国、古巴、美国官方都指报道不正确、毫无根据、纯属造谣。 《华尔街日报》星期四(6月8日)引述知情的美国官员报道,中国和古巴达成秘密协议,将在古巴建立针对美国的电子窃听设施。 中国外交部发言人汪文斌星期五(6月9日)在例行记者会上说:``众所周知,造谣污蔑是美国的惯用伎俩,肆意干涉他国内政是美国的专利……}

\entryitemWithDescription{美参院委员会通过法案 剥夺中国``发展中国家''地位}{https://www.zaobao.com/news/china/story20230609-1402847}{美国参议院星期四(6月8日)全票通过了一项法案,要求国务院争取在美国参与的国际组织中,取消中国享有的发展中国家地位。 受访学者表示,美国这项决定又是让其他国家在中美经贸关系中选边站。但这项国内立法到底会有多大效果,取决于美国国务院要花多大力气去执行,以及其他国家和国际组织是否要改变中国的贸易地位……}

\entryitemWithDescription{人人自危 中国脱口秀演员自嘲开场前互道``祝您平安''}{https://www.zaobao.com/news/china/story20230609-1402835}{中国脱口秀演员李昊石被控辱军后,脱口秀从业者更加小心谨慎。北京东城区一家脱口秀现场上,演员们极力回避社会热点话题。(孟丹丹摄) ``演出之前,不说`祝您炸场'了,大家现在都改说`祝您平安'\,''。 5月下旬,北京东城区一家咖啡馆的脱口秀(在中国也称单口喜剧)开放麦现场,主持人在开场白中调侃李昊石事件后脱口秀从业者人人自危,以往演员们上场前,相互之间都会鼓励演出成功,现在则祈祷平安无事……}

\entryitemWithDescription{中国网络大V写色情小说 犯传播淫秽物品罪获刑八个月}{https://www.zaobao.com/news/china/story20230609-1402813}{中国网络大V曾鹏宇因在微博账号上发表色情小说,犯传播淫秽物品罪而获刑八个月。 据红星新闻星期四(6月8日)报道,曾鹏宇因涉嫌犯传播淫秽物品罪于2022年8月被抓。中国裁判文书网5月26日公布的相关刑事判决书显示,曾鹏宇获刑八个月,自2022年8月28日起至2023年4月26日止。 大V是指在新浪、腾讯、网易等微博平台上获得个人认证,拥有众多粉丝的微博用户……}

\entryitemWithDescription{新北幼儿园孩童被喂安眠药水 市长侯友宜两度道歉}{https://www.zaobao.com/news/china/story20230609-1402797}{台湾新北市板桥区一家幼儿园,让不听话的孩子服用含安眠药成分的药水。目前已有17名家长报案,涉事幼儿园园长和五名老师被检察署约谈后保释在外,其中园长以5万元(新台币,下同,约2185新元)交保。 新北市长侯友宜星期四(6月8日)两度道歉,执政的民进党狠批这位国民党总统候选人毫无诚意。新北市议会民进党团星期五召开记者会,痛批侯友宜与新北市政府不积极处理这起事件,要他请辞负责……}

\entryitemWithDescription{中国``文管''登场了?}{https://www.zaobao.com/news/china/story20230609-1402562}{继城管和农管之后,中国''文管''大队登场的话题,本星期引起了境外社交媒体关注。 ``文管''一词横空出世,导源于今年5月30日黑龙江省文化旅游厅举行的一场``文化市场综合执法队伍统一着装仪式暨规范化建设现场观摩会''。 根据主办方官网发布的消息与配图,黑龙江省各级文化市场执法队员代表共500余人参加了活动。代表们穿着整齐的制服,犹如纪律部队般出席了统一着装仪式……}

\entryitemWithDescription{侯友宜选总统民调低迷 保台论述和掌握民怨或是反弹关键}{https://www.zaobao.com/news/china/story20230608-1402564}{美国在台协会(AIT)主席罗森伯格(左一)两个月内二度访台,6月7日与侯友宜(右一)和新北市政府团队密谈一小时后,搭火车到平溪放天灯祈福,并聊到深夜。(新北市政府提供) 台湾在野的国民党总统候选人侯友宜民调低迷,半年内从领先到两度垫底,引发弃保担忧。但受访民调专家研判,美方本周专程派高官来台面谈,说明侯友宜仍具备实现政党轮替的政治能量……}

\entryitemWithDescription{美媒:中国将在古巴设立针对美国的秘密监听基地}{https://www.zaobao.com/news/china/story20230608-1402559}{民众在古巴首都哈瓦那街头的摊贩排队买食物。(法新社) 美国官员称,中国和古巴达成秘密协议,中国将在古巴建立电子窃听设施,对美国构成新的地缘政治挑战。 《华尔街日报》引述熟悉美国高度机密情报的官员说,设在古巴的电子窃听设施,距离佛罗里达州约160公里,这让中国情报部门能够窃听美国东南部的电子通信,以及监视美国的船只。美国东南部是许多军事基地的所在地……}

\entryitemWithDescription{中国国企领导与女子牵手逛街后被免职}{https://www.zaobao.com/news/china/story20230608-1402537}{中国一名国企高管在四川成都出差期间,被拍到与一名年轻女子亲密牵手逛街。视频本周在网络流传后,涉事两人均被停职调查。 一则街拍视频星期三(6月7日)短视频平台抖音上被疯转。视频显示,一名身穿玫粉色上衣的中年男子与一名衣着靓丽的年轻女子牵手说笑、漫步在成都闹市区街道……}

\entryitemWithDescription{谢锋:用``去风险''为``脱钩''掩护 给中美关系埋下更多钉子}{https://www.zaobao.com/news/china/story20230608-1402527}{中国驻美大使谢锋6月7日出席美中贸易全国委员会为他履新举行的欢迎活动时说,如果用``去风险''为``脱钩''打掩护,就会给中美关系埋下更多钉子。(中国驻美大使馆官网) 中国呼吁美国在高层交往中做好全过程管理,不能言行不一致,强调如果用``去风险''为``脱钩''打掩护,就会给中美关系埋下更多钉子……}

\entryitemWithDescription{大陆军机37架次现台周边空域 台军启动防御系统}{https://www.zaobao.com/news/china/story20230608-1402516}{台湾国防部星期四(6月8日)侦查到37架次中国大陆的军用飞机,飞入台湾防空识别区西南空域。台军随即启动防御系统,两军并没有发生冲突。 根据台湾国防部星期四发布的新闻稿,自当天早上五时起,侦查到中国大陆军机的军机,包括歼11、歼16、轰6、运油20及预警等各型军机,进入台湾的防空识别区。部分大陆编队随后飞入西太平洋,进行空中监视和远程航行训练……}

\entryitemWithDescription{美议员反对邀李家超出席APEC 港学者:特首获邀也不一定赴会}{https://www.zaobao.com/news/china/story20230608-1402510}{香港近年与美国关系紧张。美国有跨党派国会议员星期三(7日)致函国务卿布林肯,反对美国有意邀请被美方制裁的香港行政长官李家超,参加今年年底在美国举行的亚太经合组织(APEC)会议。 有受访学者认为,美国总统拜登正寻求连任,未必愿意在该议题上惹恼参众两院议员。但纵使美国邀请李家超出席,李家超也不一定赴会……}

\entryitemWithDescription{英政府部门将拆除中国制造监控设备 北京指责伦敦歧视打压中企}{https://www.zaobao.com/news/china/story20230608-1402505}{英国内阁公署要求各政府部门从敏感政府机关地点拆除中国制造的监控设备。中国驻英国大使馆指这是``歧视和打压中国企业''。 综合路透社和《金融时报》报道,英国内阁公署当地时间星期二(6月6日)在一份收紧采购条规的公告中,承诺将公布一份时间表,拆除受中国《国家情报法》约束的公司所生产的监控设备……}

\entryitemWithDescription{台湾性骚扰风波持续延烧 文坛与教育界两位重量级人物被点名}{https://www.zaobao.com/news/china/story20230608-1402464}{台湾``MeToo''风暴从政坛烧向文坛,著名现代诗人郑愁予、知名作家陈芳明,被曝曾经分别性骚扰女学生和女助理。 台湾媒体《放言》星期三(7日)报道,以一首《错误》享誉诗坛,诗歌入选海峡两岸教材的老牌诗人郑愁予,近日被指在东华大学任教时,曾对女学生有类似毛手毛脚的行为。 当事人在脸书发长文《踩在受害者的位置上,踩好踩满》,指控郑愁予在2005年担任东华大学客座教授期间性骚扰学生……}

\entryitemWithDescription{陈婧:落空的黄仁勋``登陆''行}{https://www.zaobao.com/news/china/story20230608-1402205}{人工智能芯片巨头英伟达(Nvidia)创办人黄仁勋已从台湾返回美国的消息在星期一(6月5日)得到证实,让坊间对他本周访问中国大陆的高涨期待落了空。 此前一周,关于黄仁勋``登陆''的消息已传得沸沸扬扬,连具体行程都有鼻子有眼。据说,黄仁勋会到访腾讯、字节跳动等科技公司,理想和比亚迪等电动车企,以及正进军电动车领域的小米公司……}

\entryitemWithDescription{高尔夫球场部分用地``未决定用途'' 港府新界兴建公屋计划或有变}{https://www.zaobao.com/news/china/story20230607-1402213}{香港土地长期供不应求,港府早前表示将收回有逾百年历史的粉岭高尔夫球场来兴建公营房屋。但当局近日向城市规划委员会提交的文件,却将部分用地暂时修订为``未决定用途''地带,让人察觉到建屋计划出现变数。 受访学者认为,目前香港房屋问题严峻,当局应学习新加坡政府果断收回克兰芝赛马场兴建房屋的魄力,尽快收回粉岭高尔夫球场高场并立即建公屋……}

\entryitemWithDescription{中美未否认布林肯将访华 分析:若成行意味气球事件翻篇}{https://www.zaobao.com/news/china/story20230607-1402190}{美国国务卿布林肯传出将在未来几周重启访华,白宫和中国外交部之后相继表态,虽未证实也没有否认。(路透社档案照) 美国国务卿布林肯星期二(6月6日)传出将在未来几周重启访华,白宫和中国外交部之后相继表态,虽未证实也没有否认。受访中国学者评估,布林肯今年2月因气球事件推迟访华,重启访华将意味着气球事件完全翻篇,也表明中美有意继续加强沟通对话为两国紧张降温……}

\entryitemWithDescription{英国指中国关闭秘密警察站 北京敦促英国停止诋毁中国}{https://www.zaobao.com/news/china/story20230607-1402186}{英国安全部部长图根哈特(Tom Tugendhat)称,中国已经关闭在英国各地的``警察服务站''。中国外交部加以否认,并敦促英国停止诋毁中国。 综合法新社、路透社等报道,非政府人权组织``保护卫士''(Safeguard Defenders)去年9月发表报告,指中国在英国境内三个地点设有警察站,旨在提供行政服务,但也被用来``监视和骚扰侨民社区,在某些情况下,强迫人们在合法渠道之外返回中国……}